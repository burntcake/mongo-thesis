\chapter*{Abstract}

We present a study of durability properties in MongoDB focused on equipping users and developers of distributed systems with the necessary understanding and tools to increase the durability of their systems, understand the sources of data loss and empirically evaluate durability and performance tradeoffs of different solutions. As part of this, we contribute a comprehensive categorisation of cases that occur in MongoDB which result in write losses and present an experiment design which is capable of inducing failures in a way that causes write loss to occur. We then introduce the algorithm used to detect write loss using the execution history produced by the experiment and conclude with empirical results showing write loss frequency and quantity along with relevant performance metrics. Using these metrics, we detect (known) durability and performance problems in MongoDB 3.6-rc0 and demonstrate that these were fixed in MongoDB 4.0.  We also offer the means to reason about the latency until the system reaches various levels of write durability, using a novel theoretical approach. The theory additionally allows users to estimate these quantities using client-accessible measurements. Using MongoDB as the foundation, we illustrate the theory and its practical application, by conducting an empirical study of durability in MongoDB based on the theory.
