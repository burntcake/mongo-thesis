\part{Conclusion} \label{p:conc}
\chapter{Conclusion} \label{chap:conclusion}

This thesis explored the concept of durability in MongoDB with the aim of providing system users and designers with the understanding of the sources of data loss and the means to quantitatively measure the frequency and quantity of non-durable writes. This makes it possible to empirically evaluate the tradeoffs between highly-durable and highly-performant configurations. We also equip developers with the tools to reason about the time it takes for a write to reach varying levels of durability and a method by which they can estimate durability using client-accessible measurements.

Through this thesis, we have contributed a categorisation and evaluation for the conditions needed to cause a write durability failure. We found two scenarios that result in a permanent loss of at least one write, one related to MongoDB's election process and another as a consequence of its persistence mechanism. We also uncovered a scenario that leads to a transient loss of a write, where the write appears to be applied, then disappears and reappears once more - all due to the use of the Primary Preferred read preference in a failure scenario.

We have outlined an experiment design capable of inducing the conditions required for write loss and an analysis that can quantitatively evaluate the durability of a distributed storage system by measuring the frequency and quantity of write loss under failures. Having applied this experiment to MongoDB 3.6-rc0 and 4.0, we were able to detect expected write losses in both versions, while also finding durability and performance failures that we did not expect in MongoDB 3.6-rc0. These failures were attributed to known bugs in this version of MongoDB and have since been fixed in 4.0.

Additionally, we defined the concept of $k$-durability and derived a formula for the time a write becomes $k$-durable. We showed that we can estimate 1-durability of an operation using client-accessible measurements and ran experiments to explore when an acknowledged write becomes 1-durable in MongoDB, finding that 1-durability is achieved with probability of 0.9 by 300ms.

\section{Limitations}
We recognise that our work provides only a limited insight into the field of durability in MongoDB. Our experiment harness only induced one failure per experiment, exclusively on the primary replica which prevented us from observing the effects of a failing secondary or the behaviour of the replica set under multiple failures. Our experiment observed and reported only the measurements available to the client, ignoring MongoDB's own logs. This inhibited our capacity to analyse MongoDB's behaviour in detail and to directly observe the types of write failures that occured in the experiment.

Our estimation experiments concentrated on finding time-till-durability, exclusively for the case of 1-durability. This has allowed us to understand the pattern of when a write becomes durable, however limiting the scope to 1-durability meant that we could not find a generalised estimation for a $k$-durable write.

\section{Future Work}
We believe the work outlined in this thesis can be used as the foundation for furthering the exploration of durability in distributed systems. Extending upon our experiments to induce multiple failures on primary and secondary replicas would provide a more holistic view of MongoDB's durability guarantees. Additionally, exploring the behaviour of MongoDB internals by inspecting MongoDB logs could offer a greater insight into how and why certain failures occur, perhaps finding a way to minimise them.

We also believe there is great benefit in exploring how $k$-durability can be estimated for an arbitrary $k$. This can give users and developers a better understanding of how durable their operations are, equipping them with the means to design applications and systems that can account for these failures.

\section{Closing Statements}
Currently, the field of durability in distributed systems is fairly limited in scope and exploration. In this thesis, we have contributed to the greater understanding of durability in theoretical and empirical dimensions. We have offered theoretical and empirical ways of measuring how many failures occur and which categories they fall into. We have contributed a framework that allows users to evaluate tradeoffs between durability and performance. We have provided ways of reasoning about and estimating the time until a given level of durability occurs. Overall, we have equipped developers with the means the ensure their applications can be protected from these durability failures, signalling a change in the way developers think and work around this critical aspect of data storage and safety.
