\chapter{Results} \label{chap:det-results}

In this chapter, we show that our experiment design detects expected write losses in MongoDB 4.0 and 3.6-rc0, while also finding (known) durability and performance bugs in 3.6-rc0, which have been fixed in 4.0. We present the results of our experiments and give quantitative metrics for the frequency and quantity of non-durable writes, along with performance metrics for varying MongoDB configurations. We proceed to evaluate the results of our experiments and conclude with a discussion of their limitations.

\section{Environment}

The environment used for experiments consisted of 3 virtual machines running on a single host. The host has a 2.2Ghz Intel Core i7 processor, with 16GB RAM and a 256GB SSD. It runs Ubuntu 16.04, and uses VirtualBox 5.1.28 to host the virtual machines. 

The three virtual machines were each dedicated 2GB RAM and 20GB of pre-allocated disk space, running Ubuntu Server 16.04.5. They are networked together via a Host-Only network that allows the host and the VMs to communicate with each other at very stable sub-millisecond latencies. Each virtual machine was configured to honour all disk flushes as to ensure parity between which writes MongoDB expected to be persisted and which writes are indeed persisted on disk. The three virtual machines were all installed with the same version of MongoDB and configured to run in a single replica set.

Two of these environments were generated for two versions of MongoDB. The MongoDB versions used were 4.0 and 3.6-rc0. MongoDB 4.0 is the current LTS version of MongoDB and will act as our control measurement, while MongoDB 3.6-rc0 is a version with known durability issues. The reason for testing both is to see whether our experiment can find durability failures and whether the current version of MongoDB has any such failures.

Each failure was induced via a call to VirtualBox's \texttt{VBoxManage}.

All experiments were run from the host, with the host being the sole issuer of queries and write operations to the replica set. The experiment harness is written in .NET Core, using the MongoDB Driver 2.7.0.

We achieve parallel execution by having multiple threads perform these operations, each tracking their own documents and values.

\section{Write Durability}

We ran each experiment for 5 minutes, including the failure at the 100 second point and fixing the failure at the 200 second point. 

Tables \ref{tab:res-1}-\ref{tab:res-8} present the summative results of these experiments. Overall, we see that MongoDB 4.0 performs as predicted by our theoretical analysis. Any write concern that ensures acknowledgement only on the primary does not guarantee durability if the primary fails. This applies to the Primary and Journaled write concerns.

On the other hand, MongoDB 3.6-rc0 is shown to have durability failures even when majority of the replica set have acknowledged the operation. This implies that MongoDB 3.6-rc0 has software flaws in the way it measures acknowledgement from secondary replicas and how it handles recovery and elections after a failure. It should be noted that these durability failures were only seen in experiments utilising a very heavy write load onto the replica set, which means that these durability failures may also be a result of flawed persistence mechanisms. We should add that such write-heavy workloads are unlikely to appear in real world applications, which tend to involve more read operations from the database.

\subsection{Primary Preferred Read Preference}
We also make a note of \prettyref{tab:res-4}, where the Primary Preferred, Majority, Majority case resulted in 9 lost writes. Upon inspection of the execution history, we find that \textit{all} write losses in this experiment were transient. Specifically, they followed the pattern seen in \prettyref{fig:transitive-loss}. This was the only experiment that exhibited transient losses, all other experiments either showed no write loss, or permanent write loss - where a lost write was never seen after being detected as lost.

This type of behaviour is expected, given that this read preference will force the MongoDB client to read from the secondary if the primary is unavailable, instead of waiting for the primary's response. If the client then reads from a secondary that had not yet seen the operation, the data returned from the query will not include it - leading to a loss. However, once a new primary is elected, the client will query only the primary, hence seeing the write again. Fundamentally, this shows that a Primary Preferred read preference removes all guarantees about the data queried from the replica set during a failure, \textit{unless} the write has propagated to all secondaries. Given these results, we recommend that a configuration with a Primary Preferred read preference should always have the write concern configured to "all" - that is, all secondary replicas must acknowledge the write before it is acknowledged to the client.

\begin{figure}
    \begin{CVerbatim}
write x = 1
...
read  x -> x = 1
...
read  x -> x = -1
...
read  x -> x = 1
...
    \end{CVerbatim}
    \label{fig:transitive-loss}
    \caption{Example of a transient loss observed with Primary Preferred read preference}
\end{figure}

\subsection{Journaled vs Primary Write Concern}

We observe that for non-majority-majority experiments, using the Journaled write concern does not always reduce the number of lost writes. The cases which see a reduction are only the write-heavy workloads (write probability 0.7). This implies that Journaled write concern reduces the chances of a write loss as a result of not persisting the write. 

However we see that Journaled write concern experiences increased write-loss compared to experiments with a Primary write concern in a standard write-load workload (write probability 0.3). This is a strong indication that these write losses occur as a result of a rollback due to a reelection, as the operation itself is more likely to be persisted to disk.

As such, we conclude that Journaled Write Concern acts as a tradeoff - it reduces the likelihood of persistence failures but increases the time to send the operation to the secondaries, hence increasing the probability and magnitude of a rollback failure. In contrast, the Primary write concern attempts to minimise communication delay at the cost of not providing guarantees about the persistence of an operation.

\subsection{Write Loss distribution}
Having established the principal causes for write loss, and finding examples of these during the experiments, we now turn our attention to determining \textit{when} write loss occurs. Figures \ref{fig:loss-1} and \ref{fig:loss-2} plot the number of lost writes against the time within the experiment when the write was submitted.

From Figures \ref{fig:loss-1} \& \ref{fig:loss-2} we see that only writes which are acknowledged immediately before or while a failure is being induced are prone to loss. This is intuitive as those operations are the ones in the process of being persisted and being communicated to secondary nodes. 


We should observe the relationship between the write loss distribution and error distribution (Figures \ref{fig:err-1}, \ref{fig:err-2}) of the experiments. Specifically, we note that there are few to no errors present at the time of the failure. This is a direct result of the MongoDB client waiting for responses. The timeout for a response was configured to be 5 seconds. As such, errors spike at around 105 seconds and continue to persist for approximately 20 seconds. This represents the influx of timeout errors from the MongoDB client during the election process for a new primary.

We note that during all experiments, all write losses were found on \textit{unique} documents - no document suffered multiple write losses during a single run of the experiment.

Furthermore, we see a unique pattern of errors in \prettyref{fig:err-2}, where the main peak of errors takes place immediately after the failure is induced. However smaller peaks of errors then tail-off until the failure is fixed and the failed node recovers. The reason for this error can be found by looking at the pattern of write operations for this experiment in Figures \ref{fig:writes-36pmm} \& \ref{fig:reads-36pmm}. 

We notice that there are \textit{no successful write operations} during the entire failure phase of the experiment. This is even more peculiar considering that \textit{some} read operations succeeded during this phase, returning correct values the vast majority of the time. This indicates that MongoDB 3.6-rc0 has a bug in its recovery flows which causes the new elected primary to ignore all write operations, but still accept some read operations. Considering that this behaviour was not seen in any other experiment - either in 3.6-rc0 or 4.0, we can be confident that the bug was fixed in the 4.0 version of MongoDB.

Using these results, we can now explain the tailing-off of errors in \prettyref{fig:err-2}. Soon after the failure is induced, all operations that were submitted to the replica set time out and return as a massive spike in errors. After this initial surge, a new set of write operations and queries is sent to the replica set. Some of the queries return successfuly while all write operations time out. This results in a reduction of total errors reported. The cycle keeps repeating, each time all write operations timing out and only some read operations doing the same. When the failure is fixed and the failed node joins the replica set, all operations start returning successfully, hence dropping the number of errors back to 0. 

\begin{figure}
    %% Creator: Matplotlib, PGF backend
%%
%% To include the figure in your LaTeX document, write
%%   \input{<filename>.pgf}
%%
%% Make sure the required packages are loaded in your preamble
%%   \usepackage{pgf}
%%
%% Figures using additional raster images can only be included by \input if
%% they are in the same directory as the main LaTeX file. For loading figures
%% from other directories you can use the `import` package
%%   \usepackage{import}
%% and then include the figures with
%%   \import{<path to file>}{<filename>.pgf}
%%
%% Matplotlib used the following preamble
%%   \usepackage[utf8x]{inputenc}
%%   \usepackage[T1]{fontenc}
%%   \usepackage{lmodern}
%%
\begingroup%
\makeatletter%
\begin{pgfpicture}%
\pgfpathrectangle{\pgfpointorigin}{\pgfqpoint{6.400000in}{4.800000in}}%
\pgfusepath{use as bounding box, clip}%
\begin{pgfscope}%
\pgfsetbuttcap%
\pgfsetmiterjoin%
\definecolor{currentfill}{rgb}{1.000000,1.000000,1.000000}%
\pgfsetfillcolor{currentfill}%
\pgfsetlinewidth{0.000000pt}%
\definecolor{currentstroke}{rgb}{1.000000,1.000000,1.000000}%
\pgfsetstrokecolor{currentstroke}%
\pgfsetdash{}{0pt}%
\pgfpathmoveto{\pgfqpoint{0.000000in}{0.000000in}}%
\pgfpathlineto{\pgfqpoint{6.400000in}{0.000000in}}%
\pgfpathlineto{\pgfqpoint{6.400000in}{4.800000in}}%
\pgfpathlineto{\pgfqpoint{0.000000in}{4.800000in}}%
\pgfpathclose%
\pgfusepath{fill}%
\end{pgfscope}%
\begin{pgfscope}%
\pgfsetbuttcap%
\pgfsetmiterjoin%
\definecolor{currentfill}{rgb}{1.000000,1.000000,1.000000}%
\pgfsetfillcolor{currentfill}%
\pgfsetlinewidth{0.000000pt}%
\definecolor{currentstroke}{rgb}{0.000000,0.000000,0.000000}%
\pgfsetstrokecolor{currentstroke}%
\pgfsetstrokeopacity{0.000000}%
\pgfsetdash{}{0pt}%
\pgfpathmoveto{\pgfqpoint{0.800000in}{0.528000in}}%
\pgfpathlineto{\pgfqpoint{5.760000in}{0.528000in}}%
\pgfpathlineto{\pgfqpoint{5.760000in}{4.224000in}}%
\pgfpathlineto{\pgfqpoint{0.800000in}{4.224000in}}%
\pgfpathclose%
\pgfusepath{fill}%
\end{pgfscope}%
\begin{pgfscope}%
\pgfsetbuttcap%
\pgfsetroundjoin%
\definecolor{currentfill}{rgb}{0.000000,0.000000,0.000000}%
\pgfsetfillcolor{currentfill}%
\pgfsetlinewidth{0.803000pt}%
\definecolor{currentstroke}{rgb}{0.000000,0.000000,0.000000}%
\pgfsetstrokecolor{currentstroke}%
\pgfsetdash{}{0pt}%
\pgfsys@defobject{currentmarker}{\pgfqpoint{0.000000in}{-0.048611in}}{\pgfqpoint{0.000000in}{0.000000in}}{%
\pgfpathmoveto{\pgfqpoint{0.000000in}{0.000000in}}%
\pgfpathlineto{\pgfqpoint{0.000000in}{-0.048611in}}%
\pgfusepath{stroke,fill}%
}%
\begin{pgfscope}%
\pgfsys@transformshift{1.025455in}{0.528000in}%
\pgfsys@useobject{currentmarker}{}%
\end{pgfscope}%
\end{pgfscope}%
\begin{pgfscope}%
\pgftext[x=1.025455in,y=0.430778in,,top]{\fontsize{11.000000}{13.200000}\selectfont \(\displaystyle 0\)}%
\end{pgfscope}%
\begin{pgfscope}%
\pgfsetbuttcap%
\pgfsetroundjoin%
\definecolor{currentfill}{rgb}{0.000000,0.000000,0.000000}%
\pgfsetfillcolor{currentfill}%
\pgfsetlinewidth{0.803000pt}%
\definecolor{currentstroke}{rgb}{0.000000,0.000000,0.000000}%
\pgfsetstrokecolor{currentstroke}%
\pgfsetdash{}{0pt}%
\pgfsys@defobject{currentmarker}{\pgfqpoint{0.000000in}{-0.048611in}}{\pgfqpoint{0.000000in}{0.000000in}}{%
\pgfpathmoveto{\pgfqpoint{0.000000in}{0.000000in}}%
\pgfpathlineto{\pgfqpoint{0.000000in}{-0.048611in}}%
\pgfusepath{stroke,fill}%
}%
\begin{pgfscope}%
\pgfsys@transformshift{1.762234in}{0.528000in}%
\pgfsys@useobject{currentmarker}{}%
\end{pgfscope}%
\end{pgfscope}%
\begin{pgfscope}%
\pgftext[x=1.762234in,y=0.430778in,,top]{\fontsize{11.000000}{13.200000}\selectfont \(\displaystyle 50\)}%
\end{pgfscope}%
\begin{pgfscope}%
\pgfsetbuttcap%
\pgfsetroundjoin%
\definecolor{currentfill}{rgb}{0.000000,0.000000,0.000000}%
\pgfsetfillcolor{currentfill}%
\pgfsetlinewidth{0.803000pt}%
\definecolor{currentstroke}{rgb}{0.000000,0.000000,0.000000}%
\pgfsetstrokecolor{currentstroke}%
\pgfsetdash{}{0pt}%
\pgfsys@defobject{currentmarker}{\pgfqpoint{0.000000in}{-0.048611in}}{\pgfqpoint{0.000000in}{0.000000in}}{%
\pgfpathmoveto{\pgfqpoint{0.000000in}{0.000000in}}%
\pgfpathlineto{\pgfqpoint{0.000000in}{-0.048611in}}%
\pgfusepath{stroke,fill}%
}%
\begin{pgfscope}%
\pgfsys@transformshift{2.499014in}{0.528000in}%
\pgfsys@useobject{currentmarker}{}%
\end{pgfscope}%
\end{pgfscope}%
\begin{pgfscope}%
\pgftext[x=2.499014in,y=0.430778in,,top]{\fontsize{11.000000}{13.200000}\selectfont \(\displaystyle 100\)}%
\end{pgfscope}%
\begin{pgfscope}%
\pgfsetbuttcap%
\pgfsetroundjoin%
\definecolor{currentfill}{rgb}{0.000000,0.000000,0.000000}%
\pgfsetfillcolor{currentfill}%
\pgfsetlinewidth{0.803000pt}%
\definecolor{currentstroke}{rgb}{0.000000,0.000000,0.000000}%
\pgfsetstrokecolor{currentstroke}%
\pgfsetdash{}{0pt}%
\pgfsys@defobject{currentmarker}{\pgfqpoint{0.000000in}{-0.048611in}}{\pgfqpoint{0.000000in}{0.000000in}}{%
\pgfpathmoveto{\pgfqpoint{0.000000in}{0.000000in}}%
\pgfpathlineto{\pgfqpoint{0.000000in}{-0.048611in}}%
\pgfusepath{stroke,fill}%
}%
\begin{pgfscope}%
\pgfsys@transformshift{3.235793in}{0.528000in}%
\pgfsys@useobject{currentmarker}{}%
\end{pgfscope}%
\end{pgfscope}%
\begin{pgfscope}%
\pgftext[x=3.235793in,y=0.430778in,,top]{\fontsize{11.000000}{13.200000}\selectfont \(\displaystyle 150\)}%
\end{pgfscope}%
\begin{pgfscope}%
\pgfsetbuttcap%
\pgfsetroundjoin%
\definecolor{currentfill}{rgb}{0.000000,0.000000,0.000000}%
\pgfsetfillcolor{currentfill}%
\pgfsetlinewidth{0.803000pt}%
\definecolor{currentstroke}{rgb}{0.000000,0.000000,0.000000}%
\pgfsetstrokecolor{currentstroke}%
\pgfsetdash{}{0pt}%
\pgfsys@defobject{currentmarker}{\pgfqpoint{0.000000in}{-0.048611in}}{\pgfqpoint{0.000000in}{0.000000in}}{%
\pgfpathmoveto{\pgfqpoint{0.000000in}{0.000000in}}%
\pgfpathlineto{\pgfqpoint{0.000000in}{-0.048611in}}%
\pgfusepath{stroke,fill}%
}%
\begin{pgfscope}%
\pgfsys@transformshift{3.972573in}{0.528000in}%
\pgfsys@useobject{currentmarker}{}%
\end{pgfscope}%
\end{pgfscope}%
\begin{pgfscope}%
\pgftext[x=3.972573in,y=0.430778in,,top]{\fontsize{11.000000}{13.200000}\selectfont \(\displaystyle 200\)}%
\end{pgfscope}%
\begin{pgfscope}%
\pgfsetbuttcap%
\pgfsetroundjoin%
\definecolor{currentfill}{rgb}{0.000000,0.000000,0.000000}%
\pgfsetfillcolor{currentfill}%
\pgfsetlinewidth{0.803000pt}%
\definecolor{currentstroke}{rgb}{0.000000,0.000000,0.000000}%
\pgfsetstrokecolor{currentstroke}%
\pgfsetdash{}{0pt}%
\pgfsys@defobject{currentmarker}{\pgfqpoint{0.000000in}{-0.048611in}}{\pgfqpoint{0.000000in}{0.000000in}}{%
\pgfpathmoveto{\pgfqpoint{0.000000in}{0.000000in}}%
\pgfpathlineto{\pgfqpoint{0.000000in}{-0.048611in}}%
\pgfusepath{stroke,fill}%
}%
\begin{pgfscope}%
\pgfsys@transformshift{4.709352in}{0.528000in}%
\pgfsys@useobject{currentmarker}{}%
\end{pgfscope}%
\end{pgfscope}%
\begin{pgfscope}%
\pgftext[x=4.709352in,y=0.430778in,,top]{\fontsize{11.000000}{13.200000}\selectfont \(\displaystyle 250\)}%
\end{pgfscope}%
\begin{pgfscope}%
\pgfsetbuttcap%
\pgfsetroundjoin%
\definecolor{currentfill}{rgb}{0.000000,0.000000,0.000000}%
\pgfsetfillcolor{currentfill}%
\pgfsetlinewidth{0.803000pt}%
\definecolor{currentstroke}{rgb}{0.000000,0.000000,0.000000}%
\pgfsetstrokecolor{currentstroke}%
\pgfsetdash{}{0pt}%
\pgfsys@defobject{currentmarker}{\pgfqpoint{0.000000in}{-0.048611in}}{\pgfqpoint{0.000000in}{0.000000in}}{%
\pgfpathmoveto{\pgfqpoint{0.000000in}{0.000000in}}%
\pgfpathlineto{\pgfqpoint{0.000000in}{-0.048611in}}%
\pgfusepath{stroke,fill}%
}%
\begin{pgfscope}%
\pgfsys@transformshift{5.446132in}{0.528000in}%
\pgfsys@useobject{currentmarker}{}%
\end{pgfscope}%
\end{pgfscope}%
\begin{pgfscope}%
\pgftext[x=5.446132in,y=0.430778in,,top]{\fontsize{11.000000}{13.200000}\selectfont \(\displaystyle 300\)}%
\end{pgfscope}%
\begin{pgfscope}%
\pgftext[x=3.280000in,y=0.240271in,,top]{\fontsize{11.000000}{13.200000}\selectfont Time of experiment (in seconds)}%
\end{pgfscope}%
\begin{pgfscope}%
\pgfsetbuttcap%
\pgfsetroundjoin%
\definecolor{currentfill}{rgb}{0.000000,0.000000,0.000000}%
\pgfsetfillcolor{currentfill}%
\pgfsetlinewidth{0.803000pt}%
\definecolor{currentstroke}{rgb}{0.000000,0.000000,0.000000}%
\pgfsetstrokecolor{currentstroke}%
\pgfsetdash{}{0pt}%
\pgfsys@defobject{currentmarker}{\pgfqpoint{-0.048611in}{0.000000in}}{\pgfqpoint{0.000000in}{0.000000in}}{%
\pgfpathmoveto{\pgfqpoint{0.000000in}{0.000000in}}%
\pgfpathlineto{\pgfqpoint{-0.048611in}{0.000000in}}%
\pgfusepath{stroke,fill}%
}%
\begin{pgfscope}%
\pgfsys@transformshift{0.800000in}{0.696000in}%
\pgfsys@useobject{currentmarker}{}%
\end{pgfscope}%
\end{pgfscope}%
\begin{pgfscope}%
\pgftext[x=0.627981in,y=0.643378in,left,base]{\fontsize{11.000000}{13.200000}\selectfont \(\displaystyle 0\)}%
\end{pgfscope}%
\begin{pgfscope}%
\pgfsetbuttcap%
\pgfsetroundjoin%
\definecolor{currentfill}{rgb}{0.000000,0.000000,0.000000}%
\pgfsetfillcolor{currentfill}%
\pgfsetlinewidth{0.803000pt}%
\definecolor{currentstroke}{rgb}{0.000000,0.000000,0.000000}%
\pgfsetstrokecolor{currentstroke}%
\pgfsetdash{}{0pt}%
\pgfsys@defobject{currentmarker}{\pgfqpoint{-0.048611in}{0.000000in}}{\pgfqpoint{0.000000in}{0.000000in}}{%
\pgfpathmoveto{\pgfqpoint{0.000000in}{0.000000in}}%
\pgfpathlineto{\pgfqpoint{-0.048611in}{0.000000in}}%
\pgfusepath{stroke,fill}%
}%
\begin{pgfscope}%
\pgfsys@transformshift{0.800000in}{1.318222in}%
\pgfsys@useobject{currentmarker}{}%
\end{pgfscope}%
\end{pgfscope}%
\begin{pgfscope}%
\pgftext[x=0.627981in,y=1.265600in,left,base]{\fontsize{11.000000}{13.200000}\selectfont \(\displaystyle 5\)}%
\end{pgfscope}%
\begin{pgfscope}%
\pgfsetbuttcap%
\pgfsetroundjoin%
\definecolor{currentfill}{rgb}{0.000000,0.000000,0.000000}%
\pgfsetfillcolor{currentfill}%
\pgfsetlinewidth{0.803000pt}%
\definecolor{currentstroke}{rgb}{0.000000,0.000000,0.000000}%
\pgfsetstrokecolor{currentstroke}%
\pgfsetdash{}{0pt}%
\pgfsys@defobject{currentmarker}{\pgfqpoint{-0.048611in}{0.000000in}}{\pgfqpoint{0.000000in}{0.000000in}}{%
\pgfpathmoveto{\pgfqpoint{0.000000in}{0.000000in}}%
\pgfpathlineto{\pgfqpoint{-0.048611in}{0.000000in}}%
\pgfusepath{stroke,fill}%
}%
\begin{pgfscope}%
\pgfsys@transformshift{0.800000in}{1.940444in}%
\pgfsys@useobject{currentmarker}{}%
\end{pgfscope}%
\end{pgfscope}%
\begin{pgfscope}%
\pgftext[x=0.553183in,y=1.887822in,left,base]{\fontsize{11.000000}{13.200000}\selectfont \(\displaystyle 10\)}%
\end{pgfscope}%
\begin{pgfscope}%
\pgfsetbuttcap%
\pgfsetroundjoin%
\definecolor{currentfill}{rgb}{0.000000,0.000000,0.000000}%
\pgfsetfillcolor{currentfill}%
\pgfsetlinewidth{0.803000pt}%
\definecolor{currentstroke}{rgb}{0.000000,0.000000,0.000000}%
\pgfsetstrokecolor{currentstroke}%
\pgfsetdash{}{0pt}%
\pgfsys@defobject{currentmarker}{\pgfqpoint{-0.048611in}{0.000000in}}{\pgfqpoint{0.000000in}{0.000000in}}{%
\pgfpathmoveto{\pgfqpoint{0.000000in}{0.000000in}}%
\pgfpathlineto{\pgfqpoint{-0.048611in}{0.000000in}}%
\pgfusepath{stroke,fill}%
}%
\begin{pgfscope}%
\pgfsys@transformshift{0.800000in}{2.562667in}%
\pgfsys@useobject{currentmarker}{}%
\end{pgfscope}%
\end{pgfscope}%
\begin{pgfscope}%
\pgftext[x=0.553183in,y=2.510044in,left,base]{\fontsize{11.000000}{13.200000}\selectfont \(\displaystyle 15\)}%
\end{pgfscope}%
\begin{pgfscope}%
\pgfsetbuttcap%
\pgfsetroundjoin%
\definecolor{currentfill}{rgb}{0.000000,0.000000,0.000000}%
\pgfsetfillcolor{currentfill}%
\pgfsetlinewidth{0.803000pt}%
\definecolor{currentstroke}{rgb}{0.000000,0.000000,0.000000}%
\pgfsetstrokecolor{currentstroke}%
\pgfsetdash{}{0pt}%
\pgfsys@defobject{currentmarker}{\pgfqpoint{-0.048611in}{0.000000in}}{\pgfqpoint{0.000000in}{0.000000in}}{%
\pgfpathmoveto{\pgfqpoint{0.000000in}{0.000000in}}%
\pgfpathlineto{\pgfqpoint{-0.048611in}{0.000000in}}%
\pgfusepath{stroke,fill}%
}%
\begin{pgfscope}%
\pgfsys@transformshift{0.800000in}{3.184889in}%
\pgfsys@useobject{currentmarker}{}%
\end{pgfscope}%
\end{pgfscope}%
\begin{pgfscope}%
\pgftext[x=0.553183in,y=3.132267in,left,base]{\fontsize{11.000000}{13.200000}\selectfont \(\displaystyle 20\)}%
\end{pgfscope}%
\begin{pgfscope}%
\pgfsetbuttcap%
\pgfsetroundjoin%
\definecolor{currentfill}{rgb}{0.000000,0.000000,0.000000}%
\pgfsetfillcolor{currentfill}%
\pgfsetlinewidth{0.803000pt}%
\definecolor{currentstroke}{rgb}{0.000000,0.000000,0.000000}%
\pgfsetstrokecolor{currentstroke}%
\pgfsetdash{}{0pt}%
\pgfsys@defobject{currentmarker}{\pgfqpoint{-0.048611in}{0.000000in}}{\pgfqpoint{0.000000in}{0.000000in}}{%
\pgfpathmoveto{\pgfqpoint{0.000000in}{0.000000in}}%
\pgfpathlineto{\pgfqpoint{-0.048611in}{0.000000in}}%
\pgfusepath{stroke,fill}%
}%
\begin{pgfscope}%
\pgfsys@transformshift{0.800000in}{3.807111in}%
\pgfsys@useobject{currentmarker}{}%
\end{pgfscope}%
\end{pgfscope}%
\begin{pgfscope}%
\pgftext[x=0.553183in,y=3.754489in,left,base]{\fontsize{11.000000}{13.200000}\selectfont \(\displaystyle 25\)}%
\end{pgfscope}%
\begin{pgfscope}%
\pgftext[x=0.497628in,y=2.376000in,,bottom,rotate=90.000000]{\fontsize{11.000000}{13.200000}\selectfont Number of writes lost}%
\end{pgfscope}%
\begin{pgfscope}%
\pgfpathrectangle{\pgfqpoint{0.800000in}{0.528000in}}{\pgfqpoint{4.960000in}{3.696000in}}%
\pgfusepath{clip}%
\pgfsetrectcap%
\pgfsetroundjoin%
\pgfsetlinewidth{1.505625pt}%
\definecolor{currentstroke}{rgb}{0.121569,0.466667,0.705882}%
\pgfsetstrokecolor{currentstroke}%
\pgfsetdash{}{0pt}%
\pgfpathmoveto{\pgfqpoint{1.025455in}{0.696000in}}%
\pgfpathlineto{\pgfqpoint{2.469542in}{0.696000in}}%
\pgfpathlineto{\pgfqpoint{2.484278in}{2.189333in}}%
\pgfpathlineto{\pgfqpoint{2.499014in}{4.056000in}}%
\pgfpathlineto{\pgfqpoint{2.513749in}{2.064889in}}%
\pgfpathlineto{\pgfqpoint{2.528485in}{0.696000in}}%
\pgfpathlineto{\pgfqpoint{5.534545in}{0.696000in}}%
\pgfpathlineto{\pgfqpoint{5.534545in}{0.696000in}}%
\pgfusepath{stroke}%
\end{pgfscope}%
\begin{pgfscope}%
\pgfsetrectcap%
\pgfsetmiterjoin%
\pgfsetlinewidth{0.803000pt}%
\definecolor{currentstroke}{rgb}{0.000000,0.000000,0.000000}%
\pgfsetstrokecolor{currentstroke}%
\pgfsetdash{}{0pt}%
\pgfpathmoveto{\pgfqpoint{0.800000in}{0.528000in}}%
\pgfpathlineto{\pgfqpoint{0.800000in}{4.224000in}}%
\pgfusepath{stroke}%
\end{pgfscope}%
\begin{pgfscope}%
\pgfsetrectcap%
\pgfsetmiterjoin%
\pgfsetlinewidth{0.803000pt}%
\definecolor{currentstroke}{rgb}{0.000000,0.000000,0.000000}%
\pgfsetstrokecolor{currentstroke}%
\pgfsetdash{}{0pt}%
\pgfpathmoveto{\pgfqpoint{5.760000in}{0.528000in}}%
\pgfpathlineto{\pgfqpoint{5.760000in}{4.224000in}}%
\pgfusepath{stroke}%
\end{pgfscope}%
\begin{pgfscope}%
\pgfsetrectcap%
\pgfsetmiterjoin%
\pgfsetlinewidth{0.803000pt}%
\definecolor{currentstroke}{rgb}{0.000000,0.000000,0.000000}%
\pgfsetstrokecolor{currentstroke}%
\pgfsetdash{}{0pt}%
\pgfpathmoveto{\pgfqpoint{0.800000in}{0.528000in}}%
\pgfpathlineto{\pgfqpoint{5.760000in}{0.528000in}}%
\pgfusepath{stroke}%
\end{pgfscope}%
\begin{pgfscope}%
\pgfsetrectcap%
\pgfsetmiterjoin%
\pgfsetlinewidth{0.803000pt}%
\definecolor{currentstroke}{rgb}{0.000000,0.000000,0.000000}%
\pgfsetstrokecolor{currentstroke}%
\pgfsetdash{}{0pt}%
\pgfpathmoveto{\pgfqpoint{0.800000in}{4.224000in}}%
\pgfpathlineto{\pgfqpoint{5.760000in}{4.224000in}}%
\pgfusepath{stroke}%
\end{pgfscope}%
\end{pgfpicture}%
\makeatother%
\endgroup%

    \caption{Distribution of write loss for every second of experiment 3 in \prettyref{tab:res-6}}
    \label{fig:loss-1}
\end{figure}

\begin{figure}
    \input{images/poweroff_writes.pgf}
    \caption{Distribution of write loss for every second of experiment 3 in \prettyref{tab:res-8}}
    \label{fig:loss-2}
\end{figure}

\begin{figure}
    %% Creator: Matplotlib, PGF backend
%%
%% To include the figure in your LaTeX document, write
%%   \input{<filename>.pgf}
%%
%% Make sure the required packages are loaded in your preamble
%%   \usepackage{pgf}
%%
%% Figures using additional raster images can only be included by \input if
%% they are in the same directory as the main LaTeX file. For loading figures
%% from other directories you can use the `import` package
%%   \usepackage{import}
%% and then include the figures with
%%   \import{<path to file>}{<filename>.pgf}
%%
%% Matplotlib used the following preamble
%%   \usepackage[utf8x]{inputenc}
%%   \usepackage[T1]{fontenc}
%%   \usepackage{lmodern}
%%
\begingroup%
\makeatletter%
\begin{pgfpicture}%
\pgfpathrectangle{\pgfpointorigin}{\pgfqpoint{6.400000in}{4.800000in}}%
\pgfusepath{use as bounding box, clip}%
\begin{pgfscope}%
\pgfsetbuttcap%
\pgfsetmiterjoin%
\definecolor{currentfill}{rgb}{1.000000,1.000000,1.000000}%
\pgfsetfillcolor{currentfill}%
\pgfsetlinewidth{0.000000pt}%
\definecolor{currentstroke}{rgb}{1.000000,1.000000,1.000000}%
\pgfsetstrokecolor{currentstroke}%
\pgfsetdash{}{0pt}%
\pgfpathmoveto{\pgfqpoint{0.000000in}{0.000000in}}%
\pgfpathlineto{\pgfqpoint{6.400000in}{0.000000in}}%
\pgfpathlineto{\pgfqpoint{6.400000in}{4.800000in}}%
\pgfpathlineto{\pgfqpoint{0.000000in}{4.800000in}}%
\pgfpathclose%
\pgfusepath{fill}%
\end{pgfscope}%
\begin{pgfscope}%
\pgfsetbuttcap%
\pgfsetmiterjoin%
\definecolor{currentfill}{rgb}{1.000000,1.000000,1.000000}%
\pgfsetfillcolor{currentfill}%
\pgfsetlinewidth{0.000000pt}%
\definecolor{currentstroke}{rgb}{0.000000,0.000000,0.000000}%
\pgfsetstrokecolor{currentstroke}%
\pgfsetstrokeopacity{0.000000}%
\pgfsetdash{}{0pt}%
\pgfpathmoveto{\pgfqpoint{0.800000in}{0.528000in}}%
\pgfpathlineto{\pgfqpoint{5.760000in}{0.528000in}}%
\pgfpathlineto{\pgfqpoint{5.760000in}{4.224000in}}%
\pgfpathlineto{\pgfqpoint{0.800000in}{4.224000in}}%
\pgfpathclose%
\pgfusepath{fill}%
\end{pgfscope}%
\begin{pgfscope}%
\pgfsetbuttcap%
\pgfsetroundjoin%
\definecolor{currentfill}{rgb}{0.000000,0.000000,0.000000}%
\pgfsetfillcolor{currentfill}%
\pgfsetlinewidth{0.803000pt}%
\definecolor{currentstroke}{rgb}{0.000000,0.000000,0.000000}%
\pgfsetstrokecolor{currentstroke}%
\pgfsetdash{}{0pt}%
\pgfsys@defobject{currentmarker}{\pgfqpoint{0.000000in}{-0.048611in}}{\pgfqpoint{0.000000in}{0.000000in}}{%
\pgfpathmoveto{\pgfqpoint{0.000000in}{0.000000in}}%
\pgfpathlineto{\pgfqpoint{0.000000in}{-0.048611in}}%
\pgfusepath{stroke,fill}%
}%
\begin{pgfscope}%
\pgfsys@transformshift{1.025455in}{0.528000in}%
\pgfsys@useobject{currentmarker}{}%
\end{pgfscope}%
\end{pgfscope}%
\begin{pgfscope}%
\pgftext[x=1.025455in,y=0.430778in,,top]{\fontsize{11.000000}{13.200000}\selectfont \(\displaystyle 0\)}%
\end{pgfscope}%
\begin{pgfscope}%
\pgfsetbuttcap%
\pgfsetroundjoin%
\definecolor{currentfill}{rgb}{0.000000,0.000000,0.000000}%
\pgfsetfillcolor{currentfill}%
\pgfsetlinewidth{0.803000pt}%
\definecolor{currentstroke}{rgb}{0.000000,0.000000,0.000000}%
\pgfsetstrokecolor{currentstroke}%
\pgfsetdash{}{0pt}%
\pgfsys@defobject{currentmarker}{\pgfqpoint{0.000000in}{-0.048611in}}{\pgfqpoint{0.000000in}{0.000000in}}{%
\pgfpathmoveto{\pgfqpoint{0.000000in}{0.000000in}}%
\pgfpathlineto{\pgfqpoint{0.000000in}{-0.048611in}}%
\pgfusepath{stroke,fill}%
}%
\begin{pgfscope}%
\pgfsys@transformshift{1.769529in}{0.528000in}%
\pgfsys@useobject{currentmarker}{}%
\end{pgfscope}%
\end{pgfscope}%
\begin{pgfscope}%
\pgftext[x=1.769529in,y=0.430778in,,top]{\fontsize{11.000000}{13.200000}\selectfont \(\displaystyle 50\)}%
\end{pgfscope}%
\begin{pgfscope}%
\pgfsetbuttcap%
\pgfsetroundjoin%
\definecolor{currentfill}{rgb}{0.000000,0.000000,0.000000}%
\pgfsetfillcolor{currentfill}%
\pgfsetlinewidth{0.803000pt}%
\definecolor{currentstroke}{rgb}{0.000000,0.000000,0.000000}%
\pgfsetstrokecolor{currentstroke}%
\pgfsetdash{}{0pt}%
\pgfsys@defobject{currentmarker}{\pgfqpoint{0.000000in}{-0.048611in}}{\pgfqpoint{0.000000in}{0.000000in}}{%
\pgfpathmoveto{\pgfqpoint{0.000000in}{0.000000in}}%
\pgfpathlineto{\pgfqpoint{0.000000in}{-0.048611in}}%
\pgfusepath{stroke,fill}%
}%
\begin{pgfscope}%
\pgfsys@transformshift{2.513603in}{0.528000in}%
\pgfsys@useobject{currentmarker}{}%
\end{pgfscope}%
\end{pgfscope}%
\begin{pgfscope}%
\pgftext[x=2.513603in,y=0.430778in,,top]{\fontsize{11.000000}{13.200000}\selectfont \(\displaystyle 100\)}%
\end{pgfscope}%
\begin{pgfscope}%
\pgfsetbuttcap%
\pgfsetroundjoin%
\definecolor{currentfill}{rgb}{0.000000,0.000000,0.000000}%
\pgfsetfillcolor{currentfill}%
\pgfsetlinewidth{0.803000pt}%
\definecolor{currentstroke}{rgb}{0.000000,0.000000,0.000000}%
\pgfsetstrokecolor{currentstroke}%
\pgfsetdash{}{0pt}%
\pgfsys@defobject{currentmarker}{\pgfqpoint{0.000000in}{-0.048611in}}{\pgfqpoint{0.000000in}{0.000000in}}{%
\pgfpathmoveto{\pgfqpoint{0.000000in}{0.000000in}}%
\pgfpathlineto{\pgfqpoint{0.000000in}{-0.048611in}}%
\pgfusepath{stroke,fill}%
}%
\begin{pgfscope}%
\pgfsys@transformshift{3.257678in}{0.528000in}%
\pgfsys@useobject{currentmarker}{}%
\end{pgfscope}%
\end{pgfscope}%
\begin{pgfscope}%
\pgftext[x=3.257678in,y=0.430778in,,top]{\fontsize{11.000000}{13.200000}\selectfont \(\displaystyle 150\)}%
\end{pgfscope}%
\begin{pgfscope}%
\pgfsetbuttcap%
\pgfsetroundjoin%
\definecolor{currentfill}{rgb}{0.000000,0.000000,0.000000}%
\pgfsetfillcolor{currentfill}%
\pgfsetlinewidth{0.803000pt}%
\definecolor{currentstroke}{rgb}{0.000000,0.000000,0.000000}%
\pgfsetstrokecolor{currentstroke}%
\pgfsetdash{}{0pt}%
\pgfsys@defobject{currentmarker}{\pgfqpoint{0.000000in}{-0.048611in}}{\pgfqpoint{0.000000in}{0.000000in}}{%
\pgfpathmoveto{\pgfqpoint{0.000000in}{0.000000in}}%
\pgfpathlineto{\pgfqpoint{0.000000in}{-0.048611in}}%
\pgfusepath{stroke,fill}%
}%
\begin{pgfscope}%
\pgfsys@transformshift{4.001752in}{0.528000in}%
\pgfsys@useobject{currentmarker}{}%
\end{pgfscope}%
\end{pgfscope}%
\begin{pgfscope}%
\pgftext[x=4.001752in,y=0.430778in,,top]{\fontsize{11.000000}{13.200000}\selectfont \(\displaystyle 200\)}%
\end{pgfscope}%
\begin{pgfscope}%
\pgfsetbuttcap%
\pgfsetroundjoin%
\definecolor{currentfill}{rgb}{0.000000,0.000000,0.000000}%
\pgfsetfillcolor{currentfill}%
\pgfsetlinewidth{0.803000pt}%
\definecolor{currentstroke}{rgb}{0.000000,0.000000,0.000000}%
\pgfsetstrokecolor{currentstroke}%
\pgfsetdash{}{0pt}%
\pgfsys@defobject{currentmarker}{\pgfqpoint{0.000000in}{-0.048611in}}{\pgfqpoint{0.000000in}{0.000000in}}{%
\pgfpathmoveto{\pgfqpoint{0.000000in}{0.000000in}}%
\pgfpathlineto{\pgfqpoint{0.000000in}{-0.048611in}}%
\pgfusepath{stroke,fill}%
}%
\begin{pgfscope}%
\pgfsys@transformshift{4.745827in}{0.528000in}%
\pgfsys@useobject{currentmarker}{}%
\end{pgfscope}%
\end{pgfscope}%
\begin{pgfscope}%
\pgftext[x=4.745827in,y=0.430778in,,top]{\fontsize{11.000000}{13.200000}\selectfont \(\displaystyle 250\)}%
\end{pgfscope}%
\begin{pgfscope}%
\pgfsetbuttcap%
\pgfsetroundjoin%
\definecolor{currentfill}{rgb}{0.000000,0.000000,0.000000}%
\pgfsetfillcolor{currentfill}%
\pgfsetlinewidth{0.803000pt}%
\definecolor{currentstroke}{rgb}{0.000000,0.000000,0.000000}%
\pgfsetstrokecolor{currentstroke}%
\pgfsetdash{}{0pt}%
\pgfsys@defobject{currentmarker}{\pgfqpoint{0.000000in}{-0.048611in}}{\pgfqpoint{0.000000in}{0.000000in}}{%
\pgfpathmoveto{\pgfqpoint{0.000000in}{0.000000in}}%
\pgfpathlineto{\pgfqpoint{0.000000in}{-0.048611in}}%
\pgfusepath{stroke,fill}%
}%
\begin{pgfscope}%
\pgfsys@transformshift{5.489901in}{0.528000in}%
\pgfsys@useobject{currentmarker}{}%
\end{pgfscope}%
\end{pgfscope}%
\begin{pgfscope}%
\pgftext[x=5.489901in,y=0.430778in,,top]{\fontsize{11.000000}{13.200000}\selectfont \(\displaystyle 300\)}%
\end{pgfscope}%
\begin{pgfscope}%
\pgftext[x=3.280000in,y=0.240271in,,top]{\fontsize{11.000000}{13.200000}\selectfont Time of experiment (in seconds)}%
\end{pgfscope}%
\begin{pgfscope}%
\pgfsetbuttcap%
\pgfsetroundjoin%
\definecolor{currentfill}{rgb}{0.000000,0.000000,0.000000}%
\pgfsetfillcolor{currentfill}%
\pgfsetlinewidth{0.803000pt}%
\definecolor{currentstroke}{rgb}{0.000000,0.000000,0.000000}%
\pgfsetstrokecolor{currentstroke}%
\pgfsetdash{}{0pt}%
\pgfsys@defobject{currentmarker}{\pgfqpoint{-0.048611in}{0.000000in}}{\pgfqpoint{0.000000in}{0.000000in}}{%
\pgfpathmoveto{\pgfqpoint{0.000000in}{0.000000in}}%
\pgfpathlineto{\pgfqpoint{-0.048611in}{0.000000in}}%
\pgfusepath{stroke,fill}%
}%
\begin{pgfscope}%
\pgfsys@transformshift{0.800000in}{0.696000in}%
\pgfsys@useobject{currentmarker}{}%
\end{pgfscope}%
\end{pgfscope}%
\begin{pgfscope}%
\pgftext[x=0.627981in,y=0.643378in,left,base]{\fontsize{11.000000}{13.200000}\selectfont \(\displaystyle 0\)}%
\end{pgfscope}%
\begin{pgfscope}%
\pgfsetbuttcap%
\pgfsetroundjoin%
\definecolor{currentfill}{rgb}{0.000000,0.000000,0.000000}%
\pgfsetfillcolor{currentfill}%
\pgfsetlinewidth{0.803000pt}%
\definecolor{currentstroke}{rgb}{0.000000,0.000000,0.000000}%
\pgfsetstrokecolor{currentstroke}%
\pgfsetdash{}{0pt}%
\pgfsys@defobject{currentmarker}{\pgfqpoint{-0.048611in}{0.000000in}}{\pgfqpoint{0.000000in}{0.000000in}}{%
\pgfpathmoveto{\pgfqpoint{0.000000in}{0.000000in}}%
\pgfpathlineto{\pgfqpoint{-0.048611in}{0.000000in}}%
\pgfusepath{stroke,fill}%
}%
\begin{pgfscope}%
\pgfsys@transformshift{0.800000in}{1.318799in}%
\pgfsys@useobject{currentmarker}{}%
\end{pgfscope}%
\end{pgfscope}%
\begin{pgfscope}%
\pgftext[x=0.403588in,y=1.266177in,left,base]{\fontsize{11.000000}{13.200000}\selectfont \(\displaystyle 2000\)}%
\end{pgfscope}%
\begin{pgfscope}%
\pgfsetbuttcap%
\pgfsetroundjoin%
\definecolor{currentfill}{rgb}{0.000000,0.000000,0.000000}%
\pgfsetfillcolor{currentfill}%
\pgfsetlinewidth{0.803000pt}%
\definecolor{currentstroke}{rgb}{0.000000,0.000000,0.000000}%
\pgfsetstrokecolor{currentstroke}%
\pgfsetdash{}{0pt}%
\pgfsys@defobject{currentmarker}{\pgfqpoint{-0.048611in}{0.000000in}}{\pgfqpoint{0.000000in}{0.000000in}}{%
\pgfpathmoveto{\pgfqpoint{0.000000in}{0.000000in}}%
\pgfpathlineto{\pgfqpoint{-0.048611in}{0.000000in}}%
\pgfusepath{stroke,fill}%
}%
\begin{pgfscope}%
\pgfsys@transformshift{0.800000in}{1.941598in}%
\pgfsys@useobject{currentmarker}{}%
\end{pgfscope}%
\end{pgfscope}%
\begin{pgfscope}%
\pgftext[x=0.403588in,y=1.888976in,left,base]{\fontsize{11.000000}{13.200000}\selectfont \(\displaystyle 4000\)}%
\end{pgfscope}%
\begin{pgfscope}%
\pgfsetbuttcap%
\pgfsetroundjoin%
\definecolor{currentfill}{rgb}{0.000000,0.000000,0.000000}%
\pgfsetfillcolor{currentfill}%
\pgfsetlinewidth{0.803000pt}%
\definecolor{currentstroke}{rgb}{0.000000,0.000000,0.000000}%
\pgfsetstrokecolor{currentstroke}%
\pgfsetdash{}{0pt}%
\pgfsys@defobject{currentmarker}{\pgfqpoint{-0.048611in}{0.000000in}}{\pgfqpoint{0.000000in}{0.000000in}}{%
\pgfpathmoveto{\pgfqpoint{0.000000in}{0.000000in}}%
\pgfpathlineto{\pgfqpoint{-0.048611in}{0.000000in}}%
\pgfusepath{stroke,fill}%
}%
\begin{pgfscope}%
\pgfsys@transformshift{0.800000in}{2.564397in}%
\pgfsys@useobject{currentmarker}{}%
\end{pgfscope}%
\end{pgfscope}%
\begin{pgfscope}%
\pgftext[x=0.403588in,y=2.511774in,left,base]{\fontsize{11.000000}{13.200000}\selectfont \(\displaystyle 6000\)}%
\end{pgfscope}%
\begin{pgfscope}%
\pgfsetbuttcap%
\pgfsetroundjoin%
\definecolor{currentfill}{rgb}{0.000000,0.000000,0.000000}%
\pgfsetfillcolor{currentfill}%
\pgfsetlinewidth{0.803000pt}%
\definecolor{currentstroke}{rgb}{0.000000,0.000000,0.000000}%
\pgfsetstrokecolor{currentstroke}%
\pgfsetdash{}{0pt}%
\pgfsys@defobject{currentmarker}{\pgfqpoint{-0.048611in}{0.000000in}}{\pgfqpoint{0.000000in}{0.000000in}}{%
\pgfpathmoveto{\pgfqpoint{0.000000in}{0.000000in}}%
\pgfpathlineto{\pgfqpoint{-0.048611in}{0.000000in}}%
\pgfusepath{stroke,fill}%
}%
\begin{pgfscope}%
\pgfsys@transformshift{0.800000in}{3.187196in}%
\pgfsys@useobject{currentmarker}{}%
\end{pgfscope}%
\end{pgfscope}%
\begin{pgfscope}%
\pgftext[x=0.403588in,y=3.134573in,left,base]{\fontsize{11.000000}{13.200000}\selectfont \(\displaystyle 8000\)}%
\end{pgfscope}%
\begin{pgfscope}%
\pgfsetbuttcap%
\pgfsetroundjoin%
\definecolor{currentfill}{rgb}{0.000000,0.000000,0.000000}%
\pgfsetfillcolor{currentfill}%
\pgfsetlinewidth{0.803000pt}%
\definecolor{currentstroke}{rgb}{0.000000,0.000000,0.000000}%
\pgfsetstrokecolor{currentstroke}%
\pgfsetdash{}{0pt}%
\pgfsys@defobject{currentmarker}{\pgfqpoint{-0.048611in}{0.000000in}}{\pgfqpoint{0.000000in}{0.000000in}}{%
\pgfpathmoveto{\pgfqpoint{0.000000in}{0.000000in}}%
\pgfpathlineto{\pgfqpoint{-0.048611in}{0.000000in}}%
\pgfusepath{stroke,fill}%
}%
\begin{pgfscope}%
\pgfsys@transformshift{0.800000in}{3.809994in}%
\pgfsys@useobject{currentmarker}{}%
\end{pgfscope}%
\end{pgfscope}%
\begin{pgfscope}%
\pgftext[x=0.328791in,y=3.757372in,left,base]{\fontsize{11.000000}{13.200000}\selectfont \(\displaystyle 10000\)}%
\end{pgfscope}%
\begin{pgfscope}%
\pgftext[x=0.273236in,y=2.376000in,,bottom,rotate=90.000000]{\fontsize{11.000000}{13.200000}\selectfont Number of errors}%
\end{pgfscope}%
\begin{pgfscope}%
\pgfpathrectangle{\pgfqpoint{0.800000in}{0.528000in}}{\pgfqpoint{4.960000in}{3.696000in}}%
\pgfusepath{clip}%
\pgfsetrectcap%
\pgfsetroundjoin%
\pgfsetlinewidth{1.505625pt}%
\definecolor{currentstroke}{rgb}{0.121569,0.466667,0.705882}%
\pgfsetstrokecolor{currentstroke}%
\pgfsetdash{}{0pt}%
\pgfpathmoveto{\pgfqpoint{1.025455in}{0.696000in}}%
\pgfpathlineto{\pgfqpoint{2.513603in}{0.696000in}}%
\pgfpathlineto{\pgfqpoint{2.528485in}{1.066254in}}%
\pgfpathlineto{\pgfqpoint{2.543366in}{2.844345in}}%
\pgfpathlineto{\pgfqpoint{2.558248in}{3.708790in}}%
\pgfpathlineto{\pgfqpoint{2.573129in}{3.643396in}}%
\pgfpathlineto{\pgfqpoint{2.588011in}{3.802832in}}%
\pgfpathlineto{\pgfqpoint{2.602892in}{3.741798in}}%
\pgfpathlineto{\pgfqpoint{2.617774in}{3.814977in}}%
\pgfpathlineto{\pgfqpoint{2.647537in}{4.056000in}}%
\pgfpathlineto{\pgfqpoint{2.662418in}{4.015518in}}%
\pgfpathlineto{\pgfqpoint{2.677300in}{3.928638in}}%
\pgfpathlineto{\pgfqpoint{2.692181in}{3.909954in}}%
\pgfpathlineto{\pgfqpoint{2.707063in}{3.894695in}}%
\pgfpathlineto{\pgfqpoint{2.721944in}{3.979084in}}%
\pgfpathlineto{\pgfqpoint{2.736826in}{3.903726in}}%
\pgfpathlineto{\pgfqpoint{2.751707in}{3.582673in}}%
\pgfpathlineto{\pgfqpoint{2.766589in}{3.805323in}}%
\pgfpathlineto{\pgfqpoint{2.781470in}{1.339663in}}%
\pgfpathlineto{\pgfqpoint{2.796352in}{0.696000in}}%
\pgfpathlineto{\pgfqpoint{5.534545in}{0.696000in}}%
\pgfpathlineto{\pgfqpoint{5.534545in}{0.696000in}}%
\pgfusepath{stroke}%
\end{pgfscope}%
\begin{pgfscope}%
\pgfsetrectcap%
\pgfsetmiterjoin%
\pgfsetlinewidth{0.803000pt}%
\definecolor{currentstroke}{rgb}{0.000000,0.000000,0.000000}%
\pgfsetstrokecolor{currentstroke}%
\pgfsetdash{}{0pt}%
\pgfpathmoveto{\pgfqpoint{0.800000in}{0.528000in}}%
\pgfpathlineto{\pgfqpoint{0.800000in}{4.224000in}}%
\pgfusepath{stroke}%
\end{pgfscope}%
\begin{pgfscope}%
\pgfsetrectcap%
\pgfsetmiterjoin%
\pgfsetlinewidth{0.803000pt}%
\definecolor{currentstroke}{rgb}{0.000000,0.000000,0.000000}%
\pgfsetstrokecolor{currentstroke}%
\pgfsetdash{}{0pt}%
\pgfpathmoveto{\pgfqpoint{5.760000in}{0.528000in}}%
\pgfpathlineto{\pgfqpoint{5.760000in}{4.224000in}}%
\pgfusepath{stroke}%
\end{pgfscope}%
\begin{pgfscope}%
\pgfsetrectcap%
\pgfsetmiterjoin%
\pgfsetlinewidth{0.803000pt}%
\definecolor{currentstroke}{rgb}{0.000000,0.000000,0.000000}%
\pgfsetstrokecolor{currentstroke}%
\pgfsetdash{}{0pt}%
\pgfpathmoveto{\pgfqpoint{0.800000in}{0.528000in}}%
\pgfpathlineto{\pgfqpoint{5.760000in}{0.528000in}}%
\pgfusepath{stroke}%
\end{pgfscope}%
\begin{pgfscope}%
\pgfsetrectcap%
\pgfsetmiterjoin%
\pgfsetlinewidth{0.803000pt}%
\definecolor{currentstroke}{rgb}{0.000000,0.000000,0.000000}%
\pgfsetstrokecolor{currentstroke}%
\pgfsetdash{}{0pt}%
\pgfpathmoveto{\pgfqpoint{0.800000in}{4.224000in}}%
\pgfpathlineto{\pgfqpoint{5.760000in}{4.224000in}}%
\pgfusepath{stroke}%
\end{pgfscope}%
\end{pgfpicture}%
\makeatother%
\endgroup%

    \caption{Distribution of errors for every second of experiment 3 in \prettyref{tab:res-6}}
    \label{fig:err-1}
\end{figure}

\begin{figure}
    %% Creator: Matplotlib, PGF backend
%%
%% To include the figure in your LaTeX document, write
%%   \input{<filename>.pgf}
%%
%% Make sure the required packages are loaded in your preamble
%%   \usepackage{pgf}
%%
%% Figures using additional raster images can only be included by \input if
%% they are in the same directory as the main LaTeX file. For loading figures
%% from other directories you can use the `import` package
%%   \usepackage{import}
%% and then include the figures with
%%   \import{<path to file>}{<filename>.pgf}
%%
%% Matplotlib used the following preamble
%%   \usepackage[utf8x]{inputenc}
%%   \usepackage[T1]{fontenc}
%%   \usepackage{lmodern}
%%
\begingroup%
\makeatletter%
\begin{pgfpicture}%
\pgfpathrectangle{\pgfpointorigin}{\pgfqpoint{6.400000in}{4.800000in}}%
\pgfusepath{use as bounding box, clip}%
\begin{pgfscope}%
\pgfsetbuttcap%
\pgfsetmiterjoin%
\definecolor{currentfill}{rgb}{1.000000,1.000000,1.000000}%
\pgfsetfillcolor{currentfill}%
\pgfsetlinewidth{0.000000pt}%
\definecolor{currentstroke}{rgb}{1.000000,1.000000,1.000000}%
\pgfsetstrokecolor{currentstroke}%
\pgfsetdash{}{0pt}%
\pgfpathmoveto{\pgfqpoint{0.000000in}{0.000000in}}%
\pgfpathlineto{\pgfqpoint{6.400000in}{0.000000in}}%
\pgfpathlineto{\pgfqpoint{6.400000in}{4.800000in}}%
\pgfpathlineto{\pgfqpoint{0.000000in}{4.800000in}}%
\pgfpathclose%
\pgfusepath{fill}%
\end{pgfscope}%
\begin{pgfscope}%
\pgfsetbuttcap%
\pgfsetmiterjoin%
\definecolor{currentfill}{rgb}{1.000000,1.000000,1.000000}%
\pgfsetfillcolor{currentfill}%
\pgfsetlinewidth{0.000000pt}%
\definecolor{currentstroke}{rgb}{0.000000,0.000000,0.000000}%
\pgfsetstrokecolor{currentstroke}%
\pgfsetstrokeopacity{0.000000}%
\pgfsetdash{}{0pt}%
\pgfpathmoveto{\pgfqpoint{0.800000in}{0.528000in}}%
\pgfpathlineto{\pgfqpoint{5.760000in}{0.528000in}}%
\pgfpathlineto{\pgfqpoint{5.760000in}{4.224000in}}%
\pgfpathlineto{\pgfqpoint{0.800000in}{4.224000in}}%
\pgfpathclose%
\pgfusepath{fill}%
\end{pgfscope}%
\begin{pgfscope}%
\pgfsetbuttcap%
\pgfsetroundjoin%
\definecolor{currentfill}{rgb}{0.000000,0.000000,0.000000}%
\pgfsetfillcolor{currentfill}%
\pgfsetlinewidth{0.803000pt}%
\definecolor{currentstroke}{rgb}{0.000000,0.000000,0.000000}%
\pgfsetstrokecolor{currentstroke}%
\pgfsetdash{}{0pt}%
\pgfsys@defobject{currentmarker}{\pgfqpoint{0.000000in}{-0.048611in}}{\pgfqpoint{0.000000in}{0.000000in}}{%
\pgfpathmoveto{\pgfqpoint{0.000000in}{0.000000in}}%
\pgfpathlineto{\pgfqpoint{0.000000in}{-0.048611in}}%
\pgfusepath{stroke,fill}%
}%
\begin{pgfscope}%
\pgfsys@transformshift{1.025455in}{0.528000in}%
\pgfsys@useobject{currentmarker}{}%
\end{pgfscope}%
\end{pgfscope}%
\begin{pgfscope}%
\pgftext[x=1.025455in,y=0.430778in,,top]{\fontsize{11.000000}{13.200000}\selectfont \(\displaystyle 0\)}%
\end{pgfscope}%
\begin{pgfscope}%
\pgfsetbuttcap%
\pgfsetroundjoin%
\definecolor{currentfill}{rgb}{0.000000,0.000000,0.000000}%
\pgfsetfillcolor{currentfill}%
\pgfsetlinewidth{0.803000pt}%
\definecolor{currentstroke}{rgb}{0.000000,0.000000,0.000000}%
\pgfsetstrokecolor{currentstroke}%
\pgfsetdash{}{0pt}%
\pgfsys@defobject{currentmarker}{\pgfqpoint{0.000000in}{-0.048611in}}{\pgfqpoint{0.000000in}{0.000000in}}{%
\pgfpathmoveto{\pgfqpoint{0.000000in}{0.000000in}}%
\pgfpathlineto{\pgfqpoint{0.000000in}{-0.048611in}}%
\pgfusepath{stroke,fill}%
}%
\begin{pgfscope}%
\pgfsys@transformshift{1.769529in}{0.528000in}%
\pgfsys@useobject{currentmarker}{}%
\end{pgfscope}%
\end{pgfscope}%
\begin{pgfscope}%
\pgftext[x=1.769529in,y=0.430778in,,top]{\fontsize{11.000000}{13.200000}\selectfont \(\displaystyle 50\)}%
\end{pgfscope}%
\begin{pgfscope}%
\pgfsetbuttcap%
\pgfsetroundjoin%
\definecolor{currentfill}{rgb}{0.000000,0.000000,0.000000}%
\pgfsetfillcolor{currentfill}%
\pgfsetlinewidth{0.803000pt}%
\definecolor{currentstroke}{rgb}{0.000000,0.000000,0.000000}%
\pgfsetstrokecolor{currentstroke}%
\pgfsetdash{}{0pt}%
\pgfsys@defobject{currentmarker}{\pgfqpoint{0.000000in}{-0.048611in}}{\pgfqpoint{0.000000in}{0.000000in}}{%
\pgfpathmoveto{\pgfqpoint{0.000000in}{0.000000in}}%
\pgfpathlineto{\pgfqpoint{0.000000in}{-0.048611in}}%
\pgfusepath{stroke,fill}%
}%
\begin{pgfscope}%
\pgfsys@transformshift{2.513603in}{0.528000in}%
\pgfsys@useobject{currentmarker}{}%
\end{pgfscope}%
\end{pgfscope}%
\begin{pgfscope}%
\pgftext[x=2.513603in,y=0.430778in,,top]{\fontsize{11.000000}{13.200000}\selectfont \(\displaystyle 100\)}%
\end{pgfscope}%
\begin{pgfscope}%
\pgfsetbuttcap%
\pgfsetroundjoin%
\definecolor{currentfill}{rgb}{0.000000,0.000000,0.000000}%
\pgfsetfillcolor{currentfill}%
\pgfsetlinewidth{0.803000pt}%
\definecolor{currentstroke}{rgb}{0.000000,0.000000,0.000000}%
\pgfsetstrokecolor{currentstroke}%
\pgfsetdash{}{0pt}%
\pgfsys@defobject{currentmarker}{\pgfqpoint{0.000000in}{-0.048611in}}{\pgfqpoint{0.000000in}{0.000000in}}{%
\pgfpathmoveto{\pgfqpoint{0.000000in}{0.000000in}}%
\pgfpathlineto{\pgfqpoint{0.000000in}{-0.048611in}}%
\pgfusepath{stroke,fill}%
}%
\begin{pgfscope}%
\pgfsys@transformshift{3.257678in}{0.528000in}%
\pgfsys@useobject{currentmarker}{}%
\end{pgfscope}%
\end{pgfscope}%
\begin{pgfscope}%
\pgftext[x=3.257678in,y=0.430778in,,top]{\fontsize{11.000000}{13.200000}\selectfont \(\displaystyle 150\)}%
\end{pgfscope}%
\begin{pgfscope}%
\pgfsetbuttcap%
\pgfsetroundjoin%
\definecolor{currentfill}{rgb}{0.000000,0.000000,0.000000}%
\pgfsetfillcolor{currentfill}%
\pgfsetlinewidth{0.803000pt}%
\definecolor{currentstroke}{rgb}{0.000000,0.000000,0.000000}%
\pgfsetstrokecolor{currentstroke}%
\pgfsetdash{}{0pt}%
\pgfsys@defobject{currentmarker}{\pgfqpoint{0.000000in}{-0.048611in}}{\pgfqpoint{0.000000in}{0.000000in}}{%
\pgfpathmoveto{\pgfqpoint{0.000000in}{0.000000in}}%
\pgfpathlineto{\pgfqpoint{0.000000in}{-0.048611in}}%
\pgfusepath{stroke,fill}%
}%
\begin{pgfscope}%
\pgfsys@transformshift{4.001752in}{0.528000in}%
\pgfsys@useobject{currentmarker}{}%
\end{pgfscope}%
\end{pgfscope}%
\begin{pgfscope}%
\pgftext[x=4.001752in,y=0.430778in,,top]{\fontsize{11.000000}{13.200000}\selectfont \(\displaystyle 200\)}%
\end{pgfscope}%
\begin{pgfscope}%
\pgfsetbuttcap%
\pgfsetroundjoin%
\definecolor{currentfill}{rgb}{0.000000,0.000000,0.000000}%
\pgfsetfillcolor{currentfill}%
\pgfsetlinewidth{0.803000pt}%
\definecolor{currentstroke}{rgb}{0.000000,0.000000,0.000000}%
\pgfsetstrokecolor{currentstroke}%
\pgfsetdash{}{0pt}%
\pgfsys@defobject{currentmarker}{\pgfqpoint{0.000000in}{-0.048611in}}{\pgfqpoint{0.000000in}{0.000000in}}{%
\pgfpathmoveto{\pgfqpoint{0.000000in}{0.000000in}}%
\pgfpathlineto{\pgfqpoint{0.000000in}{-0.048611in}}%
\pgfusepath{stroke,fill}%
}%
\begin{pgfscope}%
\pgfsys@transformshift{4.745827in}{0.528000in}%
\pgfsys@useobject{currentmarker}{}%
\end{pgfscope}%
\end{pgfscope}%
\begin{pgfscope}%
\pgftext[x=4.745827in,y=0.430778in,,top]{\fontsize{11.000000}{13.200000}\selectfont \(\displaystyle 250\)}%
\end{pgfscope}%
\begin{pgfscope}%
\pgfsetbuttcap%
\pgfsetroundjoin%
\definecolor{currentfill}{rgb}{0.000000,0.000000,0.000000}%
\pgfsetfillcolor{currentfill}%
\pgfsetlinewidth{0.803000pt}%
\definecolor{currentstroke}{rgb}{0.000000,0.000000,0.000000}%
\pgfsetstrokecolor{currentstroke}%
\pgfsetdash{}{0pt}%
\pgfsys@defobject{currentmarker}{\pgfqpoint{0.000000in}{-0.048611in}}{\pgfqpoint{0.000000in}{0.000000in}}{%
\pgfpathmoveto{\pgfqpoint{0.000000in}{0.000000in}}%
\pgfpathlineto{\pgfqpoint{0.000000in}{-0.048611in}}%
\pgfusepath{stroke,fill}%
}%
\begin{pgfscope}%
\pgfsys@transformshift{5.489901in}{0.528000in}%
\pgfsys@useobject{currentmarker}{}%
\end{pgfscope}%
\end{pgfscope}%
\begin{pgfscope}%
\pgftext[x=5.489901in,y=0.430778in,,top]{\fontsize{11.000000}{13.200000}\selectfont \(\displaystyle 300\)}%
\end{pgfscope}%
\begin{pgfscope}%
\pgftext[x=3.280000in,y=0.240271in,,top]{\fontsize{11.000000}{13.200000}\selectfont Time of experiment (in seconds)}%
\end{pgfscope}%
\begin{pgfscope}%
\pgfsetbuttcap%
\pgfsetroundjoin%
\definecolor{currentfill}{rgb}{0.000000,0.000000,0.000000}%
\pgfsetfillcolor{currentfill}%
\pgfsetlinewidth{0.803000pt}%
\definecolor{currentstroke}{rgb}{0.000000,0.000000,0.000000}%
\pgfsetstrokecolor{currentstroke}%
\pgfsetdash{}{0pt}%
\pgfsys@defobject{currentmarker}{\pgfqpoint{-0.048611in}{0.000000in}}{\pgfqpoint{0.000000in}{0.000000in}}{%
\pgfpathmoveto{\pgfqpoint{0.000000in}{0.000000in}}%
\pgfpathlineto{\pgfqpoint{-0.048611in}{0.000000in}}%
\pgfusepath{stroke,fill}%
}%
\begin{pgfscope}%
\pgfsys@transformshift{0.800000in}{0.696000in}%
\pgfsys@useobject{currentmarker}{}%
\end{pgfscope}%
\end{pgfscope}%
\begin{pgfscope}%
\pgftext[x=0.627981in,y=0.643378in,left,base]{\fontsize{11.000000}{13.200000}\selectfont \(\displaystyle 0\)}%
\end{pgfscope}%
\begin{pgfscope}%
\pgfsetbuttcap%
\pgfsetroundjoin%
\definecolor{currentfill}{rgb}{0.000000,0.000000,0.000000}%
\pgfsetfillcolor{currentfill}%
\pgfsetlinewidth{0.803000pt}%
\definecolor{currentstroke}{rgb}{0.000000,0.000000,0.000000}%
\pgfsetstrokecolor{currentstroke}%
\pgfsetdash{}{0pt}%
\pgfsys@defobject{currentmarker}{\pgfqpoint{-0.048611in}{0.000000in}}{\pgfqpoint{0.000000in}{0.000000in}}{%
\pgfpathmoveto{\pgfqpoint{0.000000in}{0.000000in}}%
\pgfpathlineto{\pgfqpoint{-0.048611in}{0.000000in}}%
\pgfusepath{stroke,fill}%
}%
\begin{pgfscope}%
\pgfsys@transformshift{0.800000in}{1.297773in}%
\pgfsys@useobject{currentmarker}{}%
\end{pgfscope}%
\end{pgfscope}%
\begin{pgfscope}%
\pgftext[x=0.403588in,y=1.245151in,left,base]{\fontsize{11.000000}{13.200000}\selectfont \(\displaystyle 2000\)}%
\end{pgfscope}%
\begin{pgfscope}%
\pgfsetbuttcap%
\pgfsetroundjoin%
\definecolor{currentfill}{rgb}{0.000000,0.000000,0.000000}%
\pgfsetfillcolor{currentfill}%
\pgfsetlinewidth{0.803000pt}%
\definecolor{currentstroke}{rgb}{0.000000,0.000000,0.000000}%
\pgfsetstrokecolor{currentstroke}%
\pgfsetdash{}{0pt}%
\pgfsys@defobject{currentmarker}{\pgfqpoint{-0.048611in}{0.000000in}}{\pgfqpoint{0.000000in}{0.000000in}}{%
\pgfpathmoveto{\pgfqpoint{0.000000in}{0.000000in}}%
\pgfpathlineto{\pgfqpoint{-0.048611in}{0.000000in}}%
\pgfusepath{stroke,fill}%
}%
\begin{pgfscope}%
\pgfsys@transformshift{0.800000in}{1.899546in}%
\pgfsys@useobject{currentmarker}{}%
\end{pgfscope}%
\end{pgfscope}%
\begin{pgfscope}%
\pgftext[x=0.403588in,y=1.846924in,left,base]{\fontsize{11.000000}{13.200000}\selectfont \(\displaystyle 4000\)}%
\end{pgfscope}%
\begin{pgfscope}%
\pgfsetbuttcap%
\pgfsetroundjoin%
\definecolor{currentfill}{rgb}{0.000000,0.000000,0.000000}%
\pgfsetfillcolor{currentfill}%
\pgfsetlinewidth{0.803000pt}%
\definecolor{currentstroke}{rgb}{0.000000,0.000000,0.000000}%
\pgfsetstrokecolor{currentstroke}%
\pgfsetdash{}{0pt}%
\pgfsys@defobject{currentmarker}{\pgfqpoint{-0.048611in}{0.000000in}}{\pgfqpoint{0.000000in}{0.000000in}}{%
\pgfpathmoveto{\pgfqpoint{0.000000in}{0.000000in}}%
\pgfpathlineto{\pgfqpoint{-0.048611in}{0.000000in}}%
\pgfusepath{stroke,fill}%
}%
\begin{pgfscope}%
\pgfsys@transformshift{0.800000in}{2.501319in}%
\pgfsys@useobject{currentmarker}{}%
\end{pgfscope}%
\end{pgfscope}%
\begin{pgfscope}%
\pgftext[x=0.403588in,y=2.448697in,left,base]{\fontsize{11.000000}{13.200000}\selectfont \(\displaystyle 6000\)}%
\end{pgfscope}%
\begin{pgfscope}%
\pgfsetbuttcap%
\pgfsetroundjoin%
\definecolor{currentfill}{rgb}{0.000000,0.000000,0.000000}%
\pgfsetfillcolor{currentfill}%
\pgfsetlinewidth{0.803000pt}%
\definecolor{currentstroke}{rgb}{0.000000,0.000000,0.000000}%
\pgfsetstrokecolor{currentstroke}%
\pgfsetdash{}{0pt}%
\pgfsys@defobject{currentmarker}{\pgfqpoint{-0.048611in}{0.000000in}}{\pgfqpoint{0.000000in}{0.000000in}}{%
\pgfpathmoveto{\pgfqpoint{0.000000in}{0.000000in}}%
\pgfpathlineto{\pgfqpoint{-0.048611in}{0.000000in}}%
\pgfusepath{stroke,fill}%
}%
\begin{pgfscope}%
\pgfsys@transformshift{0.800000in}{3.103092in}%
\pgfsys@useobject{currentmarker}{}%
\end{pgfscope}%
\end{pgfscope}%
\begin{pgfscope}%
\pgftext[x=0.403588in,y=3.050470in,left,base]{\fontsize{11.000000}{13.200000}\selectfont \(\displaystyle 8000\)}%
\end{pgfscope}%
\begin{pgfscope}%
\pgfsetbuttcap%
\pgfsetroundjoin%
\definecolor{currentfill}{rgb}{0.000000,0.000000,0.000000}%
\pgfsetfillcolor{currentfill}%
\pgfsetlinewidth{0.803000pt}%
\definecolor{currentstroke}{rgb}{0.000000,0.000000,0.000000}%
\pgfsetstrokecolor{currentstroke}%
\pgfsetdash{}{0pt}%
\pgfsys@defobject{currentmarker}{\pgfqpoint{-0.048611in}{0.000000in}}{\pgfqpoint{0.000000in}{0.000000in}}{%
\pgfpathmoveto{\pgfqpoint{0.000000in}{0.000000in}}%
\pgfpathlineto{\pgfqpoint{-0.048611in}{0.000000in}}%
\pgfusepath{stroke,fill}%
}%
\begin{pgfscope}%
\pgfsys@transformshift{0.800000in}{3.704865in}%
\pgfsys@useobject{currentmarker}{}%
\end{pgfscope}%
\end{pgfscope}%
\begin{pgfscope}%
\pgftext[x=0.328791in,y=3.652243in,left,base]{\fontsize{11.000000}{13.200000}\selectfont \(\displaystyle 10000\)}%
\end{pgfscope}%
\begin{pgfscope}%
\pgftext[x=0.273236in,y=2.376000in,,bottom,rotate=90.000000]{\fontsize{11.000000}{13.200000}\selectfont Number of errors}%
\end{pgfscope}%
\begin{pgfscope}%
\pgfpathrectangle{\pgfqpoint{0.800000in}{0.528000in}}{\pgfqpoint{4.960000in}{3.696000in}}%
\pgfusepath{clip}%
\pgfsetrectcap%
\pgfsetroundjoin%
\pgfsetlinewidth{1.505625pt}%
\definecolor{currentstroke}{rgb}{0.121569,0.466667,0.705882}%
\pgfsetstrokecolor{currentstroke}%
\pgfsetdash{}{0pt}%
\pgfpathmoveto{\pgfqpoint{1.025455in}{0.696000in}}%
\pgfpathlineto{\pgfqpoint{2.573129in}{0.696000in}}%
\pgfpathlineto{\pgfqpoint{2.588011in}{0.939116in}}%
\pgfpathlineto{\pgfqpoint{2.602892in}{3.216226in}}%
\pgfpathlineto{\pgfqpoint{2.617774in}{3.618210in}}%
\pgfpathlineto{\pgfqpoint{2.632655in}{3.425944in}}%
\pgfpathlineto{\pgfqpoint{2.647537in}{3.492139in}}%
\pgfpathlineto{\pgfqpoint{2.662418in}{3.836353in}}%
\pgfpathlineto{\pgfqpoint{2.677300in}{3.815291in}}%
\pgfpathlineto{\pgfqpoint{2.692181in}{3.900743in}}%
\pgfpathlineto{\pgfqpoint{2.707063in}{3.651007in}}%
\pgfpathlineto{\pgfqpoint{2.721944in}{3.633555in}}%
\pgfpathlineto{\pgfqpoint{2.736826in}{3.752706in}}%
\pgfpathlineto{\pgfqpoint{2.751707in}{3.854406in}}%
\pgfpathlineto{\pgfqpoint{2.766589in}{3.827025in}}%
\pgfpathlineto{\pgfqpoint{2.796352in}{3.955203in}}%
\pgfpathlineto{\pgfqpoint{2.811233in}{3.959716in}}%
\pgfpathlineto{\pgfqpoint{2.826115in}{4.056000in}}%
\pgfpathlineto{\pgfqpoint{2.840996in}{1.255649in}}%
\pgfpathlineto{\pgfqpoint{2.855878in}{0.696000in}}%
\pgfpathlineto{\pgfqpoint{2.900522in}{0.696000in}}%
\pgfpathlineto{\pgfqpoint{2.915404in}{1.168091in}}%
\pgfpathlineto{\pgfqpoint{2.930285in}{0.696000in}}%
\pgfpathlineto{\pgfqpoint{2.974929in}{0.696000in}}%
\pgfpathlineto{\pgfqpoint{2.989811in}{0.717363in}}%
\pgfpathlineto{\pgfqpoint{3.004692in}{0.815753in}}%
\pgfpathlineto{\pgfqpoint{3.019574in}{0.708938in}}%
\pgfpathlineto{\pgfqpoint{3.034455in}{0.696000in}}%
\pgfpathlineto{\pgfqpoint{3.049337in}{0.696000in}}%
\pgfpathlineto{\pgfqpoint{3.064218in}{0.698106in}}%
\pgfpathlineto{\pgfqpoint{3.079100in}{0.752266in}}%
\pgfpathlineto{\pgfqpoint{3.093981in}{0.791682in}}%
\pgfpathlineto{\pgfqpoint{3.108863in}{0.696000in}}%
\pgfpathlineto{\pgfqpoint{3.123744in}{0.696000in}}%
\pgfpathlineto{\pgfqpoint{3.138626in}{0.697504in}}%
\pgfpathlineto{\pgfqpoint{3.153507in}{0.714956in}}%
\pgfpathlineto{\pgfqpoint{3.168389in}{0.761292in}}%
\pgfpathlineto{\pgfqpoint{3.183270in}{0.759788in}}%
\pgfpathlineto{\pgfqpoint{3.198152in}{0.700513in}}%
\pgfpathlineto{\pgfqpoint{3.213033in}{0.696903in}}%
\pgfpathlineto{\pgfqpoint{3.227915in}{0.697504in}}%
\pgfpathlineto{\pgfqpoint{3.242796in}{0.726991in}}%
\pgfpathlineto{\pgfqpoint{3.257678in}{0.791983in}}%
\pgfpathlineto{\pgfqpoint{3.272559in}{0.720673in}}%
\pgfpathlineto{\pgfqpoint{3.287441in}{0.696903in}}%
\pgfpathlineto{\pgfqpoint{3.302322in}{0.697504in}}%
\pgfpathlineto{\pgfqpoint{3.317204in}{0.711947in}}%
\pgfpathlineto{\pgfqpoint{3.332085in}{0.742637in}}%
\pgfpathlineto{\pgfqpoint{3.346967in}{0.785062in}}%
\pgfpathlineto{\pgfqpoint{3.361848in}{0.696903in}}%
\pgfpathlineto{\pgfqpoint{3.391611in}{0.697204in}}%
\pgfpathlineto{\pgfqpoint{3.406493in}{0.715558in}}%
\pgfpathlineto{\pgfqpoint{3.421374in}{0.751664in}}%
\pgfpathlineto{\pgfqpoint{3.436256in}{0.772124in}}%
\pgfpathlineto{\pgfqpoint{3.451137in}{0.697504in}}%
\pgfpathlineto{\pgfqpoint{3.466019in}{0.697204in}}%
\pgfpathlineto{\pgfqpoint{3.495782in}{0.724283in}}%
\pgfpathlineto{\pgfqpoint{3.510663in}{0.776638in}}%
\pgfpathlineto{\pgfqpoint{3.525545in}{0.723682in}}%
\pgfpathlineto{\pgfqpoint{3.540426in}{0.698106in}}%
\pgfpathlineto{\pgfqpoint{3.555308in}{0.699611in}}%
\pgfpathlineto{\pgfqpoint{3.585071in}{0.733611in}}%
\pgfpathlineto{\pgfqpoint{3.599952in}{0.784762in}}%
\pgfpathlineto{\pgfqpoint{3.614833in}{0.699912in}}%
\pgfpathlineto{\pgfqpoint{3.629715in}{0.696301in}}%
\pgfpathlineto{\pgfqpoint{3.644596in}{0.708035in}}%
\pgfpathlineto{\pgfqpoint{3.659478in}{0.722177in}}%
\pgfpathlineto{\pgfqpoint{3.674359in}{0.741735in}}%
\pgfpathlineto{\pgfqpoint{3.689241in}{0.755275in}}%
\pgfpathlineto{\pgfqpoint{3.704122in}{0.706832in}}%
\pgfpathlineto{\pgfqpoint{3.719004in}{0.708035in}}%
\pgfpathlineto{\pgfqpoint{3.733885in}{0.712248in}}%
\pgfpathlineto{\pgfqpoint{3.778530in}{0.747752in}}%
\pgfpathlineto{\pgfqpoint{3.793411in}{0.704726in}}%
\pgfpathlineto{\pgfqpoint{3.808293in}{0.707735in}}%
\pgfpathlineto{\pgfqpoint{3.823174in}{0.721575in}}%
\pgfpathlineto{\pgfqpoint{3.838056in}{0.722177in}}%
\pgfpathlineto{\pgfqpoint{3.852937in}{0.726089in}}%
\pgfpathlineto{\pgfqpoint{3.867819in}{0.751062in}}%
\pgfpathlineto{\pgfqpoint{3.882700in}{0.701717in}}%
\pgfpathlineto{\pgfqpoint{3.897582in}{0.733009in}}%
\pgfpathlineto{\pgfqpoint{3.912463in}{0.707735in}}%
\pgfpathlineto{\pgfqpoint{3.927345in}{0.729398in}}%
\pgfpathlineto{\pgfqpoint{3.942226in}{0.732407in}}%
\pgfpathlineto{\pgfqpoint{3.957108in}{0.730903in}}%
\pgfpathlineto{\pgfqpoint{3.971989in}{0.721274in}}%
\pgfpathlineto{\pgfqpoint{3.986871in}{0.720071in}}%
\pgfpathlineto{\pgfqpoint{4.001752in}{0.721274in}}%
\pgfpathlineto{\pgfqpoint{4.016634in}{0.720071in}}%
\pgfpathlineto{\pgfqpoint{4.031515in}{0.738124in}}%
\pgfpathlineto{\pgfqpoint{4.046397in}{0.719469in}}%
\pgfpathlineto{\pgfqpoint{4.061278in}{0.710443in}}%
\pgfpathlineto{\pgfqpoint{4.076160in}{0.735115in}}%
\pgfpathlineto{\pgfqpoint{4.091041in}{0.721876in}}%
\pgfpathlineto{\pgfqpoint{4.105923in}{0.733310in}}%
\pgfpathlineto{\pgfqpoint{4.120804in}{0.733009in}}%
\pgfpathlineto{\pgfqpoint{4.135686in}{0.710142in}}%
\pgfpathlineto{\pgfqpoint{4.150567in}{0.715558in}}%
\pgfpathlineto{\pgfqpoint{4.165449in}{0.729398in}}%
\pgfpathlineto{\pgfqpoint{4.180330in}{0.697504in}}%
\pgfpathlineto{\pgfqpoint{4.195212in}{0.699310in}}%
\pgfpathlineto{\pgfqpoint{4.210093in}{0.696000in}}%
\pgfpathlineto{\pgfqpoint{5.534545in}{0.696000in}}%
\pgfpathlineto{\pgfqpoint{5.534545in}{0.696000in}}%
\pgfusepath{stroke}%
\end{pgfscope}%
\begin{pgfscope}%
\pgfsetrectcap%
\pgfsetmiterjoin%
\pgfsetlinewidth{0.803000pt}%
\definecolor{currentstroke}{rgb}{0.000000,0.000000,0.000000}%
\pgfsetstrokecolor{currentstroke}%
\pgfsetdash{}{0pt}%
\pgfpathmoveto{\pgfqpoint{0.800000in}{0.528000in}}%
\pgfpathlineto{\pgfqpoint{0.800000in}{4.224000in}}%
\pgfusepath{stroke}%
\end{pgfscope}%
\begin{pgfscope}%
\pgfsetrectcap%
\pgfsetmiterjoin%
\pgfsetlinewidth{0.803000pt}%
\definecolor{currentstroke}{rgb}{0.000000,0.000000,0.000000}%
\pgfsetstrokecolor{currentstroke}%
\pgfsetdash{}{0pt}%
\pgfpathmoveto{\pgfqpoint{5.760000in}{0.528000in}}%
\pgfpathlineto{\pgfqpoint{5.760000in}{4.224000in}}%
\pgfusepath{stroke}%
\end{pgfscope}%
\begin{pgfscope}%
\pgfsetrectcap%
\pgfsetmiterjoin%
\pgfsetlinewidth{0.803000pt}%
\definecolor{currentstroke}{rgb}{0.000000,0.000000,0.000000}%
\pgfsetstrokecolor{currentstroke}%
\pgfsetdash{}{0pt}%
\pgfpathmoveto{\pgfqpoint{0.800000in}{0.528000in}}%
\pgfpathlineto{\pgfqpoint{5.760000in}{0.528000in}}%
\pgfusepath{stroke}%
\end{pgfscope}%
\begin{pgfscope}%
\pgfsetrectcap%
\pgfsetmiterjoin%
\pgfsetlinewidth{0.803000pt}%
\definecolor{currentstroke}{rgb}{0.000000,0.000000,0.000000}%
\pgfsetstrokecolor{currentstroke}%
\pgfsetdash{}{0pt}%
\pgfpathmoveto{\pgfqpoint{0.800000in}{4.224000in}}%
\pgfpathlineto{\pgfqpoint{5.760000in}{4.224000in}}%
\pgfusepath{stroke}%
\end{pgfscope}%
\end{pgfpicture}%
\makeatother%
\endgroup%

    \caption{Distribution of errors for every second of experiment 3 in \prettyref{tab:res-8}}
    \label{fig:err-2}
\end{figure}

\begin{figure}
    %% Creator: Matplotlib, PGF backend
%%
%% To include the figure in your LaTeX document, write
%%   \input{<filename>.pgf}
%%
%% Make sure the required packages are loaded in your preamble
%%   \usepackage{pgf}
%%
%% Figures using additional raster images can only be included by \input if
%% they are in the same directory as the main LaTeX file. For loading figures
%% from other directories you can use the `import` package
%%   \usepackage{import}
%% and then include the figures with
%%   \import{<path to file>}{<filename>.pgf}
%%
%% Matplotlib used the following preamble
%%   \usepackage[utf8x]{inputenc}
%%   \usepackage[T1]{fontenc}
%%   \usepackage{lmodern}
%%
\begingroup%
\makeatletter%
\begin{pgfpicture}%
\pgfpathrectangle{\pgfpointorigin}{\pgfqpoint{6.400000in}{4.800000in}}%
\pgfusepath{use as bounding box, clip}%
\begin{pgfscope}%
\pgfsetbuttcap%
\pgfsetmiterjoin%
\definecolor{currentfill}{rgb}{1.000000,1.000000,1.000000}%
\pgfsetfillcolor{currentfill}%
\pgfsetlinewidth{0.000000pt}%
\definecolor{currentstroke}{rgb}{1.000000,1.000000,1.000000}%
\pgfsetstrokecolor{currentstroke}%
\pgfsetdash{}{0pt}%
\pgfpathmoveto{\pgfqpoint{0.000000in}{0.000000in}}%
\pgfpathlineto{\pgfqpoint{6.400000in}{0.000000in}}%
\pgfpathlineto{\pgfqpoint{6.400000in}{4.800000in}}%
\pgfpathlineto{\pgfqpoint{0.000000in}{4.800000in}}%
\pgfpathclose%
\pgfusepath{fill}%
\end{pgfscope}%
\begin{pgfscope}%
\pgfsetbuttcap%
\pgfsetmiterjoin%
\definecolor{currentfill}{rgb}{1.000000,1.000000,1.000000}%
\pgfsetfillcolor{currentfill}%
\pgfsetlinewidth{0.000000pt}%
\definecolor{currentstroke}{rgb}{0.000000,0.000000,0.000000}%
\pgfsetstrokecolor{currentstroke}%
\pgfsetstrokeopacity{0.000000}%
\pgfsetdash{}{0pt}%
\pgfpathmoveto{\pgfqpoint{0.800000in}{0.528000in}}%
\pgfpathlineto{\pgfqpoint{5.760000in}{0.528000in}}%
\pgfpathlineto{\pgfqpoint{5.760000in}{4.224000in}}%
\pgfpathlineto{\pgfqpoint{0.800000in}{4.224000in}}%
\pgfpathclose%
\pgfusepath{fill}%
\end{pgfscope}%
\begin{pgfscope}%
\pgfsetbuttcap%
\pgfsetroundjoin%
\definecolor{currentfill}{rgb}{0.000000,0.000000,0.000000}%
\pgfsetfillcolor{currentfill}%
\pgfsetlinewidth{0.803000pt}%
\definecolor{currentstroke}{rgb}{0.000000,0.000000,0.000000}%
\pgfsetstrokecolor{currentstroke}%
\pgfsetdash{}{0pt}%
\pgfsys@defobject{currentmarker}{\pgfqpoint{0.000000in}{-0.048611in}}{\pgfqpoint{0.000000in}{0.000000in}}{%
\pgfpathmoveto{\pgfqpoint{0.000000in}{0.000000in}}%
\pgfpathlineto{\pgfqpoint{0.000000in}{-0.048611in}}%
\pgfusepath{stroke,fill}%
}%
\begin{pgfscope}%
\pgfsys@transformshift{1.025455in}{0.528000in}%
\pgfsys@useobject{currentmarker}{}%
\end{pgfscope}%
\end{pgfscope}%
\begin{pgfscope}%
\pgftext[x=1.025455in,y=0.430778in,,top]{\fontsize{11.000000}{13.200000}\selectfont \(\displaystyle 0\)}%
\end{pgfscope}%
\begin{pgfscope}%
\pgfsetbuttcap%
\pgfsetroundjoin%
\definecolor{currentfill}{rgb}{0.000000,0.000000,0.000000}%
\pgfsetfillcolor{currentfill}%
\pgfsetlinewidth{0.803000pt}%
\definecolor{currentstroke}{rgb}{0.000000,0.000000,0.000000}%
\pgfsetstrokecolor{currentstroke}%
\pgfsetdash{}{0pt}%
\pgfsys@defobject{currentmarker}{\pgfqpoint{0.000000in}{-0.048611in}}{\pgfqpoint{0.000000in}{0.000000in}}{%
\pgfpathmoveto{\pgfqpoint{0.000000in}{0.000000in}}%
\pgfpathlineto{\pgfqpoint{0.000000in}{-0.048611in}}%
\pgfusepath{stroke,fill}%
}%
\begin{pgfscope}%
\pgfsys@transformshift{1.769529in}{0.528000in}%
\pgfsys@useobject{currentmarker}{}%
\end{pgfscope}%
\end{pgfscope}%
\begin{pgfscope}%
\pgftext[x=1.769529in,y=0.430778in,,top]{\fontsize{11.000000}{13.200000}\selectfont \(\displaystyle 50\)}%
\end{pgfscope}%
\begin{pgfscope}%
\pgfsetbuttcap%
\pgfsetroundjoin%
\definecolor{currentfill}{rgb}{0.000000,0.000000,0.000000}%
\pgfsetfillcolor{currentfill}%
\pgfsetlinewidth{0.803000pt}%
\definecolor{currentstroke}{rgb}{0.000000,0.000000,0.000000}%
\pgfsetstrokecolor{currentstroke}%
\pgfsetdash{}{0pt}%
\pgfsys@defobject{currentmarker}{\pgfqpoint{0.000000in}{-0.048611in}}{\pgfqpoint{0.000000in}{0.000000in}}{%
\pgfpathmoveto{\pgfqpoint{0.000000in}{0.000000in}}%
\pgfpathlineto{\pgfqpoint{0.000000in}{-0.048611in}}%
\pgfusepath{stroke,fill}%
}%
\begin{pgfscope}%
\pgfsys@transformshift{2.513603in}{0.528000in}%
\pgfsys@useobject{currentmarker}{}%
\end{pgfscope}%
\end{pgfscope}%
\begin{pgfscope}%
\pgftext[x=2.513603in,y=0.430778in,,top]{\fontsize{11.000000}{13.200000}\selectfont \(\displaystyle 100\)}%
\end{pgfscope}%
\begin{pgfscope}%
\pgfsetbuttcap%
\pgfsetroundjoin%
\definecolor{currentfill}{rgb}{0.000000,0.000000,0.000000}%
\pgfsetfillcolor{currentfill}%
\pgfsetlinewidth{0.803000pt}%
\definecolor{currentstroke}{rgb}{0.000000,0.000000,0.000000}%
\pgfsetstrokecolor{currentstroke}%
\pgfsetdash{}{0pt}%
\pgfsys@defobject{currentmarker}{\pgfqpoint{0.000000in}{-0.048611in}}{\pgfqpoint{0.000000in}{0.000000in}}{%
\pgfpathmoveto{\pgfqpoint{0.000000in}{0.000000in}}%
\pgfpathlineto{\pgfqpoint{0.000000in}{-0.048611in}}%
\pgfusepath{stroke,fill}%
}%
\begin{pgfscope}%
\pgfsys@transformshift{3.257678in}{0.528000in}%
\pgfsys@useobject{currentmarker}{}%
\end{pgfscope}%
\end{pgfscope}%
\begin{pgfscope}%
\pgftext[x=3.257678in,y=0.430778in,,top]{\fontsize{11.000000}{13.200000}\selectfont \(\displaystyle 150\)}%
\end{pgfscope}%
\begin{pgfscope}%
\pgfsetbuttcap%
\pgfsetroundjoin%
\definecolor{currentfill}{rgb}{0.000000,0.000000,0.000000}%
\pgfsetfillcolor{currentfill}%
\pgfsetlinewidth{0.803000pt}%
\definecolor{currentstroke}{rgb}{0.000000,0.000000,0.000000}%
\pgfsetstrokecolor{currentstroke}%
\pgfsetdash{}{0pt}%
\pgfsys@defobject{currentmarker}{\pgfqpoint{0.000000in}{-0.048611in}}{\pgfqpoint{0.000000in}{0.000000in}}{%
\pgfpathmoveto{\pgfqpoint{0.000000in}{0.000000in}}%
\pgfpathlineto{\pgfqpoint{0.000000in}{-0.048611in}}%
\pgfusepath{stroke,fill}%
}%
\begin{pgfscope}%
\pgfsys@transformshift{4.001752in}{0.528000in}%
\pgfsys@useobject{currentmarker}{}%
\end{pgfscope}%
\end{pgfscope}%
\begin{pgfscope}%
\pgftext[x=4.001752in,y=0.430778in,,top]{\fontsize{11.000000}{13.200000}\selectfont \(\displaystyle 200\)}%
\end{pgfscope}%
\begin{pgfscope}%
\pgfsetbuttcap%
\pgfsetroundjoin%
\definecolor{currentfill}{rgb}{0.000000,0.000000,0.000000}%
\pgfsetfillcolor{currentfill}%
\pgfsetlinewidth{0.803000pt}%
\definecolor{currentstroke}{rgb}{0.000000,0.000000,0.000000}%
\pgfsetstrokecolor{currentstroke}%
\pgfsetdash{}{0pt}%
\pgfsys@defobject{currentmarker}{\pgfqpoint{0.000000in}{-0.048611in}}{\pgfqpoint{0.000000in}{0.000000in}}{%
\pgfpathmoveto{\pgfqpoint{0.000000in}{0.000000in}}%
\pgfpathlineto{\pgfqpoint{0.000000in}{-0.048611in}}%
\pgfusepath{stroke,fill}%
}%
\begin{pgfscope}%
\pgfsys@transformshift{4.745827in}{0.528000in}%
\pgfsys@useobject{currentmarker}{}%
\end{pgfscope}%
\end{pgfscope}%
\begin{pgfscope}%
\pgftext[x=4.745827in,y=0.430778in,,top]{\fontsize{11.000000}{13.200000}\selectfont \(\displaystyle 250\)}%
\end{pgfscope}%
\begin{pgfscope}%
\pgfsetbuttcap%
\pgfsetroundjoin%
\definecolor{currentfill}{rgb}{0.000000,0.000000,0.000000}%
\pgfsetfillcolor{currentfill}%
\pgfsetlinewidth{0.803000pt}%
\definecolor{currentstroke}{rgb}{0.000000,0.000000,0.000000}%
\pgfsetstrokecolor{currentstroke}%
\pgfsetdash{}{0pt}%
\pgfsys@defobject{currentmarker}{\pgfqpoint{0.000000in}{-0.048611in}}{\pgfqpoint{0.000000in}{0.000000in}}{%
\pgfpathmoveto{\pgfqpoint{0.000000in}{0.000000in}}%
\pgfpathlineto{\pgfqpoint{0.000000in}{-0.048611in}}%
\pgfusepath{stroke,fill}%
}%
\begin{pgfscope}%
\pgfsys@transformshift{5.489901in}{0.528000in}%
\pgfsys@useobject{currentmarker}{}%
\end{pgfscope}%
\end{pgfscope}%
\begin{pgfscope}%
\pgftext[x=5.489901in,y=0.430778in,,top]{\fontsize{11.000000}{13.200000}\selectfont \(\displaystyle 300\)}%
\end{pgfscope}%
\begin{pgfscope}%
\pgftext[x=3.280000in,y=0.240271in,,top]{\fontsize{11.000000}{13.200000}\selectfont Time of experiment (in seconds)}%
\end{pgfscope}%
\begin{pgfscope}%
\pgfsetbuttcap%
\pgfsetroundjoin%
\definecolor{currentfill}{rgb}{0.000000,0.000000,0.000000}%
\pgfsetfillcolor{currentfill}%
\pgfsetlinewidth{0.803000pt}%
\definecolor{currentstroke}{rgb}{0.000000,0.000000,0.000000}%
\pgfsetstrokecolor{currentstroke}%
\pgfsetdash{}{0pt}%
\pgfsys@defobject{currentmarker}{\pgfqpoint{-0.048611in}{0.000000in}}{\pgfqpoint{0.000000in}{0.000000in}}{%
\pgfpathmoveto{\pgfqpoint{0.000000in}{0.000000in}}%
\pgfpathlineto{\pgfqpoint{-0.048611in}{0.000000in}}%
\pgfusepath{stroke,fill}%
}%
\begin{pgfscope}%
\pgfsys@transformshift{0.800000in}{0.696000in}%
\pgfsys@useobject{currentmarker}{}%
\end{pgfscope}%
\end{pgfscope}%
\begin{pgfscope}%
\pgftext[x=0.627981in,y=0.643378in,left,base]{\fontsize{11.000000}{13.200000}\selectfont \(\displaystyle 0\)}%
\end{pgfscope}%
\begin{pgfscope}%
\pgfsetbuttcap%
\pgfsetroundjoin%
\definecolor{currentfill}{rgb}{0.000000,0.000000,0.000000}%
\pgfsetfillcolor{currentfill}%
\pgfsetlinewidth{0.803000pt}%
\definecolor{currentstroke}{rgb}{0.000000,0.000000,0.000000}%
\pgfsetstrokecolor{currentstroke}%
\pgfsetdash{}{0pt}%
\pgfsys@defobject{currentmarker}{\pgfqpoint{-0.048611in}{0.000000in}}{\pgfqpoint{0.000000in}{0.000000in}}{%
\pgfpathmoveto{\pgfqpoint{0.000000in}{0.000000in}}%
\pgfpathlineto{\pgfqpoint{-0.048611in}{0.000000in}}%
\pgfusepath{stroke,fill}%
}%
\begin{pgfscope}%
\pgfsys@transformshift{0.800000in}{1.208821in}%
\pgfsys@useobject{currentmarker}{}%
\end{pgfscope}%
\end{pgfscope}%
\begin{pgfscope}%
\pgftext[x=0.478386in,y=1.156198in,left,base]{\fontsize{11.000000}{13.200000}\selectfont \(\displaystyle 250\)}%
\end{pgfscope}%
\begin{pgfscope}%
\pgfsetbuttcap%
\pgfsetroundjoin%
\definecolor{currentfill}{rgb}{0.000000,0.000000,0.000000}%
\pgfsetfillcolor{currentfill}%
\pgfsetlinewidth{0.803000pt}%
\definecolor{currentstroke}{rgb}{0.000000,0.000000,0.000000}%
\pgfsetstrokecolor{currentstroke}%
\pgfsetdash{}{0pt}%
\pgfsys@defobject{currentmarker}{\pgfqpoint{-0.048611in}{0.000000in}}{\pgfqpoint{0.000000in}{0.000000in}}{%
\pgfpathmoveto{\pgfqpoint{0.000000in}{0.000000in}}%
\pgfpathlineto{\pgfqpoint{-0.048611in}{0.000000in}}%
\pgfusepath{stroke,fill}%
}%
\begin{pgfscope}%
\pgfsys@transformshift{0.800000in}{1.721641in}%
\pgfsys@useobject{currentmarker}{}%
\end{pgfscope}%
\end{pgfscope}%
\begin{pgfscope}%
\pgftext[x=0.478386in,y=1.669019in,left,base]{\fontsize{11.000000}{13.200000}\selectfont \(\displaystyle 500\)}%
\end{pgfscope}%
\begin{pgfscope}%
\pgfsetbuttcap%
\pgfsetroundjoin%
\definecolor{currentfill}{rgb}{0.000000,0.000000,0.000000}%
\pgfsetfillcolor{currentfill}%
\pgfsetlinewidth{0.803000pt}%
\definecolor{currentstroke}{rgb}{0.000000,0.000000,0.000000}%
\pgfsetstrokecolor{currentstroke}%
\pgfsetdash{}{0pt}%
\pgfsys@defobject{currentmarker}{\pgfqpoint{-0.048611in}{0.000000in}}{\pgfqpoint{0.000000in}{0.000000in}}{%
\pgfpathmoveto{\pgfqpoint{0.000000in}{0.000000in}}%
\pgfpathlineto{\pgfqpoint{-0.048611in}{0.000000in}}%
\pgfusepath{stroke,fill}%
}%
\begin{pgfscope}%
\pgfsys@transformshift{0.800000in}{2.234462in}%
\pgfsys@useobject{currentmarker}{}%
\end{pgfscope}%
\end{pgfscope}%
\begin{pgfscope}%
\pgftext[x=0.478386in,y=2.181839in,left,base]{\fontsize{11.000000}{13.200000}\selectfont \(\displaystyle 750\)}%
\end{pgfscope}%
\begin{pgfscope}%
\pgfsetbuttcap%
\pgfsetroundjoin%
\definecolor{currentfill}{rgb}{0.000000,0.000000,0.000000}%
\pgfsetfillcolor{currentfill}%
\pgfsetlinewidth{0.803000pt}%
\definecolor{currentstroke}{rgb}{0.000000,0.000000,0.000000}%
\pgfsetstrokecolor{currentstroke}%
\pgfsetdash{}{0pt}%
\pgfsys@defobject{currentmarker}{\pgfqpoint{-0.048611in}{0.000000in}}{\pgfqpoint{0.000000in}{0.000000in}}{%
\pgfpathmoveto{\pgfqpoint{0.000000in}{0.000000in}}%
\pgfpathlineto{\pgfqpoint{-0.048611in}{0.000000in}}%
\pgfusepath{stroke,fill}%
}%
\begin{pgfscope}%
\pgfsys@transformshift{0.800000in}{2.747282in}%
\pgfsys@useobject{currentmarker}{}%
\end{pgfscope}%
\end{pgfscope}%
\begin{pgfscope}%
\pgftext[x=0.403588in,y=2.694660in,left,base]{\fontsize{11.000000}{13.200000}\selectfont \(\displaystyle 1000\)}%
\end{pgfscope}%
\begin{pgfscope}%
\pgfsetbuttcap%
\pgfsetroundjoin%
\definecolor{currentfill}{rgb}{0.000000,0.000000,0.000000}%
\pgfsetfillcolor{currentfill}%
\pgfsetlinewidth{0.803000pt}%
\definecolor{currentstroke}{rgb}{0.000000,0.000000,0.000000}%
\pgfsetstrokecolor{currentstroke}%
\pgfsetdash{}{0pt}%
\pgfsys@defobject{currentmarker}{\pgfqpoint{-0.048611in}{0.000000in}}{\pgfqpoint{0.000000in}{0.000000in}}{%
\pgfpathmoveto{\pgfqpoint{0.000000in}{0.000000in}}%
\pgfpathlineto{\pgfqpoint{-0.048611in}{0.000000in}}%
\pgfusepath{stroke,fill}%
}%
\begin{pgfscope}%
\pgfsys@transformshift{0.800000in}{3.260103in}%
\pgfsys@useobject{currentmarker}{}%
\end{pgfscope}%
\end{pgfscope}%
\begin{pgfscope}%
\pgftext[x=0.403588in,y=3.207480in,left,base]{\fontsize{11.000000}{13.200000}\selectfont \(\displaystyle 1250\)}%
\end{pgfscope}%
\begin{pgfscope}%
\pgfsetbuttcap%
\pgfsetroundjoin%
\definecolor{currentfill}{rgb}{0.000000,0.000000,0.000000}%
\pgfsetfillcolor{currentfill}%
\pgfsetlinewidth{0.803000pt}%
\definecolor{currentstroke}{rgb}{0.000000,0.000000,0.000000}%
\pgfsetstrokecolor{currentstroke}%
\pgfsetdash{}{0pt}%
\pgfsys@defobject{currentmarker}{\pgfqpoint{-0.048611in}{0.000000in}}{\pgfqpoint{0.000000in}{0.000000in}}{%
\pgfpathmoveto{\pgfqpoint{0.000000in}{0.000000in}}%
\pgfpathlineto{\pgfqpoint{-0.048611in}{0.000000in}}%
\pgfusepath{stroke,fill}%
}%
\begin{pgfscope}%
\pgfsys@transformshift{0.800000in}{3.772923in}%
\pgfsys@useobject{currentmarker}{}%
\end{pgfscope}%
\end{pgfscope}%
\begin{pgfscope}%
\pgftext[x=0.403588in,y=3.720301in,left,base]{\fontsize{11.000000}{13.200000}\selectfont \(\displaystyle 1500\)}%
\end{pgfscope}%
\begin{pgfscope}%
\pgftext[x=0.348033in,y=2.376000in,,bottom,rotate=90.000000]{\fontsize{11.000000}{13.200000}\selectfont Number of successful writes}%
\end{pgfscope}%
\begin{pgfscope}%
\pgfpathrectangle{\pgfqpoint{0.800000in}{0.528000in}}{\pgfqpoint{4.960000in}{3.696000in}}%
\pgfusepath{clip}%
\pgfsetrectcap%
\pgfsetroundjoin%
\pgfsetlinewidth{1.505625pt}%
\definecolor{currentstroke}{rgb}{0.121569,0.466667,0.705882}%
\pgfsetstrokecolor{currentstroke}%
\pgfsetdash{}{0pt}%
\pgfpathmoveto{\pgfqpoint{1.025455in}{2.037538in}}%
\pgfpathlineto{\pgfqpoint{1.040336in}{2.587282in}}%
\pgfpathlineto{\pgfqpoint{1.055218in}{3.360615in}}%
\pgfpathlineto{\pgfqpoint{1.070099in}{3.050872in}}%
\pgfpathlineto{\pgfqpoint{1.084980in}{3.342154in}}%
\pgfpathlineto{\pgfqpoint{1.099862in}{3.307282in}}%
\pgfpathlineto{\pgfqpoint{1.114743in}{3.629333in}}%
\pgfpathlineto{\pgfqpoint{1.129625in}{4.056000in}}%
\pgfpathlineto{\pgfqpoint{1.144506in}{3.522667in}}%
\pgfpathlineto{\pgfqpoint{1.159388in}{3.457026in}}%
\pgfpathlineto{\pgfqpoint{1.174269in}{3.549333in}}%
\pgfpathlineto{\pgfqpoint{1.189151in}{3.633436in}}%
\pgfpathlineto{\pgfqpoint{1.204032in}{3.537026in}}%
\pgfpathlineto{\pgfqpoint{1.218914in}{3.473436in}}%
\pgfpathlineto{\pgfqpoint{1.233795in}{3.524718in}}%
\pgfpathlineto{\pgfqpoint{1.248677in}{3.446769in}}%
\pgfpathlineto{\pgfqpoint{1.263558in}{3.752410in}}%
\pgfpathlineto{\pgfqpoint{1.278440in}{3.633436in}}%
\pgfpathlineto{\pgfqpoint{1.293321in}{3.578051in}}%
\pgfpathlineto{\pgfqpoint{1.308203in}{3.344205in}}%
\pgfpathlineto{\pgfqpoint{1.323084in}{3.401641in}}%
\pgfpathlineto{\pgfqpoint{1.337966in}{3.342154in}}%
\pgfpathlineto{\pgfqpoint{1.352847in}{3.610872in}}%
\pgfpathlineto{\pgfqpoint{1.367729in}{3.446769in}}%
\pgfpathlineto{\pgfqpoint{1.382610in}{3.664205in}}%
\pgfpathlineto{\pgfqpoint{1.397492in}{3.781128in}}%
\pgfpathlineto{\pgfqpoint{1.412373in}{3.440615in}}%
\pgfpathlineto{\pgfqpoint{1.427255in}{3.498051in}}%
\pgfpathlineto{\pgfqpoint{1.442136in}{3.512410in}}%
\pgfpathlineto{\pgfqpoint{1.457018in}{3.352410in}}%
\pgfpathlineto{\pgfqpoint{1.471899in}{3.208821in}}%
\pgfpathlineto{\pgfqpoint{1.486781in}{3.477538in}}%
\pgfpathlineto{\pgfqpoint{1.501662in}{3.781128in}}%
\pgfpathlineto{\pgfqpoint{1.516544in}{3.278564in}}%
\pgfpathlineto{\pgfqpoint{1.531425in}{3.516513in}}%
\pgfpathlineto{\pgfqpoint{1.546307in}{3.461128in}}%
\pgfpathlineto{\pgfqpoint{1.561188in}{3.578051in}}%
\pgfpathlineto{\pgfqpoint{1.576070in}{3.859077in}}%
\pgfpathlineto{\pgfqpoint{1.590951in}{3.411897in}}%
\pgfpathlineto{\pgfqpoint{1.605833in}{3.590359in}}%
\pgfpathlineto{\pgfqpoint{1.620714in}{3.643692in}}%
\pgfpathlineto{\pgfqpoint{1.635596in}{3.504205in}}%
\pgfpathlineto{\pgfqpoint{1.650477in}{3.184205in}}%
\pgfpathlineto{\pgfqpoint{1.665359in}{3.619077in}}%
\pgfpathlineto{\pgfqpoint{1.680240in}{3.368821in}}%
\pgfpathlineto{\pgfqpoint{1.695122in}{3.697026in}}%
\pgfpathlineto{\pgfqpoint{1.710003in}{3.879590in}}%
\pgfpathlineto{\pgfqpoint{1.724884in}{3.426256in}}%
\pgfpathlineto{\pgfqpoint{1.739766in}{3.282667in}}%
\pgfpathlineto{\pgfqpoint{1.754647in}{3.768821in}}%
\pgfpathlineto{\pgfqpoint{1.769529in}{4.006769in}}%
\pgfpathlineto{\pgfqpoint{1.784410in}{3.729846in}}%
\pgfpathlineto{\pgfqpoint{1.799292in}{3.383179in}}%
\pgfpathlineto{\pgfqpoint{1.814173in}{3.522667in}}%
\pgfpathlineto{\pgfqpoint{1.829055in}{3.649846in}}%
\pgfpathlineto{\pgfqpoint{1.843936in}{3.520615in}}%
\pgfpathlineto{\pgfqpoint{1.858818in}{3.733949in}}%
\pgfpathlineto{\pgfqpoint{1.873699in}{3.658051in}}%
\pgfpathlineto{\pgfqpoint{1.888581in}{3.407795in}}%
\pgfpathlineto{\pgfqpoint{1.903462in}{3.399590in}}%
\pgfpathlineto{\pgfqpoint{1.918344in}{3.309333in}}%
\pgfpathlineto{\pgfqpoint{1.933225in}{3.528821in}}%
\pgfpathlineto{\pgfqpoint{1.948107in}{3.450872in}}%
\pgfpathlineto{\pgfqpoint{1.962988in}{3.791385in}}%
\pgfpathlineto{\pgfqpoint{1.977870in}{3.354462in}}%
\pgfpathlineto{\pgfqpoint{1.992751in}{3.801641in}}%
\pgfpathlineto{\pgfqpoint{2.007633in}{3.485744in}}%
\pgfpathlineto{\pgfqpoint{2.022514in}{3.532923in}}%
\pgfpathlineto{\pgfqpoint{2.037396in}{3.465231in}}%
\pgfpathlineto{\pgfqpoint{2.052277in}{3.787282in}}%
\pgfpathlineto{\pgfqpoint{2.067159in}{3.247795in}}%
\pgfpathlineto{\pgfqpoint{2.082040in}{3.653949in}}%
\pgfpathlineto{\pgfqpoint{2.096922in}{3.625231in}}%
\pgfpathlineto{\pgfqpoint{2.111803in}{3.680615in}}%
\pgfpathlineto{\pgfqpoint{2.126685in}{3.830359in}}%
\pgfpathlineto{\pgfqpoint{2.141566in}{3.307282in}}%
\pgfpathlineto{\pgfqpoint{2.156448in}{3.891897in}}%
\pgfpathlineto{\pgfqpoint{2.171329in}{3.522667in}}%
\pgfpathlineto{\pgfqpoint{2.186211in}{3.331897in}}%
\pgfpathlineto{\pgfqpoint{2.201092in}{3.258051in}}%
\pgfpathlineto{\pgfqpoint{2.215974in}{3.762667in}}%
\pgfpathlineto{\pgfqpoint{2.230855in}{3.407795in}}%
\pgfpathlineto{\pgfqpoint{2.245737in}{3.614974in}}%
\pgfpathlineto{\pgfqpoint{2.260618in}{3.526769in}}%
\pgfpathlineto{\pgfqpoint{2.275500in}{3.803692in}}%
\pgfpathlineto{\pgfqpoint{2.290381in}{3.569846in}}%
\pgfpathlineto{\pgfqpoint{2.305263in}{3.871385in}}%
\pgfpathlineto{\pgfqpoint{2.320144in}{3.364718in}}%
\pgfpathlineto{\pgfqpoint{2.335026in}{3.602667in}}%
\pgfpathlineto{\pgfqpoint{2.349907in}{3.420103in}}%
\pgfpathlineto{\pgfqpoint{2.364788in}{3.553436in}}%
\pgfpathlineto{\pgfqpoint{2.379670in}{3.545231in}}%
\pgfpathlineto{\pgfqpoint{2.394551in}{3.660103in}}%
\pgfpathlineto{\pgfqpoint{2.409433in}{3.707282in}}%
\pgfpathlineto{\pgfqpoint{2.424314in}{3.945231in}}%
\pgfpathlineto{\pgfqpoint{2.439196in}{3.208821in}}%
\pgfpathlineto{\pgfqpoint{2.454077in}{3.688821in}}%
\pgfpathlineto{\pgfqpoint{2.468959in}{3.520615in}}%
\pgfpathlineto{\pgfqpoint{2.483840in}{3.658051in}}%
\pgfpathlineto{\pgfqpoint{2.498722in}{3.063179in}}%
\pgfpathlineto{\pgfqpoint{2.826115in}{0.696000in}}%
\pgfpathlineto{\pgfqpoint{4.135686in}{0.696000in}}%
\pgfpathlineto{\pgfqpoint{4.150567in}{1.171897in}}%
\pgfpathlineto{\pgfqpoint{4.165449in}{0.700103in}}%
\pgfpathlineto{\pgfqpoint{4.180330in}{1.454974in}}%
\pgfpathlineto{\pgfqpoint{4.195212in}{0.954462in}}%
\pgfpathlineto{\pgfqpoint{4.210093in}{1.537026in}}%
\pgfpathlineto{\pgfqpoint{4.224974in}{0.905231in}}%
\pgfpathlineto{\pgfqpoint{4.254737in}{0.927795in}}%
\pgfpathlineto{\pgfqpoint{4.269619in}{1.528821in}}%
\pgfpathlineto{\pgfqpoint{4.284500in}{0.913436in}}%
\pgfpathlineto{\pgfqpoint{4.299382in}{1.598564in}}%
\pgfpathlineto{\pgfqpoint{4.314263in}{1.457026in}}%
\pgfpathlineto{\pgfqpoint{4.329145in}{1.576000in}}%
\pgfpathlineto{\pgfqpoint{4.344026in}{1.916513in}}%
\pgfpathlineto{\pgfqpoint{4.358908in}{1.631385in}}%
\pgfpathlineto{\pgfqpoint{4.373789in}{3.379077in}}%
\pgfpathlineto{\pgfqpoint{4.388671in}{3.537026in}}%
\pgfpathlineto{\pgfqpoint{4.403552in}{3.721641in}}%
\pgfpathlineto{\pgfqpoint{4.418434in}{3.506256in}}%
\pgfpathlineto{\pgfqpoint{4.433315in}{3.594462in}}%
\pgfpathlineto{\pgfqpoint{4.448197in}{3.606769in}}%
\pgfpathlineto{\pgfqpoint{4.463078in}{3.715487in}}%
\pgfpathlineto{\pgfqpoint{4.477960in}{3.036513in}}%
\pgfpathlineto{\pgfqpoint{4.492841in}{3.941128in}}%
\pgfpathlineto{\pgfqpoint{4.507723in}{3.379077in}}%
\pgfpathlineto{\pgfqpoint{4.522604in}{3.128821in}}%
\pgfpathlineto{\pgfqpoint{4.537486in}{3.582154in}}%
\pgfpathlineto{\pgfqpoint{4.552367in}{3.481641in}}%
\pgfpathlineto{\pgfqpoint{4.567249in}{3.711385in}}%
\pgfpathlineto{\pgfqpoint{4.582130in}{3.202667in}}%
\pgfpathlineto{\pgfqpoint{4.597012in}{3.791385in}}%
\pgfpathlineto{\pgfqpoint{4.611893in}{3.182154in}}%
\pgfpathlineto{\pgfqpoint{4.626775in}{3.746256in}}%
\pgfpathlineto{\pgfqpoint{4.656538in}{3.481641in}}%
\pgfpathlineto{\pgfqpoint{4.671419in}{3.325744in}}%
\pgfpathlineto{\pgfqpoint{4.686301in}{3.672410in}}%
\pgfpathlineto{\pgfqpoint{4.701182in}{3.307282in}}%
\pgfpathlineto{\pgfqpoint{4.716064in}{2.886769in}}%
\pgfpathlineto{\pgfqpoint{4.730945in}{3.274462in}}%
\pgfpathlineto{\pgfqpoint{4.745827in}{2.544205in}}%
\pgfpathlineto{\pgfqpoint{4.760708in}{3.485744in}}%
\pgfpathlineto{\pgfqpoint{4.775590in}{3.524718in}}%
\pgfpathlineto{\pgfqpoint{4.790471in}{3.087795in}}%
\pgfpathlineto{\pgfqpoint{4.805353in}{3.448821in}}%
\pgfpathlineto{\pgfqpoint{4.820234in}{3.625231in}}%
\pgfpathlineto{\pgfqpoint{4.835116in}{3.389333in}}%
\pgfpathlineto{\pgfqpoint{4.849997in}{3.614974in}}%
\pgfpathlineto{\pgfqpoint{4.864878in}{3.813949in}}%
\pgfpathlineto{\pgfqpoint{4.879760in}{3.155487in}}%
\pgfpathlineto{\pgfqpoint{4.894641in}{3.541128in}}%
\pgfpathlineto{\pgfqpoint{4.909523in}{3.457026in}}%
\pgfpathlineto{\pgfqpoint{4.924404in}{3.465231in}}%
\pgfpathlineto{\pgfqpoint{4.939286in}{3.610872in}}%
\pgfpathlineto{\pgfqpoint{4.954167in}{3.498051in}}%
\pgfpathlineto{\pgfqpoint{4.969049in}{3.407795in}}%
\pgfpathlineto{\pgfqpoint{4.983930in}{3.643692in}}%
\pgfpathlineto{\pgfqpoint{4.998812in}{3.424205in}}%
\pgfpathlineto{\pgfqpoint{5.013693in}{3.350359in}}%
\pgfpathlineto{\pgfqpoint{5.028575in}{3.699077in}}%
\pgfpathlineto{\pgfqpoint{5.043456in}{3.434462in}}%
\pgfpathlineto{\pgfqpoint{5.058338in}{3.446769in}}%
\pgfpathlineto{\pgfqpoint{5.073219in}{3.432410in}}%
\pgfpathlineto{\pgfqpoint{5.088101in}{3.422154in}}%
\pgfpathlineto{\pgfqpoint{5.102982in}{3.450872in}}%
\pgfpathlineto{\pgfqpoint{5.117864in}{3.319590in}}%
\pgfpathlineto{\pgfqpoint{5.132745in}{3.219077in}}%
\pgfpathlineto{\pgfqpoint{5.147627in}{3.364718in}}%
\pgfpathlineto{\pgfqpoint{5.162508in}{3.813949in}}%
\pgfpathlineto{\pgfqpoint{5.177390in}{3.264205in}}%
\pgfpathlineto{\pgfqpoint{5.192271in}{3.405744in}}%
\pgfpathlineto{\pgfqpoint{5.207153in}{3.760615in}}%
\pgfpathlineto{\pgfqpoint{5.222034in}{3.116513in}}%
\pgfpathlineto{\pgfqpoint{5.236916in}{3.301128in}}%
\pgfpathlineto{\pgfqpoint{5.251797in}{3.551385in}}%
\pgfpathlineto{\pgfqpoint{5.266679in}{3.588308in}}%
\pgfpathlineto{\pgfqpoint{5.281560in}{3.475487in}}%
\pgfpathlineto{\pgfqpoint{5.296442in}{3.038564in}}%
\pgfpathlineto{\pgfqpoint{5.311323in}{3.424205in}}%
\pgfpathlineto{\pgfqpoint{5.326205in}{3.731897in}}%
\pgfpathlineto{\pgfqpoint{5.341086in}{3.249846in}}%
\pgfpathlineto{\pgfqpoint{5.355968in}{3.321641in}}%
\pgfpathlineto{\pgfqpoint{5.370849in}{3.590359in}}%
\pgfpathlineto{\pgfqpoint{5.385731in}{3.680615in}}%
\pgfpathlineto{\pgfqpoint{5.400612in}{3.409846in}}%
\pgfpathlineto{\pgfqpoint{5.415494in}{3.260103in}}%
\pgfpathlineto{\pgfqpoint{5.430375in}{3.561641in}}%
\pgfpathlineto{\pgfqpoint{5.445257in}{3.282667in}}%
\pgfpathlineto{\pgfqpoint{5.460138in}{3.467282in}}%
\pgfpathlineto{\pgfqpoint{5.475020in}{3.498051in}}%
\pgfpathlineto{\pgfqpoint{5.489901in}{3.479590in}}%
\pgfpathlineto{\pgfqpoint{5.504782in}{3.496000in}}%
\pgfpathlineto{\pgfqpoint{5.519664in}{3.500103in}}%
\pgfpathlineto{\pgfqpoint{5.534545in}{2.923692in}}%
\pgfpathlineto{\pgfqpoint{5.534545in}{2.923692in}}%
\pgfusepath{stroke}%
\end{pgfscope}%
\begin{pgfscope}%
\pgfsetrectcap%
\pgfsetmiterjoin%
\pgfsetlinewidth{0.803000pt}%
\definecolor{currentstroke}{rgb}{0.000000,0.000000,0.000000}%
\pgfsetstrokecolor{currentstroke}%
\pgfsetdash{}{0pt}%
\pgfpathmoveto{\pgfqpoint{0.800000in}{0.528000in}}%
\pgfpathlineto{\pgfqpoint{0.800000in}{4.224000in}}%
\pgfusepath{stroke}%
\end{pgfscope}%
\begin{pgfscope}%
\pgfsetrectcap%
\pgfsetmiterjoin%
\pgfsetlinewidth{0.803000pt}%
\definecolor{currentstroke}{rgb}{0.000000,0.000000,0.000000}%
\pgfsetstrokecolor{currentstroke}%
\pgfsetdash{}{0pt}%
\pgfpathmoveto{\pgfqpoint{5.760000in}{0.528000in}}%
\pgfpathlineto{\pgfqpoint{5.760000in}{4.224000in}}%
\pgfusepath{stroke}%
\end{pgfscope}%
\begin{pgfscope}%
\pgfsetrectcap%
\pgfsetmiterjoin%
\pgfsetlinewidth{0.803000pt}%
\definecolor{currentstroke}{rgb}{0.000000,0.000000,0.000000}%
\pgfsetstrokecolor{currentstroke}%
\pgfsetdash{}{0pt}%
\pgfpathmoveto{\pgfqpoint{0.800000in}{0.528000in}}%
\pgfpathlineto{\pgfqpoint{5.760000in}{0.528000in}}%
\pgfusepath{stroke}%
\end{pgfscope}%
\begin{pgfscope}%
\pgfsetrectcap%
\pgfsetmiterjoin%
\pgfsetlinewidth{0.803000pt}%
\definecolor{currentstroke}{rgb}{0.000000,0.000000,0.000000}%
\pgfsetstrokecolor{currentstroke}%
\pgfsetdash{}{0pt}%
\pgfpathmoveto{\pgfqpoint{0.800000in}{4.224000in}}%
\pgfpathlineto{\pgfqpoint{5.760000in}{4.224000in}}%
\pgfusepath{stroke}%
\end{pgfscope}%
\begin{pgfscope}%
\pgfsetbuttcap%
\pgfsetmiterjoin%
\definecolor{currentfill}{rgb}{1.000000,1.000000,1.000000}%
\pgfsetfillcolor{currentfill}%
\pgfsetfillopacity{0.800000}%
\pgfsetlinewidth{1.003750pt}%
\definecolor{currentstroke}{rgb}{0.800000,0.800000,0.800000}%
\pgfsetstrokecolor{currentstroke}%
\pgfsetstrokeopacity{0.800000}%
\pgfsetdash{}{0pt}%
\pgfpathmoveto{\pgfqpoint{5.591944in}{4.025389in}}%
\pgfpathlineto{\pgfqpoint{5.653056in}{4.025389in}}%
\pgfpathquadraticcurveto{\pgfqpoint{5.683611in}{4.025389in}}{\pgfqpoint{5.683611in}{4.055944in}}%
\pgfpathlineto{\pgfqpoint{5.683611in}{4.117056in}}%
\pgfpathquadraticcurveto{\pgfqpoint{5.683611in}{4.147611in}}{\pgfqpoint{5.653056in}{4.147611in}}%
\pgfpathlineto{\pgfqpoint{5.591944in}{4.147611in}}%
\pgfpathquadraticcurveto{\pgfqpoint{5.561389in}{4.147611in}}{\pgfqpoint{5.561389in}{4.117056in}}%
\pgfpathlineto{\pgfqpoint{5.561389in}{4.055944in}}%
\pgfpathquadraticcurveto{\pgfqpoint{5.561389in}{4.025389in}}{\pgfqpoint{5.591944in}{4.025389in}}%
\pgfpathclose%
\pgfusepath{stroke,fill}%
\end{pgfscope}%
\end{pgfpicture}%
\makeatother%
\endgroup%

    \caption{Distribution of successful write operations of experiment 3 in \prettyref{tab:res-8}}
    \label{fig:writes-36pmm}
\end{figure}

\begin{figure}
    %% Creator: Matplotlib, PGF backend
%%
%% To include the figure in your LaTeX document, write
%%   \input{<filename>.pgf}
%%
%% Make sure the required packages are loaded in your preamble
%%   \usepackage{pgf}
%%
%% Figures using additional raster images can only be included by \input if
%% they are in the same directory as the main LaTeX file. For loading figures
%% from other directories you can use the `import` package
%%   \usepackage{import}
%% and then include the figures with
%%   \import{<path to file>}{<filename>.pgf}
%%
%% Matplotlib used the following preamble
%%   \usepackage[utf8x]{inputenc}
%%   \usepackage[T1]{fontenc}
%%   \usepackage{lmodern}
%%
\begingroup%
\makeatletter%
\begin{pgfpicture}%
\pgfpathrectangle{\pgfpointorigin}{\pgfqpoint{6.400000in}{4.800000in}}%
\pgfusepath{use as bounding box, clip}%
\begin{pgfscope}%
\pgfsetbuttcap%
\pgfsetmiterjoin%
\definecolor{currentfill}{rgb}{1.000000,1.000000,1.000000}%
\pgfsetfillcolor{currentfill}%
\pgfsetlinewidth{0.000000pt}%
\definecolor{currentstroke}{rgb}{1.000000,1.000000,1.000000}%
\pgfsetstrokecolor{currentstroke}%
\pgfsetdash{}{0pt}%
\pgfpathmoveto{\pgfqpoint{0.000000in}{0.000000in}}%
\pgfpathlineto{\pgfqpoint{6.400000in}{0.000000in}}%
\pgfpathlineto{\pgfqpoint{6.400000in}{4.800000in}}%
\pgfpathlineto{\pgfqpoint{0.000000in}{4.800000in}}%
\pgfpathclose%
\pgfusepath{fill}%
\end{pgfscope}%
\begin{pgfscope}%
\pgfsetbuttcap%
\pgfsetmiterjoin%
\definecolor{currentfill}{rgb}{1.000000,1.000000,1.000000}%
\pgfsetfillcolor{currentfill}%
\pgfsetlinewidth{0.000000pt}%
\definecolor{currentstroke}{rgb}{0.000000,0.000000,0.000000}%
\pgfsetstrokecolor{currentstroke}%
\pgfsetstrokeopacity{0.000000}%
\pgfsetdash{}{0pt}%
\pgfpathmoveto{\pgfqpoint{0.800000in}{0.528000in}}%
\pgfpathlineto{\pgfqpoint{5.760000in}{0.528000in}}%
\pgfpathlineto{\pgfqpoint{5.760000in}{4.224000in}}%
\pgfpathlineto{\pgfqpoint{0.800000in}{4.224000in}}%
\pgfpathclose%
\pgfusepath{fill}%
\end{pgfscope}%
\begin{pgfscope}%
\pgfsetbuttcap%
\pgfsetroundjoin%
\definecolor{currentfill}{rgb}{0.000000,0.000000,0.000000}%
\pgfsetfillcolor{currentfill}%
\pgfsetlinewidth{0.803000pt}%
\definecolor{currentstroke}{rgb}{0.000000,0.000000,0.000000}%
\pgfsetstrokecolor{currentstroke}%
\pgfsetdash{}{0pt}%
\pgfsys@defobject{currentmarker}{\pgfqpoint{0.000000in}{-0.048611in}}{\pgfqpoint{0.000000in}{0.000000in}}{%
\pgfpathmoveto{\pgfqpoint{0.000000in}{0.000000in}}%
\pgfpathlineto{\pgfqpoint{0.000000in}{-0.048611in}}%
\pgfusepath{stroke,fill}%
}%
\begin{pgfscope}%
\pgfsys@transformshift{1.025455in}{0.528000in}%
\pgfsys@useobject{currentmarker}{}%
\end{pgfscope}%
\end{pgfscope}%
\begin{pgfscope}%
\pgftext[x=1.025455in,y=0.430778in,,top]{\fontsize{11.000000}{13.200000}\selectfont \(\displaystyle 0\)}%
\end{pgfscope}%
\begin{pgfscope}%
\pgfsetbuttcap%
\pgfsetroundjoin%
\definecolor{currentfill}{rgb}{0.000000,0.000000,0.000000}%
\pgfsetfillcolor{currentfill}%
\pgfsetlinewidth{0.803000pt}%
\definecolor{currentstroke}{rgb}{0.000000,0.000000,0.000000}%
\pgfsetstrokecolor{currentstroke}%
\pgfsetdash{}{0pt}%
\pgfsys@defobject{currentmarker}{\pgfqpoint{0.000000in}{-0.048611in}}{\pgfqpoint{0.000000in}{0.000000in}}{%
\pgfpathmoveto{\pgfqpoint{0.000000in}{0.000000in}}%
\pgfpathlineto{\pgfqpoint{0.000000in}{-0.048611in}}%
\pgfusepath{stroke,fill}%
}%
\begin{pgfscope}%
\pgfsys@transformshift{1.769529in}{0.528000in}%
\pgfsys@useobject{currentmarker}{}%
\end{pgfscope}%
\end{pgfscope}%
\begin{pgfscope}%
\pgftext[x=1.769529in,y=0.430778in,,top]{\fontsize{11.000000}{13.200000}\selectfont \(\displaystyle 50\)}%
\end{pgfscope}%
\begin{pgfscope}%
\pgfsetbuttcap%
\pgfsetroundjoin%
\definecolor{currentfill}{rgb}{0.000000,0.000000,0.000000}%
\pgfsetfillcolor{currentfill}%
\pgfsetlinewidth{0.803000pt}%
\definecolor{currentstroke}{rgb}{0.000000,0.000000,0.000000}%
\pgfsetstrokecolor{currentstroke}%
\pgfsetdash{}{0pt}%
\pgfsys@defobject{currentmarker}{\pgfqpoint{0.000000in}{-0.048611in}}{\pgfqpoint{0.000000in}{0.000000in}}{%
\pgfpathmoveto{\pgfqpoint{0.000000in}{0.000000in}}%
\pgfpathlineto{\pgfqpoint{0.000000in}{-0.048611in}}%
\pgfusepath{stroke,fill}%
}%
\begin{pgfscope}%
\pgfsys@transformshift{2.513603in}{0.528000in}%
\pgfsys@useobject{currentmarker}{}%
\end{pgfscope}%
\end{pgfscope}%
\begin{pgfscope}%
\pgftext[x=2.513603in,y=0.430778in,,top]{\fontsize{11.000000}{13.200000}\selectfont \(\displaystyle 100\)}%
\end{pgfscope}%
\begin{pgfscope}%
\pgfsetbuttcap%
\pgfsetroundjoin%
\definecolor{currentfill}{rgb}{0.000000,0.000000,0.000000}%
\pgfsetfillcolor{currentfill}%
\pgfsetlinewidth{0.803000pt}%
\definecolor{currentstroke}{rgb}{0.000000,0.000000,0.000000}%
\pgfsetstrokecolor{currentstroke}%
\pgfsetdash{}{0pt}%
\pgfsys@defobject{currentmarker}{\pgfqpoint{0.000000in}{-0.048611in}}{\pgfqpoint{0.000000in}{0.000000in}}{%
\pgfpathmoveto{\pgfqpoint{0.000000in}{0.000000in}}%
\pgfpathlineto{\pgfqpoint{0.000000in}{-0.048611in}}%
\pgfusepath{stroke,fill}%
}%
\begin{pgfscope}%
\pgfsys@transformshift{3.257678in}{0.528000in}%
\pgfsys@useobject{currentmarker}{}%
\end{pgfscope}%
\end{pgfscope}%
\begin{pgfscope}%
\pgftext[x=3.257678in,y=0.430778in,,top]{\fontsize{11.000000}{13.200000}\selectfont \(\displaystyle 150\)}%
\end{pgfscope}%
\begin{pgfscope}%
\pgfsetbuttcap%
\pgfsetroundjoin%
\definecolor{currentfill}{rgb}{0.000000,0.000000,0.000000}%
\pgfsetfillcolor{currentfill}%
\pgfsetlinewidth{0.803000pt}%
\definecolor{currentstroke}{rgb}{0.000000,0.000000,0.000000}%
\pgfsetstrokecolor{currentstroke}%
\pgfsetdash{}{0pt}%
\pgfsys@defobject{currentmarker}{\pgfqpoint{0.000000in}{-0.048611in}}{\pgfqpoint{0.000000in}{0.000000in}}{%
\pgfpathmoveto{\pgfqpoint{0.000000in}{0.000000in}}%
\pgfpathlineto{\pgfqpoint{0.000000in}{-0.048611in}}%
\pgfusepath{stroke,fill}%
}%
\begin{pgfscope}%
\pgfsys@transformshift{4.001752in}{0.528000in}%
\pgfsys@useobject{currentmarker}{}%
\end{pgfscope}%
\end{pgfscope}%
\begin{pgfscope}%
\pgftext[x=4.001752in,y=0.430778in,,top]{\fontsize{11.000000}{13.200000}\selectfont \(\displaystyle 200\)}%
\end{pgfscope}%
\begin{pgfscope}%
\pgfsetbuttcap%
\pgfsetroundjoin%
\definecolor{currentfill}{rgb}{0.000000,0.000000,0.000000}%
\pgfsetfillcolor{currentfill}%
\pgfsetlinewidth{0.803000pt}%
\definecolor{currentstroke}{rgb}{0.000000,0.000000,0.000000}%
\pgfsetstrokecolor{currentstroke}%
\pgfsetdash{}{0pt}%
\pgfsys@defobject{currentmarker}{\pgfqpoint{0.000000in}{-0.048611in}}{\pgfqpoint{0.000000in}{0.000000in}}{%
\pgfpathmoveto{\pgfqpoint{0.000000in}{0.000000in}}%
\pgfpathlineto{\pgfqpoint{0.000000in}{-0.048611in}}%
\pgfusepath{stroke,fill}%
}%
\begin{pgfscope}%
\pgfsys@transformshift{4.745827in}{0.528000in}%
\pgfsys@useobject{currentmarker}{}%
\end{pgfscope}%
\end{pgfscope}%
\begin{pgfscope}%
\pgftext[x=4.745827in,y=0.430778in,,top]{\fontsize{11.000000}{13.200000}\selectfont \(\displaystyle 250\)}%
\end{pgfscope}%
\begin{pgfscope}%
\pgfsetbuttcap%
\pgfsetroundjoin%
\definecolor{currentfill}{rgb}{0.000000,0.000000,0.000000}%
\pgfsetfillcolor{currentfill}%
\pgfsetlinewidth{0.803000pt}%
\definecolor{currentstroke}{rgb}{0.000000,0.000000,0.000000}%
\pgfsetstrokecolor{currentstroke}%
\pgfsetdash{}{0pt}%
\pgfsys@defobject{currentmarker}{\pgfqpoint{0.000000in}{-0.048611in}}{\pgfqpoint{0.000000in}{0.000000in}}{%
\pgfpathmoveto{\pgfqpoint{0.000000in}{0.000000in}}%
\pgfpathlineto{\pgfqpoint{0.000000in}{-0.048611in}}%
\pgfusepath{stroke,fill}%
}%
\begin{pgfscope}%
\pgfsys@transformshift{5.489901in}{0.528000in}%
\pgfsys@useobject{currentmarker}{}%
\end{pgfscope}%
\end{pgfscope}%
\begin{pgfscope}%
\pgftext[x=5.489901in,y=0.430778in,,top]{\fontsize{11.000000}{13.200000}\selectfont \(\displaystyle 300\)}%
\end{pgfscope}%
\begin{pgfscope}%
\pgftext[x=3.280000in,y=0.240271in,,top]{\fontsize{11.000000}{13.200000}\selectfont Time of experiment (in seconds)}%
\end{pgfscope}%
\begin{pgfscope}%
\pgfsetbuttcap%
\pgfsetroundjoin%
\definecolor{currentfill}{rgb}{0.000000,0.000000,0.000000}%
\pgfsetfillcolor{currentfill}%
\pgfsetlinewidth{0.803000pt}%
\definecolor{currentstroke}{rgb}{0.000000,0.000000,0.000000}%
\pgfsetstrokecolor{currentstroke}%
\pgfsetdash{}{0pt}%
\pgfsys@defobject{currentmarker}{\pgfqpoint{-0.048611in}{0.000000in}}{\pgfqpoint{0.000000in}{0.000000in}}{%
\pgfpathmoveto{\pgfqpoint{0.000000in}{0.000000in}}%
\pgfpathlineto{\pgfqpoint{-0.048611in}{0.000000in}}%
\pgfusepath{stroke,fill}%
}%
\begin{pgfscope}%
\pgfsys@transformshift{0.800000in}{0.691472in}%
\pgfsys@useobject{currentmarker}{}%
\end{pgfscope}%
\end{pgfscope}%
\begin{pgfscope}%
\pgftext[x=0.627981in,y=0.638849in,left,base]{\fontsize{11.000000}{13.200000}\selectfont \(\displaystyle 0\)}%
\end{pgfscope}%
\begin{pgfscope}%
\pgfsetbuttcap%
\pgfsetroundjoin%
\definecolor{currentfill}{rgb}{0.000000,0.000000,0.000000}%
\pgfsetfillcolor{currentfill}%
\pgfsetlinewidth{0.803000pt}%
\definecolor{currentstroke}{rgb}{0.000000,0.000000,0.000000}%
\pgfsetstrokecolor{currentstroke}%
\pgfsetdash{}{0pt}%
\pgfsys@defobject{currentmarker}{\pgfqpoint{-0.048611in}{0.000000in}}{\pgfqpoint{0.000000in}{0.000000in}}{%
\pgfpathmoveto{\pgfqpoint{0.000000in}{0.000000in}}%
\pgfpathlineto{\pgfqpoint{-0.048611in}{0.000000in}}%
\pgfusepath{stroke,fill}%
}%
\begin{pgfscope}%
\pgfsys@transformshift{0.800000in}{1.144302in}%
\pgfsys@useobject{currentmarker}{}%
\end{pgfscope}%
\end{pgfscope}%
\begin{pgfscope}%
\pgftext[x=0.478386in,y=1.091680in,left,base]{\fontsize{11.000000}{13.200000}\selectfont \(\displaystyle 100\)}%
\end{pgfscope}%
\begin{pgfscope}%
\pgfsetbuttcap%
\pgfsetroundjoin%
\definecolor{currentfill}{rgb}{0.000000,0.000000,0.000000}%
\pgfsetfillcolor{currentfill}%
\pgfsetlinewidth{0.803000pt}%
\definecolor{currentstroke}{rgb}{0.000000,0.000000,0.000000}%
\pgfsetstrokecolor{currentstroke}%
\pgfsetdash{}{0pt}%
\pgfsys@defobject{currentmarker}{\pgfqpoint{-0.048611in}{0.000000in}}{\pgfqpoint{0.000000in}{0.000000in}}{%
\pgfpathmoveto{\pgfqpoint{0.000000in}{0.000000in}}%
\pgfpathlineto{\pgfqpoint{-0.048611in}{0.000000in}}%
\pgfusepath{stroke,fill}%
}%
\begin{pgfscope}%
\pgfsys@transformshift{0.800000in}{1.597132in}%
\pgfsys@useobject{currentmarker}{}%
\end{pgfscope}%
\end{pgfscope}%
\begin{pgfscope}%
\pgftext[x=0.478386in,y=1.544510in,left,base]{\fontsize{11.000000}{13.200000}\selectfont \(\displaystyle 200\)}%
\end{pgfscope}%
\begin{pgfscope}%
\pgfsetbuttcap%
\pgfsetroundjoin%
\definecolor{currentfill}{rgb}{0.000000,0.000000,0.000000}%
\pgfsetfillcolor{currentfill}%
\pgfsetlinewidth{0.803000pt}%
\definecolor{currentstroke}{rgb}{0.000000,0.000000,0.000000}%
\pgfsetstrokecolor{currentstroke}%
\pgfsetdash{}{0pt}%
\pgfsys@defobject{currentmarker}{\pgfqpoint{-0.048611in}{0.000000in}}{\pgfqpoint{0.000000in}{0.000000in}}{%
\pgfpathmoveto{\pgfqpoint{0.000000in}{0.000000in}}%
\pgfpathlineto{\pgfqpoint{-0.048611in}{0.000000in}}%
\pgfusepath{stroke,fill}%
}%
\begin{pgfscope}%
\pgfsys@transformshift{0.800000in}{2.049962in}%
\pgfsys@useobject{currentmarker}{}%
\end{pgfscope}%
\end{pgfscope}%
\begin{pgfscope}%
\pgftext[x=0.478386in,y=1.997340in,left,base]{\fontsize{11.000000}{13.200000}\selectfont \(\displaystyle 300\)}%
\end{pgfscope}%
\begin{pgfscope}%
\pgfsetbuttcap%
\pgfsetroundjoin%
\definecolor{currentfill}{rgb}{0.000000,0.000000,0.000000}%
\pgfsetfillcolor{currentfill}%
\pgfsetlinewidth{0.803000pt}%
\definecolor{currentstroke}{rgb}{0.000000,0.000000,0.000000}%
\pgfsetstrokecolor{currentstroke}%
\pgfsetdash{}{0pt}%
\pgfsys@defobject{currentmarker}{\pgfqpoint{-0.048611in}{0.000000in}}{\pgfqpoint{0.000000in}{0.000000in}}{%
\pgfpathmoveto{\pgfqpoint{0.000000in}{0.000000in}}%
\pgfpathlineto{\pgfqpoint{-0.048611in}{0.000000in}}%
\pgfusepath{stroke,fill}%
}%
\begin{pgfscope}%
\pgfsys@transformshift{0.800000in}{2.502792in}%
\pgfsys@useobject{currentmarker}{}%
\end{pgfscope}%
\end{pgfscope}%
\begin{pgfscope}%
\pgftext[x=0.478386in,y=2.450170in,left,base]{\fontsize{11.000000}{13.200000}\selectfont \(\displaystyle 400\)}%
\end{pgfscope}%
\begin{pgfscope}%
\pgfsetbuttcap%
\pgfsetroundjoin%
\definecolor{currentfill}{rgb}{0.000000,0.000000,0.000000}%
\pgfsetfillcolor{currentfill}%
\pgfsetlinewidth{0.803000pt}%
\definecolor{currentstroke}{rgb}{0.000000,0.000000,0.000000}%
\pgfsetstrokecolor{currentstroke}%
\pgfsetdash{}{0pt}%
\pgfsys@defobject{currentmarker}{\pgfqpoint{-0.048611in}{0.000000in}}{\pgfqpoint{0.000000in}{0.000000in}}{%
\pgfpathmoveto{\pgfqpoint{0.000000in}{0.000000in}}%
\pgfpathlineto{\pgfqpoint{-0.048611in}{0.000000in}}%
\pgfusepath{stroke,fill}%
}%
\begin{pgfscope}%
\pgfsys@transformshift{0.800000in}{2.955623in}%
\pgfsys@useobject{currentmarker}{}%
\end{pgfscope}%
\end{pgfscope}%
\begin{pgfscope}%
\pgftext[x=0.478386in,y=2.903000in,left,base]{\fontsize{11.000000}{13.200000}\selectfont \(\displaystyle 500\)}%
\end{pgfscope}%
\begin{pgfscope}%
\pgfsetbuttcap%
\pgfsetroundjoin%
\definecolor{currentfill}{rgb}{0.000000,0.000000,0.000000}%
\pgfsetfillcolor{currentfill}%
\pgfsetlinewidth{0.803000pt}%
\definecolor{currentstroke}{rgb}{0.000000,0.000000,0.000000}%
\pgfsetstrokecolor{currentstroke}%
\pgfsetdash{}{0pt}%
\pgfsys@defobject{currentmarker}{\pgfqpoint{-0.048611in}{0.000000in}}{\pgfqpoint{0.000000in}{0.000000in}}{%
\pgfpathmoveto{\pgfqpoint{0.000000in}{0.000000in}}%
\pgfpathlineto{\pgfqpoint{-0.048611in}{0.000000in}}%
\pgfusepath{stroke,fill}%
}%
\begin{pgfscope}%
\pgfsys@transformshift{0.800000in}{3.408453in}%
\pgfsys@useobject{currentmarker}{}%
\end{pgfscope}%
\end{pgfscope}%
\begin{pgfscope}%
\pgftext[x=0.478386in,y=3.355831in,left,base]{\fontsize{11.000000}{13.200000}\selectfont \(\displaystyle 600\)}%
\end{pgfscope}%
\begin{pgfscope}%
\pgfsetbuttcap%
\pgfsetroundjoin%
\definecolor{currentfill}{rgb}{0.000000,0.000000,0.000000}%
\pgfsetfillcolor{currentfill}%
\pgfsetlinewidth{0.803000pt}%
\definecolor{currentstroke}{rgb}{0.000000,0.000000,0.000000}%
\pgfsetstrokecolor{currentstroke}%
\pgfsetdash{}{0pt}%
\pgfsys@defobject{currentmarker}{\pgfqpoint{-0.048611in}{0.000000in}}{\pgfqpoint{0.000000in}{0.000000in}}{%
\pgfpathmoveto{\pgfqpoint{0.000000in}{0.000000in}}%
\pgfpathlineto{\pgfqpoint{-0.048611in}{0.000000in}}%
\pgfusepath{stroke,fill}%
}%
\begin{pgfscope}%
\pgfsys@transformshift{0.800000in}{3.861283in}%
\pgfsys@useobject{currentmarker}{}%
\end{pgfscope}%
\end{pgfscope}%
\begin{pgfscope}%
\pgftext[x=0.478386in,y=3.808661in,left,base]{\fontsize{11.000000}{13.200000}\selectfont \(\displaystyle 700\)}%
\end{pgfscope}%
\begin{pgfscope}%
\pgftext[x=0.422830in,y=2.376000in,,bottom,rotate=90.000000]{\fontsize{11.000000}{13.200000}\selectfont Number of successful reads}%
\end{pgfscope}%
\begin{pgfscope}%
\pgfpathrectangle{\pgfqpoint{0.800000in}{0.528000in}}{\pgfqpoint{4.960000in}{3.696000in}}%
\pgfusepath{clip}%
\pgfsetrectcap%
\pgfsetroundjoin%
\pgfsetlinewidth{1.505625pt}%
\definecolor{currentstroke}{rgb}{0.121569,0.466667,0.705882}%
\pgfsetstrokecolor{currentstroke}%
\pgfsetdash{}{0pt}%
\pgfpathmoveto{\pgfqpoint{1.025455in}{0.990340in}}%
\pgfpathlineto{\pgfqpoint{1.040336in}{1.248453in}}%
\pgfpathlineto{\pgfqpoint{1.055218in}{1.751094in}}%
\pgfpathlineto{\pgfqpoint{1.070099in}{2.955623in}}%
\pgfpathlineto{\pgfqpoint{1.084980in}{2.928453in}}%
\pgfpathlineto{\pgfqpoint{1.099862in}{3.422038in}}%
\pgfpathlineto{\pgfqpoint{1.114743in}{3.761660in}}%
\pgfpathlineto{\pgfqpoint{1.129625in}{3.689208in}}%
\pgfpathlineto{\pgfqpoint{1.144506in}{3.132226in}}%
\pgfpathlineto{\pgfqpoint{1.159388in}{3.757132in}}%
\pgfpathlineto{\pgfqpoint{1.174269in}{3.426566in}}%
\pgfpathlineto{\pgfqpoint{1.189151in}{2.987321in}}%
\pgfpathlineto{\pgfqpoint{1.204032in}{3.109585in}}%
\pgfpathlineto{\pgfqpoint{1.218914in}{3.254491in}}%
\pgfpathlineto{\pgfqpoint{1.233795in}{3.775245in}}%
\pgfpathlineto{\pgfqpoint{1.248677in}{3.231849in}}%
\pgfpathlineto{\pgfqpoint{1.263558in}{2.991849in}}%
\pgfpathlineto{\pgfqpoint{1.278440in}{3.544302in}}%
\pgfpathlineto{\pgfqpoint{1.293321in}{3.272604in}}%
\pgfpathlineto{\pgfqpoint{1.308203in}{4.056000in}}%
\pgfpathlineto{\pgfqpoint{1.323084in}{3.209208in}}%
\pgfpathlineto{\pgfqpoint{1.337966in}{3.159396in}}%
\pgfpathlineto{\pgfqpoint{1.352847in}{3.394868in}}%
\pgfpathlineto{\pgfqpoint{1.367729in}{3.376755in}}%
\pgfpathlineto{\pgfqpoint{1.382610in}{3.182038in}}%
\pgfpathlineto{\pgfqpoint{1.397492in}{3.431094in}}%
\pgfpathlineto{\pgfqpoint{1.412373in}{3.195623in}}%
\pgfpathlineto{\pgfqpoint{1.427255in}{3.313358in}}%
\pgfpathlineto{\pgfqpoint{1.442136in}{3.757132in}}%
\pgfpathlineto{\pgfqpoint{1.457018in}{2.932981in}}%
\pgfpathlineto{\pgfqpoint{1.471899in}{3.000906in}}%
\pgfpathlineto{\pgfqpoint{1.486781in}{3.249962in}}%
\pgfpathlineto{\pgfqpoint{1.516544in}{3.544302in}}%
\pgfpathlineto{\pgfqpoint{1.531425in}{3.249962in}}%
\pgfpathlineto{\pgfqpoint{1.546307in}{3.249962in}}%
\pgfpathlineto{\pgfqpoint{1.561188in}{3.077887in}}%
\pgfpathlineto{\pgfqpoint{1.590951in}{3.403925in}}%
\pgfpathlineto{\pgfqpoint{1.605833in}{3.680151in}}%
\pgfpathlineto{\pgfqpoint{1.620714in}{2.892226in}}%
\pgfpathlineto{\pgfqpoint{1.635596in}{3.209208in}}%
\pgfpathlineto{\pgfqpoint{1.650477in}{3.734491in}}%
\pgfpathlineto{\pgfqpoint{1.665359in}{3.073358in}}%
\pgfpathlineto{\pgfqpoint{1.680240in}{3.530717in}}%
\pgfpathlineto{\pgfqpoint{1.695122in}{3.354113in}}%
\pgfpathlineto{\pgfqpoint{1.710003in}{3.431094in}}%
\pgfpathlineto{\pgfqpoint{1.724884in}{3.349585in}}%
\pgfpathlineto{\pgfqpoint{1.739766in}{3.689208in}}%
\pgfpathlineto{\pgfqpoint{1.754647in}{3.426566in}}%
\pgfpathlineto{\pgfqpoint{1.769529in}{3.716377in}}%
\pgfpathlineto{\pgfqpoint{1.784410in}{3.399396in}}%
\pgfpathlineto{\pgfqpoint{1.799292in}{3.630340in}}%
\pgfpathlineto{\pgfqpoint{1.814173in}{3.200151in}}%
\pgfpathlineto{\pgfqpoint{1.829055in}{3.693736in}}%
\pgfpathlineto{\pgfqpoint{1.843936in}{3.657509in}}%
\pgfpathlineto{\pgfqpoint{1.858818in}{3.213736in}}%
\pgfpathlineto{\pgfqpoint{1.873699in}{3.462792in}}%
\pgfpathlineto{\pgfqpoint{1.888581in}{3.435623in}}%
\pgfpathlineto{\pgfqpoint{1.903462in}{3.422038in}}%
\pgfpathlineto{\pgfqpoint{1.918344in}{3.625811in}}%
\pgfpathlineto{\pgfqpoint{1.933225in}{3.145811in}}%
\pgfpathlineto{\pgfqpoint{1.948107in}{3.299774in}}%
\pgfpathlineto{\pgfqpoint{1.962988in}{3.367698in}}%
\pgfpathlineto{\pgfqpoint{1.977870in}{3.566943in}}%
\pgfpathlineto{\pgfqpoint{1.992751in}{3.340528in}}%
\pgfpathlineto{\pgfqpoint{2.022514in}{3.172981in}}%
\pgfpathlineto{\pgfqpoint{2.037396in}{3.245434in}}%
\pgfpathlineto{\pgfqpoint{2.052277in}{3.376755in}}%
\pgfpathlineto{\pgfqpoint{2.067159in}{3.761660in}}%
\pgfpathlineto{\pgfqpoint{2.082040in}{3.684679in}}%
\pgfpathlineto{\pgfqpoint{2.111803in}{3.163925in}}%
\pgfpathlineto{\pgfqpoint{2.126685in}{3.240906in}}%
\pgfpathlineto{\pgfqpoint{2.141566in}{3.734491in}}%
\pgfpathlineto{\pgfqpoint{2.156448in}{3.159396in}}%
\pgfpathlineto{\pgfqpoint{2.171329in}{3.331472in}}%
\pgfpathlineto{\pgfqpoint{2.186211in}{3.299774in}}%
\pgfpathlineto{\pgfqpoint{2.201092in}{3.372226in}}%
\pgfpathlineto{\pgfqpoint{2.215974in}{2.901283in}}%
\pgfpathlineto{\pgfqpoint{2.230855in}{3.385811in}}%
\pgfpathlineto{\pgfqpoint{2.245737in}{3.757132in}}%
\pgfpathlineto{\pgfqpoint{2.260618in}{3.245434in}}%
\pgfpathlineto{\pgfqpoint{2.275500in}{3.394868in}}%
\pgfpathlineto{\pgfqpoint{2.290381in}{3.295245in}}%
\pgfpathlineto{\pgfqpoint{2.305263in}{3.748075in}}%
\pgfpathlineto{\pgfqpoint{2.320144in}{3.521660in}}%
\pgfpathlineto{\pgfqpoint{2.335026in}{3.748075in}}%
\pgfpathlineto{\pgfqpoint{2.349907in}{3.105057in}}%
\pgfpathlineto{\pgfqpoint{2.364788in}{3.385811in}}%
\pgfpathlineto{\pgfqpoint{2.379670in}{3.467321in}}%
\pgfpathlineto{\pgfqpoint{2.394551in}{3.304302in}}%
\pgfpathlineto{\pgfqpoint{2.409433in}{3.408453in}}%
\pgfpathlineto{\pgfqpoint{2.424314in}{3.236377in}}%
\pgfpathlineto{\pgfqpoint{2.439196in}{3.788830in}}%
\pgfpathlineto{\pgfqpoint{2.468959in}{3.322415in}}%
\pgfpathlineto{\pgfqpoint{2.483840in}{3.313358in}}%
\pgfpathlineto{\pgfqpoint{2.498722in}{3.050717in}}%
\pgfpathlineto{\pgfqpoint{2.826115in}{1.624302in}}%
\pgfpathlineto{\pgfqpoint{2.900522in}{0.859019in}}%
\pgfpathlineto{\pgfqpoint{2.915404in}{1.506566in}}%
\pgfpathlineto{\pgfqpoint{2.930285in}{0.827321in}}%
\pgfpathlineto{\pgfqpoint{2.974929in}{0.714113in}}%
\pgfpathlineto{\pgfqpoint{2.989811in}{1.153358in}}%
\pgfpathlineto{\pgfqpoint{3.004692in}{1.293736in}}%
\pgfpathlineto{\pgfqpoint{3.049337in}{0.696000in}}%
\pgfpathlineto{\pgfqpoint{3.064218in}{0.849962in}}%
\pgfpathlineto{\pgfqpoint{3.079100in}{1.139774in}}%
\pgfpathlineto{\pgfqpoint{3.093981in}{1.144302in}}%
\pgfpathlineto{\pgfqpoint{3.108863in}{0.736755in}}%
\pgfpathlineto{\pgfqpoint{3.123744in}{0.696000in}}%
\pgfpathlineto{\pgfqpoint{3.153507in}{0.859019in}}%
\pgfpathlineto{\pgfqpoint{3.168389in}{1.361660in}}%
\pgfpathlineto{\pgfqpoint{3.183270in}{0.877132in}}%
\pgfpathlineto{\pgfqpoint{3.198152in}{0.705057in}}%
\pgfpathlineto{\pgfqpoint{3.213033in}{0.727698in}}%
\pgfpathlineto{\pgfqpoint{3.227915in}{0.777509in}}%
\pgfpathlineto{\pgfqpoint{3.242796in}{0.981283in}}%
\pgfpathlineto{\pgfqpoint{3.257678in}{1.225811in}}%
\pgfpathlineto{\pgfqpoint{3.272559in}{0.714113in}}%
\pgfpathlineto{\pgfqpoint{3.287441in}{0.709585in}}%
\pgfpathlineto{\pgfqpoint{3.302322in}{0.727698in}}%
\pgfpathlineto{\pgfqpoint{3.317204in}{0.791094in}}%
\pgfpathlineto{\pgfqpoint{3.332085in}{1.062792in}}%
\pgfpathlineto{\pgfqpoint{3.346967in}{1.180528in}}%
\pgfpathlineto{\pgfqpoint{3.361848in}{0.700528in}}%
\pgfpathlineto{\pgfqpoint{3.391611in}{0.795623in}}%
\pgfpathlineto{\pgfqpoint{3.406493in}{0.899774in}}%
\pgfpathlineto{\pgfqpoint{3.421374in}{1.157887in}}%
\pgfpathlineto{\pgfqpoint{3.436256in}{0.976755in}}%
\pgfpathlineto{\pgfqpoint{3.451137in}{0.714113in}}%
\pgfpathlineto{\pgfqpoint{3.466019in}{0.709585in}}%
\pgfpathlineto{\pgfqpoint{3.480900in}{0.859019in}}%
\pgfpathlineto{\pgfqpoint{3.495782in}{0.985811in}}%
\pgfpathlineto{\pgfqpoint{3.510663in}{1.293736in}}%
\pgfpathlineto{\pgfqpoint{3.525545in}{0.696000in}}%
\pgfpathlineto{\pgfqpoint{3.555308in}{0.772981in}}%
\pgfpathlineto{\pgfqpoint{3.570189in}{0.849962in}}%
\pgfpathlineto{\pgfqpoint{3.585071in}{0.899774in}}%
\pgfpathlineto{\pgfqpoint{3.599952in}{1.031094in}}%
\pgfpathlineto{\pgfqpoint{3.614833in}{0.768453in}}%
\pgfpathlineto{\pgfqpoint{3.629715in}{0.727698in}}%
\pgfpathlineto{\pgfqpoint{3.644596in}{0.795623in}}%
\pgfpathlineto{\pgfqpoint{3.659478in}{0.872604in}}%
\pgfpathlineto{\pgfqpoint{3.674359in}{0.936000in}}%
\pgfpathlineto{\pgfqpoint{3.689241in}{0.958642in}}%
\pgfpathlineto{\pgfqpoint{3.704122in}{0.754868in}}%
\pgfpathlineto{\pgfqpoint{3.719004in}{0.763925in}}%
\pgfpathlineto{\pgfqpoint{3.733885in}{0.845434in}}%
\pgfpathlineto{\pgfqpoint{3.748767in}{0.963170in}}%
\pgfpathlineto{\pgfqpoint{3.763648in}{0.831849in}}%
\pgfpathlineto{\pgfqpoint{3.778530in}{1.171472in}}%
\pgfpathlineto{\pgfqpoint{3.793411in}{0.723170in}}%
\pgfpathlineto{\pgfqpoint{3.808293in}{0.926943in}}%
\pgfpathlineto{\pgfqpoint{3.823174in}{0.750340in}}%
\pgfpathlineto{\pgfqpoint{3.838056in}{0.940528in}}%
\pgfpathlineto{\pgfqpoint{3.852937in}{0.877132in}}%
\pgfpathlineto{\pgfqpoint{3.867819in}{0.899774in}}%
\pgfpathlineto{\pgfqpoint{3.882700in}{0.890717in}}%
\pgfpathlineto{\pgfqpoint{3.897582in}{0.795623in}}%
\pgfpathlineto{\pgfqpoint{3.912463in}{0.836377in}}%
\pgfpathlineto{\pgfqpoint{3.927345in}{0.804679in}}%
\pgfpathlineto{\pgfqpoint{3.942226in}{0.917887in}}%
\pgfpathlineto{\pgfqpoint{3.957108in}{0.908830in}}%
\pgfpathlineto{\pgfqpoint{3.971989in}{0.795623in}}%
\pgfpathlineto{\pgfqpoint{3.986871in}{0.958642in}}%
\pgfpathlineto{\pgfqpoint{4.001752in}{0.881660in}}%
\pgfpathlineto{\pgfqpoint{4.016634in}{0.958642in}}%
\pgfpathlineto{\pgfqpoint{4.046397in}{0.804679in}}%
\pgfpathlineto{\pgfqpoint{4.061278in}{0.800151in}}%
\pgfpathlineto{\pgfqpoint{4.076160in}{0.922415in}}%
\pgfpathlineto{\pgfqpoint{4.091041in}{0.895245in}}%
\pgfpathlineto{\pgfqpoint{4.105923in}{0.890717in}}%
\pgfpathlineto{\pgfqpoint{4.120804in}{0.913358in}}%
\pgfpathlineto{\pgfqpoint{4.135686in}{0.827321in}}%
\pgfpathlineto{\pgfqpoint{4.150567in}{0.845434in}}%
\pgfpathlineto{\pgfqpoint{4.165449in}{1.325434in}}%
\pgfpathlineto{\pgfqpoint{4.180330in}{0.750340in}}%
\pgfpathlineto{\pgfqpoint{4.195212in}{1.479396in}}%
\pgfpathlineto{\pgfqpoint{4.210093in}{1.008453in}}%
\pgfpathlineto{\pgfqpoint{4.224974in}{1.751094in}}%
\pgfpathlineto{\pgfqpoint{4.254737in}{1.805434in}}%
\pgfpathlineto{\pgfqpoint{4.269619in}{0.714113in}}%
\pgfpathlineto{\pgfqpoint{4.284500in}{1.678642in}}%
\pgfpathlineto{\pgfqpoint{4.299382in}{0.714113in}}%
\pgfpathlineto{\pgfqpoint{4.314263in}{2.194868in}}%
\pgfpathlineto{\pgfqpoint{4.329145in}{0.745811in}}%
\pgfpathlineto{\pgfqpoint{4.344026in}{2.122415in}}%
\pgfpathlineto{\pgfqpoint{4.358908in}{1.461283in}}%
\pgfpathlineto{\pgfqpoint{4.373789in}{3.503547in}}%
\pgfpathlineto{\pgfqpoint{4.388671in}{3.797887in}}%
\pgfpathlineto{\pgfqpoint{4.403552in}{3.322415in}}%
\pgfpathlineto{\pgfqpoint{4.418434in}{3.259019in}}%
\pgfpathlineto{\pgfqpoint{4.433315in}{3.422038in}}%
\pgfpathlineto{\pgfqpoint{4.448197in}{3.376755in}}%
\pgfpathlineto{\pgfqpoint{4.463078in}{3.150340in}}%
\pgfpathlineto{\pgfqpoint{4.477960in}{3.435623in}}%
\pgfpathlineto{\pgfqpoint{4.492841in}{3.064302in}}%
\pgfpathlineto{\pgfqpoint{4.507723in}{3.390340in}}%
\pgfpathlineto{\pgfqpoint{4.522604in}{3.295245in}}%
\pgfpathlineto{\pgfqpoint{4.537486in}{3.308830in}}%
\pgfpathlineto{\pgfqpoint{4.552367in}{3.136755in}}%
\pgfpathlineto{\pgfqpoint{4.567249in}{3.489962in}}%
\pgfpathlineto{\pgfqpoint{4.582130in}{3.299774in}}%
\pgfpathlineto{\pgfqpoint{4.597012in}{2.955623in}}%
\pgfpathlineto{\pgfqpoint{4.611893in}{3.711849in}}%
\pgfpathlineto{\pgfqpoint{4.626775in}{3.209208in}}%
\pgfpathlineto{\pgfqpoint{4.641656in}{3.589585in}}%
\pgfpathlineto{\pgfqpoint{4.656538in}{3.222792in}}%
\pgfpathlineto{\pgfqpoint{4.671419in}{3.630340in}}%
\pgfpathlineto{\pgfqpoint{4.686301in}{3.394868in}}%
\pgfpathlineto{\pgfqpoint{4.701182in}{3.268075in}}%
\pgfpathlineto{\pgfqpoint{4.716064in}{2.946566in}}%
\pgfpathlineto{\pgfqpoint{4.730945in}{2.683925in}}%
\pgfpathlineto{\pgfqpoint{4.760708in}{3.222792in}}%
\pgfpathlineto{\pgfqpoint{4.775590in}{3.209208in}}%
\pgfpathlineto{\pgfqpoint{4.790471in}{3.308830in}}%
\pgfpathlineto{\pgfqpoint{4.805353in}{3.299774in}}%
\pgfpathlineto{\pgfqpoint{4.820234in}{2.960151in}}%
\pgfpathlineto{\pgfqpoint{4.835116in}{3.403925in}}%
\pgfpathlineto{\pgfqpoint{4.849997in}{3.775245in}}%
\pgfpathlineto{\pgfqpoint{4.864878in}{3.385811in}}%
\pgfpathlineto{\pgfqpoint{4.879760in}{3.100528in}}%
\pgfpathlineto{\pgfqpoint{4.894641in}{3.114113in}}%
\pgfpathlineto{\pgfqpoint{4.909523in}{3.295245in}}%
\pgfpathlineto{\pgfqpoint{4.924404in}{3.277132in}}%
\pgfpathlineto{\pgfqpoint{4.939286in}{3.625811in}}%
\pgfpathlineto{\pgfqpoint{4.954167in}{2.584302in}}%
\pgfpathlineto{\pgfqpoint{4.969049in}{3.372226in}}%
\pgfpathlineto{\pgfqpoint{4.983930in}{3.268075in}}%
\pgfpathlineto{\pgfqpoint{4.998812in}{3.557887in}}%
\pgfpathlineto{\pgfqpoint{5.013693in}{3.009962in}}%
\pgfpathlineto{\pgfqpoint{5.028575in}{3.616755in}}%
\pgfpathlineto{\pgfqpoint{5.043456in}{3.145811in}}%
\pgfpathlineto{\pgfqpoint{5.058338in}{3.354113in}}%
\pgfpathlineto{\pgfqpoint{5.073219in}{3.096000in}}%
\pgfpathlineto{\pgfqpoint{5.088101in}{3.702792in}}%
\pgfpathlineto{\pgfqpoint{5.102982in}{3.068830in}}%
\pgfpathlineto{\pgfqpoint{5.117864in}{3.172981in}}%
\pgfpathlineto{\pgfqpoint{5.132745in}{3.037132in}}%
\pgfpathlineto{\pgfqpoint{5.147627in}{3.480906in}}%
\pgfpathlineto{\pgfqpoint{5.162508in}{3.444679in}}%
\pgfpathlineto{\pgfqpoint{5.177390in}{3.616755in}}%
\pgfpathlineto{\pgfqpoint{5.192271in}{3.105057in}}%
\pgfpathlineto{\pgfqpoint{5.207153in}{3.005434in}}%
\pgfpathlineto{\pgfqpoint{5.222034in}{3.231849in}}%
\pgfpathlineto{\pgfqpoint{5.236916in}{3.657509in}}%
\pgfpathlineto{\pgfqpoint{5.251797in}{3.086943in}}%
\pgfpathlineto{\pgfqpoint{5.266679in}{3.023547in}}%
\pgfpathlineto{\pgfqpoint{5.281560in}{3.322415in}}%
\pgfpathlineto{\pgfqpoint{5.296442in}{3.417509in}}%
\pgfpathlineto{\pgfqpoint{5.311323in}{3.317887in}}%
\pgfpathlineto{\pgfqpoint{5.326205in}{2.914868in}}%
\pgfpathlineto{\pgfqpoint{5.341086in}{3.571472in}}%
\pgfpathlineto{\pgfqpoint{5.355968in}{3.580528in}}%
\pgfpathlineto{\pgfqpoint{5.370849in}{3.000906in}}%
\pgfpathlineto{\pgfqpoint{5.385731in}{3.652981in}}%
\pgfpathlineto{\pgfqpoint{5.400612in}{3.014491in}}%
\pgfpathlineto{\pgfqpoint{5.415494in}{3.254491in}}%
\pgfpathlineto{\pgfqpoint{5.430375in}{3.082415in}}%
\pgfpathlineto{\pgfqpoint{5.445257in}{3.218264in}}%
\pgfpathlineto{\pgfqpoint{5.460138in}{2.905811in}}%
\pgfpathlineto{\pgfqpoint{5.475020in}{3.698264in}}%
\pgfpathlineto{\pgfqpoint{5.489901in}{3.367698in}}%
\pgfpathlineto{\pgfqpoint{5.504782in}{3.349585in}}%
\pgfpathlineto{\pgfqpoint{5.519664in}{3.218264in}}%
\pgfpathlineto{\pgfqpoint{5.534545in}{2.344302in}}%
\pgfpathlineto{\pgfqpoint{5.534545in}{2.344302in}}%
\pgfusepath{stroke}%
\end{pgfscope}%
\begin{pgfscope}%
\pgfsetrectcap%
\pgfsetmiterjoin%
\pgfsetlinewidth{0.803000pt}%
\definecolor{currentstroke}{rgb}{0.000000,0.000000,0.000000}%
\pgfsetstrokecolor{currentstroke}%
\pgfsetdash{}{0pt}%
\pgfpathmoveto{\pgfqpoint{0.800000in}{0.528000in}}%
\pgfpathlineto{\pgfqpoint{0.800000in}{4.224000in}}%
\pgfusepath{stroke}%
\end{pgfscope}%
\begin{pgfscope}%
\pgfsetrectcap%
\pgfsetmiterjoin%
\pgfsetlinewidth{0.803000pt}%
\definecolor{currentstroke}{rgb}{0.000000,0.000000,0.000000}%
\pgfsetstrokecolor{currentstroke}%
\pgfsetdash{}{0pt}%
\pgfpathmoveto{\pgfqpoint{5.760000in}{0.528000in}}%
\pgfpathlineto{\pgfqpoint{5.760000in}{4.224000in}}%
\pgfusepath{stroke}%
\end{pgfscope}%
\begin{pgfscope}%
\pgfsetrectcap%
\pgfsetmiterjoin%
\pgfsetlinewidth{0.803000pt}%
\definecolor{currentstroke}{rgb}{0.000000,0.000000,0.000000}%
\pgfsetstrokecolor{currentstroke}%
\pgfsetdash{}{0pt}%
\pgfpathmoveto{\pgfqpoint{0.800000in}{0.528000in}}%
\pgfpathlineto{\pgfqpoint{5.760000in}{0.528000in}}%
\pgfusepath{stroke}%
\end{pgfscope}%
\begin{pgfscope}%
\pgfsetrectcap%
\pgfsetmiterjoin%
\pgfsetlinewidth{0.803000pt}%
\definecolor{currentstroke}{rgb}{0.000000,0.000000,0.000000}%
\pgfsetstrokecolor{currentstroke}%
\pgfsetdash{}{0pt}%
\pgfpathmoveto{\pgfqpoint{0.800000in}{4.224000in}}%
\pgfpathlineto{\pgfqpoint{5.760000in}{4.224000in}}%
\pgfusepath{stroke}%
\end{pgfscope}%
\begin{pgfscope}%
\pgfsetbuttcap%
\pgfsetmiterjoin%
\definecolor{currentfill}{rgb}{1.000000,1.000000,1.000000}%
\pgfsetfillcolor{currentfill}%
\pgfsetfillopacity{0.800000}%
\pgfsetlinewidth{1.003750pt}%
\definecolor{currentstroke}{rgb}{0.800000,0.800000,0.800000}%
\pgfsetstrokecolor{currentstroke}%
\pgfsetstrokeopacity{0.800000}%
\pgfsetdash{}{0pt}%
\pgfpathmoveto{\pgfqpoint{5.591944in}{4.025389in}}%
\pgfpathlineto{\pgfqpoint{5.653056in}{4.025389in}}%
\pgfpathquadraticcurveto{\pgfqpoint{5.683611in}{4.025389in}}{\pgfqpoint{5.683611in}{4.055944in}}%
\pgfpathlineto{\pgfqpoint{5.683611in}{4.117056in}}%
\pgfpathquadraticcurveto{\pgfqpoint{5.683611in}{4.147611in}}{\pgfqpoint{5.653056in}{4.147611in}}%
\pgfpathlineto{\pgfqpoint{5.591944in}{4.147611in}}%
\pgfpathquadraticcurveto{\pgfqpoint{5.561389in}{4.147611in}}{\pgfqpoint{5.561389in}{4.117056in}}%
\pgfpathlineto{\pgfqpoint{5.561389in}{4.055944in}}%
\pgfpathquadraticcurveto{\pgfqpoint{5.561389in}{4.025389in}}{\pgfqpoint{5.591944in}{4.025389in}}%
\pgfpathclose%
\pgfusepath{stroke,fill}%
\end{pgfscope}%
\end{pgfpicture}%
\makeatother%
\endgroup%

    \caption{Distribution of successful read operations of experiment 3 in \prettyref{tab:res-8}}
    \label{fig:reads-36pmm}
\end{figure}

\subsection{Performance}

Having established the durability properties of the MongoDB configurations used in these experiments, we will now address performance as the main tradeoff to the durability guarantees these configurations provide. 

\subsubsection{Write Concerns}

Figures \ref{fig:lat-plp}, \ref{fig:lat-plj} \& \ref{fig:lat-pmm} show the average latencies of write operations for the first three experiments performed in \prettyref{tab:res-4} . We will consider any operation with latency of 2000ms or more as timed out. We found that read latencies were not affected by write concerns and so we did not include them in this analysis.

We can immediately observe that as we increase the strength of write concern (where Primary is the weakest and Majority is strongest), we also increase the average latency of the operations performed with that write concern. Recall from \prettyref{sec:writeconcern} that as the strength the write concern increases, MongoDB has to perform more actions prior to acknowledging the operation. This is exactly the reason for the increasing average latencies.

However we make another observation in these results. Figures \ref{fig:lat-plj} and \ref{fig:lat-pmm} show a noticeable but temporary increase in latency soon after the 200 second mark - the recovery phase of the experiment. In our experiments, the recovery phase is signified by turning on the failed replica and letting it rejoin the replica set as a secondary, which would also involve performing synchronisation and rollbacks to be in sync with the new primary. It is likely that this extra load on the primary replica reduces the resources available for it to respond to queries and update its copy of the data. Another possibility is that synchronisation may require that certain documents be locked while they're being synchronised, causing a delay to acknowledging any operations on those documents until after the primary has confirmed that they've been synchronised with the secondary. 

\subsubsection{Read Preferences}

Another dimension of our experiments concerns the effect of read preference on how the client reacts to failures in the replica set. Figures \ref{fig:rlat-plp} \& \ref{fig:rlat-pplp} show the average latencies of read operations for two equivalent experiments, only varying the read preference between Primary and Primary Preferred. During this analysis, we will consider any operations that took 2000ms or longer as timed out.

We immediately observe that Primary Preferred and Primary reads behave differently under a failure. Primary reads exhibit the behaviour indicative of timeouts for approximately 20 seconds after a failure is induced and return to a relatively stable state afterwards.

On the other hand, Primary Preferred reads tend to return very quickly after a failure for only 10 seconds after it is induced, returning back to normal operation soon after. The small spike immediately at the point of failure implies that the client was trying to query from the Primary while it was failing, resulting in a small number of timeouts. This implies that the client was able to identify the primary replica as failing and start querying secondaries as per its read preference. The reasoning behind the significant drop in latency is unknown but may be attributed to the significantly reduced load on the replica set, as no write operations are being applied.

The more peculiar result is found around the time the failed node rejoins the replica set, between 180 seconds and 230 seconds into the experiment. During this time frame, the execution exhibits a very unstable latency pattern, with latencies jumping between 100ms at the lowest and 900ms at the highest. We could not determine the source of this anomaly, but postulate that it may be affected by the process in charge of adding a new secondary to the replica set.

\begin{figure}
    %% Creator: Matplotlib, PGF backend
%%
%% To include the figure in your LaTeX document, write
%%   \input{<filename>.pgf}
%%
%% Make sure the required packages are loaded in your preamble
%%   \usepackage{pgf}
%%
%% Figures using additional raster images can only be included by \input if
%% they are in the same directory as the main LaTeX file. For loading figures
%% from other directories you can use the `import` package
%%   \usepackage{import}
%% and then include the figures with
%%   \import{<path to file>}{<filename>.pgf}
%%
%% Matplotlib used the following preamble
%%   \usepackage[utf8x]{inputenc}
%%   \usepackage[T1]{fontenc}
%%   \usepackage{lmodern}
%%
\begingroup%
\makeatletter%
\begin{pgfpicture}%
\pgfpathrectangle{\pgfpointorigin}{\pgfqpoint{6.400000in}{4.800000in}}%
\pgfusepath{use as bounding box, clip}%
\begin{pgfscope}%
\pgfsetbuttcap%
\pgfsetmiterjoin%
\definecolor{currentfill}{rgb}{1.000000,1.000000,1.000000}%
\pgfsetfillcolor{currentfill}%
\pgfsetlinewidth{0.000000pt}%
\definecolor{currentstroke}{rgb}{1.000000,1.000000,1.000000}%
\pgfsetstrokecolor{currentstroke}%
\pgfsetdash{}{0pt}%
\pgfpathmoveto{\pgfqpoint{0.000000in}{0.000000in}}%
\pgfpathlineto{\pgfqpoint{6.400000in}{0.000000in}}%
\pgfpathlineto{\pgfqpoint{6.400000in}{4.800000in}}%
\pgfpathlineto{\pgfqpoint{0.000000in}{4.800000in}}%
\pgfpathclose%
\pgfusepath{fill}%
\end{pgfscope}%
\begin{pgfscope}%
\pgfsetbuttcap%
\pgfsetmiterjoin%
\definecolor{currentfill}{rgb}{1.000000,1.000000,1.000000}%
\pgfsetfillcolor{currentfill}%
\pgfsetlinewidth{0.000000pt}%
\definecolor{currentstroke}{rgb}{0.000000,0.000000,0.000000}%
\pgfsetstrokecolor{currentstroke}%
\pgfsetstrokeopacity{0.000000}%
\pgfsetdash{}{0pt}%
\pgfpathmoveto{\pgfqpoint{0.800000in}{0.528000in}}%
\pgfpathlineto{\pgfqpoint{5.760000in}{0.528000in}}%
\pgfpathlineto{\pgfqpoint{5.760000in}{4.224000in}}%
\pgfpathlineto{\pgfqpoint{0.800000in}{4.224000in}}%
\pgfpathclose%
\pgfusepath{fill}%
\end{pgfscope}%
\begin{pgfscope}%
\pgfsetbuttcap%
\pgfsetroundjoin%
\definecolor{currentfill}{rgb}{0.000000,0.000000,0.000000}%
\pgfsetfillcolor{currentfill}%
\pgfsetlinewidth{0.803000pt}%
\definecolor{currentstroke}{rgb}{0.000000,0.000000,0.000000}%
\pgfsetstrokecolor{currentstroke}%
\pgfsetdash{}{0pt}%
\pgfsys@defobject{currentmarker}{\pgfqpoint{0.000000in}{-0.048611in}}{\pgfqpoint{0.000000in}{0.000000in}}{%
\pgfpathmoveto{\pgfqpoint{0.000000in}{0.000000in}}%
\pgfpathlineto{\pgfqpoint{0.000000in}{-0.048611in}}%
\pgfusepath{stroke,fill}%
}%
\begin{pgfscope}%
\pgfsys@transformshift{1.025455in}{0.528000in}%
\pgfsys@useobject{currentmarker}{}%
\end{pgfscope}%
\end{pgfscope}%
\begin{pgfscope}%
\pgftext[x=1.025455in,y=0.430778in,,top]{\fontsize{11.000000}{13.200000}\selectfont \(\displaystyle 0\)}%
\end{pgfscope}%
\begin{pgfscope}%
\pgfsetbuttcap%
\pgfsetroundjoin%
\definecolor{currentfill}{rgb}{0.000000,0.000000,0.000000}%
\pgfsetfillcolor{currentfill}%
\pgfsetlinewidth{0.803000pt}%
\definecolor{currentstroke}{rgb}{0.000000,0.000000,0.000000}%
\pgfsetstrokecolor{currentstroke}%
\pgfsetdash{}{0pt}%
\pgfsys@defobject{currentmarker}{\pgfqpoint{0.000000in}{-0.048611in}}{\pgfqpoint{0.000000in}{0.000000in}}{%
\pgfpathmoveto{\pgfqpoint{0.000000in}{0.000000in}}%
\pgfpathlineto{\pgfqpoint{0.000000in}{-0.048611in}}%
\pgfusepath{stroke,fill}%
}%
\begin{pgfscope}%
\pgfsys@transformshift{1.738918in}{0.528000in}%
\pgfsys@useobject{currentmarker}{}%
\end{pgfscope}%
\end{pgfscope}%
\begin{pgfscope}%
\pgftext[x=1.738918in,y=0.430778in,,top]{\fontsize{11.000000}{13.200000}\selectfont \(\displaystyle 50\)}%
\end{pgfscope}%
\begin{pgfscope}%
\pgfsetbuttcap%
\pgfsetroundjoin%
\definecolor{currentfill}{rgb}{0.000000,0.000000,0.000000}%
\pgfsetfillcolor{currentfill}%
\pgfsetlinewidth{0.803000pt}%
\definecolor{currentstroke}{rgb}{0.000000,0.000000,0.000000}%
\pgfsetstrokecolor{currentstroke}%
\pgfsetdash{}{0pt}%
\pgfsys@defobject{currentmarker}{\pgfqpoint{0.000000in}{-0.048611in}}{\pgfqpoint{0.000000in}{0.000000in}}{%
\pgfpathmoveto{\pgfqpoint{0.000000in}{0.000000in}}%
\pgfpathlineto{\pgfqpoint{0.000000in}{-0.048611in}}%
\pgfusepath{stroke,fill}%
}%
\begin{pgfscope}%
\pgfsys@transformshift{2.452382in}{0.528000in}%
\pgfsys@useobject{currentmarker}{}%
\end{pgfscope}%
\end{pgfscope}%
\begin{pgfscope}%
\pgftext[x=2.452382in,y=0.430778in,,top]{\fontsize{11.000000}{13.200000}\selectfont \(\displaystyle 100\)}%
\end{pgfscope}%
\begin{pgfscope}%
\pgfsetbuttcap%
\pgfsetroundjoin%
\definecolor{currentfill}{rgb}{0.000000,0.000000,0.000000}%
\pgfsetfillcolor{currentfill}%
\pgfsetlinewidth{0.803000pt}%
\definecolor{currentstroke}{rgb}{0.000000,0.000000,0.000000}%
\pgfsetstrokecolor{currentstroke}%
\pgfsetdash{}{0pt}%
\pgfsys@defobject{currentmarker}{\pgfqpoint{0.000000in}{-0.048611in}}{\pgfqpoint{0.000000in}{0.000000in}}{%
\pgfpathmoveto{\pgfqpoint{0.000000in}{0.000000in}}%
\pgfpathlineto{\pgfqpoint{0.000000in}{-0.048611in}}%
\pgfusepath{stroke,fill}%
}%
\begin{pgfscope}%
\pgfsys@transformshift{3.165846in}{0.528000in}%
\pgfsys@useobject{currentmarker}{}%
\end{pgfscope}%
\end{pgfscope}%
\begin{pgfscope}%
\pgftext[x=3.165846in,y=0.430778in,,top]{\fontsize{11.000000}{13.200000}\selectfont \(\displaystyle 150\)}%
\end{pgfscope}%
\begin{pgfscope}%
\pgfsetbuttcap%
\pgfsetroundjoin%
\definecolor{currentfill}{rgb}{0.000000,0.000000,0.000000}%
\pgfsetfillcolor{currentfill}%
\pgfsetlinewidth{0.803000pt}%
\definecolor{currentstroke}{rgb}{0.000000,0.000000,0.000000}%
\pgfsetstrokecolor{currentstroke}%
\pgfsetdash{}{0pt}%
\pgfsys@defobject{currentmarker}{\pgfqpoint{0.000000in}{-0.048611in}}{\pgfqpoint{0.000000in}{0.000000in}}{%
\pgfpathmoveto{\pgfqpoint{0.000000in}{0.000000in}}%
\pgfpathlineto{\pgfqpoint{0.000000in}{-0.048611in}}%
\pgfusepath{stroke,fill}%
}%
\begin{pgfscope}%
\pgfsys@transformshift{3.879310in}{0.528000in}%
\pgfsys@useobject{currentmarker}{}%
\end{pgfscope}%
\end{pgfscope}%
\begin{pgfscope}%
\pgftext[x=3.879310in,y=0.430778in,,top]{\fontsize{11.000000}{13.200000}\selectfont \(\displaystyle 200\)}%
\end{pgfscope}%
\begin{pgfscope}%
\pgfsetbuttcap%
\pgfsetroundjoin%
\definecolor{currentfill}{rgb}{0.000000,0.000000,0.000000}%
\pgfsetfillcolor{currentfill}%
\pgfsetlinewidth{0.803000pt}%
\definecolor{currentstroke}{rgb}{0.000000,0.000000,0.000000}%
\pgfsetstrokecolor{currentstroke}%
\pgfsetdash{}{0pt}%
\pgfsys@defobject{currentmarker}{\pgfqpoint{0.000000in}{-0.048611in}}{\pgfqpoint{0.000000in}{0.000000in}}{%
\pgfpathmoveto{\pgfqpoint{0.000000in}{0.000000in}}%
\pgfpathlineto{\pgfqpoint{0.000000in}{-0.048611in}}%
\pgfusepath{stroke,fill}%
}%
\begin{pgfscope}%
\pgfsys@transformshift{4.592773in}{0.528000in}%
\pgfsys@useobject{currentmarker}{}%
\end{pgfscope}%
\end{pgfscope}%
\begin{pgfscope}%
\pgftext[x=4.592773in,y=0.430778in,,top]{\fontsize{11.000000}{13.200000}\selectfont \(\displaystyle 250\)}%
\end{pgfscope}%
\begin{pgfscope}%
\pgfsetbuttcap%
\pgfsetroundjoin%
\definecolor{currentfill}{rgb}{0.000000,0.000000,0.000000}%
\pgfsetfillcolor{currentfill}%
\pgfsetlinewidth{0.803000pt}%
\definecolor{currentstroke}{rgb}{0.000000,0.000000,0.000000}%
\pgfsetstrokecolor{currentstroke}%
\pgfsetdash{}{0pt}%
\pgfsys@defobject{currentmarker}{\pgfqpoint{0.000000in}{-0.048611in}}{\pgfqpoint{0.000000in}{0.000000in}}{%
\pgfpathmoveto{\pgfqpoint{0.000000in}{0.000000in}}%
\pgfpathlineto{\pgfqpoint{0.000000in}{-0.048611in}}%
\pgfusepath{stroke,fill}%
}%
\begin{pgfscope}%
\pgfsys@transformshift{5.306237in}{0.528000in}%
\pgfsys@useobject{currentmarker}{}%
\end{pgfscope}%
\end{pgfscope}%
\begin{pgfscope}%
\pgftext[x=5.306237in,y=0.430778in,,top]{\fontsize{11.000000}{13.200000}\selectfont \(\displaystyle 300\)}%
\end{pgfscope}%
\begin{pgfscope}%
\pgftext[x=3.280000in,y=0.240271in,,top]{\fontsize{11.000000}{13.200000}\selectfont Time of experiment (in seconds)}%
\end{pgfscope}%
\begin{pgfscope}%
\pgfsetbuttcap%
\pgfsetroundjoin%
\definecolor{currentfill}{rgb}{0.000000,0.000000,0.000000}%
\pgfsetfillcolor{currentfill}%
\pgfsetlinewidth{0.803000pt}%
\definecolor{currentstroke}{rgb}{0.000000,0.000000,0.000000}%
\pgfsetstrokecolor{currentstroke}%
\pgfsetdash{}{0pt}%
\pgfsys@defobject{currentmarker}{\pgfqpoint{-0.048611in}{0.000000in}}{\pgfqpoint{0.000000in}{0.000000in}}{%
\pgfpathmoveto{\pgfqpoint{0.000000in}{0.000000in}}%
\pgfpathlineto{\pgfqpoint{-0.048611in}{0.000000in}}%
\pgfusepath{stroke,fill}%
}%
\begin{pgfscope}%
\pgfsys@transformshift{0.800000in}{0.694795in}%
\pgfsys@useobject{currentmarker}{}%
\end{pgfscope}%
\end{pgfscope}%
\begin{pgfscope}%
\pgftext[x=0.627981in,y=0.642173in,left,base]{\fontsize{11.000000}{13.200000}\selectfont \(\displaystyle 0\)}%
\end{pgfscope}%
\begin{pgfscope}%
\pgfsetbuttcap%
\pgfsetroundjoin%
\definecolor{currentfill}{rgb}{0.000000,0.000000,0.000000}%
\pgfsetfillcolor{currentfill}%
\pgfsetlinewidth{0.803000pt}%
\definecolor{currentstroke}{rgb}{0.000000,0.000000,0.000000}%
\pgfsetstrokecolor{currentstroke}%
\pgfsetdash{}{0pt}%
\pgfsys@defobject{currentmarker}{\pgfqpoint{-0.048611in}{0.000000in}}{\pgfqpoint{0.000000in}{0.000000in}}{%
\pgfpathmoveto{\pgfqpoint{0.000000in}{0.000000in}}%
\pgfpathlineto{\pgfqpoint{-0.048611in}{0.000000in}}%
\pgfusepath{stroke,fill}%
}%
\begin{pgfscope}%
\pgfsys@transformshift{0.800000in}{1.114946in}%
\pgfsys@useobject{currentmarker}{}%
\end{pgfscope}%
\end{pgfscope}%
\begin{pgfscope}%
\pgftext[x=0.478386in,y=1.062323in,left,base]{\fontsize{11.000000}{13.200000}\selectfont \(\displaystyle 250\)}%
\end{pgfscope}%
\begin{pgfscope}%
\pgfsetbuttcap%
\pgfsetroundjoin%
\definecolor{currentfill}{rgb}{0.000000,0.000000,0.000000}%
\pgfsetfillcolor{currentfill}%
\pgfsetlinewidth{0.803000pt}%
\definecolor{currentstroke}{rgb}{0.000000,0.000000,0.000000}%
\pgfsetstrokecolor{currentstroke}%
\pgfsetdash{}{0pt}%
\pgfsys@defobject{currentmarker}{\pgfqpoint{-0.048611in}{0.000000in}}{\pgfqpoint{0.000000in}{0.000000in}}{%
\pgfpathmoveto{\pgfqpoint{0.000000in}{0.000000in}}%
\pgfpathlineto{\pgfqpoint{-0.048611in}{0.000000in}}%
\pgfusepath{stroke,fill}%
}%
\begin{pgfscope}%
\pgfsys@transformshift{0.800000in}{1.535096in}%
\pgfsys@useobject{currentmarker}{}%
\end{pgfscope}%
\end{pgfscope}%
\begin{pgfscope}%
\pgftext[x=0.478386in,y=1.482474in,left,base]{\fontsize{11.000000}{13.200000}\selectfont \(\displaystyle 500\)}%
\end{pgfscope}%
\begin{pgfscope}%
\pgfsetbuttcap%
\pgfsetroundjoin%
\definecolor{currentfill}{rgb}{0.000000,0.000000,0.000000}%
\pgfsetfillcolor{currentfill}%
\pgfsetlinewidth{0.803000pt}%
\definecolor{currentstroke}{rgb}{0.000000,0.000000,0.000000}%
\pgfsetstrokecolor{currentstroke}%
\pgfsetdash{}{0pt}%
\pgfsys@defobject{currentmarker}{\pgfqpoint{-0.048611in}{0.000000in}}{\pgfqpoint{0.000000in}{0.000000in}}{%
\pgfpathmoveto{\pgfqpoint{0.000000in}{0.000000in}}%
\pgfpathlineto{\pgfqpoint{-0.048611in}{0.000000in}}%
\pgfusepath{stroke,fill}%
}%
\begin{pgfscope}%
\pgfsys@transformshift{0.800000in}{1.955247in}%
\pgfsys@useobject{currentmarker}{}%
\end{pgfscope}%
\end{pgfscope}%
\begin{pgfscope}%
\pgftext[x=0.478386in,y=1.902625in,left,base]{\fontsize{11.000000}{13.200000}\selectfont \(\displaystyle 750\)}%
\end{pgfscope}%
\begin{pgfscope}%
\pgfsetbuttcap%
\pgfsetroundjoin%
\definecolor{currentfill}{rgb}{0.000000,0.000000,0.000000}%
\pgfsetfillcolor{currentfill}%
\pgfsetlinewidth{0.803000pt}%
\definecolor{currentstroke}{rgb}{0.000000,0.000000,0.000000}%
\pgfsetstrokecolor{currentstroke}%
\pgfsetdash{}{0pt}%
\pgfsys@defobject{currentmarker}{\pgfqpoint{-0.048611in}{0.000000in}}{\pgfqpoint{0.000000in}{0.000000in}}{%
\pgfpathmoveto{\pgfqpoint{0.000000in}{0.000000in}}%
\pgfpathlineto{\pgfqpoint{-0.048611in}{0.000000in}}%
\pgfusepath{stroke,fill}%
}%
\begin{pgfscope}%
\pgfsys@transformshift{0.800000in}{2.375398in}%
\pgfsys@useobject{currentmarker}{}%
\end{pgfscope}%
\end{pgfscope}%
\begin{pgfscope}%
\pgftext[x=0.403588in,y=2.322775in,left,base]{\fontsize{11.000000}{13.200000}\selectfont \(\displaystyle 1000\)}%
\end{pgfscope}%
\begin{pgfscope}%
\pgfsetbuttcap%
\pgfsetroundjoin%
\definecolor{currentfill}{rgb}{0.000000,0.000000,0.000000}%
\pgfsetfillcolor{currentfill}%
\pgfsetlinewidth{0.803000pt}%
\definecolor{currentstroke}{rgb}{0.000000,0.000000,0.000000}%
\pgfsetstrokecolor{currentstroke}%
\pgfsetdash{}{0pt}%
\pgfsys@defobject{currentmarker}{\pgfqpoint{-0.048611in}{0.000000in}}{\pgfqpoint{0.000000in}{0.000000in}}{%
\pgfpathmoveto{\pgfqpoint{0.000000in}{0.000000in}}%
\pgfpathlineto{\pgfqpoint{-0.048611in}{0.000000in}}%
\pgfusepath{stroke,fill}%
}%
\begin{pgfscope}%
\pgfsys@transformshift{0.800000in}{2.795548in}%
\pgfsys@useobject{currentmarker}{}%
\end{pgfscope}%
\end{pgfscope}%
\begin{pgfscope}%
\pgftext[x=0.403588in,y=2.742926in,left,base]{\fontsize{11.000000}{13.200000}\selectfont \(\displaystyle 1250\)}%
\end{pgfscope}%
\begin{pgfscope}%
\pgfsetbuttcap%
\pgfsetroundjoin%
\definecolor{currentfill}{rgb}{0.000000,0.000000,0.000000}%
\pgfsetfillcolor{currentfill}%
\pgfsetlinewidth{0.803000pt}%
\definecolor{currentstroke}{rgb}{0.000000,0.000000,0.000000}%
\pgfsetstrokecolor{currentstroke}%
\pgfsetdash{}{0pt}%
\pgfsys@defobject{currentmarker}{\pgfqpoint{-0.048611in}{0.000000in}}{\pgfqpoint{0.000000in}{0.000000in}}{%
\pgfpathmoveto{\pgfqpoint{0.000000in}{0.000000in}}%
\pgfpathlineto{\pgfqpoint{-0.048611in}{0.000000in}}%
\pgfusepath{stroke,fill}%
}%
\begin{pgfscope}%
\pgfsys@transformshift{0.800000in}{3.215699in}%
\pgfsys@useobject{currentmarker}{}%
\end{pgfscope}%
\end{pgfscope}%
\begin{pgfscope}%
\pgftext[x=0.403588in,y=3.163077in,left,base]{\fontsize{11.000000}{13.200000}\selectfont \(\displaystyle 1500\)}%
\end{pgfscope}%
\begin{pgfscope}%
\pgfsetbuttcap%
\pgfsetroundjoin%
\definecolor{currentfill}{rgb}{0.000000,0.000000,0.000000}%
\pgfsetfillcolor{currentfill}%
\pgfsetlinewidth{0.803000pt}%
\definecolor{currentstroke}{rgb}{0.000000,0.000000,0.000000}%
\pgfsetstrokecolor{currentstroke}%
\pgfsetdash{}{0pt}%
\pgfsys@defobject{currentmarker}{\pgfqpoint{-0.048611in}{0.000000in}}{\pgfqpoint{0.000000in}{0.000000in}}{%
\pgfpathmoveto{\pgfqpoint{0.000000in}{0.000000in}}%
\pgfpathlineto{\pgfqpoint{-0.048611in}{0.000000in}}%
\pgfusepath{stroke,fill}%
}%
\begin{pgfscope}%
\pgfsys@transformshift{0.800000in}{3.635849in}%
\pgfsys@useobject{currentmarker}{}%
\end{pgfscope}%
\end{pgfscope}%
\begin{pgfscope}%
\pgftext[x=0.403588in,y=3.583227in,left,base]{\fontsize{11.000000}{13.200000}\selectfont \(\displaystyle 1750\)}%
\end{pgfscope}%
\begin{pgfscope}%
\pgfsetbuttcap%
\pgfsetroundjoin%
\definecolor{currentfill}{rgb}{0.000000,0.000000,0.000000}%
\pgfsetfillcolor{currentfill}%
\pgfsetlinewidth{0.803000pt}%
\definecolor{currentstroke}{rgb}{0.000000,0.000000,0.000000}%
\pgfsetstrokecolor{currentstroke}%
\pgfsetdash{}{0pt}%
\pgfsys@defobject{currentmarker}{\pgfqpoint{-0.048611in}{0.000000in}}{\pgfqpoint{0.000000in}{0.000000in}}{%
\pgfpathmoveto{\pgfqpoint{0.000000in}{0.000000in}}%
\pgfpathlineto{\pgfqpoint{-0.048611in}{0.000000in}}%
\pgfusepath{stroke,fill}%
}%
\begin{pgfscope}%
\pgfsys@transformshift{0.800000in}{4.056000in}%
\pgfsys@useobject{currentmarker}{}%
\end{pgfscope}%
\end{pgfscope}%
\begin{pgfscope}%
\pgftext[x=0.403588in,y=4.003378in,left,base]{\fontsize{11.000000}{13.200000}\selectfont \(\displaystyle 2000\)}%
\end{pgfscope}%
\begin{pgfscope}%
\pgftext[x=0.348033in,y=2.376000in,,bottom,rotate=90.000000]{\fontsize{11.000000}{13.200000}\selectfont Latency (in milliseconds)}%
\end{pgfscope}%
\begin{pgfscope}%
\pgfpathrectangle{\pgfqpoint{0.800000in}{0.528000in}}{\pgfqpoint{4.960000in}{3.696000in}}%
\pgfusepath{clip}%
\pgfsetrectcap%
\pgfsetroundjoin%
\pgfsetlinewidth{1.505625pt}%
\definecolor{currentstroke}{rgb}{0.121569,0.466667,0.705882}%
\pgfsetstrokecolor{currentstroke}%
\pgfsetdash{}{0pt}%
\pgfpathmoveto{\pgfqpoint{1.025455in}{0.972295in}}%
\pgfpathlineto{\pgfqpoint{1.039724in}{0.997617in}}%
\pgfpathlineto{\pgfqpoint{1.053993in}{1.052171in}}%
\pgfpathlineto{\pgfqpoint{1.068262in}{1.117593in}}%
\pgfpathlineto{\pgfqpoint{1.082532in}{1.365498in}}%
\pgfpathlineto{\pgfqpoint{1.096801in}{1.273319in}}%
\pgfpathlineto{\pgfqpoint{1.111070in}{1.211917in}}%
\pgfpathlineto{\pgfqpoint{1.125339in}{1.083321in}}%
\pgfpathlineto{\pgfqpoint{1.139609in}{0.998616in}}%
\pgfpathlineto{\pgfqpoint{1.153878in}{1.042688in}}%
\pgfpathlineto{\pgfqpoint{1.168147in}{0.934259in}}%
\pgfpathlineto{\pgfqpoint{1.182417in}{0.951813in}}%
\pgfpathlineto{\pgfqpoint{1.196686in}{0.937006in}}%
\pgfpathlineto{\pgfqpoint{1.210955in}{1.154540in}}%
\pgfpathlineto{\pgfqpoint{1.225224in}{0.966189in}}%
\pgfpathlineto{\pgfqpoint{1.239494in}{1.325017in}}%
\pgfpathlineto{\pgfqpoint{1.253763in}{1.113861in}}%
\pgfpathlineto{\pgfqpoint{1.268032in}{1.015755in}}%
\pgfpathlineto{\pgfqpoint{1.282301in}{1.111682in}}%
\pgfpathlineto{\pgfqpoint{1.296571in}{1.073035in}}%
\pgfpathlineto{\pgfqpoint{1.310840in}{0.982426in}}%
\pgfpathlineto{\pgfqpoint{1.325109in}{0.982082in}}%
\pgfpathlineto{\pgfqpoint{1.339379in}{1.100191in}}%
\pgfpathlineto{\pgfqpoint{1.353648in}{1.058201in}}%
\pgfpathlineto{\pgfqpoint{1.367917in}{0.992941in}}%
\pgfpathlineto{\pgfqpoint{1.382186in}{0.987459in}}%
\pgfpathlineto{\pgfqpoint{1.396456in}{1.033034in}}%
\pgfpathlineto{\pgfqpoint{1.410725in}{1.065615in}}%
\pgfpathlineto{\pgfqpoint{1.424994in}{1.040043in}}%
\pgfpathlineto{\pgfqpoint{1.439264in}{1.064193in}}%
\pgfpathlineto{\pgfqpoint{1.453533in}{1.159738in}}%
\pgfpathlineto{\pgfqpoint{1.467802in}{1.018547in}}%
\pgfpathlineto{\pgfqpoint{1.482071in}{1.760114in}}%
\pgfpathlineto{\pgfqpoint{1.496341in}{1.194661in}}%
\pgfpathlineto{\pgfqpoint{1.510610in}{1.122575in}}%
\pgfpathlineto{\pgfqpoint{1.524879in}{1.202413in}}%
\pgfpathlineto{\pgfqpoint{1.539148in}{1.027019in}}%
\pgfpathlineto{\pgfqpoint{1.553418in}{1.030670in}}%
\pgfpathlineto{\pgfqpoint{1.567687in}{1.148709in}}%
\pgfpathlineto{\pgfqpoint{1.581956in}{0.990747in}}%
\pgfpathlineto{\pgfqpoint{1.596226in}{1.063363in}}%
\pgfpathlineto{\pgfqpoint{1.610495in}{0.987274in}}%
\pgfpathlineto{\pgfqpoint{1.624764in}{1.130952in}}%
\pgfpathlineto{\pgfqpoint{1.639033in}{0.887645in}}%
\pgfpathlineto{\pgfqpoint{1.653303in}{1.139192in}}%
\pgfpathlineto{\pgfqpoint{1.667572in}{0.994044in}}%
\pgfpathlineto{\pgfqpoint{1.681841in}{1.110035in}}%
\pgfpathlineto{\pgfqpoint{1.696110in}{1.076454in}}%
\pgfpathlineto{\pgfqpoint{1.710380in}{0.963220in}}%
\pgfpathlineto{\pgfqpoint{1.724649in}{0.967763in}}%
\pgfpathlineto{\pgfqpoint{1.738918in}{0.961558in}}%
\pgfpathlineto{\pgfqpoint{1.753188in}{1.147735in}}%
\pgfpathlineto{\pgfqpoint{1.767457in}{0.997092in}}%
\pgfpathlineto{\pgfqpoint{1.781726in}{1.097902in}}%
\pgfpathlineto{\pgfqpoint{1.795995in}{1.017133in}}%
\pgfpathlineto{\pgfqpoint{1.810265in}{1.069446in}}%
\pgfpathlineto{\pgfqpoint{1.824534in}{0.994072in}}%
\pgfpathlineto{\pgfqpoint{1.838803in}{1.009631in}}%
\pgfpathlineto{\pgfqpoint{1.853072in}{1.034523in}}%
\pgfpathlineto{\pgfqpoint{1.867342in}{1.000015in}}%
\pgfpathlineto{\pgfqpoint{1.881611in}{1.152043in}}%
\pgfpathlineto{\pgfqpoint{1.895880in}{1.155803in}}%
\pgfpathlineto{\pgfqpoint{1.910150in}{0.993100in}}%
\pgfpathlineto{\pgfqpoint{1.924419in}{1.014407in}}%
\pgfpathlineto{\pgfqpoint{1.938688in}{1.005658in}}%
\pgfpathlineto{\pgfqpoint{1.952957in}{1.021300in}}%
\pgfpathlineto{\pgfqpoint{1.967227in}{0.994492in}}%
\pgfpathlineto{\pgfqpoint{1.981496in}{1.388091in}}%
\pgfpathlineto{\pgfqpoint{1.995765in}{1.029161in}}%
\pgfpathlineto{\pgfqpoint{2.010035in}{1.029733in}}%
\pgfpathlineto{\pgfqpoint{2.024304in}{1.052528in}}%
\pgfpathlineto{\pgfqpoint{2.038573in}{0.974169in}}%
\pgfpathlineto{\pgfqpoint{2.052842in}{0.880107in}}%
\pgfpathlineto{\pgfqpoint{2.067112in}{1.140025in}}%
\pgfpathlineto{\pgfqpoint{2.081381in}{1.088692in}}%
\pgfpathlineto{\pgfqpoint{2.095650in}{0.961647in}}%
\pgfpathlineto{\pgfqpoint{2.109919in}{0.951558in}}%
\pgfpathlineto{\pgfqpoint{2.124189in}{0.955065in}}%
\pgfpathlineto{\pgfqpoint{2.138458in}{1.207833in}}%
\pgfpathlineto{\pgfqpoint{2.152727in}{0.973815in}}%
\pgfpathlineto{\pgfqpoint{2.166997in}{1.116478in}}%
\pgfpathlineto{\pgfqpoint{2.181266in}{1.006422in}}%
\pgfpathlineto{\pgfqpoint{2.195535in}{1.040346in}}%
\pgfpathlineto{\pgfqpoint{2.209804in}{1.101044in}}%
\pgfpathlineto{\pgfqpoint{2.224074in}{0.978658in}}%
\pgfpathlineto{\pgfqpoint{2.238343in}{1.040067in}}%
\pgfpathlineto{\pgfqpoint{2.252612in}{0.963645in}}%
\pgfpathlineto{\pgfqpoint{2.266881in}{1.146728in}}%
\pgfpathlineto{\pgfqpoint{2.281151in}{1.001360in}}%
\pgfpathlineto{\pgfqpoint{2.295420in}{1.081906in}}%
\pgfpathlineto{\pgfqpoint{2.309689in}{1.001276in}}%
\pgfpathlineto{\pgfqpoint{2.323959in}{1.002440in}}%
\pgfpathlineto{\pgfqpoint{2.338228in}{1.198990in}}%
\pgfpathlineto{\pgfqpoint{2.352497in}{1.044799in}}%
\pgfpathlineto{\pgfqpoint{2.366766in}{1.037488in}}%
\pgfpathlineto{\pgfqpoint{2.381036in}{1.028391in}}%
\pgfpathlineto{\pgfqpoint{2.395305in}{0.997394in}}%
\pgfpathlineto{\pgfqpoint{2.409574in}{1.179747in}}%
\pgfpathlineto{\pgfqpoint{2.423843in}{0.960463in}}%
\pgfpathlineto{\pgfqpoint{2.438113in}{0.998447in}}%
\pgfpathlineto{\pgfqpoint{2.452382in}{1.048670in}}%
\pgfpathlineto{\pgfqpoint{2.637883in}{4.056000in}}%
\pgfpathlineto{\pgfqpoint{2.652152in}{1.145914in}}%
\pgfpathlineto{\pgfqpoint{2.666421in}{1.037506in}}%
\pgfpathlineto{\pgfqpoint{2.694960in}{0.933592in}}%
\pgfpathlineto{\pgfqpoint{2.709229in}{1.059280in}}%
\pgfpathlineto{\pgfqpoint{2.723498in}{1.155783in}}%
\pgfpathlineto{\pgfqpoint{2.737768in}{1.045930in}}%
\pgfpathlineto{\pgfqpoint{2.752037in}{0.981918in}}%
\pgfpathlineto{\pgfqpoint{2.766306in}{0.966117in}}%
\pgfpathlineto{\pgfqpoint{2.780575in}{1.060697in}}%
\pgfpathlineto{\pgfqpoint{2.794845in}{0.995466in}}%
\pgfpathlineto{\pgfqpoint{2.809114in}{0.991075in}}%
\pgfpathlineto{\pgfqpoint{2.823383in}{1.073532in}}%
\pgfpathlineto{\pgfqpoint{2.837652in}{0.972082in}}%
\pgfpathlineto{\pgfqpoint{2.851922in}{0.920409in}}%
\pgfpathlineto{\pgfqpoint{2.866191in}{1.060415in}}%
\pgfpathlineto{\pgfqpoint{2.880460in}{0.941154in}}%
\pgfpathlineto{\pgfqpoint{2.894730in}{1.058182in}}%
\pgfpathlineto{\pgfqpoint{2.908999in}{0.992651in}}%
\pgfpathlineto{\pgfqpoint{2.923268in}{1.088026in}}%
\pgfpathlineto{\pgfqpoint{2.937537in}{1.089249in}}%
\pgfpathlineto{\pgfqpoint{2.951807in}{0.997747in}}%
\pgfpathlineto{\pgfqpoint{2.966076in}{1.034706in}}%
\pgfpathlineto{\pgfqpoint{2.980345in}{0.998440in}}%
\pgfpathlineto{\pgfqpoint{2.994614in}{0.975267in}}%
\pgfpathlineto{\pgfqpoint{3.008884in}{0.978940in}}%
\pgfpathlineto{\pgfqpoint{3.023153in}{1.075402in}}%
\pgfpathlineto{\pgfqpoint{3.037422in}{1.014249in}}%
\pgfpathlineto{\pgfqpoint{3.051692in}{1.101648in}}%
\pgfpathlineto{\pgfqpoint{3.065961in}{1.059613in}}%
\pgfpathlineto{\pgfqpoint{3.080230in}{1.025317in}}%
\pgfpathlineto{\pgfqpoint{3.108769in}{0.993920in}}%
\pgfpathlineto{\pgfqpoint{3.123038in}{0.981685in}}%
\pgfpathlineto{\pgfqpoint{3.137307in}{1.005792in}}%
\pgfpathlineto{\pgfqpoint{3.151577in}{0.976255in}}%
\pgfpathlineto{\pgfqpoint{3.165846in}{1.035511in}}%
\pgfpathlineto{\pgfqpoint{3.180115in}{0.979705in}}%
\pgfpathlineto{\pgfqpoint{3.194384in}{1.017288in}}%
\pgfpathlineto{\pgfqpoint{3.208654in}{1.006632in}}%
\pgfpathlineto{\pgfqpoint{3.222923in}{0.981535in}}%
\pgfpathlineto{\pgfqpoint{3.237192in}{1.019835in}}%
\pgfpathlineto{\pgfqpoint{3.251461in}{0.949714in}}%
\pgfpathlineto{\pgfqpoint{3.265731in}{1.104096in}}%
\pgfpathlineto{\pgfqpoint{3.280000in}{1.091461in}}%
\pgfpathlineto{\pgfqpoint{3.294269in}{0.961535in}}%
\pgfpathlineto{\pgfqpoint{3.308539in}{0.954521in}}%
\pgfpathlineto{\pgfqpoint{3.322808in}{1.121257in}}%
\pgfpathlineto{\pgfqpoint{3.337077in}{0.981488in}}%
\pgfpathlineto{\pgfqpoint{3.351346in}{0.983823in}}%
\pgfpathlineto{\pgfqpoint{3.365616in}{1.036302in}}%
\pgfpathlineto{\pgfqpoint{3.379885in}{1.149280in}}%
\pgfpathlineto{\pgfqpoint{3.394154in}{1.027334in}}%
\pgfpathlineto{\pgfqpoint{3.408423in}{1.077898in}}%
\pgfpathlineto{\pgfqpoint{3.422693in}{1.308785in}}%
\pgfpathlineto{\pgfqpoint{3.436962in}{1.475213in}}%
\pgfpathlineto{\pgfqpoint{3.451231in}{1.006032in}}%
\pgfpathlineto{\pgfqpoint{3.465501in}{1.094157in}}%
\pgfpathlineto{\pgfqpoint{3.479770in}{1.048330in}}%
\pgfpathlineto{\pgfqpoint{3.494039in}{1.087956in}}%
\pgfpathlineto{\pgfqpoint{3.508308in}{0.976195in}}%
\pgfpathlineto{\pgfqpoint{3.522578in}{1.037309in}}%
\pgfpathlineto{\pgfqpoint{3.536847in}{1.160975in}}%
\pgfpathlineto{\pgfqpoint{3.551116in}{0.962686in}}%
\pgfpathlineto{\pgfqpoint{3.565386in}{1.063903in}}%
\pgfpathlineto{\pgfqpoint{3.579655in}{1.028787in}}%
\pgfpathlineto{\pgfqpoint{3.593924in}{1.084745in}}%
\pgfpathlineto{\pgfqpoint{3.608193in}{1.002243in}}%
\pgfpathlineto{\pgfqpoint{3.622463in}{0.996301in}}%
\pgfpathlineto{\pgfqpoint{3.636732in}{1.030158in}}%
\pgfpathlineto{\pgfqpoint{3.651001in}{0.948307in}}%
\pgfpathlineto{\pgfqpoint{3.665270in}{1.013693in}}%
\pgfpathlineto{\pgfqpoint{3.679540in}{1.048827in}}%
\pgfpathlineto{\pgfqpoint{3.693809in}{1.077684in}}%
\pgfpathlineto{\pgfqpoint{3.708078in}{1.037849in}}%
\pgfpathlineto{\pgfqpoint{3.722348in}{0.989854in}}%
\pgfpathlineto{\pgfqpoint{3.736617in}{1.241337in}}%
\pgfpathlineto{\pgfqpoint{3.750886in}{1.059900in}}%
\pgfpathlineto{\pgfqpoint{3.765155in}{1.078523in}}%
\pgfpathlineto{\pgfqpoint{3.779425in}{0.955397in}}%
\pgfpathlineto{\pgfqpoint{3.793694in}{0.900448in}}%
\pgfpathlineto{\pgfqpoint{3.807963in}{1.163388in}}%
\pgfpathlineto{\pgfqpoint{3.822232in}{1.045643in}}%
\pgfpathlineto{\pgfqpoint{3.836502in}{0.980881in}}%
\pgfpathlineto{\pgfqpoint{3.850771in}{0.997119in}}%
\pgfpathlineto{\pgfqpoint{3.865040in}{1.249257in}}%
\pgfpathlineto{\pgfqpoint{3.879310in}{1.041890in}}%
\pgfpathlineto{\pgfqpoint{3.893579in}{1.229546in}}%
\pgfpathlineto{\pgfqpoint{3.907848in}{1.003683in}}%
\pgfpathlineto{\pgfqpoint{3.922117in}{0.893958in}}%
\pgfpathlineto{\pgfqpoint{3.936387in}{1.106062in}}%
\pgfpathlineto{\pgfqpoint{3.950656in}{1.065805in}}%
\pgfpathlineto{\pgfqpoint{3.964925in}{1.129196in}}%
\pgfpathlineto{\pgfqpoint{3.979194in}{1.005965in}}%
\pgfpathlineto{\pgfqpoint{3.993464in}{1.233484in}}%
\pgfpathlineto{\pgfqpoint{4.007733in}{0.960690in}}%
\pgfpathlineto{\pgfqpoint{4.022002in}{1.304403in}}%
\pgfpathlineto{\pgfqpoint{4.036272in}{0.898648in}}%
\pgfpathlineto{\pgfqpoint{4.050541in}{1.110715in}}%
\pgfpathlineto{\pgfqpoint{4.064810in}{0.998863in}}%
\pgfpathlineto{\pgfqpoint{4.079079in}{1.115565in}}%
\pgfpathlineto{\pgfqpoint{4.093349in}{1.486461in}}%
\pgfpathlineto{\pgfqpoint{4.107618in}{1.054005in}}%
\pgfpathlineto{\pgfqpoint{4.121887in}{1.013742in}}%
\pgfpathlineto{\pgfqpoint{4.136157in}{1.016480in}}%
\pgfpathlineto{\pgfqpoint{4.150426in}{1.066668in}}%
\pgfpathlineto{\pgfqpoint{4.164695in}{1.063643in}}%
\pgfpathlineto{\pgfqpoint{4.178964in}{0.970546in}}%
\pgfpathlineto{\pgfqpoint{4.207503in}{1.265185in}}%
\pgfpathlineto{\pgfqpoint{4.221772in}{1.000896in}}%
\pgfpathlineto{\pgfqpoint{4.236041in}{1.192161in}}%
\pgfpathlineto{\pgfqpoint{4.264580in}{1.058987in}}%
\pgfpathlineto{\pgfqpoint{4.278849in}{1.259807in}}%
\pgfpathlineto{\pgfqpoint{4.293119in}{1.143252in}}%
\pgfpathlineto{\pgfqpoint{4.307388in}{1.010122in}}%
\pgfpathlineto{\pgfqpoint{4.321657in}{1.013052in}}%
\pgfpathlineto{\pgfqpoint{4.335926in}{1.036751in}}%
\pgfpathlineto{\pgfqpoint{4.350196in}{1.011216in}}%
\pgfpathlineto{\pgfqpoint{4.364465in}{1.008174in}}%
\pgfpathlineto{\pgfqpoint{4.378734in}{1.057601in}}%
\pgfpathlineto{\pgfqpoint{4.393003in}{1.067326in}}%
\pgfpathlineto{\pgfqpoint{4.407273in}{1.047854in}}%
\pgfpathlineto{\pgfqpoint{4.421542in}{1.040016in}}%
\pgfpathlineto{\pgfqpoint{4.435811in}{1.182487in}}%
\pgfpathlineto{\pgfqpoint{4.450081in}{1.071009in}}%
\pgfpathlineto{\pgfqpoint{4.464350in}{1.113226in}}%
\pgfpathlineto{\pgfqpoint{4.478619in}{1.088171in}}%
\pgfpathlineto{\pgfqpoint{4.492888in}{1.204050in}}%
\pgfpathlineto{\pgfqpoint{4.507158in}{1.011304in}}%
\pgfpathlineto{\pgfqpoint{4.521427in}{0.953141in}}%
\pgfpathlineto{\pgfqpoint{4.535696in}{1.050558in}}%
\pgfpathlineto{\pgfqpoint{4.549965in}{1.003143in}}%
\pgfpathlineto{\pgfqpoint{4.564235in}{1.022861in}}%
\pgfpathlineto{\pgfqpoint{4.578504in}{1.065336in}}%
\pgfpathlineto{\pgfqpoint{4.592773in}{0.983470in}}%
\pgfpathlineto{\pgfqpoint{4.607043in}{1.021488in}}%
\pgfpathlineto{\pgfqpoint{4.621312in}{0.983387in}}%
\pgfpathlineto{\pgfqpoint{4.635581in}{1.049388in}}%
\pgfpathlineto{\pgfqpoint{4.649850in}{1.029552in}}%
\pgfpathlineto{\pgfqpoint{4.664120in}{1.069389in}}%
\pgfpathlineto{\pgfqpoint{4.678389in}{1.056586in}}%
\pgfpathlineto{\pgfqpoint{4.692658in}{0.952753in}}%
\pgfpathlineto{\pgfqpoint{4.706928in}{1.022653in}}%
\pgfpathlineto{\pgfqpoint{4.721197in}{1.131842in}}%
\pgfpathlineto{\pgfqpoint{4.735466in}{1.032663in}}%
\pgfpathlineto{\pgfqpoint{4.749735in}{1.178583in}}%
\pgfpathlineto{\pgfqpoint{4.764005in}{1.004256in}}%
\pgfpathlineto{\pgfqpoint{4.778274in}{1.012923in}}%
\pgfpathlineto{\pgfqpoint{4.792543in}{0.936340in}}%
\pgfpathlineto{\pgfqpoint{4.806812in}{1.015310in}}%
\pgfpathlineto{\pgfqpoint{4.821082in}{1.230191in}}%
\pgfpathlineto{\pgfqpoint{4.835351in}{1.072422in}}%
\pgfpathlineto{\pgfqpoint{4.849620in}{1.022616in}}%
\pgfpathlineto{\pgfqpoint{4.863890in}{1.051308in}}%
\pgfpathlineto{\pgfqpoint{4.878159in}{1.146175in}}%
\pgfpathlineto{\pgfqpoint{4.906697in}{0.977426in}}%
\pgfpathlineto{\pgfqpoint{4.920967in}{1.152366in}}%
\pgfpathlineto{\pgfqpoint{4.935236in}{1.119659in}}%
\pgfpathlineto{\pgfqpoint{4.949505in}{1.001341in}}%
\pgfpathlineto{\pgfqpoint{4.963774in}{1.064711in}}%
\pgfpathlineto{\pgfqpoint{4.978044in}{1.082450in}}%
\pgfpathlineto{\pgfqpoint{4.992313in}{0.839135in}}%
\pgfpathlineto{\pgfqpoint{5.006582in}{1.238069in}}%
\pgfpathlineto{\pgfqpoint{5.020852in}{1.336210in}}%
\pgfpathlineto{\pgfqpoint{5.035121in}{0.985034in}}%
\pgfpathlineto{\pgfqpoint{5.049390in}{1.026729in}}%
\pgfpathlineto{\pgfqpoint{5.063659in}{1.110849in}}%
\pgfpathlineto{\pgfqpoint{5.077929in}{1.049832in}}%
\pgfpathlineto{\pgfqpoint{5.092198in}{1.056418in}}%
\pgfpathlineto{\pgfqpoint{5.106467in}{0.996022in}}%
\pgfpathlineto{\pgfqpoint{5.120736in}{1.117098in}}%
\pgfpathlineto{\pgfqpoint{5.135006in}{1.002495in}}%
\pgfpathlineto{\pgfqpoint{5.149275in}{1.036050in}}%
\pgfpathlineto{\pgfqpoint{5.163544in}{1.083653in}}%
\pgfpathlineto{\pgfqpoint{5.177814in}{1.051841in}}%
\pgfpathlineto{\pgfqpoint{5.192083in}{0.966611in}}%
\pgfpathlineto{\pgfqpoint{5.206352in}{1.059379in}}%
\pgfpathlineto{\pgfqpoint{5.220621in}{1.024834in}}%
\pgfpathlineto{\pgfqpoint{5.234891in}{1.021150in}}%
\pgfpathlineto{\pgfqpoint{5.249160in}{1.050648in}}%
\pgfpathlineto{\pgfqpoint{5.263429in}{1.070438in}}%
\pgfpathlineto{\pgfqpoint{5.277699in}{1.071970in}}%
\pgfpathlineto{\pgfqpoint{5.291968in}{1.000989in}}%
\pgfpathlineto{\pgfqpoint{5.306237in}{1.022292in}}%
\pgfpathlineto{\pgfqpoint{5.320506in}{0.969914in}}%
\pgfpathlineto{\pgfqpoint{5.334776in}{0.885118in}}%
\pgfpathlineto{\pgfqpoint{5.349045in}{0.790907in}}%
\pgfpathlineto{\pgfqpoint{5.363314in}{0.758108in}}%
\pgfpathlineto{\pgfqpoint{5.377583in}{0.735463in}}%
\pgfpathlineto{\pgfqpoint{5.391853in}{0.708669in}}%
\pgfpathlineto{\pgfqpoint{5.406122in}{0.700020in}}%
\pgfpathlineto{\pgfqpoint{5.420391in}{0.697422in}}%
\pgfpathlineto{\pgfqpoint{5.520276in}{0.696470in}}%
\pgfpathlineto{\pgfqpoint{5.534545in}{0.696000in}}%
\pgfpathlineto{\pgfqpoint{5.534545in}{0.696000in}}%
\pgfusepath{stroke}%
\end{pgfscope}%
\begin{pgfscope}%
\pgfpathrectangle{\pgfqpoint{0.800000in}{0.528000in}}{\pgfqpoint{4.960000in}{3.696000in}}%
\pgfusepath{clip}%
\pgfsetrectcap%
\pgfsetroundjoin%
\pgfsetlinewidth{1.505625pt}%
\definecolor{currentstroke}{rgb}{1.000000,0.498039,0.054902}%
\pgfsetstrokecolor{currentstroke}%
\pgfsetdash{}{0pt}%
\pgfpathmoveto{\pgfqpoint{1.025455in}{1.039992in}}%
\pgfpathlineto{\pgfqpoint{5.534545in}{1.039992in}}%
\pgfpathlineto{\pgfqpoint{5.534545in}{1.039992in}}%
\pgfusepath{stroke}%
\end{pgfscope}%
\begin{pgfscope}%
\pgfsetrectcap%
\pgfsetmiterjoin%
\pgfsetlinewidth{0.803000pt}%
\definecolor{currentstroke}{rgb}{0.000000,0.000000,0.000000}%
\pgfsetstrokecolor{currentstroke}%
\pgfsetdash{}{0pt}%
\pgfpathmoveto{\pgfqpoint{0.800000in}{0.528000in}}%
\pgfpathlineto{\pgfqpoint{0.800000in}{4.224000in}}%
\pgfusepath{stroke}%
\end{pgfscope}%
\begin{pgfscope}%
\pgfsetrectcap%
\pgfsetmiterjoin%
\pgfsetlinewidth{0.803000pt}%
\definecolor{currentstroke}{rgb}{0.000000,0.000000,0.000000}%
\pgfsetstrokecolor{currentstroke}%
\pgfsetdash{}{0pt}%
\pgfpathmoveto{\pgfqpoint{5.760000in}{0.528000in}}%
\pgfpathlineto{\pgfqpoint{5.760000in}{4.224000in}}%
\pgfusepath{stroke}%
\end{pgfscope}%
\begin{pgfscope}%
\pgfsetrectcap%
\pgfsetmiterjoin%
\pgfsetlinewidth{0.803000pt}%
\definecolor{currentstroke}{rgb}{0.000000,0.000000,0.000000}%
\pgfsetstrokecolor{currentstroke}%
\pgfsetdash{}{0pt}%
\pgfpathmoveto{\pgfqpoint{0.800000in}{0.528000in}}%
\pgfpathlineto{\pgfqpoint{5.760000in}{0.528000in}}%
\pgfusepath{stroke}%
\end{pgfscope}%
\begin{pgfscope}%
\pgfsetrectcap%
\pgfsetmiterjoin%
\pgfsetlinewidth{0.803000pt}%
\definecolor{currentstroke}{rgb}{0.000000,0.000000,0.000000}%
\pgfsetstrokecolor{currentstroke}%
\pgfsetdash{}{0pt}%
\pgfpathmoveto{\pgfqpoint{0.800000in}{4.224000in}}%
\pgfpathlineto{\pgfqpoint{5.760000in}{4.224000in}}%
\pgfusepath{stroke}%
\end{pgfscope}%
\begin{pgfscope}%
\pgfsetbuttcap%
\pgfsetmiterjoin%
\definecolor{currentfill}{rgb}{1.000000,1.000000,1.000000}%
\pgfsetfillcolor{currentfill}%
\pgfsetfillopacity{0.800000}%
\pgfsetlinewidth{1.003750pt}%
\definecolor{currentstroke}{rgb}{0.800000,0.800000,0.800000}%
\pgfsetstrokecolor{currentstroke}%
\pgfsetstrokeopacity{0.800000}%
\pgfsetdash{}{0pt}%
\pgfpathmoveto{\pgfqpoint{3.522024in}{3.675698in}}%
\pgfpathlineto{\pgfqpoint{5.653056in}{3.675698in}}%
\pgfpathquadraticcurveto{\pgfqpoint{5.683611in}{3.675698in}}{\pgfqpoint{5.683611in}{3.706254in}}%
\pgfpathlineto{\pgfqpoint{5.683611in}{4.117056in}}%
\pgfpathquadraticcurveto{\pgfqpoint{5.683611in}{4.147611in}}{\pgfqpoint{5.653056in}{4.147611in}}%
\pgfpathlineto{\pgfqpoint{3.522024in}{4.147611in}}%
\pgfpathquadraticcurveto{\pgfqpoint{3.491468in}{4.147611in}}{\pgfqpoint{3.491468in}{4.117056in}}%
\pgfpathlineto{\pgfqpoint{3.491468in}{3.706254in}}%
\pgfpathquadraticcurveto{\pgfqpoint{3.491468in}{3.675698in}}{\pgfqpoint{3.522024in}{3.675698in}}%
\pgfpathclose%
\pgfusepath{stroke,fill}%
\end{pgfscope}%
\begin{pgfscope}%
\pgfsetrectcap%
\pgfsetroundjoin%
\pgfsetlinewidth{1.505625pt}%
\definecolor{currentstroke}{rgb}{0.121569,0.466667,0.705882}%
\pgfsetstrokecolor{currentstroke}%
\pgfsetdash{}{0pt}%
\pgfpathmoveto{\pgfqpoint{3.552579in}{4.033028in}}%
\pgfpathlineto{\pgfqpoint{3.858135in}{4.033028in}}%
\pgfusepath{stroke}%
\end{pgfscope}%
\begin{pgfscope}%
\pgftext[x=3.980357in,y=3.979556in,left,base]{\fontsize{11.000000}{13.200000}\selectfont Latency of writes}%
\end{pgfscope}%
\begin{pgfscope}%
\pgfsetrectcap%
\pgfsetroundjoin%
\pgfsetlinewidth{1.505625pt}%
\definecolor{currentstroke}{rgb}{1.000000,0.498039,0.054902}%
\pgfsetstrokecolor{currentstroke}%
\pgfsetdash{}{0pt}%
\pgfpathmoveto{\pgfqpoint{3.552579in}{3.819988in}}%
\pgfpathlineto{\pgfqpoint{3.858135in}{3.819988in}}%
\pgfusepath{stroke}%
\end{pgfscope}%
\begin{pgfscope}%
\pgftext[x=3.980357in,y=3.766516in,left,base]{\fontsize{11.000000}{13.200000}\selectfont Average latency of writes}%
\end{pgfscope}%
\end{pgfpicture}%
\makeatother%
\endgroup%

    \caption{Latency of write operations for every second of experiment 1 in \prettyref{tab:res-4}}
    \label{fig:lat-plp}
\end{figure}

\begin{figure}
    %% Creator: Matplotlib, PGF backend
%%
%% To include the figure in your LaTeX document, write
%%   \input{<filename>.pgf}
%%
%% Make sure the required packages are loaded in your preamble
%%   \usepackage{pgf}
%%
%% Figures using additional raster images can only be included by \input if
%% they are in the same directory as the main LaTeX file. For loading figures
%% from other directories you can use the `import` package
%%   \usepackage{import}
%% and then include the figures with
%%   \import{<path to file>}{<filename>.pgf}
%%
%% Matplotlib used the following preamble
%%   \usepackage[utf8x]{inputenc}
%%   \usepackage[T1]{fontenc}
%%   \usepackage{lmodern}
%%
\begingroup%
\makeatletter%
\begin{pgfpicture}%
\pgfpathrectangle{\pgfpointorigin}{\pgfqpoint{6.400000in}{4.800000in}}%
\pgfusepath{use as bounding box, clip}%
\begin{pgfscope}%
\pgfsetbuttcap%
\pgfsetmiterjoin%
\definecolor{currentfill}{rgb}{1.000000,1.000000,1.000000}%
\pgfsetfillcolor{currentfill}%
\pgfsetlinewidth{0.000000pt}%
\definecolor{currentstroke}{rgb}{1.000000,1.000000,1.000000}%
\pgfsetstrokecolor{currentstroke}%
\pgfsetdash{}{0pt}%
\pgfpathmoveto{\pgfqpoint{0.000000in}{0.000000in}}%
\pgfpathlineto{\pgfqpoint{6.400000in}{0.000000in}}%
\pgfpathlineto{\pgfqpoint{6.400000in}{4.800000in}}%
\pgfpathlineto{\pgfqpoint{0.000000in}{4.800000in}}%
\pgfpathclose%
\pgfusepath{fill}%
\end{pgfscope}%
\begin{pgfscope}%
\pgfsetbuttcap%
\pgfsetmiterjoin%
\definecolor{currentfill}{rgb}{1.000000,1.000000,1.000000}%
\pgfsetfillcolor{currentfill}%
\pgfsetlinewidth{0.000000pt}%
\definecolor{currentstroke}{rgb}{0.000000,0.000000,0.000000}%
\pgfsetstrokecolor{currentstroke}%
\pgfsetstrokeopacity{0.000000}%
\pgfsetdash{}{0pt}%
\pgfpathmoveto{\pgfqpoint{0.800000in}{0.528000in}}%
\pgfpathlineto{\pgfqpoint{5.760000in}{0.528000in}}%
\pgfpathlineto{\pgfqpoint{5.760000in}{4.224000in}}%
\pgfpathlineto{\pgfqpoint{0.800000in}{4.224000in}}%
\pgfpathclose%
\pgfusepath{fill}%
\end{pgfscope}%
\begin{pgfscope}%
\pgfsetbuttcap%
\pgfsetroundjoin%
\definecolor{currentfill}{rgb}{0.000000,0.000000,0.000000}%
\pgfsetfillcolor{currentfill}%
\pgfsetlinewidth{0.803000pt}%
\definecolor{currentstroke}{rgb}{0.000000,0.000000,0.000000}%
\pgfsetstrokecolor{currentstroke}%
\pgfsetdash{}{0pt}%
\pgfsys@defobject{currentmarker}{\pgfqpoint{0.000000in}{-0.048611in}}{\pgfqpoint{0.000000in}{0.000000in}}{%
\pgfpathmoveto{\pgfqpoint{0.000000in}{0.000000in}}%
\pgfpathlineto{\pgfqpoint{0.000000in}{-0.048611in}}%
\pgfusepath{stroke,fill}%
}%
\begin{pgfscope}%
\pgfsys@transformshift{1.025455in}{0.528000in}%
\pgfsys@useobject{currentmarker}{}%
\end{pgfscope}%
\end{pgfscope}%
\begin{pgfscope}%
\pgftext[x=1.025455in,y=0.430778in,,top]{\fontsize{11.000000}{13.200000}\selectfont \(\displaystyle 0\)}%
\end{pgfscope}%
\begin{pgfscope}%
\pgfsetbuttcap%
\pgfsetroundjoin%
\definecolor{currentfill}{rgb}{0.000000,0.000000,0.000000}%
\pgfsetfillcolor{currentfill}%
\pgfsetlinewidth{0.803000pt}%
\definecolor{currentstroke}{rgb}{0.000000,0.000000,0.000000}%
\pgfsetstrokecolor{currentstroke}%
\pgfsetdash{}{0pt}%
\pgfsys@defobject{currentmarker}{\pgfqpoint{0.000000in}{-0.048611in}}{\pgfqpoint{0.000000in}{0.000000in}}{%
\pgfpathmoveto{\pgfqpoint{0.000000in}{0.000000in}}%
\pgfpathlineto{\pgfqpoint{0.000000in}{-0.048611in}}%
\pgfusepath{stroke,fill}%
}%
\begin{pgfscope}%
\pgfsys@transformshift{1.767081in}{0.528000in}%
\pgfsys@useobject{currentmarker}{}%
\end{pgfscope}%
\end{pgfscope}%
\begin{pgfscope}%
\pgftext[x=1.767081in,y=0.430778in,,top]{\fontsize{11.000000}{13.200000}\selectfont \(\displaystyle 50\)}%
\end{pgfscope}%
\begin{pgfscope}%
\pgfsetbuttcap%
\pgfsetroundjoin%
\definecolor{currentfill}{rgb}{0.000000,0.000000,0.000000}%
\pgfsetfillcolor{currentfill}%
\pgfsetlinewidth{0.803000pt}%
\definecolor{currentstroke}{rgb}{0.000000,0.000000,0.000000}%
\pgfsetstrokecolor{currentstroke}%
\pgfsetdash{}{0pt}%
\pgfsys@defobject{currentmarker}{\pgfqpoint{0.000000in}{-0.048611in}}{\pgfqpoint{0.000000in}{0.000000in}}{%
\pgfpathmoveto{\pgfqpoint{0.000000in}{0.000000in}}%
\pgfpathlineto{\pgfqpoint{0.000000in}{-0.048611in}}%
\pgfusepath{stroke,fill}%
}%
\begin{pgfscope}%
\pgfsys@transformshift{2.508708in}{0.528000in}%
\pgfsys@useobject{currentmarker}{}%
\end{pgfscope}%
\end{pgfscope}%
\begin{pgfscope}%
\pgftext[x=2.508708in,y=0.430778in,,top]{\fontsize{11.000000}{13.200000}\selectfont \(\displaystyle 100\)}%
\end{pgfscope}%
\begin{pgfscope}%
\pgfsetbuttcap%
\pgfsetroundjoin%
\definecolor{currentfill}{rgb}{0.000000,0.000000,0.000000}%
\pgfsetfillcolor{currentfill}%
\pgfsetlinewidth{0.803000pt}%
\definecolor{currentstroke}{rgb}{0.000000,0.000000,0.000000}%
\pgfsetstrokecolor{currentstroke}%
\pgfsetdash{}{0pt}%
\pgfsys@defobject{currentmarker}{\pgfqpoint{0.000000in}{-0.048611in}}{\pgfqpoint{0.000000in}{0.000000in}}{%
\pgfpathmoveto{\pgfqpoint{0.000000in}{0.000000in}}%
\pgfpathlineto{\pgfqpoint{0.000000in}{-0.048611in}}%
\pgfusepath{stroke,fill}%
}%
\begin{pgfscope}%
\pgfsys@transformshift{3.250335in}{0.528000in}%
\pgfsys@useobject{currentmarker}{}%
\end{pgfscope}%
\end{pgfscope}%
\begin{pgfscope}%
\pgftext[x=3.250335in,y=0.430778in,,top]{\fontsize{11.000000}{13.200000}\selectfont \(\displaystyle 150\)}%
\end{pgfscope}%
\begin{pgfscope}%
\pgfsetbuttcap%
\pgfsetroundjoin%
\definecolor{currentfill}{rgb}{0.000000,0.000000,0.000000}%
\pgfsetfillcolor{currentfill}%
\pgfsetlinewidth{0.803000pt}%
\definecolor{currentstroke}{rgb}{0.000000,0.000000,0.000000}%
\pgfsetstrokecolor{currentstroke}%
\pgfsetdash{}{0pt}%
\pgfsys@defobject{currentmarker}{\pgfqpoint{0.000000in}{-0.048611in}}{\pgfqpoint{0.000000in}{0.000000in}}{%
\pgfpathmoveto{\pgfqpoint{0.000000in}{0.000000in}}%
\pgfpathlineto{\pgfqpoint{0.000000in}{-0.048611in}}%
\pgfusepath{stroke,fill}%
}%
\begin{pgfscope}%
\pgfsys@transformshift{3.991962in}{0.528000in}%
\pgfsys@useobject{currentmarker}{}%
\end{pgfscope}%
\end{pgfscope}%
\begin{pgfscope}%
\pgftext[x=3.991962in,y=0.430778in,,top]{\fontsize{11.000000}{13.200000}\selectfont \(\displaystyle 200\)}%
\end{pgfscope}%
\begin{pgfscope}%
\pgfsetbuttcap%
\pgfsetroundjoin%
\definecolor{currentfill}{rgb}{0.000000,0.000000,0.000000}%
\pgfsetfillcolor{currentfill}%
\pgfsetlinewidth{0.803000pt}%
\definecolor{currentstroke}{rgb}{0.000000,0.000000,0.000000}%
\pgfsetstrokecolor{currentstroke}%
\pgfsetdash{}{0pt}%
\pgfsys@defobject{currentmarker}{\pgfqpoint{0.000000in}{-0.048611in}}{\pgfqpoint{0.000000in}{0.000000in}}{%
\pgfpathmoveto{\pgfqpoint{0.000000in}{0.000000in}}%
\pgfpathlineto{\pgfqpoint{0.000000in}{-0.048611in}}%
\pgfusepath{stroke,fill}%
}%
\begin{pgfscope}%
\pgfsys@transformshift{4.733589in}{0.528000in}%
\pgfsys@useobject{currentmarker}{}%
\end{pgfscope}%
\end{pgfscope}%
\begin{pgfscope}%
\pgftext[x=4.733589in,y=0.430778in,,top]{\fontsize{11.000000}{13.200000}\selectfont \(\displaystyle 250\)}%
\end{pgfscope}%
\begin{pgfscope}%
\pgfsetbuttcap%
\pgfsetroundjoin%
\definecolor{currentfill}{rgb}{0.000000,0.000000,0.000000}%
\pgfsetfillcolor{currentfill}%
\pgfsetlinewidth{0.803000pt}%
\definecolor{currentstroke}{rgb}{0.000000,0.000000,0.000000}%
\pgfsetstrokecolor{currentstroke}%
\pgfsetdash{}{0pt}%
\pgfsys@defobject{currentmarker}{\pgfqpoint{0.000000in}{-0.048611in}}{\pgfqpoint{0.000000in}{0.000000in}}{%
\pgfpathmoveto{\pgfqpoint{0.000000in}{0.000000in}}%
\pgfpathlineto{\pgfqpoint{0.000000in}{-0.048611in}}%
\pgfusepath{stroke,fill}%
}%
\begin{pgfscope}%
\pgfsys@transformshift{5.475215in}{0.528000in}%
\pgfsys@useobject{currentmarker}{}%
\end{pgfscope}%
\end{pgfscope}%
\begin{pgfscope}%
\pgftext[x=5.475215in,y=0.430778in,,top]{\fontsize{11.000000}{13.200000}\selectfont \(\displaystyle 300\)}%
\end{pgfscope}%
\begin{pgfscope}%
\pgftext[x=3.280000in,y=0.240271in,,top]{\fontsize{11.000000}{13.200000}\selectfont Time of experiment (in seconds)}%
\end{pgfscope}%
\begin{pgfscope}%
\pgfsetbuttcap%
\pgfsetroundjoin%
\definecolor{currentfill}{rgb}{0.000000,0.000000,0.000000}%
\pgfsetfillcolor{currentfill}%
\pgfsetlinewidth{0.803000pt}%
\definecolor{currentstroke}{rgb}{0.000000,0.000000,0.000000}%
\pgfsetstrokecolor{currentstroke}%
\pgfsetdash{}{0pt}%
\pgfsys@defobject{currentmarker}{\pgfqpoint{-0.048611in}{0.000000in}}{\pgfqpoint{0.000000in}{0.000000in}}{%
\pgfpathmoveto{\pgfqpoint{0.000000in}{0.000000in}}%
\pgfpathlineto{\pgfqpoint{-0.048611in}{0.000000in}}%
\pgfusepath{stroke,fill}%
}%
\begin{pgfscope}%
\pgfsys@transformshift{0.800000in}{0.691007in}%
\pgfsys@useobject{currentmarker}{}%
\end{pgfscope}%
\end{pgfscope}%
\begin{pgfscope}%
\pgftext[x=0.627981in,y=0.638385in,left,base]{\fontsize{11.000000}{13.200000}\selectfont \(\displaystyle 0\)}%
\end{pgfscope}%
\begin{pgfscope}%
\pgfsetbuttcap%
\pgfsetroundjoin%
\definecolor{currentfill}{rgb}{0.000000,0.000000,0.000000}%
\pgfsetfillcolor{currentfill}%
\pgfsetlinewidth{0.803000pt}%
\definecolor{currentstroke}{rgb}{0.000000,0.000000,0.000000}%
\pgfsetstrokecolor{currentstroke}%
\pgfsetdash{}{0pt}%
\pgfsys@defobject{currentmarker}{\pgfqpoint{-0.048611in}{0.000000in}}{\pgfqpoint{0.000000in}{0.000000in}}{%
\pgfpathmoveto{\pgfqpoint{0.000000in}{0.000000in}}%
\pgfpathlineto{\pgfqpoint{-0.048611in}{0.000000in}}%
\pgfusepath{stroke,fill}%
}%
\begin{pgfscope}%
\pgfsys@transformshift{0.800000in}{1.111631in}%
\pgfsys@useobject{currentmarker}{}%
\end{pgfscope}%
\end{pgfscope}%
\begin{pgfscope}%
\pgftext[x=0.478386in,y=1.059009in,left,base]{\fontsize{11.000000}{13.200000}\selectfont \(\displaystyle 250\)}%
\end{pgfscope}%
\begin{pgfscope}%
\pgfsetbuttcap%
\pgfsetroundjoin%
\definecolor{currentfill}{rgb}{0.000000,0.000000,0.000000}%
\pgfsetfillcolor{currentfill}%
\pgfsetlinewidth{0.803000pt}%
\definecolor{currentstroke}{rgb}{0.000000,0.000000,0.000000}%
\pgfsetstrokecolor{currentstroke}%
\pgfsetdash{}{0pt}%
\pgfsys@defobject{currentmarker}{\pgfqpoint{-0.048611in}{0.000000in}}{\pgfqpoint{0.000000in}{0.000000in}}{%
\pgfpathmoveto{\pgfqpoint{0.000000in}{0.000000in}}%
\pgfpathlineto{\pgfqpoint{-0.048611in}{0.000000in}}%
\pgfusepath{stroke,fill}%
}%
\begin{pgfscope}%
\pgfsys@transformshift{0.800000in}{1.532256in}%
\pgfsys@useobject{currentmarker}{}%
\end{pgfscope}%
\end{pgfscope}%
\begin{pgfscope}%
\pgftext[x=0.478386in,y=1.479633in,left,base]{\fontsize{11.000000}{13.200000}\selectfont \(\displaystyle 500\)}%
\end{pgfscope}%
\begin{pgfscope}%
\pgfsetbuttcap%
\pgfsetroundjoin%
\definecolor{currentfill}{rgb}{0.000000,0.000000,0.000000}%
\pgfsetfillcolor{currentfill}%
\pgfsetlinewidth{0.803000pt}%
\definecolor{currentstroke}{rgb}{0.000000,0.000000,0.000000}%
\pgfsetstrokecolor{currentstroke}%
\pgfsetdash{}{0pt}%
\pgfsys@defobject{currentmarker}{\pgfqpoint{-0.048611in}{0.000000in}}{\pgfqpoint{0.000000in}{0.000000in}}{%
\pgfpathmoveto{\pgfqpoint{0.000000in}{0.000000in}}%
\pgfpathlineto{\pgfqpoint{-0.048611in}{0.000000in}}%
\pgfusepath{stroke,fill}%
}%
\begin{pgfscope}%
\pgfsys@transformshift{0.800000in}{1.952880in}%
\pgfsys@useobject{currentmarker}{}%
\end{pgfscope}%
\end{pgfscope}%
\begin{pgfscope}%
\pgftext[x=0.478386in,y=1.900257in,left,base]{\fontsize{11.000000}{13.200000}\selectfont \(\displaystyle 750\)}%
\end{pgfscope}%
\begin{pgfscope}%
\pgfsetbuttcap%
\pgfsetroundjoin%
\definecolor{currentfill}{rgb}{0.000000,0.000000,0.000000}%
\pgfsetfillcolor{currentfill}%
\pgfsetlinewidth{0.803000pt}%
\definecolor{currentstroke}{rgb}{0.000000,0.000000,0.000000}%
\pgfsetstrokecolor{currentstroke}%
\pgfsetdash{}{0pt}%
\pgfsys@defobject{currentmarker}{\pgfqpoint{-0.048611in}{0.000000in}}{\pgfqpoint{0.000000in}{0.000000in}}{%
\pgfpathmoveto{\pgfqpoint{0.000000in}{0.000000in}}%
\pgfpathlineto{\pgfqpoint{-0.048611in}{0.000000in}}%
\pgfusepath{stroke,fill}%
}%
\begin{pgfscope}%
\pgfsys@transformshift{0.800000in}{2.373504in}%
\pgfsys@useobject{currentmarker}{}%
\end{pgfscope}%
\end{pgfscope}%
\begin{pgfscope}%
\pgftext[x=0.403588in,y=2.320881in,left,base]{\fontsize{11.000000}{13.200000}\selectfont \(\displaystyle 1000\)}%
\end{pgfscope}%
\begin{pgfscope}%
\pgfsetbuttcap%
\pgfsetroundjoin%
\definecolor{currentfill}{rgb}{0.000000,0.000000,0.000000}%
\pgfsetfillcolor{currentfill}%
\pgfsetlinewidth{0.803000pt}%
\definecolor{currentstroke}{rgb}{0.000000,0.000000,0.000000}%
\pgfsetstrokecolor{currentstroke}%
\pgfsetdash{}{0pt}%
\pgfsys@defobject{currentmarker}{\pgfqpoint{-0.048611in}{0.000000in}}{\pgfqpoint{0.000000in}{0.000000in}}{%
\pgfpathmoveto{\pgfqpoint{0.000000in}{0.000000in}}%
\pgfpathlineto{\pgfqpoint{-0.048611in}{0.000000in}}%
\pgfusepath{stroke,fill}%
}%
\begin{pgfscope}%
\pgfsys@transformshift{0.800000in}{2.794128in}%
\pgfsys@useobject{currentmarker}{}%
\end{pgfscope}%
\end{pgfscope}%
\begin{pgfscope}%
\pgftext[x=0.403588in,y=2.741506in,left,base]{\fontsize{11.000000}{13.200000}\selectfont \(\displaystyle 1250\)}%
\end{pgfscope}%
\begin{pgfscope}%
\pgfsetbuttcap%
\pgfsetroundjoin%
\definecolor{currentfill}{rgb}{0.000000,0.000000,0.000000}%
\pgfsetfillcolor{currentfill}%
\pgfsetlinewidth{0.803000pt}%
\definecolor{currentstroke}{rgb}{0.000000,0.000000,0.000000}%
\pgfsetstrokecolor{currentstroke}%
\pgfsetdash{}{0pt}%
\pgfsys@defobject{currentmarker}{\pgfqpoint{-0.048611in}{0.000000in}}{\pgfqpoint{0.000000in}{0.000000in}}{%
\pgfpathmoveto{\pgfqpoint{0.000000in}{0.000000in}}%
\pgfpathlineto{\pgfqpoint{-0.048611in}{0.000000in}}%
\pgfusepath{stroke,fill}%
}%
\begin{pgfscope}%
\pgfsys@transformshift{0.800000in}{3.214752in}%
\pgfsys@useobject{currentmarker}{}%
\end{pgfscope}%
\end{pgfscope}%
\begin{pgfscope}%
\pgftext[x=0.403588in,y=3.162130in,left,base]{\fontsize{11.000000}{13.200000}\selectfont \(\displaystyle 1500\)}%
\end{pgfscope}%
\begin{pgfscope}%
\pgfsetbuttcap%
\pgfsetroundjoin%
\definecolor{currentfill}{rgb}{0.000000,0.000000,0.000000}%
\pgfsetfillcolor{currentfill}%
\pgfsetlinewidth{0.803000pt}%
\definecolor{currentstroke}{rgb}{0.000000,0.000000,0.000000}%
\pgfsetstrokecolor{currentstroke}%
\pgfsetdash{}{0pt}%
\pgfsys@defobject{currentmarker}{\pgfqpoint{-0.048611in}{0.000000in}}{\pgfqpoint{0.000000in}{0.000000in}}{%
\pgfpathmoveto{\pgfqpoint{0.000000in}{0.000000in}}%
\pgfpathlineto{\pgfqpoint{-0.048611in}{0.000000in}}%
\pgfusepath{stroke,fill}%
}%
\begin{pgfscope}%
\pgfsys@transformshift{0.800000in}{3.635376in}%
\pgfsys@useobject{currentmarker}{}%
\end{pgfscope}%
\end{pgfscope}%
\begin{pgfscope}%
\pgftext[x=0.403588in,y=3.582754in,left,base]{\fontsize{11.000000}{13.200000}\selectfont \(\displaystyle 1750\)}%
\end{pgfscope}%
\begin{pgfscope}%
\pgfsetbuttcap%
\pgfsetroundjoin%
\definecolor{currentfill}{rgb}{0.000000,0.000000,0.000000}%
\pgfsetfillcolor{currentfill}%
\pgfsetlinewidth{0.803000pt}%
\definecolor{currentstroke}{rgb}{0.000000,0.000000,0.000000}%
\pgfsetstrokecolor{currentstroke}%
\pgfsetdash{}{0pt}%
\pgfsys@defobject{currentmarker}{\pgfqpoint{-0.048611in}{0.000000in}}{\pgfqpoint{0.000000in}{0.000000in}}{%
\pgfpathmoveto{\pgfqpoint{0.000000in}{0.000000in}}%
\pgfpathlineto{\pgfqpoint{-0.048611in}{0.000000in}}%
\pgfusepath{stroke,fill}%
}%
\begin{pgfscope}%
\pgfsys@transformshift{0.800000in}{4.056000in}%
\pgfsys@useobject{currentmarker}{}%
\end{pgfscope}%
\end{pgfscope}%
\begin{pgfscope}%
\pgftext[x=0.403588in,y=4.003378in,left,base]{\fontsize{11.000000}{13.200000}\selectfont \(\displaystyle 2000\)}%
\end{pgfscope}%
\begin{pgfscope}%
\pgftext[x=0.348033in,y=2.376000in,,bottom,rotate=90.000000]{\fontsize{11.000000}{13.200000}\selectfont Latency (in milliseconds)}%
\end{pgfscope}%
\begin{pgfscope}%
\pgfpathrectangle{\pgfqpoint{0.800000in}{0.528000in}}{\pgfqpoint{4.960000in}{3.696000in}}%
\pgfusepath{clip}%
\pgfsetrectcap%
\pgfsetroundjoin%
\pgfsetlinewidth{1.505625pt}%
\definecolor{currentstroke}{rgb}{0.121569,0.466667,0.705882}%
\pgfsetstrokecolor{currentstroke}%
\pgfsetdash{}{0pt}%
\pgfpathmoveto{\pgfqpoint{1.025455in}{1.205673in}}%
\pgfpathlineto{\pgfqpoint{1.040287in}{1.225235in}}%
\pgfpathlineto{\pgfqpoint{1.055120in}{1.585488in}}%
\pgfpathlineto{\pgfqpoint{1.069952in}{1.211887in}}%
\pgfpathlineto{\pgfqpoint{1.084785in}{1.196636in}}%
\pgfpathlineto{\pgfqpoint{1.099617in}{1.167278in}}%
\pgfpathlineto{\pgfqpoint{1.114450in}{1.166567in}}%
\pgfpathlineto{\pgfqpoint{1.129282in}{1.161027in}}%
\pgfpathlineto{\pgfqpoint{1.144115in}{1.123387in}}%
\pgfpathlineto{\pgfqpoint{1.158947in}{1.245510in}}%
\pgfpathlineto{\pgfqpoint{1.173780in}{1.219034in}}%
\pgfpathlineto{\pgfqpoint{1.188612in}{1.277163in}}%
\pgfpathlineto{\pgfqpoint{1.203445in}{1.226173in}}%
\pgfpathlineto{\pgfqpoint{1.218278in}{1.247401in}}%
\pgfpathlineto{\pgfqpoint{1.233110in}{1.203724in}}%
\pgfpathlineto{\pgfqpoint{1.247943in}{1.120436in}}%
\pgfpathlineto{\pgfqpoint{1.262775in}{1.121307in}}%
\pgfpathlineto{\pgfqpoint{1.277608in}{1.220711in}}%
\pgfpathlineto{\pgfqpoint{1.292440in}{1.220452in}}%
\pgfpathlineto{\pgfqpoint{1.307273in}{1.283993in}}%
\pgfpathlineto{\pgfqpoint{1.322105in}{1.170200in}}%
\pgfpathlineto{\pgfqpoint{1.336938in}{1.268457in}}%
\pgfpathlineto{\pgfqpoint{1.351770in}{1.328810in}}%
\pgfpathlineto{\pgfqpoint{1.366603in}{1.198332in}}%
\pgfpathlineto{\pgfqpoint{1.381435in}{1.292317in}}%
\pgfpathlineto{\pgfqpoint{1.396268in}{1.208094in}}%
\pgfpathlineto{\pgfqpoint{1.411100in}{1.177075in}}%
\pgfpathlineto{\pgfqpoint{1.425933in}{1.209201in}}%
\pgfpathlineto{\pgfqpoint{1.440766in}{1.204674in}}%
\pgfpathlineto{\pgfqpoint{1.455598in}{1.168928in}}%
\pgfpathlineto{\pgfqpoint{1.470431in}{1.266855in}}%
\pgfpathlineto{\pgfqpoint{1.485263in}{1.216258in}}%
\pgfpathlineto{\pgfqpoint{1.500096in}{1.341442in}}%
\pgfpathlineto{\pgfqpoint{1.514928in}{1.211057in}}%
\pgfpathlineto{\pgfqpoint{1.529761in}{1.195724in}}%
\pgfpathlineto{\pgfqpoint{1.544593in}{1.170967in}}%
\pgfpathlineto{\pgfqpoint{1.559426in}{1.211975in}}%
\pgfpathlineto{\pgfqpoint{1.574258in}{0.945282in}}%
\pgfpathlineto{\pgfqpoint{1.589091in}{0.960050in}}%
\pgfpathlineto{\pgfqpoint{1.603923in}{1.424967in}}%
\pgfpathlineto{\pgfqpoint{1.618756in}{1.228174in}}%
\pgfpathlineto{\pgfqpoint{1.633589in}{1.304389in}}%
\pgfpathlineto{\pgfqpoint{1.648421in}{1.169410in}}%
\pgfpathlineto{\pgfqpoint{1.663254in}{1.002188in}}%
\pgfpathlineto{\pgfqpoint{1.678086in}{1.126681in}}%
\pgfpathlineto{\pgfqpoint{1.692919in}{1.200333in}}%
\pgfpathlineto{\pgfqpoint{1.707751in}{1.204327in}}%
\pgfpathlineto{\pgfqpoint{1.722584in}{1.226563in}}%
\pgfpathlineto{\pgfqpoint{1.737416in}{1.159129in}}%
\pgfpathlineto{\pgfqpoint{1.752249in}{1.206843in}}%
\pgfpathlineto{\pgfqpoint{1.767081in}{1.273513in}}%
\pgfpathlineto{\pgfqpoint{1.781914in}{1.236812in}}%
\pgfpathlineto{\pgfqpoint{1.796746in}{1.206147in}}%
\pgfpathlineto{\pgfqpoint{1.826411in}{1.274333in}}%
\pgfpathlineto{\pgfqpoint{1.841244in}{1.202166in}}%
\pgfpathlineto{\pgfqpoint{1.856077in}{1.222998in}}%
\pgfpathlineto{\pgfqpoint{1.870909in}{1.132442in}}%
\pgfpathlineto{\pgfqpoint{1.885742in}{1.317155in}}%
\pgfpathlineto{\pgfqpoint{1.900574in}{1.551300in}}%
\pgfpathlineto{\pgfqpoint{1.915407in}{1.324954in}}%
\pgfpathlineto{\pgfqpoint{1.930239in}{1.140953in}}%
\pgfpathlineto{\pgfqpoint{1.959904in}{1.194579in}}%
\pgfpathlineto{\pgfqpoint{1.974737in}{1.090391in}}%
\pgfpathlineto{\pgfqpoint{1.989569in}{1.086843in}}%
\pgfpathlineto{\pgfqpoint{2.004402in}{1.274663in}}%
\pgfpathlineto{\pgfqpoint{2.019234in}{1.259346in}}%
\pgfpathlineto{\pgfqpoint{2.034067in}{1.214106in}}%
\pgfpathlineto{\pgfqpoint{2.048900in}{1.302450in}}%
\pgfpathlineto{\pgfqpoint{2.063732in}{1.234110in}}%
\pgfpathlineto{\pgfqpoint{2.078565in}{1.207390in}}%
\pgfpathlineto{\pgfqpoint{2.093397in}{1.189373in}}%
\pgfpathlineto{\pgfqpoint{2.108230in}{1.216362in}}%
\pgfpathlineto{\pgfqpoint{2.123062in}{1.232815in}}%
\pgfpathlineto{\pgfqpoint{2.137895in}{1.183482in}}%
\pgfpathlineto{\pgfqpoint{2.152727in}{1.172759in}}%
\pgfpathlineto{\pgfqpoint{2.167560in}{1.129620in}}%
\pgfpathlineto{\pgfqpoint{2.182392in}{1.199538in}}%
\pgfpathlineto{\pgfqpoint{2.197225in}{1.162850in}}%
\pgfpathlineto{\pgfqpoint{2.226890in}{1.175545in}}%
\pgfpathlineto{\pgfqpoint{2.241722in}{1.256160in}}%
\pgfpathlineto{\pgfqpoint{2.256555in}{1.248657in}}%
\pgfpathlineto{\pgfqpoint{2.271388in}{1.293631in}}%
\pgfpathlineto{\pgfqpoint{2.286220in}{1.180400in}}%
\pgfpathlineto{\pgfqpoint{2.301053in}{1.235752in}}%
\pgfpathlineto{\pgfqpoint{2.315885in}{1.217196in}}%
\pgfpathlineto{\pgfqpoint{2.330718in}{1.253514in}}%
\pgfpathlineto{\pgfqpoint{2.345550in}{1.180489in}}%
\pgfpathlineto{\pgfqpoint{2.360383in}{1.281447in}}%
\pgfpathlineto{\pgfqpoint{2.375215in}{1.140590in}}%
\pgfpathlineto{\pgfqpoint{2.390048in}{1.156301in}}%
\pgfpathlineto{\pgfqpoint{2.404880in}{1.185909in}}%
\pgfpathlineto{\pgfqpoint{2.419713in}{1.191885in}}%
\pgfpathlineto{\pgfqpoint{2.434545in}{1.142755in}}%
\pgfpathlineto{\pgfqpoint{2.449378in}{1.308243in}}%
\pgfpathlineto{\pgfqpoint{2.464211in}{1.319831in}}%
\pgfpathlineto{\pgfqpoint{2.479043in}{1.252139in}}%
\pgfpathlineto{\pgfqpoint{2.493876in}{1.102951in}}%
\pgfpathlineto{\pgfqpoint{2.671866in}{4.056000in}}%
\pgfpathlineto{\pgfqpoint{2.686699in}{2.193740in}}%
\pgfpathlineto{\pgfqpoint{2.701531in}{1.218804in}}%
\pgfpathlineto{\pgfqpoint{2.716364in}{1.119280in}}%
\pgfpathlineto{\pgfqpoint{2.731196in}{1.131917in}}%
\pgfpathlineto{\pgfqpoint{2.746029in}{1.191625in}}%
\pgfpathlineto{\pgfqpoint{2.760861in}{1.123958in}}%
\pgfpathlineto{\pgfqpoint{2.775694in}{1.212829in}}%
\pgfpathlineto{\pgfqpoint{2.790526in}{1.147098in}}%
\pgfpathlineto{\pgfqpoint{2.805359in}{1.142832in}}%
\pgfpathlineto{\pgfqpoint{2.820191in}{1.247802in}}%
\pgfpathlineto{\pgfqpoint{2.835024in}{1.203227in}}%
\pgfpathlineto{\pgfqpoint{2.849856in}{1.181508in}}%
\pgfpathlineto{\pgfqpoint{2.864689in}{1.145622in}}%
\pgfpathlineto{\pgfqpoint{2.879522in}{1.196373in}}%
\pgfpathlineto{\pgfqpoint{2.909187in}{1.080516in}}%
\pgfpathlineto{\pgfqpoint{2.924019in}{1.292785in}}%
\pgfpathlineto{\pgfqpoint{2.938852in}{1.181455in}}%
\pgfpathlineto{\pgfqpoint{2.953684in}{1.180541in}}%
\pgfpathlineto{\pgfqpoint{2.968517in}{1.183736in}}%
\pgfpathlineto{\pgfqpoint{2.983349in}{1.108363in}}%
\pgfpathlineto{\pgfqpoint{2.998182in}{1.294935in}}%
\pgfpathlineto{\pgfqpoint{3.013014in}{1.142866in}}%
\pgfpathlineto{\pgfqpoint{3.027847in}{1.165174in}}%
\pgfpathlineto{\pgfqpoint{3.042679in}{1.219388in}}%
\pgfpathlineto{\pgfqpoint{3.057512in}{1.128413in}}%
\pgfpathlineto{\pgfqpoint{3.072344in}{1.132650in}}%
\pgfpathlineto{\pgfqpoint{3.087177in}{1.226406in}}%
\pgfpathlineto{\pgfqpoint{3.102010in}{1.129963in}}%
\pgfpathlineto{\pgfqpoint{3.116842in}{1.156977in}}%
\pgfpathlineto{\pgfqpoint{3.131675in}{1.160820in}}%
\pgfpathlineto{\pgfqpoint{3.146507in}{1.131477in}}%
\pgfpathlineto{\pgfqpoint{3.161340in}{1.181797in}}%
\pgfpathlineto{\pgfqpoint{3.176172in}{1.129997in}}%
\pgfpathlineto{\pgfqpoint{3.191005in}{1.188159in}}%
\pgfpathlineto{\pgfqpoint{3.205837in}{1.222374in}}%
\pgfpathlineto{\pgfqpoint{3.220670in}{1.128188in}}%
\pgfpathlineto{\pgfqpoint{3.235502in}{1.141226in}}%
\pgfpathlineto{\pgfqpoint{3.250335in}{1.163499in}}%
\pgfpathlineto{\pgfqpoint{3.265167in}{1.122092in}}%
\pgfpathlineto{\pgfqpoint{3.280000in}{1.224767in}}%
\pgfpathlineto{\pgfqpoint{3.294833in}{1.138813in}}%
\pgfpathlineto{\pgfqpoint{3.309665in}{1.182803in}}%
\pgfpathlineto{\pgfqpoint{3.324498in}{1.158412in}}%
\pgfpathlineto{\pgfqpoint{3.339330in}{1.054211in}}%
\pgfpathlineto{\pgfqpoint{3.354163in}{1.142602in}}%
\pgfpathlineto{\pgfqpoint{3.368995in}{1.134534in}}%
\pgfpathlineto{\pgfqpoint{3.383828in}{1.163561in}}%
\pgfpathlineto{\pgfqpoint{3.398660in}{1.102147in}}%
\pgfpathlineto{\pgfqpoint{3.413493in}{1.149262in}}%
\pgfpathlineto{\pgfqpoint{3.428325in}{1.174463in}}%
\pgfpathlineto{\pgfqpoint{3.443158in}{1.135226in}}%
\pgfpathlineto{\pgfqpoint{3.457990in}{1.069049in}}%
\pgfpathlineto{\pgfqpoint{3.472823in}{1.138088in}}%
\pgfpathlineto{\pgfqpoint{3.487656in}{1.168172in}}%
\pgfpathlineto{\pgfqpoint{3.502488in}{1.149651in}}%
\pgfpathlineto{\pgfqpoint{3.517321in}{1.121098in}}%
\pgfpathlineto{\pgfqpoint{3.532153in}{1.224950in}}%
\pgfpathlineto{\pgfqpoint{3.546986in}{1.106579in}}%
\pgfpathlineto{\pgfqpoint{3.561818in}{1.117414in}}%
\pgfpathlineto{\pgfqpoint{3.576651in}{1.115020in}}%
\pgfpathlineto{\pgfqpoint{3.591483in}{1.121515in}}%
\pgfpathlineto{\pgfqpoint{3.606316in}{1.138375in}}%
\pgfpathlineto{\pgfqpoint{3.621148in}{1.139198in}}%
\pgfpathlineto{\pgfqpoint{3.635981in}{1.146655in}}%
\pgfpathlineto{\pgfqpoint{3.650813in}{1.059095in}}%
\pgfpathlineto{\pgfqpoint{3.665646in}{1.195609in}}%
\pgfpathlineto{\pgfqpoint{3.680478in}{1.136171in}}%
\pgfpathlineto{\pgfqpoint{3.695311in}{0.950764in}}%
\pgfpathlineto{\pgfqpoint{3.710144in}{1.419001in}}%
\pgfpathlineto{\pgfqpoint{3.724976in}{0.988203in}}%
\pgfpathlineto{\pgfqpoint{3.739809in}{1.311383in}}%
\pgfpathlineto{\pgfqpoint{3.754641in}{1.149078in}}%
\pgfpathlineto{\pgfqpoint{3.769474in}{1.169239in}}%
\pgfpathlineto{\pgfqpoint{3.784306in}{1.207953in}}%
\pgfpathlineto{\pgfqpoint{3.799139in}{1.181014in}}%
\pgfpathlineto{\pgfqpoint{3.813971in}{1.198121in}}%
\pgfpathlineto{\pgfqpoint{3.828804in}{1.168619in}}%
\pgfpathlineto{\pgfqpoint{3.843636in}{0.997827in}}%
\pgfpathlineto{\pgfqpoint{3.858469in}{1.346585in}}%
\pgfpathlineto{\pgfqpoint{3.873301in}{1.167787in}}%
\pgfpathlineto{\pgfqpoint{3.888134in}{1.179199in}}%
\pgfpathlineto{\pgfqpoint{3.902967in}{1.170036in}}%
\pgfpathlineto{\pgfqpoint{3.917799in}{1.210343in}}%
\pgfpathlineto{\pgfqpoint{3.932632in}{1.151897in}}%
\pgfpathlineto{\pgfqpoint{3.947464in}{1.210696in}}%
\pgfpathlineto{\pgfqpoint{3.962297in}{1.157574in}}%
\pgfpathlineto{\pgfqpoint{3.977129in}{1.187386in}}%
\pgfpathlineto{\pgfqpoint{3.991962in}{1.739842in}}%
\pgfpathlineto{\pgfqpoint{4.006794in}{1.302051in}}%
\pgfpathlineto{\pgfqpoint{4.021627in}{1.181513in}}%
\pgfpathlineto{\pgfqpoint{4.036459in}{1.292584in}}%
\pgfpathlineto{\pgfqpoint{4.051292in}{1.327679in}}%
\pgfpathlineto{\pgfqpoint{4.066124in}{1.318687in}}%
\pgfpathlineto{\pgfqpoint{4.080957in}{1.144105in}}%
\pgfpathlineto{\pgfqpoint{4.095789in}{1.142504in}}%
\pgfpathlineto{\pgfqpoint{4.110622in}{1.178324in}}%
\pgfpathlineto{\pgfqpoint{4.125455in}{1.170878in}}%
\pgfpathlineto{\pgfqpoint{4.140287in}{1.218655in}}%
\pgfpathlineto{\pgfqpoint{4.155120in}{1.227258in}}%
\pgfpathlineto{\pgfqpoint{4.169952in}{1.249964in}}%
\pgfpathlineto{\pgfqpoint{4.184785in}{1.219943in}}%
\pgfpathlineto{\pgfqpoint{4.199617in}{1.347921in}}%
\pgfpathlineto{\pgfqpoint{4.214450in}{1.220356in}}%
\pgfpathlineto{\pgfqpoint{4.229282in}{1.214351in}}%
\pgfpathlineto{\pgfqpoint{4.244115in}{1.291578in}}%
\pgfpathlineto{\pgfqpoint{4.258947in}{1.170448in}}%
\pgfpathlineto{\pgfqpoint{4.273780in}{1.163360in}}%
\pgfpathlineto{\pgfqpoint{4.288612in}{1.163414in}}%
\pgfpathlineto{\pgfqpoint{4.303445in}{1.155128in}}%
\pgfpathlineto{\pgfqpoint{4.318278in}{1.113975in}}%
\pgfpathlineto{\pgfqpoint{4.333110in}{1.182021in}}%
\pgfpathlineto{\pgfqpoint{4.347943in}{1.181945in}}%
\pgfpathlineto{\pgfqpoint{4.362775in}{1.151253in}}%
\pgfpathlineto{\pgfqpoint{4.377608in}{1.393324in}}%
\pgfpathlineto{\pgfqpoint{4.392440in}{1.201459in}}%
\pgfpathlineto{\pgfqpoint{4.407273in}{1.425506in}}%
\pgfpathlineto{\pgfqpoint{4.422105in}{1.316801in}}%
\pgfpathlineto{\pgfqpoint{4.436938in}{1.666245in}}%
\pgfpathlineto{\pgfqpoint{4.451770in}{1.308568in}}%
\pgfpathlineto{\pgfqpoint{4.466603in}{1.334797in}}%
\pgfpathlineto{\pgfqpoint{4.481435in}{1.182463in}}%
\pgfpathlineto{\pgfqpoint{4.496268in}{1.413765in}}%
\pgfpathlineto{\pgfqpoint{4.511100in}{1.178751in}}%
\pgfpathlineto{\pgfqpoint{4.525933in}{1.192485in}}%
\pgfpathlineto{\pgfqpoint{4.540766in}{1.155128in}}%
\pgfpathlineto{\pgfqpoint{4.555598in}{1.069513in}}%
\pgfpathlineto{\pgfqpoint{4.570431in}{1.176383in}}%
\pgfpathlineto{\pgfqpoint{4.585263in}{1.113967in}}%
\pgfpathlineto{\pgfqpoint{4.600096in}{1.135632in}}%
\pgfpathlineto{\pgfqpoint{4.614928in}{1.354588in}}%
\pgfpathlineto{\pgfqpoint{4.629761in}{1.196906in}}%
\pgfpathlineto{\pgfqpoint{4.644593in}{1.232099in}}%
\pgfpathlineto{\pgfqpoint{4.659426in}{1.234889in}}%
\pgfpathlineto{\pgfqpoint{4.674258in}{1.235648in}}%
\pgfpathlineto{\pgfqpoint{4.689091in}{1.098587in}}%
\pgfpathlineto{\pgfqpoint{4.703923in}{1.182492in}}%
\pgfpathlineto{\pgfqpoint{4.718756in}{1.163015in}}%
\pgfpathlineto{\pgfqpoint{4.733589in}{1.206833in}}%
\pgfpathlineto{\pgfqpoint{4.748421in}{1.263819in}}%
\pgfpathlineto{\pgfqpoint{4.763254in}{1.163337in}}%
\pgfpathlineto{\pgfqpoint{4.778086in}{1.133342in}}%
\pgfpathlineto{\pgfqpoint{4.792919in}{1.182363in}}%
\pgfpathlineto{\pgfqpoint{4.807751in}{1.276154in}}%
\pgfpathlineto{\pgfqpoint{4.822584in}{1.193609in}}%
\pgfpathlineto{\pgfqpoint{4.837416in}{1.345551in}}%
\pgfpathlineto{\pgfqpoint{4.852249in}{1.170007in}}%
\pgfpathlineto{\pgfqpoint{4.867081in}{1.167014in}}%
\pgfpathlineto{\pgfqpoint{4.881914in}{1.185338in}}%
\pgfpathlineto{\pgfqpoint{4.896746in}{1.280669in}}%
\pgfpathlineto{\pgfqpoint{4.911579in}{1.123428in}}%
\pgfpathlineto{\pgfqpoint{4.941244in}{0.955241in}}%
\pgfpathlineto{\pgfqpoint{4.956077in}{1.355480in}}%
\pgfpathlineto{\pgfqpoint{4.970909in}{1.110109in}}%
\pgfpathlineto{\pgfqpoint{4.985742in}{1.014511in}}%
\pgfpathlineto{\pgfqpoint{5.000574in}{1.192717in}}%
\pgfpathlineto{\pgfqpoint{5.015407in}{1.158768in}}%
\pgfpathlineto{\pgfqpoint{5.030239in}{1.180938in}}%
\pgfpathlineto{\pgfqpoint{5.045072in}{1.163223in}}%
\pgfpathlineto{\pgfqpoint{5.059904in}{1.113848in}}%
\pgfpathlineto{\pgfqpoint{5.074737in}{1.161415in}}%
\pgfpathlineto{\pgfqpoint{5.089569in}{1.119422in}}%
\pgfpathlineto{\pgfqpoint{5.104402in}{1.118876in}}%
\pgfpathlineto{\pgfqpoint{5.119234in}{1.058759in}}%
\pgfpathlineto{\pgfqpoint{5.134067in}{1.124255in}}%
\pgfpathlineto{\pgfqpoint{5.148900in}{1.112520in}}%
\pgfpathlineto{\pgfqpoint{5.163732in}{1.063794in}}%
\pgfpathlineto{\pgfqpoint{5.178565in}{1.131419in}}%
\pgfpathlineto{\pgfqpoint{5.193397in}{1.104173in}}%
\pgfpathlineto{\pgfqpoint{5.208230in}{1.139584in}}%
\pgfpathlineto{\pgfqpoint{5.223062in}{1.154705in}}%
\pgfpathlineto{\pgfqpoint{5.237895in}{1.140741in}}%
\pgfpathlineto{\pgfqpoint{5.252727in}{1.176732in}}%
\pgfpathlineto{\pgfqpoint{5.267560in}{1.152015in}}%
\pgfpathlineto{\pgfqpoint{5.282392in}{1.131512in}}%
\pgfpathlineto{\pgfqpoint{5.297225in}{1.160677in}}%
\pgfpathlineto{\pgfqpoint{5.312057in}{1.254366in}}%
\pgfpathlineto{\pgfqpoint{5.326890in}{1.211887in}}%
\pgfpathlineto{\pgfqpoint{5.341722in}{1.239338in}}%
\pgfpathlineto{\pgfqpoint{5.356555in}{1.202191in}}%
\pgfpathlineto{\pgfqpoint{5.371388in}{1.201754in}}%
\pgfpathlineto{\pgfqpoint{5.386220in}{1.182038in}}%
\pgfpathlineto{\pgfqpoint{5.401053in}{1.185655in}}%
\pgfpathlineto{\pgfqpoint{5.415885in}{1.091963in}}%
\pgfpathlineto{\pgfqpoint{5.430718in}{1.228906in}}%
\pgfpathlineto{\pgfqpoint{5.445550in}{1.123875in}}%
\pgfpathlineto{\pgfqpoint{5.460383in}{1.180283in}}%
\pgfpathlineto{\pgfqpoint{5.475215in}{0.968024in}}%
\pgfpathlineto{\pgfqpoint{5.490048in}{0.898463in}}%
\pgfpathlineto{\pgfqpoint{5.504880in}{0.849440in}}%
\pgfpathlineto{\pgfqpoint{5.519713in}{0.763383in}}%
\pgfpathlineto{\pgfqpoint{5.534545in}{0.696000in}}%
\pgfpathlineto{\pgfqpoint{5.534545in}{0.696000in}}%
\pgfusepath{stroke}%
\end{pgfscope}%
\begin{pgfscope}%
\pgfpathrectangle{\pgfqpoint{0.800000in}{0.528000in}}{\pgfqpoint{4.960000in}{3.696000in}}%
\pgfusepath{clip}%
\pgfsetrectcap%
\pgfsetroundjoin%
\pgfsetlinewidth{1.505625pt}%
\definecolor{currentstroke}{rgb}{1.000000,0.498039,0.054902}%
\pgfsetstrokecolor{currentstroke}%
\pgfsetdash{}{0pt}%
\pgfpathmoveto{\pgfqpoint{1.025455in}{1.190835in}}%
\pgfpathlineto{\pgfqpoint{5.534545in}{1.190835in}}%
\pgfpathlineto{\pgfqpoint{5.534545in}{1.190835in}}%
\pgfusepath{stroke}%
\end{pgfscope}%
\begin{pgfscope}%
\pgfsetrectcap%
\pgfsetmiterjoin%
\pgfsetlinewidth{0.803000pt}%
\definecolor{currentstroke}{rgb}{0.000000,0.000000,0.000000}%
\pgfsetstrokecolor{currentstroke}%
\pgfsetdash{}{0pt}%
\pgfpathmoveto{\pgfqpoint{0.800000in}{0.528000in}}%
\pgfpathlineto{\pgfqpoint{0.800000in}{4.224000in}}%
\pgfusepath{stroke}%
\end{pgfscope}%
\begin{pgfscope}%
\pgfsetrectcap%
\pgfsetmiterjoin%
\pgfsetlinewidth{0.803000pt}%
\definecolor{currentstroke}{rgb}{0.000000,0.000000,0.000000}%
\pgfsetstrokecolor{currentstroke}%
\pgfsetdash{}{0pt}%
\pgfpathmoveto{\pgfqpoint{5.760000in}{0.528000in}}%
\pgfpathlineto{\pgfqpoint{5.760000in}{4.224000in}}%
\pgfusepath{stroke}%
\end{pgfscope}%
\begin{pgfscope}%
\pgfsetrectcap%
\pgfsetmiterjoin%
\pgfsetlinewidth{0.803000pt}%
\definecolor{currentstroke}{rgb}{0.000000,0.000000,0.000000}%
\pgfsetstrokecolor{currentstroke}%
\pgfsetdash{}{0pt}%
\pgfpathmoveto{\pgfqpoint{0.800000in}{0.528000in}}%
\pgfpathlineto{\pgfqpoint{5.760000in}{0.528000in}}%
\pgfusepath{stroke}%
\end{pgfscope}%
\begin{pgfscope}%
\pgfsetrectcap%
\pgfsetmiterjoin%
\pgfsetlinewidth{0.803000pt}%
\definecolor{currentstroke}{rgb}{0.000000,0.000000,0.000000}%
\pgfsetstrokecolor{currentstroke}%
\pgfsetdash{}{0pt}%
\pgfpathmoveto{\pgfqpoint{0.800000in}{4.224000in}}%
\pgfpathlineto{\pgfqpoint{5.760000in}{4.224000in}}%
\pgfusepath{stroke}%
\end{pgfscope}%
\begin{pgfscope}%
\pgfsetbuttcap%
\pgfsetmiterjoin%
\definecolor{currentfill}{rgb}{1.000000,1.000000,1.000000}%
\pgfsetfillcolor{currentfill}%
\pgfsetfillopacity{0.800000}%
\pgfsetlinewidth{1.003750pt}%
\definecolor{currentstroke}{rgb}{0.800000,0.800000,0.800000}%
\pgfsetstrokecolor{currentstroke}%
\pgfsetstrokeopacity{0.800000}%
\pgfsetdash{}{0pt}%
\pgfpathmoveto{\pgfqpoint{3.522024in}{3.675698in}}%
\pgfpathlineto{\pgfqpoint{5.653056in}{3.675698in}}%
\pgfpathquadraticcurveto{\pgfqpoint{5.683611in}{3.675698in}}{\pgfqpoint{5.683611in}{3.706254in}}%
\pgfpathlineto{\pgfqpoint{5.683611in}{4.117056in}}%
\pgfpathquadraticcurveto{\pgfqpoint{5.683611in}{4.147611in}}{\pgfqpoint{5.653056in}{4.147611in}}%
\pgfpathlineto{\pgfqpoint{3.522024in}{4.147611in}}%
\pgfpathquadraticcurveto{\pgfqpoint{3.491468in}{4.147611in}}{\pgfqpoint{3.491468in}{4.117056in}}%
\pgfpathlineto{\pgfqpoint{3.491468in}{3.706254in}}%
\pgfpathquadraticcurveto{\pgfqpoint{3.491468in}{3.675698in}}{\pgfqpoint{3.522024in}{3.675698in}}%
\pgfpathclose%
\pgfusepath{stroke,fill}%
\end{pgfscope}%
\begin{pgfscope}%
\pgfsetrectcap%
\pgfsetroundjoin%
\pgfsetlinewidth{1.505625pt}%
\definecolor{currentstroke}{rgb}{0.121569,0.466667,0.705882}%
\pgfsetstrokecolor{currentstroke}%
\pgfsetdash{}{0pt}%
\pgfpathmoveto{\pgfqpoint{3.552579in}{4.033028in}}%
\pgfpathlineto{\pgfqpoint{3.858135in}{4.033028in}}%
\pgfusepath{stroke}%
\end{pgfscope}%
\begin{pgfscope}%
\pgftext[x=3.980357in,y=3.979556in,left,base]{\fontsize{11.000000}{13.200000}\selectfont Latency of writes}%
\end{pgfscope}%
\begin{pgfscope}%
\pgfsetrectcap%
\pgfsetroundjoin%
\pgfsetlinewidth{1.505625pt}%
\definecolor{currentstroke}{rgb}{1.000000,0.498039,0.054902}%
\pgfsetstrokecolor{currentstroke}%
\pgfsetdash{}{0pt}%
\pgfpathmoveto{\pgfqpoint{3.552579in}{3.819988in}}%
\pgfpathlineto{\pgfqpoint{3.858135in}{3.819988in}}%
\pgfusepath{stroke}%
\end{pgfscope}%
\begin{pgfscope}%
\pgftext[x=3.980357in,y=3.766516in,left,base]{\fontsize{11.000000}{13.200000}\selectfont Average latency of writes}%
\end{pgfscope}%
\end{pgfpicture}%
\makeatother%
\endgroup%

    \caption{Latency of write operations for every second of experiment 2 in \prettyref{tab:res-4}}
    \label{fig:lat-plj}
\end{figure}

\begin{figure}
    %% Creator: Matplotlib, PGF backend
%%
%% To include the figure in your LaTeX document, write
%%   \input{<filename>.pgf}
%%
%% Make sure the required packages are loaded in your preamble
%%   \usepackage{pgf}
%%
%% Figures using additional raster images can only be included by \input if
%% they are in the same directory as the main LaTeX file. For loading figures
%% from other directories you can use the `import` package
%%   \usepackage{import}
%% and then include the figures with
%%   \import{<path to file>}{<filename>.pgf}
%%
%% Matplotlib used the following preamble
%%   \usepackage[utf8x]{inputenc}
%%   \usepackage[T1]{fontenc}
%%   \usepackage{lmodern}
%%
\begingroup%
\makeatletter%
\begin{pgfpicture}%
\pgfpathrectangle{\pgfpointorigin}{\pgfqpoint{6.400000in}{4.800000in}}%
\pgfusepath{use as bounding box, clip}%
\begin{pgfscope}%
\pgfsetbuttcap%
\pgfsetmiterjoin%
\definecolor{currentfill}{rgb}{1.000000,1.000000,1.000000}%
\pgfsetfillcolor{currentfill}%
\pgfsetlinewidth{0.000000pt}%
\definecolor{currentstroke}{rgb}{1.000000,1.000000,1.000000}%
\pgfsetstrokecolor{currentstroke}%
\pgfsetdash{}{0pt}%
\pgfpathmoveto{\pgfqpoint{0.000000in}{0.000000in}}%
\pgfpathlineto{\pgfqpoint{6.400000in}{0.000000in}}%
\pgfpathlineto{\pgfqpoint{6.400000in}{4.800000in}}%
\pgfpathlineto{\pgfqpoint{0.000000in}{4.800000in}}%
\pgfpathclose%
\pgfusepath{fill}%
\end{pgfscope}%
\begin{pgfscope}%
\pgfsetbuttcap%
\pgfsetmiterjoin%
\definecolor{currentfill}{rgb}{1.000000,1.000000,1.000000}%
\pgfsetfillcolor{currentfill}%
\pgfsetlinewidth{0.000000pt}%
\definecolor{currentstroke}{rgb}{0.000000,0.000000,0.000000}%
\pgfsetstrokecolor{currentstroke}%
\pgfsetstrokeopacity{0.000000}%
\pgfsetdash{}{0pt}%
\pgfpathmoveto{\pgfqpoint{0.800000in}{0.528000in}}%
\pgfpathlineto{\pgfqpoint{5.760000in}{0.528000in}}%
\pgfpathlineto{\pgfqpoint{5.760000in}{4.224000in}}%
\pgfpathlineto{\pgfqpoint{0.800000in}{4.224000in}}%
\pgfpathclose%
\pgfusepath{fill}%
\end{pgfscope}%
\begin{pgfscope}%
\pgfsetbuttcap%
\pgfsetroundjoin%
\definecolor{currentfill}{rgb}{0.000000,0.000000,0.000000}%
\pgfsetfillcolor{currentfill}%
\pgfsetlinewidth{0.803000pt}%
\definecolor{currentstroke}{rgb}{0.000000,0.000000,0.000000}%
\pgfsetstrokecolor{currentstroke}%
\pgfsetdash{}{0pt}%
\pgfsys@defobject{currentmarker}{\pgfqpoint{0.000000in}{-0.048611in}}{\pgfqpoint{0.000000in}{0.000000in}}{%
\pgfpathmoveto{\pgfqpoint{0.000000in}{0.000000in}}%
\pgfpathlineto{\pgfqpoint{0.000000in}{-0.048611in}}%
\pgfusepath{stroke,fill}%
}%
\begin{pgfscope}%
\pgfsys@transformshift{1.025455in}{0.528000in}%
\pgfsys@useobject{currentmarker}{}%
\end{pgfscope}%
\end{pgfscope}%
\begin{pgfscope}%
\pgftext[x=1.025455in,y=0.430778in,,top]{\fontsize{11.000000}{13.200000}\selectfont \(\displaystyle 0\)}%
\end{pgfscope}%
\begin{pgfscope}%
\pgfsetbuttcap%
\pgfsetroundjoin%
\definecolor{currentfill}{rgb}{0.000000,0.000000,0.000000}%
\pgfsetfillcolor{currentfill}%
\pgfsetlinewidth{0.803000pt}%
\definecolor{currentstroke}{rgb}{0.000000,0.000000,0.000000}%
\pgfsetstrokecolor{currentstroke}%
\pgfsetdash{}{0pt}%
\pgfsys@defobject{currentmarker}{\pgfqpoint{0.000000in}{-0.048611in}}{\pgfqpoint{0.000000in}{0.000000in}}{%
\pgfpathmoveto{\pgfqpoint{0.000000in}{0.000000in}}%
\pgfpathlineto{\pgfqpoint{0.000000in}{-0.048611in}}%
\pgfusepath{stroke,fill}%
}%
\begin{pgfscope}%
\pgfsys@transformshift{1.769529in}{0.528000in}%
\pgfsys@useobject{currentmarker}{}%
\end{pgfscope}%
\end{pgfscope}%
\begin{pgfscope}%
\pgftext[x=1.769529in,y=0.430778in,,top]{\fontsize{11.000000}{13.200000}\selectfont \(\displaystyle 50\)}%
\end{pgfscope}%
\begin{pgfscope}%
\pgfsetbuttcap%
\pgfsetroundjoin%
\definecolor{currentfill}{rgb}{0.000000,0.000000,0.000000}%
\pgfsetfillcolor{currentfill}%
\pgfsetlinewidth{0.803000pt}%
\definecolor{currentstroke}{rgb}{0.000000,0.000000,0.000000}%
\pgfsetstrokecolor{currentstroke}%
\pgfsetdash{}{0pt}%
\pgfsys@defobject{currentmarker}{\pgfqpoint{0.000000in}{-0.048611in}}{\pgfqpoint{0.000000in}{0.000000in}}{%
\pgfpathmoveto{\pgfqpoint{0.000000in}{0.000000in}}%
\pgfpathlineto{\pgfqpoint{0.000000in}{-0.048611in}}%
\pgfusepath{stroke,fill}%
}%
\begin{pgfscope}%
\pgfsys@transformshift{2.513603in}{0.528000in}%
\pgfsys@useobject{currentmarker}{}%
\end{pgfscope}%
\end{pgfscope}%
\begin{pgfscope}%
\pgftext[x=2.513603in,y=0.430778in,,top]{\fontsize{11.000000}{13.200000}\selectfont \(\displaystyle 100\)}%
\end{pgfscope}%
\begin{pgfscope}%
\pgfsetbuttcap%
\pgfsetroundjoin%
\definecolor{currentfill}{rgb}{0.000000,0.000000,0.000000}%
\pgfsetfillcolor{currentfill}%
\pgfsetlinewidth{0.803000pt}%
\definecolor{currentstroke}{rgb}{0.000000,0.000000,0.000000}%
\pgfsetstrokecolor{currentstroke}%
\pgfsetdash{}{0pt}%
\pgfsys@defobject{currentmarker}{\pgfqpoint{0.000000in}{-0.048611in}}{\pgfqpoint{0.000000in}{0.000000in}}{%
\pgfpathmoveto{\pgfqpoint{0.000000in}{0.000000in}}%
\pgfpathlineto{\pgfqpoint{0.000000in}{-0.048611in}}%
\pgfusepath{stroke,fill}%
}%
\begin{pgfscope}%
\pgfsys@transformshift{3.257678in}{0.528000in}%
\pgfsys@useobject{currentmarker}{}%
\end{pgfscope}%
\end{pgfscope}%
\begin{pgfscope}%
\pgftext[x=3.257678in,y=0.430778in,,top]{\fontsize{11.000000}{13.200000}\selectfont \(\displaystyle 150\)}%
\end{pgfscope}%
\begin{pgfscope}%
\pgfsetbuttcap%
\pgfsetroundjoin%
\definecolor{currentfill}{rgb}{0.000000,0.000000,0.000000}%
\pgfsetfillcolor{currentfill}%
\pgfsetlinewidth{0.803000pt}%
\definecolor{currentstroke}{rgb}{0.000000,0.000000,0.000000}%
\pgfsetstrokecolor{currentstroke}%
\pgfsetdash{}{0pt}%
\pgfsys@defobject{currentmarker}{\pgfqpoint{0.000000in}{-0.048611in}}{\pgfqpoint{0.000000in}{0.000000in}}{%
\pgfpathmoveto{\pgfqpoint{0.000000in}{0.000000in}}%
\pgfpathlineto{\pgfqpoint{0.000000in}{-0.048611in}}%
\pgfusepath{stroke,fill}%
}%
\begin{pgfscope}%
\pgfsys@transformshift{4.001752in}{0.528000in}%
\pgfsys@useobject{currentmarker}{}%
\end{pgfscope}%
\end{pgfscope}%
\begin{pgfscope}%
\pgftext[x=4.001752in,y=0.430778in,,top]{\fontsize{11.000000}{13.200000}\selectfont \(\displaystyle 200\)}%
\end{pgfscope}%
\begin{pgfscope}%
\pgfsetbuttcap%
\pgfsetroundjoin%
\definecolor{currentfill}{rgb}{0.000000,0.000000,0.000000}%
\pgfsetfillcolor{currentfill}%
\pgfsetlinewidth{0.803000pt}%
\definecolor{currentstroke}{rgb}{0.000000,0.000000,0.000000}%
\pgfsetstrokecolor{currentstroke}%
\pgfsetdash{}{0pt}%
\pgfsys@defobject{currentmarker}{\pgfqpoint{0.000000in}{-0.048611in}}{\pgfqpoint{0.000000in}{0.000000in}}{%
\pgfpathmoveto{\pgfqpoint{0.000000in}{0.000000in}}%
\pgfpathlineto{\pgfqpoint{0.000000in}{-0.048611in}}%
\pgfusepath{stroke,fill}%
}%
\begin{pgfscope}%
\pgfsys@transformshift{4.745827in}{0.528000in}%
\pgfsys@useobject{currentmarker}{}%
\end{pgfscope}%
\end{pgfscope}%
\begin{pgfscope}%
\pgftext[x=4.745827in,y=0.430778in,,top]{\fontsize{11.000000}{13.200000}\selectfont \(\displaystyle 250\)}%
\end{pgfscope}%
\begin{pgfscope}%
\pgfsetbuttcap%
\pgfsetroundjoin%
\definecolor{currentfill}{rgb}{0.000000,0.000000,0.000000}%
\pgfsetfillcolor{currentfill}%
\pgfsetlinewidth{0.803000pt}%
\definecolor{currentstroke}{rgb}{0.000000,0.000000,0.000000}%
\pgfsetstrokecolor{currentstroke}%
\pgfsetdash{}{0pt}%
\pgfsys@defobject{currentmarker}{\pgfqpoint{0.000000in}{-0.048611in}}{\pgfqpoint{0.000000in}{0.000000in}}{%
\pgfpathmoveto{\pgfqpoint{0.000000in}{0.000000in}}%
\pgfpathlineto{\pgfqpoint{0.000000in}{-0.048611in}}%
\pgfusepath{stroke,fill}%
}%
\begin{pgfscope}%
\pgfsys@transformshift{5.489901in}{0.528000in}%
\pgfsys@useobject{currentmarker}{}%
\end{pgfscope}%
\end{pgfscope}%
\begin{pgfscope}%
\pgftext[x=5.489901in,y=0.430778in,,top]{\fontsize{11.000000}{13.200000}\selectfont \(\displaystyle 300\)}%
\end{pgfscope}%
\begin{pgfscope}%
\pgftext[x=3.280000in,y=0.240271in,,top]{\fontsize{11.000000}{13.200000}\selectfont Time of experiment (in seconds)}%
\end{pgfscope}%
\begin{pgfscope}%
\pgfsetbuttcap%
\pgfsetroundjoin%
\definecolor{currentfill}{rgb}{0.000000,0.000000,0.000000}%
\pgfsetfillcolor{currentfill}%
\pgfsetlinewidth{0.803000pt}%
\definecolor{currentstroke}{rgb}{0.000000,0.000000,0.000000}%
\pgfsetstrokecolor{currentstroke}%
\pgfsetdash{}{0pt}%
\pgfsys@defobject{currentmarker}{\pgfqpoint{-0.048611in}{0.000000in}}{\pgfqpoint{0.000000in}{0.000000in}}{%
\pgfpathmoveto{\pgfqpoint{0.000000in}{0.000000in}}%
\pgfpathlineto{\pgfqpoint{-0.048611in}{0.000000in}}%
\pgfusepath{stroke,fill}%
}%
\begin{pgfscope}%
\pgfsys@transformshift{0.800000in}{0.696000in}%
\pgfsys@useobject{currentmarker}{}%
\end{pgfscope}%
\end{pgfscope}%
\begin{pgfscope}%
\pgftext[x=0.627981in,y=0.643378in,left,base]{\fontsize{11.000000}{13.200000}\selectfont \(\displaystyle 0\)}%
\end{pgfscope}%
\begin{pgfscope}%
\pgfsetbuttcap%
\pgfsetroundjoin%
\definecolor{currentfill}{rgb}{0.000000,0.000000,0.000000}%
\pgfsetfillcolor{currentfill}%
\pgfsetlinewidth{0.803000pt}%
\definecolor{currentstroke}{rgb}{0.000000,0.000000,0.000000}%
\pgfsetstrokecolor{currentstroke}%
\pgfsetdash{}{0pt}%
\pgfsys@defobject{currentmarker}{\pgfqpoint{-0.048611in}{0.000000in}}{\pgfqpoint{0.000000in}{0.000000in}}{%
\pgfpathmoveto{\pgfqpoint{0.000000in}{0.000000in}}%
\pgfpathlineto{\pgfqpoint{-0.048611in}{0.000000in}}%
\pgfusepath{stroke,fill}%
}%
\begin{pgfscope}%
\pgfsys@transformshift{0.800000in}{1.116000in}%
\pgfsys@useobject{currentmarker}{}%
\end{pgfscope}%
\end{pgfscope}%
\begin{pgfscope}%
\pgftext[x=0.478386in,y=1.063378in,left,base]{\fontsize{11.000000}{13.200000}\selectfont \(\displaystyle 250\)}%
\end{pgfscope}%
\begin{pgfscope}%
\pgfsetbuttcap%
\pgfsetroundjoin%
\definecolor{currentfill}{rgb}{0.000000,0.000000,0.000000}%
\pgfsetfillcolor{currentfill}%
\pgfsetlinewidth{0.803000pt}%
\definecolor{currentstroke}{rgb}{0.000000,0.000000,0.000000}%
\pgfsetstrokecolor{currentstroke}%
\pgfsetdash{}{0pt}%
\pgfsys@defobject{currentmarker}{\pgfqpoint{-0.048611in}{0.000000in}}{\pgfqpoint{0.000000in}{0.000000in}}{%
\pgfpathmoveto{\pgfqpoint{0.000000in}{0.000000in}}%
\pgfpathlineto{\pgfqpoint{-0.048611in}{0.000000in}}%
\pgfusepath{stroke,fill}%
}%
\begin{pgfscope}%
\pgfsys@transformshift{0.800000in}{1.536000in}%
\pgfsys@useobject{currentmarker}{}%
\end{pgfscope}%
\end{pgfscope}%
\begin{pgfscope}%
\pgftext[x=0.478386in,y=1.483378in,left,base]{\fontsize{11.000000}{13.200000}\selectfont \(\displaystyle 500\)}%
\end{pgfscope}%
\begin{pgfscope}%
\pgfsetbuttcap%
\pgfsetroundjoin%
\definecolor{currentfill}{rgb}{0.000000,0.000000,0.000000}%
\pgfsetfillcolor{currentfill}%
\pgfsetlinewidth{0.803000pt}%
\definecolor{currentstroke}{rgb}{0.000000,0.000000,0.000000}%
\pgfsetstrokecolor{currentstroke}%
\pgfsetdash{}{0pt}%
\pgfsys@defobject{currentmarker}{\pgfqpoint{-0.048611in}{0.000000in}}{\pgfqpoint{0.000000in}{0.000000in}}{%
\pgfpathmoveto{\pgfqpoint{0.000000in}{0.000000in}}%
\pgfpathlineto{\pgfqpoint{-0.048611in}{0.000000in}}%
\pgfusepath{stroke,fill}%
}%
\begin{pgfscope}%
\pgfsys@transformshift{0.800000in}{1.956000in}%
\pgfsys@useobject{currentmarker}{}%
\end{pgfscope}%
\end{pgfscope}%
\begin{pgfscope}%
\pgftext[x=0.478386in,y=1.903378in,left,base]{\fontsize{11.000000}{13.200000}\selectfont \(\displaystyle 750\)}%
\end{pgfscope}%
\begin{pgfscope}%
\pgfsetbuttcap%
\pgfsetroundjoin%
\definecolor{currentfill}{rgb}{0.000000,0.000000,0.000000}%
\pgfsetfillcolor{currentfill}%
\pgfsetlinewidth{0.803000pt}%
\definecolor{currentstroke}{rgb}{0.000000,0.000000,0.000000}%
\pgfsetstrokecolor{currentstroke}%
\pgfsetdash{}{0pt}%
\pgfsys@defobject{currentmarker}{\pgfqpoint{-0.048611in}{0.000000in}}{\pgfqpoint{0.000000in}{0.000000in}}{%
\pgfpathmoveto{\pgfqpoint{0.000000in}{0.000000in}}%
\pgfpathlineto{\pgfqpoint{-0.048611in}{0.000000in}}%
\pgfusepath{stroke,fill}%
}%
\begin{pgfscope}%
\pgfsys@transformshift{0.800000in}{2.376000in}%
\pgfsys@useobject{currentmarker}{}%
\end{pgfscope}%
\end{pgfscope}%
\begin{pgfscope}%
\pgftext[x=0.403588in,y=2.323378in,left,base]{\fontsize{11.000000}{13.200000}\selectfont \(\displaystyle 1000\)}%
\end{pgfscope}%
\begin{pgfscope}%
\pgfsetbuttcap%
\pgfsetroundjoin%
\definecolor{currentfill}{rgb}{0.000000,0.000000,0.000000}%
\pgfsetfillcolor{currentfill}%
\pgfsetlinewidth{0.803000pt}%
\definecolor{currentstroke}{rgb}{0.000000,0.000000,0.000000}%
\pgfsetstrokecolor{currentstroke}%
\pgfsetdash{}{0pt}%
\pgfsys@defobject{currentmarker}{\pgfqpoint{-0.048611in}{0.000000in}}{\pgfqpoint{0.000000in}{0.000000in}}{%
\pgfpathmoveto{\pgfqpoint{0.000000in}{0.000000in}}%
\pgfpathlineto{\pgfqpoint{-0.048611in}{0.000000in}}%
\pgfusepath{stroke,fill}%
}%
\begin{pgfscope}%
\pgfsys@transformshift{0.800000in}{2.796000in}%
\pgfsys@useobject{currentmarker}{}%
\end{pgfscope}%
\end{pgfscope}%
\begin{pgfscope}%
\pgftext[x=0.403588in,y=2.743378in,left,base]{\fontsize{11.000000}{13.200000}\selectfont \(\displaystyle 1250\)}%
\end{pgfscope}%
\begin{pgfscope}%
\pgfsetbuttcap%
\pgfsetroundjoin%
\definecolor{currentfill}{rgb}{0.000000,0.000000,0.000000}%
\pgfsetfillcolor{currentfill}%
\pgfsetlinewidth{0.803000pt}%
\definecolor{currentstroke}{rgb}{0.000000,0.000000,0.000000}%
\pgfsetstrokecolor{currentstroke}%
\pgfsetdash{}{0pt}%
\pgfsys@defobject{currentmarker}{\pgfqpoint{-0.048611in}{0.000000in}}{\pgfqpoint{0.000000in}{0.000000in}}{%
\pgfpathmoveto{\pgfqpoint{0.000000in}{0.000000in}}%
\pgfpathlineto{\pgfqpoint{-0.048611in}{0.000000in}}%
\pgfusepath{stroke,fill}%
}%
\begin{pgfscope}%
\pgfsys@transformshift{0.800000in}{3.216000in}%
\pgfsys@useobject{currentmarker}{}%
\end{pgfscope}%
\end{pgfscope}%
\begin{pgfscope}%
\pgftext[x=0.403588in,y=3.163378in,left,base]{\fontsize{11.000000}{13.200000}\selectfont \(\displaystyle 1500\)}%
\end{pgfscope}%
\begin{pgfscope}%
\pgfsetbuttcap%
\pgfsetroundjoin%
\definecolor{currentfill}{rgb}{0.000000,0.000000,0.000000}%
\pgfsetfillcolor{currentfill}%
\pgfsetlinewidth{0.803000pt}%
\definecolor{currentstroke}{rgb}{0.000000,0.000000,0.000000}%
\pgfsetstrokecolor{currentstroke}%
\pgfsetdash{}{0pt}%
\pgfsys@defobject{currentmarker}{\pgfqpoint{-0.048611in}{0.000000in}}{\pgfqpoint{0.000000in}{0.000000in}}{%
\pgfpathmoveto{\pgfqpoint{0.000000in}{0.000000in}}%
\pgfpathlineto{\pgfqpoint{-0.048611in}{0.000000in}}%
\pgfusepath{stroke,fill}%
}%
\begin{pgfscope}%
\pgfsys@transformshift{0.800000in}{3.636000in}%
\pgfsys@useobject{currentmarker}{}%
\end{pgfscope}%
\end{pgfscope}%
\begin{pgfscope}%
\pgftext[x=0.403588in,y=3.583378in,left,base]{\fontsize{11.000000}{13.200000}\selectfont \(\displaystyle 1750\)}%
\end{pgfscope}%
\begin{pgfscope}%
\pgfsetbuttcap%
\pgfsetroundjoin%
\definecolor{currentfill}{rgb}{0.000000,0.000000,0.000000}%
\pgfsetfillcolor{currentfill}%
\pgfsetlinewidth{0.803000pt}%
\definecolor{currentstroke}{rgb}{0.000000,0.000000,0.000000}%
\pgfsetstrokecolor{currentstroke}%
\pgfsetdash{}{0pt}%
\pgfsys@defobject{currentmarker}{\pgfqpoint{-0.048611in}{0.000000in}}{\pgfqpoint{0.000000in}{0.000000in}}{%
\pgfpathmoveto{\pgfqpoint{0.000000in}{0.000000in}}%
\pgfpathlineto{\pgfqpoint{-0.048611in}{0.000000in}}%
\pgfusepath{stroke,fill}%
}%
\begin{pgfscope}%
\pgfsys@transformshift{0.800000in}{4.056000in}%
\pgfsys@useobject{currentmarker}{}%
\end{pgfscope}%
\end{pgfscope}%
\begin{pgfscope}%
\pgftext[x=0.403588in,y=4.003378in,left,base]{\fontsize{11.000000}{13.200000}\selectfont \(\displaystyle 2000\)}%
\end{pgfscope}%
\begin{pgfscope}%
\pgftext[x=0.348033in,y=2.376000in,,bottom,rotate=90.000000]{\fontsize{11.000000}{13.200000}\selectfont Latency (in milliseconds)}%
\end{pgfscope}%
\begin{pgfscope}%
\pgfpathrectangle{\pgfqpoint{0.800000in}{0.528000in}}{\pgfqpoint{4.960000in}{3.696000in}}%
\pgfusepath{clip}%
\pgfsetrectcap%
\pgfsetroundjoin%
\pgfsetlinewidth{1.505625pt}%
\definecolor{currentstroke}{rgb}{0.121569,0.466667,0.705882}%
\pgfsetstrokecolor{currentstroke}%
\pgfsetdash{}{0pt}%
\pgfpathmoveto{\pgfqpoint{1.025455in}{1.275597in}}%
\pgfpathlineto{\pgfqpoint{1.040336in}{1.653087in}}%
\pgfpathlineto{\pgfqpoint{1.055218in}{1.261718in}}%
\pgfpathlineto{\pgfqpoint{1.070099in}{1.286845in}}%
\pgfpathlineto{\pgfqpoint{1.084980in}{1.211268in}}%
\pgfpathlineto{\pgfqpoint{1.099862in}{1.198870in}}%
\pgfpathlineto{\pgfqpoint{1.129625in}{1.226609in}}%
\pgfpathlineto{\pgfqpoint{1.144506in}{1.233717in}}%
\pgfpathlineto{\pgfqpoint{1.159388in}{1.222864in}}%
\pgfpathlineto{\pgfqpoint{1.174269in}{1.252776in}}%
\pgfpathlineto{\pgfqpoint{1.189151in}{1.179905in}}%
\pgfpathlineto{\pgfqpoint{1.204032in}{1.219125in}}%
\pgfpathlineto{\pgfqpoint{1.218914in}{1.312149in}}%
\pgfpathlineto{\pgfqpoint{1.233795in}{1.356083in}}%
\pgfpathlineto{\pgfqpoint{1.248677in}{1.209519in}}%
\pgfpathlineto{\pgfqpoint{1.263558in}{1.324156in}}%
\pgfpathlineto{\pgfqpoint{1.278440in}{1.282176in}}%
\pgfpathlineto{\pgfqpoint{1.293321in}{1.187092in}}%
\pgfpathlineto{\pgfqpoint{1.308203in}{1.309164in}}%
\pgfpathlineto{\pgfqpoint{1.323084in}{1.249649in}}%
\pgfpathlineto{\pgfqpoint{1.337966in}{1.290080in}}%
\pgfpathlineto{\pgfqpoint{1.352847in}{1.465924in}}%
\pgfpathlineto{\pgfqpoint{1.367729in}{1.397062in}}%
\pgfpathlineto{\pgfqpoint{1.382610in}{1.576995in}}%
\pgfpathlineto{\pgfqpoint{1.397492in}{3.326287in}}%
\pgfpathlineto{\pgfqpoint{1.412373in}{3.393828in}}%
\pgfpathlineto{\pgfqpoint{1.427255in}{4.056000in}}%
\pgfpathlineto{\pgfqpoint{1.442136in}{1.632971in}}%
\pgfpathlineto{\pgfqpoint{1.457018in}{1.531782in}}%
\pgfpathlineto{\pgfqpoint{1.471899in}{1.477478in}}%
\pgfpathlineto{\pgfqpoint{1.486781in}{1.321890in}}%
\pgfpathlineto{\pgfqpoint{1.501662in}{1.453720in}}%
\pgfpathlineto{\pgfqpoint{1.516544in}{1.258059in}}%
\pgfpathlineto{\pgfqpoint{1.531425in}{1.218129in}}%
\pgfpathlineto{\pgfqpoint{1.546307in}{1.307819in}}%
\pgfpathlineto{\pgfqpoint{1.561188in}{1.217627in}}%
\pgfpathlineto{\pgfqpoint{1.576070in}{1.215310in}}%
\pgfpathlineto{\pgfqpoint{1.590951in}{1.193993in}}%
\pgfpathlineto{\pgfqpoint{1.605833in}{1.353544in}}%
\pgfpathlineto{\pgfqpoint{1.620714in}{1.368912in}}%
\pgfpathlineto{\pgfqpoint{1.635596in}{1.360244in}}%
\pgfpathlineto{\pgfqpoint{1.650477in}{1.333806in}}%
\pgfpathlineto{\pgfqpoint{1.665359in}{1.413707in}}%
\pgfpathlineto{\pgfqpoint{1.680240in}{1.550112in}}%
\pgfpathlineto{\pgfqpoint{1.710003in}{1.364546in}}%
\pgfpathlineto{\pgfqpoint{1.724884in}{1.317662in}}%
\pgfpathlineto{\pgfqpoint{1.739766in}{1.420594in}}%
\pgfpathlineto{\pgfqpoint{1.754647in}{1.189701in}}%
\pgfpathlineto{\pgfqpoint{1.769529in}{1.234756in}}%
\pgfpathlineto{\pgfqpoint{1.784410in}{1.210522in}}%
\pgfpathlineto{\pgfqpoint{1.799292in}{1.231210in}}%
\pgfpathlineto{\pgfqpoint{1.814173in}{1.461494in}}%
\pgfpathlineto{\pgfqpoint{1.829055in}{1.261138in}}%
\pgfpathlineto{\pgfqpoint{1.843936in}{1.249554in}}%
\pgfpathlineto{\pgfqpoint{1.858818in}{1.184403in}}%
\pgfpathlineto{\pgfqpoint{1.873699in}{1.258943in}}%
\pgfpathlineto{\pgfqpoint{1.888581in}{1.282539in}}%
\pgfpathlineto{\pgfqpoint{1.903462in}{1.334347in}}%
\pgfpathlineto{\pgfqpoint{1.918344in}{1.379011in}}%
\pgfpathlineto{\pgfqpoint{1.933225in}{1.416337in}}%
\pgfpathlineto{\pgfqpoint{1.948107in}{1.431155in}}%
\pgfpathlineto{\pgfqpoint{1.962988in}{1.401547in}}%
\pgfpathlineto{\pgfqpoint{1.977870in}{1.310064in}}%
\pgfpathlineto{\pgfqpoint{1.992751in}{1.206448in}}%
\pgfpathlineto{\pgfqpoint{2.007633in}{1.248238in}}%
\pgfpathlineto{\pgfqpoint{2.022514in}{1.321743in}}%
\pgfpathlineto{\pgfqpoint{2.037396in}{1.264212in}}%
\pgfpathlineto{\pgfqpoint{2.052277in}{1.316039in}}%
\pgfpathlineto{\pgfqpoint{2.067159in}{1.263071in}}%
\pgfpathlineto{\pgfqpoint{2.096922in}{1.335711in}}%
\pgfpathlineto{\pgfqpoint{2.111803in}{1.293831in}}%
\pgfpathlineto{\pgfqpoint{2.126685in}{1.176313in}}%
\pgfpathlineto{\pgfqpoint{2.141566in}{1.203287in}}%
\pgfpathlineto{\pgfqpoint{2.156448in}{1.383440in}}%
\pgfpathlineto{\pgfqpoint{2.171329in}{1.364090in}}%
\pgfpathlineto{\pgfqpoint{2.186211in}{1.389547in}}%
\pgfpathlineto{\pgfqpoint{2.201092in}{1.275684in}}%
\pgfpathlineto{\pgfqpoint{2.215974in}{1.364236in}}%
\pgfpathlineto{\pgfqpoint{2.230855in}{1.275545in}}%
\pgfpathlineto{\pgfqpoint{2.245737in}{1.178812in}}%
\pgfpathlineto{\pgfqpoint{2.260618in}{1.209401in}}%
\pgfpathlineto{\pgfqpoint{2.275500in}{1.203763in}}%
\pgfpathlineto{\pgfqpoint{2.290381in}{1.192003in}}%
\pgfpathlineto{\pgfqpoint{2.305263in}{1.235040in}}%
\pgfpathlineto{\pgfqpoint{2.320144in}{1.200514in}}%
\pgfpathlineto{\pgfqpoint{2.335026in}{1.225331in}}%
\pgfpathlineto{\pgfqpoint{2.349907in}{1.263698in}}%
\pgfpathlineto{\pgfqpoint{2.364788in}{1.211771in}}%
\pgfpathlineto{\pgfqpoint{2.379670in}{1.408205in}}%
\pgfpathlineto{\pgfqpoint{2.394551in}{1.309452in}}%
\pgfpathlineto{\pgfqpoint{2.439196in}{1.416842in}}%
\pgfpathlineto{\pgfqpoint{2.454077in}{1.405470in}}%
\pgfpathlineto{\pgfqpoint{2.468959in}{1.297736in}}%
\pgfpathlineto{\pgfqpoint{2.483840in}{1.369279in}}%
\pgfpathlineto{\pgfqpoint{2.498722in}{1.455086in}}%
\pgfpathlineto{\pgfqpoint{2.677300in}{0.696000in}}%
\pgfpathlineto{\pgfqpoint{2.692181in}{4.056000in}}%
\pgfpathlineto{\pgfqpoint{2.707063in}{1.264689in}}%
\pgfpathlineto{\pgfqpoint{2.721944in}{1.177520in}}%
\pgfpathlineto{\pgfqpoint{2.736826in}{1.206166in}}%
\pgfpathlineto{\pgfqpoint{2.751707in}{1.238758in}}%
\pgfpathlineto{\pgfqpoint{2.766589in}{1.243332in}}%
\pgfpathlineto{\pgfqpoint{2.781470in}{1.317637in}}%
\pgfpathlineto{\pgfqpoint{2.796352in}{1.262343in}}%
\pgfpathlineto{\pgfqpoint{2.811233in}{1.280265in}}%
\pgfpathlineto{\pgfqpoint{2.826115in}{1.260713in}}%
\pgfpathlineto{\pgfqpoint{2.840996in}{1.306636in}}%
\pgfpathlineto{\pgfqpoint{2.855878in}{1.308616in}}%
\pgfpathlineto{\pgfqpoint{2.870759in}{1.252664in}}%
\pgfpathlineto{\pgfqpoint{2.885641in}{1.181686in}}%
\pgfpathlineto{\pgfqpoint{2.900522in}{1.250090in}}%
\pgfpathlineto{\pgfqpoint{2.915404in}{1.195109in}}%
\pgfpathlineto{\pgfqpoint{2.930285in}{1.276919in}}%
\pgfpathlineto{\pgfqpoint{2.945167in}{1.218889in}}%
\pgfpathlineto{\pgfqpoint{2.960048in}{1.220785in}}%
\pgfpathlineto{\pgfqpoint{2.974929in}{1.200666in}}%
\pgfpathlineto{\pgfqpoint{2.989811in}{1.251635in}}%
\pgfpathlineto{\pgfqpoint{3.004692in}{1.186912in}}%
\pgfpathlineto{\pgfqpoint{3.019574in}{1.159778in}}%
\pgfpathlineto{\pgfqpoint{3.034455in}{1.234322in}}%
\pgfpathlineto{\pgfqpoint{3.049337in}{1.259989in}}%
\pgfpathlineto{\pgfqpoint{3.064218in}{1.291952in}}%
\pgfpathlineto{\pgfqpoint{3.079100in}{1.229887in}}%
\pgfpathlineto{\pgfqpoint{3.093981in}{1.442229in}}%
\pgfpathlineto{\pgfqpoint{3.108863in}{1.543094in}}%
\pgfpathlineto{\pgfqpoint{3.123744in}{1.278545in}}%
\pgfpathlineto{\pgfqpoint{3.138626in}{1.256044in}}%
\pgfpathlineto{\pgfqpoint{3.153507in}{1.350053in}}%
\pgfpathlineto{\pgfqpoint{3.168389in}{1.299859in}}%
\pgfpathlineto{\pgfqpoint{3.183270in}{1.295251in}}%
\pgfpathlineto{\pgfqpoint{3.198152in}{1.321732in}}%
\pgfpathlineto{\pgfqpoint{3.213033in}{1.303359in}}%
\pgfpathlineto{\pgfqpoint{3.227915in}{1.360237in}}%
\pgfpathlineto{\pgfqpoint{3.242796in}{1.216471in}}%
\pgfpathlineto{\pgfqpoint{3.257678in}{1.194632in}}%
\pgfpathlineto{\pgfqpoint{3.272559in}{1.155247in}}%
\pgfpathlineto{\pgfqpoint{3.287441in}{1.139258in}}%
\pgfpathlineto{\pgfqpoint{3.302322in}{1.169934in}}%
\pgfpathlineto{\pgfqpoint{3.317204in}{1.210308in}}%
\pgfpathlineto{\pgfqpoint{3.332085in}{1.238792in}}%
\pgfpathlineto{\pgfqpoint{3.346967in}{1.255571in}}%
\pgfpathlineto{\pgfqpoint{3.361848in}{1.251046in}}%
\pgfpathlineto{\pgfqpoint{3.376730in}{1.149704in}}%
\pgfpathlineto{\pgfqpoint{3.391611in}{1.142180in}}%
\pgfpathlineto{\pgfqpoint{3.406493in}{1.173836in}}%
\pgfpathlineto{\pgfqpoint{3.421374in}{1.195378in}}%
\pgfpathlineto{\pgfqpoint{3.436256in}{1.142352in}}%
\pgfpathlineto{\pgfqpoint{3.451137in}{1.148267in}}%
\pgfpathlineto{\pgfqpoint{3.466019in}{1.086141in}}%
\pgfpathlineto{\pgfqpoint{3.480900in}{1.167868in}}%
\pgfpathlineto{\pgfqpoint{3.495782in}{1.175672in}}%
\pgfpathlineto{\pgfqpoint{3.510663in}{1.168180in}}%
\pgfpathlineto{\pgfqpoint{3.525545in}{1.186874in}}%
\pgfpathlineto{\pgfqpoint{3.540426in}{1.259915in}}%
\pgfpathlineto{\pgfqpoint{3.555308in}{1.345136in}}%
\pgfpathlineto{\pgfqpoint{3.570189in}{1.346933in}}%
\pgfpathlineto{\pgfqpoint{3.585071in}{1.249188in}}%
\pgfpathlineto{\pgfqpoint{3.599952in}{1.318454in}}%
\pgfpathlineto{\pgfqpoint{3.614833in}{1.312832in}}%
\pgfpathlineto{\pgfqpoint{3.629715in}{1.404236in}}%
\pgfpathlineto{\pgfqpoint{3.644596in}{1.324054in}}%
\pgfpathlineto{\pgfqpoint{3.659478in}{1.215925in}}%
\pgfpathlineto{\pgfqpoint{3.674359in}{1.340741in}}%
\pgfpathlineto{\pgfqpoint{3.689241in}{1.127182in}}%
\pgfpathlineto{\pgfqpoint{3.704122in}{1.183019in}}%
\pgfpathlineto{\pgfqpoint{3.719004in}{1.190887in}}%
\pgfpathlineto{\pgfqpoint{3.733885in}{1.157157in}}%
\pgfpathlineto{\pgfqpoint{3.748767in}{1.169515in}}%
\pgfpathlineto{\pgfqpoint{3.763648in}{1.170227in}}%
\pgfpathlineto{\pgfqpoint{3.778530in}{1.216534in}}%
\pgfpathlineto{\pgfqpoint{3.793411in}{1.336087in}}%
\pgfpathlineto{\pgfqpoint{3.808293in}{1.282375in}}%
\pgfpathlineto{\pgfqpoint{3.823174in}{1.158789in}}%
\pgfpathlineto{\pgfqpoint{3.838056in}{1.209199in}}%
\pgfpathlineto{\pgfqpoint{3.852937in}{1.216376in}}%
\pgfpathlineto{\pgfqpoint{3.867819in}{1.375783in}}%
\pgfpathlineto{\pgfqpoint{3.882700in}{1.302658in}}%
\pgfpathlineto{\pgfqpoint{3.897582in}{1.368024in}}%
\pgfpathlineto{\pgfqpoint{3.912463in}{1.291441in}}%
\pgfpathlineto{\pgfqpoint{3.927345in}{1.174389in}}%
\pgfpathlineto{\pgfqpoint{3.942226in}{1.227952in}}%
\pgfpathlineto{\pgfqpoint{3.957108in}{1.306438in}}%
\pgfpathlineto{\pgfqpoint{3.971989in}{1.263167in}}%
\pgfpathlineto{\pgfqpoint{3.986871in}{1.231399in}}%
\pgfpathlineto{\pgfqpoint{4.001752in}{1.179412in}}%
\pgfpathlineto{\pgfqpoint{4.016634in}{1.191758in}}%
\pgfpathlineto{\pgfqpoint{4.031515in}{1.254278in}}%
\pgfpathlineto{\pgfqpoint{4.046397in}{1.357095in}}%
\pgfpathlineto{\pgfqpoint{4.061278in}{1.252092in}}%
\pgfpathlineto{\pgfqpoint{4.076160in}{1.285455in}}%
\pgfpathlineto{\pgfqpoint{4.091041in}{1.338444in}}%
\pgfpathlineto{\pgfqpoint{4.105923in}{1.249293in}}%
\pgfpathlineto{\pgfqpoint{4.120804in}{1.285524in}}%
\pgfpathlineto{\pgfqpoint{4.135686in}{1.263558in}}%
\pgfpathlineto{\pgfqpoint{4.150567in}{1.239455in}}%
\pgfpathlineto{\pgfqpoint{4.165449in}{1.320850in}}%
\pgfpathlineto{\pgfqpoint{4.180330in}{1.296893in}}%
\pgfpathlineto{\pgfqpoint{4.195212in}{1.311045in}}%
\pgfpathlineto{\pgfqpoint{4.210093in}{1.355112in}}%
\pgfpathlineto{\pgfqpoint{4.224974in}{1.493411in}}%
\pgfpathlineto{\pgfqpoint{4.239856in}{1.312244in}}%
\pgfpathlineto{\pgfqpoint{4.254737in}{1.387170in}}%
\pgfpathlineto{\pgfqpoint{4.269619in}{1.340394in}}%
\pgfpathlineto{\pgfqpoint{4.284500in}{1.416679in}}%
\pgfpathlineto{\pgfqpoint{4.299382in}{1.367216in}}%
\pgfpathlineto{\pgfqpoint{4.314263in}{1.453377in}}%
\pgfpathlineto{\pgfqpoint{4.329145in}{1.559031in}}%
\pgfpathlineto{\pgfqpoint{4.344026in}{1.477932in}}%
\pgfpathlineto{\pgfqpoint{4.358908in}{1.599140in}}%
\pgfpathlineto{\pgfqpoint{4.373789in}{1.303089in}}%
\pgfpathlineto{\pgfqpoint{4.388671in}{1.467017in}}%
\pgfpathlineto{\pgfqpoint{4.403552in}{1.320014in}}%
\pgfpathlineto{\pgfqpoint{4.418434in}{1.340721in}}%
\pgfpathlineto{\pgfqpoint{4.448197in}{1.162706in}}%
\pgfpathlineto{\pgfqpoint{4.463078in}{1.225462in}}%
\pgfpathlineto{\pgfqpoint{4.477960in}{1.189117in}}%
\pgfpathlineto{\pgfqpoint{4.492841in}{1.358714in}}%
\pgfpathlineto{\pgfqpoint{4.507723in}{1.392696in}}%
\pgfpathlineto{\pgfqpoint{4.522604in}{1.226338in}}%
\pgfpathlineto{\pgfqpoint{4.537486in}{1.229532in}}%
\pgfpathlineto{\pgfqpoint{4.552367in}{1.189841in}}%
\pgfpathlineto{\pgfqpoint{4.567249in}{1.166778in}}%
\pgfpathlineto{\pgfqpoint{4.582130in}{1.348352in}}%
\pgfpathlineto{\pgfqpoint{4.597012in}{1.300442in}}%
\pgfpathlineto{\pgfqpoint{4.611893in}{1.348891in}}%
\pgfpathlineto{\pgfqpoint{4.626775in}{1.230196in}}%
\pgfpathlineto{\pgfqpoint{4.641656in}{1.200460in}}%
\pgfpathlineto{\pgfqpoint{4.656538in}{1.254189in}}%
\pgfpathlineto{\pgfqpoint{4.671419in}{1.210221in}}%
\pgfpathlineto{\pgfqpoint{4.686301in}{1.199057in}}%
\pgfpathlineto{\pgfqpoint{4.701182in}{1.236110in}}%
\pgfpathlineto{\pgfqpoint{4.716064in}{1.242313in}}%
\pgfpathlineto{\pgfqpoint{4.730945in}{1.169126in}}%
\pgfpathlineto{\pgfqpoint{4.745827in}{1.101692in}}%
\pgfpathlineto{\pgfqpoint{4.760708in}{1.162792in}}%
\pgfpathlineto{\pgfqpoint{4.775590in}{1.196802in}}%
\pgfpathlineto{\pgfqpoint{4.790471in}{1.179068in}}%
\pgfpathlineto{\pgfqpoint{4.805353in}{1.173523in}}%
\pgfpathlineto{\pgfqpoint{4.820234in}{1.143235in}}%
\pgfpathlineto{\pgfqpoint{4.835116in}{1.185241in}}%
\pgfpathlineto{\pgfqpoint{4.849997in}{1.193344in}}%
\pgfpathlineto{\pgfqpoint{4.864878in}{1.194833in}}%
\pgfpathlineto{\pgfqpoint{4.879760in}{1.263989in}}%
\pgfpathlineto{\pgfqpoint{4.894641in}{1.298118in}}%
\pgfpathlineto{\pgfqpoint{4.909523in}{1.355668in}}%
\pgfpathlineto{\pgfqpoint{4.924404in}{1.270921in}}%
\pgfpathlineto{\pgfqpoint{4.939286in}{1.380398in}}%
\pgfpathlineto{\pgfqpoint{4.954167in}{1.263850in}}%
\pgfpathlineto{\pgfqpoint{4.969049in}{1.295963in}}%
\pgfpathlineto{\pgfqpoint{4.983930in}{1.236920in}}%
\pgfpathlineto{\pgfqpoint{4.998812in}{1.232240in}}%
\pgfpathlineto{\pgfqpoint{5.013693in}{1.182231in}}%
\pgfpathlineto{\pgfqpoint{5.028575in}{1.090365in}}%
\pgfpathlineto{\pgfqpoint{5.043456in}{1.156175in}}%
\pgfpathlineto{\pgfqpoint{5.058338in}{1.200357in}}%
\pgfpathlineto{\pgfqpoint{5.073219in}{1.158092in}}%
\pgfpathlineto{\pgfqpoint{5.088101in}{1.296055in}}%
\pgfpathlineto{\pgfqpoint{5.102982in}{1.358828in}}%
\pgfpathlineto{\pgfqpoint{5.117864in}{1.212534in}}%
\pgfpathlineto{\pgfqpoint{5.132745in}{1.269734in}}%
\pgfpathlineto{\pgfqpoint{5.147627in}{1.167426in}}%
\pgfpathlineto{\pgfqpoint{5.162508in}{1.233053in}}%
\pgfpathlineto{\pgfqpoint{5.177390in}{1.241410in}}%
\pgfpathlineto{\pgfqpoint{5.192271in}{1.260565in}}%
\pgfpathlineto{\pgfqpoint{5.207153in}{1.285886in}}%
\pgfpathlineto{\pgfqpoint{5.222034in}{1.210533in}}%
\pgfpathlineto{\pgfqpoint{5.236916in}{1.337360in}}%
\pgfpathlineto{\pgfqpoint{5.251797in}{1.204016in}}%
\pgfpathlineto{\pgfqpoint{5.266679in}{1.159061in}}%
\pgfpathlineto{\pgfqpoint{5.281560in}{1.138703in}}%
\pgfpathlineto{\pgfqpoint{5.296442in}{1.142006in}}%
\pgfpathlineto{\pgfqpoint{5.311323in}{1.162582in}}%
\pgfpathlineto{\pgfqpoint{5.326205in}{1.268506in}}%
\pgfpathlineto{\pgfqpoint{5.341086in}{1.184070in}}%
\pgfpathlineto{\pgfqpoint{5.355968in}{1.129326in}}%
\pgfpathlineto{\pgfqpoint{5.370849in}{1.217365in}}%
\pgfpathlineto{\pgfqpoint{5.385731in}{1.435247in}}%
\pgfpathlineto{\pgfqpoint{5.400612in}{1.370515in}}%
\pgfpathlineto{\pgfqpoint{5.415494in}{1.314255in}}%
\pgfpathlineto{\pgfqpoint{5.430375in}{1.310374in}}%
\pgfpathlineto{\pgfqpoint{5.445257in}{1.349564in}}%
\pgfpathlineto{\pgfqpoint{5.460138in}{1.216127in}}%
\pgfpathlineto{\pgfqpoint{5.475020in}{1.188597in}}%
\pgfpathlineto{\pgfqpoint{5.519664in}{0.907334in}}%
\pgfpathlineto{\pgfqpoint{5.534545in}{0.704603in}}%
\pgfpathlineto{\pgfqpoint{5.534545in}{0.704603in}}%
\pgfusepath{stroke}%
\end{pgfscope}%
\begin{pgfscope}%
\pgfpathrectangle{\pgfqpoint{0.800000in}{0.528000in}}{\pgfqpoint{4.960000in}{3.696000in}}%
\pgfusepath{clip}%
\pgfsetrectcap%
\pgfsetroundjoin%
\pgfsetlinewidth{1.505625pt}%
\definecolor{currentstroke}{rgb}{1.000000,0.498039,0.054902}%
\pgfsetstrokecolor{currentstroke}%
\pgfsetdash{}{0pt}%
\pgfpathmoveto{\pgfqpoint{1.025455in}{1.283602in}}%
\pgfpathlineto{\pgfqpoint{5.534545in}{1.283602in}}%
\pgfpathlineto{\pgfqpoint{5.534545in}{1.283602in}}%
\pgfusepath{stroke}%
\end{pgfscope}%
\begin{pgfscope}%
\pgfsetrectcap%
\pgfsetmiterjoin%
\pgfsetlinewidth{0.803000pt}%
\definecolor{currentstroke}{rgb}{0.000000,0.000000,0.000000}%
\pgfsetstrokecolor{currentstroke}%
\pgfsetdash{}{0pt}%
\pgfpathmoveto{\pgfqpoint{0.800000in}{0.528000in}}%
\pgfpathlineto{\pgfqpoint{0.800000in}{4.224000in}}%
\pgfusepath{stroke}%
\end{pgfscope}%
\begin{pgfscope}%
\pgfsetrectcap%
\pgfsetmiterjoin%
\pgfsetlinewidth{0.803000pt}%
\definecolor{currentstroke}{rgb}{0.000000,0.000000,0.000000}%
\pgfsetstrokecolor{currentstroke}%
\pgfsetdash{}{0pt}%
\pgfpathmoveto{\pgfqpoint{5.760000in}{0.528000in}}%
\pgfpathlineto{\pgfqpoint{5.760000in}{4.224000in}}%
\pgfusepath{stroke}%
\end{pgfscope}%
\begin{pgfscope}%
\pgfsetrectcap%
\pgfsetmiterjoin%
\pgfsetlinewidth{0.803000pt}%
\definecolor{currentstroke}{rgb}{0.000000,0.000000,0.000000}%
\pgfsetstrokecolor{currentstroke}%
\pgfsetdash{}{0pt}%
\pgfpathmoveto{\pgfqpoint{0.800000in}{0.528000in}}%
\pgfpathlineto{\pgfqpoint{5.760000in}{0.528000in}}%
\pgfusepath{stroke}%
\end{pgfscope}%
\begin{pgfscope}%
\pgfsetrectcap%
\pgfsetmiterjoin%
\pgfsetlinewidth{0.803000pt}%
\definecolor{currentstroke}{rgb}{0.000000,0.000000,0.000000}%
\pgfsetstrokecolor{currentstroke}%
\pgfsetdash{}{0pt}%
\pgfpathmoveto{\pgfqpoint{0.800000in}{4.224000in}}%
\pgfpathlineto{\pgfqpoint{5.760000in}{4.224000in}}%
\pgfusepath{stroke}%
\end{pgfscope}%
\begin{pgfscope}%
\pgfsetbuttcap%
\pgfsetmiterjoin%
\definecolor{currentfill}{rgb}{1.000000,1.000000,1.000000}%
\pgfsetfillcolor{currentfill}%
\pgfsetfillopacity{0.800000}%
\pgfsetlinewidth{1.003750pt}%
\definecolor{currentstroke}{rgb}{0.800000,0.800000,0.800000}%
\pgfsetstrokecolor{currentstroke}%
\pgfsetstrokeopacity{0.800000}%
\pgfsetdash{}{0pt}%
\pgfpathmoveto{\pgfqpoint{3.522024in}{3.675698in}}%
\pgfpathlineto{\pgfqpoint{5.653056in}{3.675698in}}%
\pgfpathquadraticcurveto{\pgfqpoint{5.683611in}{3.675698in}}{\pgfqpoint{5.683611in}{3.706254in}}%
\pgfpathlineto{\pgfqpoint{5.683611in}{4.117056in}}%
\pgfpathquadraticcurveto{\pgfqpoint{5.683611in}{4.147611in}}{\pgfqpoint{5.653056in}{4.147611in}}%
\pgfpathlineto{\pgfqpoint{3.522024in}{4.147611in}}%
\pgfpathquadraticcurveto{\pgfqpoint{3.491468in}{4.147611in}}{\pgfqpoint{3.491468in}{4.117056in}}%
\pgfpathlineto{\pgfqpoint{3.491468in}{3.706254in}}%
\pgfpathquadraticcurveto{\pgfqpoint{3.491468in}{3.675698in}}{\pgfqpoint{3.522024in}{3.675698in}}%
\pgfpathclose%
\pgfusepath{stroke,fill}%
\end{pgfscope}%
\begin{pgfscope}%
\pgfsetrectcap%
\pgfsetroundjoin%
\pgfsetlinewidth{1.505625pt}%
\definecolor{currentstroke}{rgb}{0.121569,0.466667,0.705882}%
\pgfsetstrokecolor{currentstroke}%
\pgfsetdash{}{0pt}%
\pgfpathmoveto{\pgfqpoint{3.552579in}{4.033028in}}%
\pgfpathlineto{\pgfqpoint{3.858135in}{4.033028in}}%
\pgfusepath{stroke}%
\end{pgfscope}%
\begin{pgfscope}%
\pgftext[x=3.980357in,y=3.979556in,left,base]{\fontsize{11.000000}{13.200000}\selectfont Latency of writes}%
\end{pgfscope}%
\begin{pgfscope}%
\pgfsetrectcap%
\pgfsetroundjoin%
\pgfsetlinewidth{1.505625pt}%
\definecolor{currentstroke}{rgb}{1.000000,0.498039,0.054902}%
\pgfsetstrokecolor{currentstroke}%
\pgfsetdash{}{0pt}%
\pgfpathmoveto{\pgfqpoint{3.552579in}{3.819988in}}%
\pgfpathlineto{\pgfqpoint{3.858135in}{3.819988in}}%
\pgfusepath{stroke}%
\end{pgfscope}%
\begin{pgfscope}%
\pgftext[x=3.980357in,y=3.766516in,left,base]{\fontsize{11.000000}{13.200000}\selectfont Average latency of writes}%
\end{pgfscope}%
\end{pgfpicture}%
\makeatother%
\endgroup%

    \caption{Latency of write operations for every second of experiment 3 in \prettyref{tab:res-4}}
    \label{fig:lat-pmm}
\end{figure}

\begin{figure}
    %% Creator: Matplotlib, PGF backend
%%
%% To include the figure in your LaTeX document, write
%%   \input{<filename>.pgf}
%%
%% Make sure the required packages are loaded in your preamble
%%   \usepackage{pgf}
%%
%% Figures using additional raster images can only be included by \input if
%% they are in the same directory as the main LaTeX file. For loading figures
%% from other directories you can use the `import` package
%%   \usepackage{import}
%% and then include the figures with
%%   \import{<path to file>}{<filename>.pgf}
%%
%% Matplotlib used the following preamble
%%   \usepackage[utf8x]{inputenc}
%%   \usepackage[T1]{fontenc}
%%   \usepackage{lmodern}
%%
\begingroup%
\makeatletter%
\begin{pgfpicture}%
\pgfpathrectangle{\pgfpointorigin}{\pgfqpoint{6.400000in}{4.800000in}}%
\pgfusepath{use as bounding box, clip}%
\begin{pgfscope}%
\pgfsetbuttcap%
\pgfsetmiterjoin%
\definecolor{currentfill}{rgb}{1.000000,1.000000,1.000000}%
\pgfsetfillcolor{currentfill}%
\pgfsetlinewidth{0.000000pt}%
\definecolor{currentstroke}{rgb}{1.000000,1.000000,1.000000}%
\pgfsetstrokecolor{currentstroke}%
\pgfsetdash{}{0pt}%
\pgfpathmoveto{\pgfqpoint{0.000000in}{0.000000in}}%
\pgfpathlineto{\pgfqpoint{6.400000in}{0.000000in}}%
\pgfpathlineto{\pgfqpoint{6.400000in}{4.800000in}}%
\pgfpathlineto{\pgfqpoint{0.000000in}{4.800000in}}%
\pgfpathclose%
\pgfusepath{fill}%
\end{pgfscope}%
\begin{pgfscope}%
\pgfsetbuttcap%
\pgfsetmiterjoin%
\definecolor{currentfill}{rgb}{1.000000,1.000000,1.000000}%
\pgfsetfillcolor{currentfill}%
\pgfsetlinewidth{0.000000pt}%
\definecolor{currentstroke}{rgb}{0.000000,0.000000,0.000000}%
\pgfsetstrokecolor{currentstroke}%
\pgfsetstrokeopacity{0.000000}%
\pgfsetdash{}{0pt}%
\pgfpathmoveto{\pgfqpoint{0.800000in}{0.528000in}}%
\pgfpathlineto{\pgfqpoint{5.760000in}{0.528000in}}%
\pgfpathlineto{\pgfqpoint{5.760000in}{4.224000in}}%
\pgfpathlineto{\pgfqpoint{0.800000in}{4.224000in}}%
\pgfpathclose%
\pgfusepath{fill}%
\end{pgfscope}%
\begin{pgfscope}%
\pgfsetbuttcap%
\pgfsetroundjoin%
\definecolor{currentfill}{rgb}{0.000000,0.000000,0.000000}%
\pgfsetfillcolor{currentfill}%
\pgfsetlinewidth{0.803000pt}%
\definecolor{currentstroke}{rgb}{0.000000,0.000000,0.000000}%
\pgfsetstrokecolor{currentstroke}%
\pgfsetdash{}{0pt}%
\pgfsys@defobject{currentmarker}{\pgfqpoint{0.000000in}{-0.048611in}}{\pgfqpoint{0.000000in}{0.000000in}}{%
\pgfpathmoveto{\pgfqpoint{0.000000in}{0.000000in}}%
\pgfpathlineto{\pgfqpoint{0.000000in}{-0.048611in}}%
\pgfusepath{stroke,fill}%
}%
\begin{pgfscope}%
\pgfsys@transformshift{1.025455in}{0.528000in}%
\pgfsys@useobject{currentmarker}{}%
\end{pgfscope}%
\end{pgfscope}%
\begin{pgfscope}%
\pgftext[x=1.025455in,y=0.430778in,,top]{\fontsize{11.000000}{13.200000}\selectfont \(\displaystyle 0\)}%
\end{pgfscope}%
\begin{pgfscope}%
\pgfsetbuttcap%
\pgfsetroundjoin%
\definecolor{currentfill}{rgb}{0.000000,0.000000,0.000000}%
\pgfsetfillcolor{currentfill}%
\pgfsetlinewidth{0.803000pt}%
\definecolor{currentstroke}{rgb}{0.000000,0.000000,0.000000}%
\pgfsetstrokecolor{currentstroke}%
\pgfsetdash{}{0pt}%
\pgfsys@defobject{currentmarker}{\pgfqpoint{0.000000in}{-0.048611in}}{\pgfqpoint{0.000000in}{0.000000in}}{%
\pgfpathmoveto{\pgfqpoint{0.000000in}{0.000000in}}%
\pgfpathlineto{\pgfqpoint{0.000000in}{-0.048611in}}%
\pgfusepath{stroke,fill}%
}%
\begin{pgfscope}%
\pgfsys@transformshift{1.743463in}{0.528000in}%
\pgfsys@useobject{currentmarker}{}%
\end{pgfscope}%
\end{pgfscope}%
\begin{pgfscope}%
\pgftext[x=1.743463in,y=0.430778in,,top]{\fontsize{11.000000}{13.200000}\selectfont \(\displaystyle 50\)}%
\end{pgfscope}%
\begin{pgfscope}%
\pgfsetbuttcap%
\pgfsetroundjoin%
\definecolor{currentfill}{rgb}{0.000000,0.000000,0.000000}%
\pgfsetfillcolor{currentfill}%
\pgfsetlinewidth{0.803000pt}%
\definecolor{currentstroke}{rgb}{0.000000,0.000000,0.000000}%
\pgfsetstrokecolor{currentstroke}%
\pgfsetdash{}{0pt}%
\pgfsys@defobject{currentmarker}{\pgfqpoint{0.000000in}{-0.048611in}}{\pgfqpoint{0.000000in}{0.000000in}}{%
\pgfpathmoveto{\pgfqpoint{0.000000in}{0.000000in}}%
\pgfpathlineto{\pgfqpoint{0.000000in}{-0.048611in}}%
\pgfusepath{stroke,fill}%
}%
\begin{pgfscope}%
\pgfsys@transformshift{2.461471in}{0.528000in}%
\pgfsys@useobject{currentmarker}{}%
\end{pgfscope}%
\end{pgfscope}%
\begin{pgfscope}%
\pgftext[x=2.461471in,y=0.430778in,,top]{\fontsize{11.000000}{13.200000}\selectfont \(\displaystyle 100\)}%
\end{pgfscope}%
\begin{pgfscope}%
\pgfsetbuttcap%
\pgfsetroundjoin%
\definecolor{currentfill}{rgb}{0.000000,0.000000,0.000000}%
\pgfsetfillcolor{currentfill}%
\pgfsetlinewidth{0.803000pt}%
\definecolor{currentstroke}{rgb}{0.000000,0.000000,0.000000}%
\pgfsetstrokecolor{currentstroke}%
\pgfsetdash{}{0pt}%
\pgfsys@defobject{currentmarker}{\pgfqpoint{0.000000in}{-0.048611in}}{\pgfqpoint{0.000000in}{0.000000in}}{%
\pgfpathmoveto{\pgfqpoint{0.000000in}{0.000000in}}%
\pgfpathlineto{\pgfqpoint{0.000000in}{-0.048611in}}%
\pgfusepath{stroke,fill}%
}%
\begin{pgfscope}%
\pgfsys@transformshift{3.179479in}{0.528000in}%
\pgfsys@useobject{currentmarker}{}%
\end{pgfscope}%
\end{pgfscope}%
\begin{pgfscope}%
\pgftext[x=3.179479in,y=0.430778in,,top]{\fontsize{11.000000}{13.200000}\selectfont \(\displaystyle 150\)}%
\end{pgfscope}%
\begin{pgfscope}%
\pgfsetbuttcap%
\pgfsetroundjoin%
\definecolor{currentfill}{rgb}{0.000000,0.000000,0.000000}%
\pgfsetfillcolor{currentfill}%
\pgfsetlinewidth{0.803000pt}%
\definecolor{currentstroke}{rgb}{0.000000,0.000000,0.000000}%
\pgfsetstrokecolor{currentstroke}%
\pgfsetdash{}{0pt}%
\pgfsys@defobject{currentmarker}{\pgfqpoint{0.000000in}{-0.048611in}}{\pgfqpoint{0.000000in}{0.000000in}}{%
\pgfpathmoveto{\pgfqpoint{0.000000in}{0.000000in}}%
\pgfpathlineto{\pgfqpoint{0.000000in}{-0.048611in}}%
\pgfusepath{stroke,fill}%
}%
\begin{pgfscope}%
\pgfsys@transformshift{3.897487in}{0.528000in}%
\pgfsys@useobject{currentmarker}{}%
\end{pgfscope}%
\end{pgfscope}%
\begin{pgfscope}%
\pgftext[x=3.897487in,y=0.430778in,,top]{\fontsize{11.000000}{13.200000}\selectfont \(\displaystyle 200\)}%
\end{pgfscope}%
\begin{pgfscope}%
\pgfsetbuttcap%
\pgfsetroundjoin%
\definecolor{currentfill}{rgb}{0.000000,0.000000,0.000000}%
\pgfsetfillcolor{currentfill}%
\pgfsetlinewidth{0.803000pt}%
\definecolor{currentstroke}{rgb}{0.000000,0.000000,0.000000}%
\pgfsetstrokecolor{currentstroke}%
\pgfsetdash{}{0pt}%
\pgfsys@defobject{currentmarker}{\pgfqpoint{0.000000in}{-0.048611in}}{\pgfqpoint{0.000000in}{0.000000in}}{%
\pgfpathmoveto{\pgfqpoint{0.000000in}{0.000000in}}%
\pgfpathlineto{\pgfqpoint{0.000000in}{-0.048611in}}%
\pgfusepath{stroke,fill}%
}%
\begin{pgfscope}%
\pgfsys@transformshift{4.615495in}{0.528000in}%
\pgfsys@useobject{currentmarker}{}%
\end{pgfscope}%
\end{pgfscope}%
\begin{pgfscope}%
\pgftext[x=4.615495in,y=0.430778in,,top]{\fontsize{11.000000}{13.200000}\selectfont \(\displaystyle 250\)}%
\end{pgfscope}%
\begin{pgfscope}%
\pgfsetbuttcap%
\pgfsetroundjoin%
\definecolor{currentfill}{rgb}{0.000000,0.000000,0.000000}%
\pgfsetfillcolor{currentfill}%
\pgfsetlinewidth{0.803000pt}%
\definecolor{currentstroke}{rgb}{0.000000,0.000000,0.000000}%
\pgfsetstrokecolor{currentstroke}%
\pgfsetdash{}{0pt}%
\pgfsys@defobject{currentmarker}{\pgfqpoint{0.000000in}{-0.048611in}}{\pgfqpoint{0.000000in}{0.000000in}}{%
\pgfpathmoveto{\pgfqpoint{0.000000in}{0.000000in}}%
\pgfpathlineto{\pgfqpoint{0.000000in}{-0.048611in}}%
\pgfusepath{stroke,fill}%
}%
\begin{pgfscope}%
\pgfsys@transformshift{5.333503in}{0.528000in}%
\pgfsys@useobject{currentmarker}{}%
\end{pgfscope}%
\end{pgfscope}%
\begin{pgfscope}%
\pgftext[x=5.333503in,y=0.430778in,,top]{\fontsize{11.000000}{13.200000}\selectfont \(\displaystyle 300\)}%
\end{pgfscope}%
\begin{pgfscope}%
\pgftext[x=3.280000in,y=0.240271in,,top]{\fontsize{11.000000}{13.200000}\selectfont Time of experiment (in seconds)}%
\end{pgfscope}%
\begin{pgfscope}%
\pgfsetbuttcap%
\pgfsetroundjoin%
\definecolor{currentfill}{rgb}{0.000000,0.000000,0.000000}%
\pgfsetfillcolor{currentfill}%
\pgfsetlinewidth{0.803000pt}%
\definecolor{currentstroke}{rgb}{0.000000,0.000000,0.000000}%
\pgfsetstrokecolor{currentstroke}%
\pgfsetdash{}{0pt}%
\pgfsys@defobject{currentmarker}{\pgfqpoint{-0.048611in}{0.000000in}}{\pgfqpoint{0.000000in}{0.000000in}}{%
\pgfpathmoveto{\pgfqpoint{0.000000in}{0.000000in}}%
\pgfpathlineto{\pgfqpoint{-0.048611in}{0.000000in}}%
\pgfusepath{stroke,fill}%
}%
\begin{pgfscope}%
\pgfsys@transformshift{0.800000in}{0.692189in}%
\pgfsys@useobject{currentmarker}{}%
\end{pgfscope}%
\end{pgfscope}%
\begin{pgfscope}%
\pgftext[x=0.627981in,y=0.639566in,left,base]{\fontsize{11.000000}{13.200000}\selectfont \(\displaystyle 0\)}%
\end{pgfscope}%
\begin{pgfscope}%
\pgfsetbuttcap%
\pgfsetroundjoin%
\definecolor{currentfill}{rgb}{0.000000,0.000000,0.000000}%
\pgfsetfillcolor{currentfill}%
\pgfsetlinewidth{0.803000pt}%
\definecolor{currentstroke}{rgb}{0.000000,0.000000,0.000000}%
\pgfsetstrokecolor{currentstroke}%
\pgfsetdash{}{0pt}%
\pgfsys@defobject{currentmarker}{\pgfqpoint{-0.048611in}{0.000000in}}{\pgfqpoint{0.000000in}{0.000000in}}{%
\pgfpathmoveto{\pgfqpoint{0.000000in}{0.000000in}}%
\pgfpathlineto{\pgfqpoint{-0.048611in}{0.000000in}}%
\pgfusepath{stroke,fill}%
}%
\begin{pgfscope}%
\pgfsys@transformshift{0.800000in}{1.112665in}%
\pgfsys@useobject{currentmarker}{}%
\end{pgfscope}%
\end{pgfscope}%
\begin{pgfscope}%
\pgftext[x=0.478386in,y=1.060043in,left,base]{\fontsize{11.000000}{13.200000}\selectfont \(\displaystyle 250\)}%
\end{pgfscope}%
\begin{pgfscope}%
\pgfsetbuttcap%
\pgfsetroundjoin%
\definecolor{currentfill}{rgb}{0.000000,0.000000,0.000000}%
\pgfsetfillcolor{currentfill}%
\pgfsetlinewidth{0.803000pt}%
\definecolor{currentstroke}{rgb}{0.000000,0.000000,0.000000}%
\pgfsetstrokecolor{currentstroke}%
\pgfsetdash{}{0pt}%
\pgfsys@defobject{currentmarker}{\pgfqpoint{-0.048611in}{0.000000in}}{\pgfqpoint{0.000000in}{0.000000in}}{%
\pgfpathmoveto{\pgfqpoint{0.000000in}{0.000000in}}%
\pgfpathlineto{\pgfqpoint{-0.048611in}{0.000000in}}%
\pgfusepath{stroke,fill}%
}%
\begin{pgfscope}%
\pgfsys@transformshift{0.800000in}{1.533141in}%
\pgfsys@useobject{currentmarker}{}%
\end{pgfscope}%
\end{pgfscope}%
\begin{pgfscope}%
\pgftext[x=0.478386in,y=1.480519in,left,base]{\fontsize{11.000000}{13.200000}\selectfont \(\displaystyle 500\)}%
\end{pgfscope}%
\begin{pgfscope}%
\pgfsetbuttcap%
\pgfsetroundjoin%
\definecolor{currentfill}{rgb}{0.000000,0.000000,0.000000}%
\pgfsetfillcolor{currentfill}%
\pgfsetlinewidth{0.803000pt}%
\definecolor{currentstroke}{rgb}{0.000000,0.000000,0.000000}%
\pgfsetstrokecolor{currentstroke}%
\pgfsetdash{}{0pt}%
\pgfsys@defobject{currentmarker}{\pgfqpoint{-0.048611in}{0.000000in}}{\pgfqpoint{0.000000in}{0.000000in}}{%
\pgfpathmoveto{\pgfqpoint{0.000000in}{0.000000in}}%
\pgfpathlineto{\pgfqpoint{-0.048611in}{0.000000in}}%
\pgfusepath{stroke,fill}%
}%
\begin{pgfscope}%
\pgfsys@transformshift{0.800000in}{1.953618in}%
\pgfsys@useobject{currentmarker}{}%
\end{pgfscope}%
\end{pgfscope}%
\begin{pgfscope}%
\pgftext[x=0.478386in,y=1.900996in,left,base]{\fontsize{11.000000}{13.200000}\selectfont \(\displaystyle 750\)}%
\end{pgfscope}%
\begin{pgfscope}%
\pgfsetbuttcap%
\pgfsetroundjoin%
\definecolor{currentfill}{rgb}{0.000000,0.000000,0.000000}%
\pgfsetfillcolor{currentfill}%
\pgfsetlinewidth{0.803000pt}%
\definecolor{currentstroke}{rgb}{0.000000,0.000000,0.000000}%
\pgfsetstrokecolor{currentstroke}%
\pgfsetdash{}{0pt}%
\pgfsys@defobject{currentmarker}{\pgfqpoint{-0.048611in}{0.000000in}}{\pgfqpoint{0.000000in}{0.000000in}}{%
\pgfpathmoveto{\pgfqpoint{0.000000in}{0.000000in}}%
\pgfpathlineto{\pgfqpoint{-0.048611in}{0.000000in}}%
\pgfusepath{stroke,fill}%
}%
\begin{pgfscope}%
\pgfsys@transformshift{0.800000in}{2.374094in}%
\pgfsys@useobject{currentmarker}{}%
\end{pgfscope}%
\end{pgfscope}%
\begin{pgfscope}%
\pgftext[x=0.403588in,y=2.321472in,left,base]{\fontsize{11.000000}{13.200000}\selectfont \(\displaystyle 1000\)}%
\end{pgfscope}%
\begin{pgfscope}%
\pgfsetbuttcap%
\pgfsetroundjoin%
\definecolor{currentfill}{rgb}{0.000000,0.000000,0.000000}%
\pgfsetfillcolor{currentfill}%
\pgfsetlinewidth{0.803000pt}%
\definecolor{currentstroke}{rgb}{0.000000,0.000000,0.000000}%
\pgfsetstrokecolor{currentstroke}%
\pgfsetdash{}{0pt}%
\pgfsys@defobject{currentmarker}{\pgfqpoint{-0.048611in}{0.000000in}}{\pgfqpoint{0.000000in}{0.000000in}}{%
\pgfpathmoveto{\pgfqpoint{0.000000in}{0.000000in}}%
\pgfpathlineto{\pgfqpoint{-0.048611in}{0.000000in}}%
\pgfusepath{stroke,fill}%
}%
\begin{pgfscope}%
\pgfsys@transformshift{0.800000in}{2.794571in}%
\pgfsys@useobject{currentmarker}{}%
\end{pgfscope}%
\end{pgfscope}%
\begin{pgfscope}%
\pgftext[x=0.403588in,y=2.741948in,left,base]{\fontsize{11.000000}{13.200000}\selectfont \(\displaystyle 1250\)}%
\end{pgfscope}%
\begin{pgfscope}%
\pgfsetbuttcap%
\pgfsetroundjoin%
\definecolor{currentfill}{rgb}{0.000000,0.000000,0.000000}%
\pgfsetfillcolor{currentfill}%
\pgfsetlinewidth{0.803000pt}%
\definecolor{currentstroke}{rgb}{0.000000,0.000000,0.000000}%
\pgfsetstrokecolor{currentstroke}%
\pgfsetdash{}{0pt}%
\pgfsys@defobject{currentmarker}{\pgfqpoint{-0.048611in}{0.000000in}}{\pgfqpoint{0.000000in}{0.000000in}}{%
\pgfpathmoveto{\pgfqpoint{0.000000in}{0.000000in}}%
\pgfpathlineto{\pgfqpoint{-0.048611in}{0.000000in}}%
\pgfusepath{stroke,fill}%
}%
\begin{pgfscope}%
\pgfsys@transformshift{0.800000in}{3.215047in}%
\pgfsys@useobject{currentmarker}{}%
\end{pgfscope}%
\end{pgfscope}%
\begin{pgfscope}%
\pgftext[x=0.403588in,y=3.162425in,left,base]{\fontsize{11.000000}{13.200000}\selectfont \(\displaystyle 1500\)}%
\end{pgfscope}%
\begin{pgfscope}%
\pgfsetbuttcap%
\pgfsetroundjoin%
\definecolor{currentfill}{rgb}{0.000000,0.000000,0.000000}%
\pgfsetfillcolor{currentfill}%
\pgfsetlinewidth{0.803000pt}%
\definecolor{currentstroke}{rgb}{0.000000,0.000000,0.000000}%
\pgfsetstrokecolor{currentstroke}%
\pgfsetdash{}{0pt}%
\pgfsys@defobject{currentmarker}{\pgfqpoint{-0.048611in}{0.000000in}}{\pgfqpoint{0.000000in}{0.000000in}}{%
\pgfpathmoveto{\pgfqpoint{0.000000in}{0.000000in}}%
\pgfpathlineto{\pgfqpoint{-0.048611in}{0.000000in}}%
\pgfusepath{stroke,fill}%
}%
\begin{pgfscope}%
\pgfsys@transformshift{0.800000in}{3.635524in}%
\pgfsys@useobject{currentmarker}{}%
\end{pgfscope}%
\end{pgfscope}%
\begin{pgfscope}%
\pgftext[x=0.403588in,y=3.582901in,left,base]{\fontsize{11.000000}{13.200000}\selectfont \(\displaystyle 1750\)}%
\end{pgfscope}%
\begin{pgfscope}%
\pgfsetbuttcap%
\pgfsetroundjoin%
\definecolor{currentfill}{rgb}{0.000000,0.000000,0.000000}%
\pgfsetfillcolor{currentfill}%
\pgfsetlinewidth{0.803000pt}%
\definecolor{currentstroke}{rgb}{0.000000,0.000000,0.000000}%
\pgfsetstrokecolor{currentstroke}%
\pgfsetdash{}{0pt}%
\pgfsys@defobject{currentmarker}{\pgfqpoint{-0.048611in}{0.000000in}}{\pgfqpoint{0.000000in}{0.000000in}}{%
\pgfpathmoveto{\pgfqpoint{0.000000in}{0.000000in}}%
\pgfpathlineto{\pgfqpoint{-0.048611in}{0.000000in}}%
\pgfusepath{stroke,fill}%
}%
\begin{pgfscope}%
\pgfsys@transformshift{0.800000in}{4.056000in}%
\pgfsys@useobject{currentmarker}{}%
\end{pgfscope}%
\end{pgfscope}%
\begin{pgfscope}%
\pgftext[x=0.403588in,y=4.003378in,left,base]{\fontsize{11.000000}{13.200000}\selectfont \(\displaystyle 2000\)}%
\end{pgfscope}%
\begin{pgfscope}%
\pgftext[x=0.348033in,y=2.376000in,,bottom,rotate=90.000000]{\fontsize{11.000000}{13.200000}\selectfont Latency of reads (in milliseconds)}%
\end{pgfscope}%
\begin{pgfscope}%
\pgfpathrectangle{\pgfqpoint{0.800000in}{0.528000in}}{\pgfqpoint{4.960000in}{3.696000in}}%
\pgfusepath{clip}%
\pgfsetrectcap%
\pgfsetroundjoin%
\pgfsetlinewidth{1.505625pt}%
\definecolor{currentstroke}{rgb}{0.121569,0.466667,0.705882}%
\pgfsetstrokecolor{currentstroke}%
\pgfsetdash{}{0pt}%
\pgfpathmoveto{\pgfqpoint{1.025455in}{0.881444in}}%
\pgfpathlineto{\pgfqpoint{1.039815in}{0.811604in}}%
\pgfpathlineto{\pgfqpoint{1.054175in}{0.879414in}}%
\pgfpathlineto{\pgfqpoint{1.068535in}{0.848499in}}%
\pgfpathlineto{\pgfqpoint{1.082895in}{0.921770in}}%
\pgfpathlineto{\pgfqpoint{1.097255in}{0.912890in}}%
\pgfpathlineto{\pgfqpoint{1.111616in}{0.980285in}}%
\pgfpathlineto{\pgfqpoint{1.125976in}{0.934925in}}%
\pgfpathlineto{\pgfqpoint{1.140336in}{0.916828in}}%
\pgfpathlineto{\pgfqpoint{1.154696in}{0.992357in}}%
\pgfpathlineto{\pgfqpoint{1.169056in}{0.946353in}}%
\pgfpathlineto{\pgfqpoint{1.183416in}{1.008264in}}%
\pgfpathlineto{\pgfqpoint{1.197776in}{0.889469in}}%
\pgfpathlineto{\pgfqpoint{1.212137in}{1.003858in}}%
\pgfpathlineto{\pgfqpoint{1.226497in}{1.050466in}}%
\pgfpathlineto{\pgfqpoint{1.240857in}{0.978806in}}%
\pgfpathlineto{\pgfqpoint{1.255217in}{0.999074in}}%
\pgfpathlineto{\pgfqpoint{1.269577in}{0.922577in}}%
\pgfpathlineto{\pgfqpoint{1.283937in}{0.996900in}}%
\pgfpathlineto{\pgfqpoint{1.298298in}{1.013650in}}%
\pgfpathlineto{\pgfqpoint{1.312658in}{1.007608in}}%
\pgfpathlineto{\pgfqpoint{1.327018in}{1.220465in}}%
\pgfpathlineto{\pgfqpoint{1.341378in}{0.972714in}}%
\pgfpathlineto{\pgfqpoint{1.355738in}{0.899592in}}%
\pgfpathlineto{\pgfqpoint{1.370098in}{0.902715in}}%
\pgfpathlineto{\pgfqpoint{1.384459in}{1.145165in}}%
\pgfpathlineto{\pgfqpoint{1.398819in}{1.005452in}}%
\pgfpathlineto{\pgfqpoint{1.413179in}{1.176871in}}%
\pgfpathlineto{\pgfqpoint{1.427539in}{0.958298in}}%
\pgfpathlineto{\pgfqpoint{1.441899in}{0.878172in}}%
\pgfpathlineto{\pgfqpoint{1.456259in}{1.002224in}}%
\pgfpathlineto{\pgfqpoint{1.470620in}{1.061472in}}%
\pgfpathlineto{\pgfqpoint{1.484980in}{0.986964in}}%
\pgfpathlineto{\pgfqpoint{1.499340in}{0.959050in}}%
\pgfpathlineto{\pgfqpoint{1.513700in}{0.964665in}}%
\pgfpathlineto{\pgfqpoint{1.528060in}{0.984394in}}%
\pgfpathlineto{\pgfqpoint{1.542420in}{0.974562in}}%
\pgfpathlineto{\pgfqpoint{1.556781in}{1.020823in}}%
\pgfpathlineto{\pgfqpoint{1.571141in}{0.964510in}}%
\pgfpathlineto{\pgfqpoint{1.585501in}{0.977754in}}%
\pgfpathlineto{\pgfqpoint{1.599861in}{0.951291in}}%
\pgfpathlineto{\pgfqpoint{1.614221in}{0.941712in}}%
\pgfpathlineto{\pgfqpoint{1.628581in}{0.905393in}}%
\pgfpathlineto{\pgfqpoint{1.642942in}{0.912832in}}%
\pgfpathlineto{\pgfqpoint{1.657302in}{1.199246in}}%
\pgfpathlineto{\pgfqpoint{1.671662in}{1.013940in}}%
\pgfpathlineto{\pgfqpoint{1.686022in}{1.002795in}}%
\pgfpathlineto{\pgfqpoint{1.700382in}{0.999418in}}%
\pgfpathlineto{\pgfqpoint{1.714742in}{1.117503in}}%
\pgfpathlineto{\pgfqpoint{1.729102in}{0.856143in}}%
\pgfpathlineto{\pgfqpoint{1.743463in}{1.211876in}}%
\pgfpathlineto{\pgfqpoint{1.757823in}{1.029273in}}%
\pgfpathlineto{\pgfqpoint{1.786543in}{0.905477in}}%
\pgfpathlineto{\pgfqpoint{1.800903in}{0.993925in}}%
\pgfpathlineto{\pgfqpoint{1.815263in}{0.996080in}}%
\pgfpathlineto{\pgfqpoint{1.829624in}{0.925399in}}%
\pgfpathlineto{\pgfqpoint{1.843984in}{0.989684in}}%
\pgfpathlineto{\pgfqpoint{1.858344in}{0.988133in}}%
\pgfpathlineto{\pgfqpoint{1.872704in}{0.850376in}}%
\pgfpathlineto{\pgfqpoint{1.887064in}{0.998613in}}%
\pgfpathlineto{\pgfqpoint{1.901424in}{0.963477in}}%
\pgfpathlineto{\pgfqpoint{1.915785in}{0.979006in}}%
\pgfpathlineto{\pgfqpoint{1.930145in}{0.969150in}}%
\pgfpathlineto{\pgfqpoint{1.944505in}{0.949771in}}%
\pgfpathlineto{\pgfqpoint{1.958865in}{0.968270in}}%
\pgfpathlineto{\pgfqpoint{1.973225in}{1.042771in}}%
\pgfpathlineto{\pgfqpoint{1.987585in}{0.994540in}}%
\pgfpathlineto{\pgfqpoint{2.001946in}{0.934437in}}%
\pgfpathlineto{\pgfqpoint{2.016306in}{0.987929in}}%
\pgfpathlineto{\pgfqpoint{2.030666in}{1.012625in}}%
\pgfpathlineto{\pgfqpoint{2.045026in}{1.027029in}}%
\pgfpathlineto{\pgfqpoint{2.059386in}{0.987813in}}%
\pgfpathlineto{\pgfqpoint{2.073746in}{0.998731in}}%
\pgfpathlineto{\pgfqpoint{2.088107in}{1.018720in}}%
\pgfpathlineto{\pgfqpoint{2.102467in}{0.863186in}}%
\pgfpathlineto{\pgfqpoint{2.116827in}{1.160321in}}%
\pgfpathlineto{\pgfqpoint{2.131187in}{0.961094in}}%
\pgfpathlineto{\pgfqpoint{2.145547in}{0.929243in}}%
\pgfpathlineto{\pgfqpoint{2.159907in}{0.966940in}}%
\pgfpathlineto{\pgfqpoint{2.174268in}{0.863448in}}%
\pgfpathlineto{\pgfqpoint{2.188628in}{0.837063in}}%
\pgfpathlineto{\pgfqpoint{2.202988in}{1.014940in}}%
\pgfpathlineto{\pgfqpoint{2.217348in}{0.891176in}}%
\pgfpathlineto{\pgfqpoint{2.231708in}{0.895309in}}%
\pgfpathlineto{\pgfqpoint{2.246068in}{0.877079in}}%
\pgfpathlineto{\pgfqpoint{2.260428in}{1.037250in}}%
\pgfpathlineto{\pgfqpoint{2.274789in}{1.064536in}}%
\pgfpathlineto{\pgfqpoint{2.289149in}{0.959057in}}%
\pgfpathlineto{\pgfqpoint{2.303509in}{1.088684in}}%
\pgfpathlineto{\pgfqpoint{2.317869in}{0.942648in}}%
\pgfpathlineto{\pgfqpoint{2.332229in}{0.947227in}}%
\pgfpathlineto{\pgfqpoint{2.346589in}{0.917054in}}%
\pgfpathlineto{\pgfqpoint{2.360950in}{0.919553in}}%
\pgfpathlineto{\pgfqpoint{2.375310in}{0.909456in}}%
\pgfpathlineto{\pgfqpoint{2.389670in}{1.357658in}}%
\pgfpathlineto{\pgfqpoint{2.404030in}{0.931618in}}%
\pgfpathlineto{\pgfqpoint{2.418390in}{0.876762in}}%
\pgfpathlineto{\pgfqpoint{2.432750in}{0.978497in}}%
\pgfpathlineto{\pgfqpoint{2.447111in}{1.014103in}}%
\pgfpathlineto{\pgfqpoint{2.461471in}{0.989148in}}%
\pgfpathlineto{\pgfqpoint{2.763034in}{4.056000in}}%
\pgfpathlineto{\pgfqpoint{2.777394in}{0.972570in}}%
\pgfpathlineto{\pgfqpoint{2.791754in}{0.969424in}}%
\pgfpathlineto{\pgfqpoint{2.806115in}{0.838585in}}%
\pgfpathlineto{\pgfqpoint{2.820475in}{0.909293in}}%
\pgfpathlineto{\pgfqpoint{2.834835in}{0.877071in}}%
\pgfpathlineto{\pgfqpoint{2.849195in}{0.891202in}}%
\pgfpathlineto{\pgfqpoint{2.863555in}{0.968984in}}%
\pgfpathlineto{\pgfqpoint{2.877915in}{1.016470in}}%
\pgfpathlineto{\pgfqpoint{2.892276in}{0.943197in}}%
\pgfpathlineto{\pgfqpoint{2.906636in}{0.928750in}}%
\pgfpathlineto{\pgfqpoint{2.920996in}{0.976067in}}%
\pgfpathlineto{\pgfqpoint{2.935356in}{0.948618in}}%
\pgfpathlineto{\pgfqpoint{2.949716in}{0.967611in}}%
\pgfpathlineto{\pgfqpoint{2.964076in}{0.941118in}}%
\pgfpathlineto{\pgfqpoint{2.978437in}{0.980376in}}%
\pgfpathlineto{\pgfqpoint{2.992797in}{0.925390in}}%
\pgfpathlineto{\pgfqpoint{3.007157in}{0.978579in}}%
\pgfpathlineto{\pgfqpoint{3.021517in}{0.894732in}}%
\pgfpathlineto{\pgfqpoint{3.035877in}{1.062306in}}%
\pgfpathlineto{\pgfqpoint{3.050237in}{0.999785in}}%
\pgfpathlineto{\pgfqpoint{3.064598in}{1.006395in}}%
\pgfpathlineto{\pgfqpoint{3.078958in}{0.882497in}}%
\pgfpathlineto{\pgfqpoint{3.093318in}{1.060371in}}%
\pgfpathlineto{\pgfqpoint{3.107678in}{1.066213in}}%
\pgfpathlineto{\pgfqpoint{3.122038in}{0.968611in}}%
\pgfpathlineto{\pgfqpoint{3.136398in}{0.993326in}}%
\pgfpathlineto{\pgfqpoint{3.150759in}{0.982751in}}%
\pgfpathlineto{\pgfqpoint{3.165119in}{0.981102in}}%
\pgfpathlineto{\pgfqpoint{3.179479in}{0.981416in}}%
\pgfpathlineto{\pgfqpoint{3.193839in}{1.011519in}}%
\pgfpathlineto{\pgfqpoint{3.208199in}{1.000662in}}%
\pgfpathlineto{\pgfqpoint{3.222559in}{0.963726in}}%
\pgfpathlineto{\pgfqpoint{3.236920in}{1.049450in}}%
\pgfpathlineto{\pgfqpoint{3.251280in}{0.994778in}}%
\pgfpathlineto{\pgfqpoint{3.265640in}{0.948761in}}%
\pgfpathlineto{\pgfqpoint{3.280000in}{0.935803in}}%
\pgfpathlineto{\pgfqpoint{3.294360in}{0.945350in}}%
\pgfpathlineto{\pgfqpoint{3.308720in}{1.006399in}}%
\pgfpathlineto{\pgfqpoint{3.323080in}{0.975003in}}%
\pgfpathlineto{\pgfqpoint{3.337441in}{0.966004in}}%
\pgfpathlineto{\pgfqpoint{3.351801in}{0.856996in}}%
\pgfpathlineto{\pgfqpoint{3.366161in}{0.911584in}}%
\pgfpathlineto{\pgfqpoint{3.380521in}{0.950960in}}%
\pgfpathlineto{\pgfqpoint{3.394881in}{1.111778in}}%
\pgfpathlineto{\pgfqpoint{3.409241in}{1.406234in}}%
\pgfpathlineto{\pgfqpoint{3.423602in}{0.995905in}}%
\pgfpathlineto{\pgfqpoint{3.437962in}{0.967211in}}%
\pgfpathlineto{\pgfqpoint{3.452322in}{0.985795in}}%
\pgfpathlineto{\pgfqpoint{3.466682in}{0.891527in}}%
\pgfpathlineto{\pgfqpoint{3.495402in}{1.122651in}}%
\pgfpathlineto{\pgfqpoint{3.509763in}{0.884349in}}%
\pgfpathlineto{\pgfqpoint{3.524123in}{1.234140in}}%
\pgfpathlineto{\pgfqpoint{3.538483in}{1.011607in}}%
\pgfpathlineto{\pgfqpoint{3.552843in}{1.003371in}}%
\pgfpathlineto{\pgfqpoint{3.567203in}{1.067367in}}%
\pgfpathlineto{\pgfqpoint{3.581563in}{0.981159in}}%
\pgfpathlineto{\pgfqpoint{3.595924in}{0.971631in}}%
\pgfpathlineto{\pgfqpoint{3.610284in}{1.065425in}}%
\pgfpathlineto{\pgfqpoint{3.624644in}{0.878817in}}%
\pgfpathlineto{\pgfqpoint{3.639004in}{0.997880in}}%
\pgfpathlineto{\pgfqpoint{3.653364in}{1.102222in}}%
\pgfpathlineto{\pgfqpoint{3.667724in}{0.943473in}}%
\pgfpathlineto{\pgfqpoint{3.682085in}{1.172668in}}%
\pgfpathlineto{\pgfqpoint{3.696445in}{1.265590in}}%
\pgfpathlineto{\pgfqpoint{3.710805in}{0.932566in}}%
\pgfpathlineto{\pgfqpoint{3.725165in}{0.883117in}}%
\pgfpathlineto{\pgfqpoint{3.739525in}{0.967080in}}%
\pgfpathlineto{\pgfqpoint{3.753885in}{0.922796in}}%
\pgfpathlineto{\pgfqpoint{3.768246in}{0.963908in}}%
\pgfpathlineto{\pgfqpoint{3.782606in}{0.957249in}}%
\pgfpathlineto{\pgfqpoint{3.796966in}{1.013403in}}%
\pgfpathlineto{\pgfqpoint{3.811326in}{0.982294in}}%
\pgfpathlineto{\pgfqpoint{3.825686in}{0.981542in}}%
\pgfpathlineto{\pgfqpoint{3.840046in}{0.931951in}}%
\pgfpathlineto{\pgfqpoint{3.854406in}{1.105501in}}%
\pgfpathlineto{\pgfqpoint{3.868767in}{0.967146in}}%
\pgfpathlineto{\pgfqpoint{3.883127in}{1.015076in}}%
\pgfpathlineto{\pgfqpoint{3.897487in}{0.988048in}}%
\pgfpathlineto{\pgfqpoint{3.911847in}{1.108404in}}%
\pgfpathlineto{\pgfqpoint{3.926207in}{1.090437in}}%
\pgfpathlineto{\pgfqpoint{3.940567in}{0.979270in}}%
\pgfpathlineto{\pgfqpoint{3.954928in}{0.929920in}}%
\pgfpathlineto{\pgfqpoint{3.969288in}{1.018034in}}%
\pgfpathlineto{\pgfqpoint{3.983648in}{1.048009in}}%
\pgfpathlineto{\pgfqpoint{3.998008in}{1.139355in}}%
\pgfpathlineto{\pgfqpoint{4.012368in}{0.990587in}}%
\pgfpathlineto{\pgfqpoint{4.026728in}{0.958984in}}%
\pgfpathlineto{\pgfqpoint{4.041089in}{0.983684in}}%
\pgfpathlineto{\pgfqpoint{4.055449in}{0.987086in}}%
\pgfpathlineto{\pgfqpoint{4.069809in}{1.234625in}}%
\pgfpathlineto{\pgfqpoint{4.084169in}{0.977844in}}%
\pgfpathlineto{\pgfqpoint{4.112889in}{1.010914in}}%
\pgfpathlineto{\pgfqpoint{4.127250in}{0.930959in}}%
\pgfpathlineto{\pgfqpoint{4.141610in}{0.966831in}}%
\pgfpathlineto{\pgfqpoint{4.155970in}{1.104985in}}%
\pgfpathlineto{\pgfqpoint{4.170330in}{1.016436in}}%
\pgfpathlineto{\pgfqpoint{4.184690in}{0.993038in}}%
\pgfpathlineto{\pgfqpoint{4.199050in}{0.935063in}}%
\pgfpathlineto{\pgfqpoint{4.213411in}{1.221452in}}%
\pgfpathlineto{\pgfqpoint{4.227771in}{0.921262in}}%
\pgfpathlineto{\pgfqpoint{4.242131in}{0.856148in}}%
\pgfpathlineto{\pgfqpoint{4.256491in}{0.862485in}}%
\pgfpathlineto{\pgfqpoint{4.270851in}{0.973364in}}%
\pgfpathlineto{\pgfqpoint{4.285211in}{1.049730in}}%
\pgfpathlineto{\pgfqpoint{4.299572in}{0.898346in}}%
\pgfpathlineto{\pgfqpoint{4.313932in}{1.195609in}}%
\pgfpathlineto{\pgfqpoint{4.328292in}{0.995555in}}%
\pgfpathlineto{\pgfqpoint{4.342652in}{0.970141in}}%
\pgfpathlineto{\pgfqpoint{4.357012in}{1.138320in}}%
\pgfpathlineto{\pgfqpoint{4.371372in}{0.838239in}}%
\pgfpathlineto{\pgfqpoint{4.385732in}{1.225462in}}%
\pgfpathlineto{\pgfqpoint{4.400093in}{1.316735in}}%
\pgfpathlineto{\pgfqpoint{4.414453in}{0.979946in}}%
\pgfpathlineto{\pgfqpoint{4.428813in}{0.895322in}}%
\pgfpathlineto{\pgfqpoint{4.443173in}{1.069323in}}%
\pgfpathlineto{\pgfqpoint{4.457533in}{0.947945in}}%
\pgfpathlineto{\pgfqpoint{4.471893in}{0.985947in}}%
\pgfpathlineto{\pgfqpoint{4.486254in}{0.999117in}}%
\pgfpathlineto{\pgfqpoint{4.500614in}{0.966487in}}%
\pgfpathlineto{\pgfqpoint{4.514974in}{0.947814in}}%
\pgfpathlineto{\pgfqpoint{4.529334in}{1.199186in}}%
\pgfpathlineto{\pgfqpoint{4.543694in}{1.117542in}}%
\pgfpathlineto{\pgfqpoint{4.558054in}{0.988540in}}%
\pgfpathlineto{\pgfqpoint{4.572415in}{0.934093in}}%
\pgfpathlineto{\pgfqpoint{4.586775in}{0.853027in}}%
\pgfpathlineto{\pgfqpoint{4.601135in}{1.051063in}}%
\pgfpathlineto{\pgfqpoint{4.615495in}{0.970645in}}%
\pgfpathlineto{\pgfqpoint{4.629855in}{0.989690in}}%
\pgfpathlineto{\pgfqpoint{4.644215in}{0.864560in}}%
\pgfpathlineto{\pgfqpoint{4.658576in}{1.124448in}}%
\pgfpathlineto{\pgfqpoint{4.672936in}{1.127088in}}%
\pgfpathlineto{\pgfqpoint{4.687296in}{1.012347in}}%
\pgfpathlineto{\pgfqpoint{4.701656in}{0.980696in}}%
\pgfpathlineto{\pgfqpoint{4.716016in}{1.020407in}}%
\pgfpathlineto{\pgfqpoint{4.730376in}{1.016778in}}%
\pgfpathlineto{\pgfqpoint{4.744737in}{0.952357in}}%
\pgfpathlineto{\pgfqpoint{4.759097in}{1.026879in}}%
\pgfpathlineto{\pgfqpoint{4.773457in}{0.975313in}}%
\pgfpathlineto{\pgfqpoint{4.787817in}{0.882818in}}%
\pgfpathlineto{\pgfqpoint{4.802177in}{0.907924in}}%
\pgfpathlineto{\pgfqpoint{4.816537in}{1.124329in}}%
\pgfpathlineto{\pgfqpoint{4.830898in}{1.313525in}}%
\pgfpathlineto{\pgfqpoint{4.845258in}{0.965626in}}%
\pgfpathlineto{\pgfqpoint{4.859618in}{0.908101in}}%
\pgfpathlineto{\pgfqpoint{4.873978in}{0.964468in}}%
\pgfpathlineto{\pgfqpoint{4.888338in}{1.026733in}}%
\pgfpathlineto{\pgfqpoint{4.902698in}{1.131135in}}%
\pgfpathlineto{\pgfqpoint{4.917058in}{1.010054in}}%
\pgfpathlineto{\pgfqpoint{4.931419in}{0.952265in}}%
\pgfpathlineto{\pgfqpoint{4.945779in}{0.822312in}}%
\pgfpathlineto{\pgfqpoint{4.960139in}{1.069936in}}%
\pgfpathlineto{\pgfqpoint{4.974499in}{1.006355in}}%
\pgfpathlineto{\pgfqpoint{4.988859in}{1.016139in}}%
\pgfpathlineto{\pgfqpoint{5.003219in}{0.963734in}}%
\pgfpathlineto{\pgfqpoint{5.017580in}{1.145689in}}%
\pgfpathlineto{\pgfqpoint{5.031940in}{1.076011in}}%
\pgfpathlineto{\pgfqpoint{5.046300in}{0.995329in}}%
\pgfpathlineto{\pgfqpoint{5.060660in}{1.118565in}}%
\pgfpathlineto{\pgfqpoint{5.075020in}{0.899646in}}%
\pgfpathlineto{\pgfqpoint{5.089380in}{0.852964in}}%
\pgfpathlineto{\pgfqpoint{5.103741in}{0.856549in}}%
\pgfpathlineto{\pgfqpoint{5.118101in}{1.227599in}}%
\pgfpathlineto{\pgfqpoint{5.132461in}{1.278474in}}%
\pgfpathlineto{\pgfqpoint{5.146821in}{1.014288in}}%
\pgfpathlineto{\pgfqpoint{5.161181in}{0.991122in}}%
\pgfpathlineto{\pgfqpoint{5.175541in}{1.381173in}}%
\pgfpathlineto{\pgfqpoint{5.189902in}{1.062664in}}%
\pgfpathlineto{\pgfqpoint{5.204262in}{1.175255in}}%
\pgfpathlineto{\pgfqpoint{5.218622in}{0.890744in}}%
\pgfpathlineto{\pgfqpoint{5.232982in}{0.874876in}}%
\pgfpathlineto{\pgfqpoint{5.247342in}{1.805047in}}%
\pgfpathlineto{\pgfqpoint{5.261702in}{1.017375in}}%
\pgfpathlineto{\pgfqpoint{5.276063in}{1.029875in}}%
\pgfpathlineto{\pgfqpoint{5.304783in}{0.924074in}}%
\pgfpathlineto{\pgfqpoint{5.319143in}{1.087736in}}%
\pgfpathlineto{\pgfqpoint{5.333503in}{0.936259in}}%
\pgfpathlineto{\pgfqpoint{5.347863in}{0.816937in}}%
\pgfpathlineto{\pgfqpoint{5.362224in}{1.047468in}}%
\pgfpathlineto{\pgfqpoint{5.376584in}{1.049656in}}%
\pgfpathlineto{\pgfqpoint{5.390944in}{0.872654in}}%
\pgfpathlineto{\pgfqpoint{5.405304in}{0.945963in}}%
\pgfpathlineto{\pgfqpoint{5.419664in}{0.784413in}}%
\pgfpathlineto{\pgfqpoint{5.434024in}{0.723006in}}%
\pgfpathlineto{\pgfqpoint{5.462745in}{0.710662in}}%
\pgfpathlineto{\pgfqpoint{5.505825in}{0.706353in}}%
\pgfpathlineto{\pgfqpoint{5.534545in}{0.696000in}}%
\pgfpathlineto{\pgfqpoint{5.534545in}{0.696000in}}%
\pgfusepath{stroke}%
\end{pgfscope}%
\begin{pgfscope}%
\pgfsetrectcap%
\pgfsetmiterjoin%
\pgfsetlinewidth{0.803000pt}%
\definecolor{currentstroke}{rgb}{0.000000,0.000000,0.000000}%
\pgfsetstrokecolor{currentstroke}%
\pgfsetdash{}{0pt}%
\pgfpathmoveto{\pgfqpoint{0.800000in}{0.528000in}}%
\pgfpathlineto{\pgfqpoint{0.800000in}{4.224000in}}%
\pgfusepath{stroke}%
\end{pgfscope}%
\begin{pgfscope}%
\pgfsetrectcap%
\pgfsetmiterjoin%
\pgfsetlinewidth{0.803000pt}%
\definecolor{currentstroke}{rgb}{0.000000,0.000000,0.000000}%
\pgfsetstrokecolor{currentstroke}%
\pgfsetdash{}{0pt}%
\pgfpathmoveto{\pgfqpoint{5.760000in}{0.528000in}}%
\pgfpathlineto{\pgfqpoint{5.760000in}{4.224000in}}%
\pgfusepath{stroke}%
\end{pgfscope}%
\begin{pgfscope}%
\pgfsetrectcap%
\pgfsetmiterjoin%
\pgfsetlinewidth{0.803000pt}%
\definecolor{currentstroke}{rgb}{0.000000,0.000000,0.000000}%
\pgfsetstrokecolor{currentstroke}%
\pgfsetdash{}{0pt}%
\pgfpathmoveto{\pgfqpoint{0.800000in}{0.528000in}}%
\pgfpathlineto{\pgfqpoint{5.760000in}{0.528000in}}%
\pgfusepath{stroke}%
\end{pgfscope}%
\begin{pgfscope}%
\pgfsetrectcap%
\pgfsetmiterjoin%
\pgfsetlinewidth{0.803000pt}%
\definecolor{currentstroke}{rgb}{0.000000,0.000000,0.000000}%
\pgfsetstrokecolor{currentstroke}%
\pgfsetdash{}{0pt}%
\pgfpathmoveto{\pgfqpoint{0.800000in}{4.224000in}}%
\pgfpathlineto{\pgfqpoint{5.760000in}{4.224000in}}%
\pgfusepath{stroke}%
\end{pgfscope}%
\begin{pgfscope}%
\pgfsetbuttcap%
\pgfsetmiterjoin%
\definecolor{currentfill}{rgb}{1.000000,1.000000,1.000000}%
\pgfsetfillcolor{currentfill}%
\pgfsetfillopacity{0.800000}%
\pgfsetlinewidth{1.003750pt}%
\definecolor{currentstroke}{rgb}{0.800000,0.800000,0.800000}%
\pgfsetstrokecolor{currentstroke}%
\pgfsetstrokeopacity{0.800000}%
\pgfsetdash{}{0pt}%
\pgfpathmoveto{\pgfqpoint{5.591944in}{4.025389in}}%
\pgfpathlineto{\pgfqpoint{5.653056in}{4.025389in}}%
\pgfpathquadraticcurveto{\pgfqpoint{5.683611in}{4.025389in}}{\pgfqpoint{5.683611in}{4.055944in}}%
\pgfpathlineto{\pgfqpoint{5.683611in}{4.117056in}}%
\pgfpathquadraticcurveto{\pgfqpoint{5.683611in}{4.147611in}}{\pgfqpoint{5.653056in}{4.147611in}}%
\pgfpathlineto{\pgfqpoint{5.591944in}{4.147611in}}%
\pgfpathquadraticcurveto{\pgfqpoint{5.561389in}{4.147611in}}{\pgfqpoint{5.561389in}{4.117056in}}%
\pgfpathlineto{\pgfqpoint{5.561389in}{4.055944in}}%
\pgfpathquadraticcurveto{\pgfqpoint{5.561389in}{4.025389in}}{\pgfqpoint{5.591944in}{4.025389in}}%
\pgfpathclose%
\pgfusepath{stroke,fill}%
\end{pgfscope}%
\end{pgfpicture}%
\makeatother%
\endgroup%

    \caption{Latency of read operations for every second of experiment 1 in \prettyref{tab:res-3}}
    \label{fig:rlat-plp}
\end{figure}

\begin{figure}
    %% Creator: Matplotlib, PGF backend
%%
%% To include the figure in your LaTeX document, write
%%   \input{<filename>.pgf}
%%
%% Make sure the required packages are loaded in your preamble
%%   \usepackage{pgf}
%%
%% Figures using additional raster images can only be included by \input if
%% they are in the same directory as the main LaTeX file. For loading figures
%% from other directories you can use the `import` package
%%   \usepackage{import}
%% and then include the figures with
%%   \import{<path to file>}{<filename>.pgf}
%%
%% Matplotlib used the following preamble
%%   \usepackage[utf8x]{inputenc}
%%   \usepackage[T1]{fontenc}
%%   \usepackage{lmodern}
%%
\begingroup%
\makeatletter%
\begin{pgfpicture}%
\pgfpathrectangle{\pgfpointorigin}{\pgfqpoint{6.400000in}{4.800000in}}%
\pgfusepath{use as bounding box, clip}%
\begin{pgfscope}%
\pgfsetbuttcap%
\pgfsetmiterjoin%
\definecolor{currentfill}{rgb}{1.000000,1.000000,1.000000}%
\pgfsetfillcolor{currentfill}%
\pgfsetlinewidth{0.000000pt}%
\definecolor{currentstroke}{rgb}{1.000000,1.000000,1.000000}%
\pgfsetstrokecolor{currentstroke}%
\pgfsetdash{}{0pt}%
\pgfpathmoveto{\pgfqpoint{0.000000in}{0.000000in}}%
\pgfpathlineto{\pgfqpoint{6.400000in}{0.000000in}}%
\pgfpathlineto{\pgfqpoint{6.400000in}{4.800000in}}%
\pgfpathlineto{\pgfqpoint{0.000000in}{4.800000in}}%
\pgfpathclose%
\pgfusepath{fill}%
\end{pgfscope}%
\begin{pgfscope}%
\pgfsetbuttcap%
\pgfsetmiterjoin%
\definecolor{currentfill}{rgb}{1.000000,1.000000,1.000000}%
\pgfsetfillcolor{currentfill}%
\pgfsetlinewidth{0.000000pt}%
\definecolor{currentstroke}{rgb}{0.000000,0.000000,0.000000}%
\pgfsetstrokecolor{currentstroke}%
\pgfsetstrokeopacity{0.000000}%
\pgfsetdash{}{0pt}%
\pgfpathmoveto{\pgfqpoint{0.800000in}{0.528000in}}%
\pgfpathlineto{\pgfqpoint{5.760000in}{0.528000in}}%
\pgfpathlineto{\pgfqpoint{5.760000in}{4.224000in}}%
\pgfpathlineto{\pgfqpoint{0.800000in}{4.224000in}}%
\pgfpathclose%
\pgfusepath{fill}%
\end{pgfscope}%
\begin{pgfscope}%
\pgfsetbuttcap%
\pgfsetroundjoin%
\definecolor{currentfill}{rgb}{0.000000,0.000000,0.000000}%
\pgfsetfillcolor{currentfill}%
\pgfsetlinewidth{0.803000pt}%
\definecolor{currentstroke}{rgb}{0.000000,0.000000,0.000000}%
\pgfsetstrokecolor{currentstroke}%
\pgfsetdash{}{0pt}%
\pgfsys@defobject{currentmarker}{\pgfqpoint{0.000000in}{-0.048611in}}{\pgfqpoint{0.000000in}{0.000000in}}{%
\pgfpathmoveto{\pgfqpoint{0.000000in}{0.000000in}}%
\pgfpathlineto{\pgfqpoint{0.000000in}{-0.048611in}}%
\pgfusepath{stroke,fill}%
}%
\begin{pgfscope}%
\pgfsys@transformshift{1.025455in}{0.528000in}%
\pgfsys@useobject{currentmarker}{}%
\end{pgfscope}%
\end{pgfscope}%
\begin{pgfscope}%
\pgftext[x=1.025455in,y=0.430778in,,top]{\fontsize{11.000000}{13.200000}\selectfont \(\displaystyle 0\)}%
\end{pgfscope}%
\begin{pgfscope}%
\pgfsetbuttcap%
\pgfsetroundjoin%
\definecolor{currentfill}{rgb}{0.000000,0.000000,0.000000}%
\pgfsetfillcolor{currentfill}%
\pgfsetlinewidth{0.803000pt}%
\definecolor{currentstroke}{rgb}{0.000000,0.000000,0.000000}%
\pgfsetstrokecolor{currentstroke}%
\pgfsetdash{}{0pt}%
\pgfsys@defobject{currentmarker}{\pgfqpoint{0.000000in}{-0.048611in}}{\pgfqpoint{0.000000in}{0.000000in}}{%
\pgfpathmoveto{\pgfqpoint{0.000000in}{0.000000in}}%
\pgfpathlineto{\pgfqpoint{0.000000in}{-0.048611in}}%
\pgfusepath{stroke,fill}%
}%
\begin{pgfscope}%
\pgfsys@transformshift{1.745757in}{0.528000in}%
\pgfsys@useobject{currentmarker}{}%
\end{pgfscope}%
\end{pgfscope}%
\begin{pgfscope}%
\pgftext[x=1.745757in,y=0.430778in,,top]{\fontsize{11.000000}{13.200000}\selectfont \(\displaystyle 50\)}%
\end{pgfscope}%
\begin{pgfscope}%
\pgfsetbuttcap%
\pgfsetroundjoin%
\definecolor{currentfill}{rgb}{0.000000,0.000000,0.000000}%
\pgfsetfillcolor{currentfill}%
\pgfsetlinewidth{0.803000pt}%
\definecolor{currentstroke}{rgb}{0.000000,0.000000,0.000000}%
\pgfsetstrokecolor{currentstroke}%
\pgfsetdash{}{0pt}%
\pgfsys@defobject{currentmarker}{\pgfqpoint{0.000000in}{-0.048611in}}{\pgfqpoint{0.000000in}{0.000000in}}{%
\pgfpathmoveto{\pgfqpoint{0.000000in}{0.000000in}}%
\pgfpathlineto{\pgfqpoint{0.000000in}{-0.048611in}}%
\pgfusepath{stroke,fill}%
}%
\begin{pgfscope}%
\pgfsys@transformshift{2.466059in}{0.528000in}%
\pgfsys@useobject{currentmarker}{}%
\end{pgfscope}%
\end{pgfscope}%
\begin{pgfscope}%
\pgftext[x=2.466059in,y=0.430778in,,top]{\fontsize{11.000000}{13.200000}\selectfont \(\displaystyle 100\)}%
\end{pgfscope}%
\begin{pgfscope}%
\pgfsetbuttcap%
\pgfsetroundjoin%
\definecolor{currentfill}{rgb}{0.000000,0.000000,0.000000}%
\pgfsetfillcolor{currentfill}%
\pgfsetlinewidth{0.803000pt}%
\definecolor{currentstroke}{rgb}{0.000000,0.000000,0.000000}%
\pgfsetstrokecolor{currentstroke}%
\pgfsetdash{}{0pt}%
\pgfsys@defobject{currentmarker}{\pgfqpoint{0.000000in}{-0.048611in}}{\pgfqpoint{0.000000in}{0.000000in}}{%
\pgfpathmoveto{\pgfqpoint{0.000000in}{0.000000in}}%
\pgfpathlineto{\pgfqpoint{0.000000in}{-0.048611in}}%
\pgfusepath{stroke,fill}%
}%
\begin{pgfscope}%
\pgfsys@transformshift{3.186361in}{0.528000in}%
\pgfsys@useobject{currentmarker}{}%
\end{pgfscope}%
\end{pgfscope}%
\begin{pgfscope}%
\pgftext[x=3.186361in,y=0.430778in,,top]{\fontsize{11.000000}{13.200000}\selectfont \(\displaystyle 150\)}%
\end{pgfscope}%
\begin{pgfscope}%
\pgfsetbuttcap%
\pgfsetroundjoin%
\definecolor{currentfill}{rgb}{0.000000,0.000000,0.000000}%
\pgfsetfillcolor{currentfill}%
\pgfsetlinewidth{0.803000pt}%
\definecolor{currentstroke}{rgb}{0.000000,0.000000,0.000000}%
\pgfsetstrokecolor{currentstroke}%
\pgfsetdash{}{0pt}%
\pgfsys@defobject{currentmarker}{\pgfqpoint{0.000000in}{-0.048611in}}{\pgfqpoint{0.000000in}{0.000000in}}{%
\pgfpathmoveto{\pgfqpoint{0.000000in}{0.000000in}}%
\pgfpathlineto{\pgfqpoint{0.000000in}{-0.048611in}}%
\pgfusepath{stroke,fill}%
}%
\begin{pgfscope}%
\pgfsys@transformshift{3.906663in}{0.528000in}%
\pgfsys@useobject{currentmarker}{}%
\end{pgfscope}%
\end{pgfscope}%
\begin{pgfscope}%
\pgftext[x=3.906663in,y=0.430778in,,top]{\fontsize{11.000000}{13.200000}\selectfont \(\displaystyle 200\)}%
\end{pgfscope}%
\begin{pgfscope}%
\pgfsetbuttcap%
\pgfsetroundjoin%
\definecolor{currentfill}{rgb}{0.000000,0.000000,0.000000}%
\pgfsetfillcolor{currentfill}%
\pgfsetlinewidth{0.803000pt}%
\definecolor{currentstroke}{rgb}{0.000000,0.000000,0.000000}%
\pgfsetstrokecolor{currentstroke}%
\pgfsetdash{}{0pt}%
\pgfsys@defobject{currentmarker}{\pgfqpoint{0.000000in}{-0.048611in}}{\pgfqpoint{0.000000in}{0.000000in}}{%
\pgfpathmoveto{\pgfqpoint{0.000000in}{0.000000in}}%
\pgfpathlineto{\pgfqpoint{0.000000in}{-0.048611in}}%
\pgfusepath{stroke,fill}%
}%
\begin{pgfscope}%
\pgfsys@transformshift{4.626965in}{0.528000in}%
\pgfsys@useobject{currentmarker}{}%
\end{pgfscope}%
\end{pgfscope}%
\begin{pgfscope}%
\pgftext[x=4.626965in,y=0.430778in,,top]{\fontsize{11.000000}{13.200000}\selectfont \(\displaystyle 250\)}%
\end{pgfscope}%
\begin{pgfscope}%
\pgfsetbuttcap%
\pgfsetroundjoin%
\definecolor{currentfill}{rgb}{0.000000,0.000000,0.000000}%
\pgfsetfillcolor{currentfill}%
\pgfsetlinewidth{0.803000pt}%
\definecolor{currentstroke}{rgb}{0.000000,0.000000,0.000000}%
\pgfsetstrokecolor{currentstroke}%
\pgfsetdash{}{0pt}%
\pgfsys@defobject{currentmarker}{\pgfqpoint{0.000000in}{-0.048611in}}{\pgfqpoint{0.000000in}{0.000000in}}{%
\pgfpathmoveto{\pgfqpoint{0.000000in}{0.000000in}}%
\pgfpathlineto{\pgfqpoint{0.000000in}{-0.048611in}}%
\pgfusepath{stroke,fill}%
}%
\begin{pgfscope}%
\pgfsys@transformshift{5.347267in}{0.528000in}%
\pgfsys@useobject{currentmarker}{}%
\end{pgfscope}%
\end{pgfscope}%
\begin{pgfscope}%
\pgftext[x=5.347267in,y=0.430778in,,top]{\fontsize{11.000000}{13.200000}\selectfont \(\displaystyle 300\)}%
\end{pgfscope}%
\begin{pgfscope}%
\pgftext[x=3.280000in,y=0.240271in,,top]{\fontsize{11.000000}{13.200000}\selectfont Time of experiment (in seconds)}%
\end{pgfscope}%
\begin{pgfscope}%
\pgfsetbuttcap%
\pgfsetroundjoin%
\definecolor{currentfill}{rgb}{0.000000,0.000000,0.000000}%
\pgfsetfillcolor{currentfill}%
\pgfsetlinewidth{0.803000pt}%
\definecolor{currentstroke}{rgb}{0.000000,0.000000,0.000000}%
\pgfsetstrokecolor{currentstroke}%
\pgfsetdash{}{0pt}%
\pgfsys@defobject{currentmarker}{\pgfqpoint{-0.048611in}{0.000000in}}{\pgfqpoint{0.000000in}{0.000000in}}{%
\pgfpathmoveto{\pgfqpoint{0.000000in}{0.000000in}}%
\pgfpathlineto{\pgfqpoint{-0.048611in}{0.000000in}}%
\pgfusepath{stroke,fill}%
}%
\begin{pgfscope}%
\pgfsys@transformshift{0.800000in}{0.688483in}%
\pgfsys@useobject{currentmarker}{}%
\end{pgfscope}%
\end{pgfscope}%
\begin{pgfscope}%
\pgftext[x=0.627981in,y=0.635861in,left,base]{\fontsize{11.000000}{13.200000}\selectfont \(\displaystyle 0\)}%
\end{pgfscope}%
\begin{pgfscope}%
\pgfsetbuttcap%
\pgfsetroundjoin%
\definecolor{currentfill}{rgb}{0.000000,0.000000,0.000000}%
\pgfsetfillcolor{currentfill}%
\pgfsetlinewidth{0.803000pt}%
\definecolor{currentstroke}{rgb}{0.000000,0.000000,0.000000}%
\pgfsetstrokecolor{currentstroke}%
\pgfsetdash{}{0pt}%
\pgfsys@defobject{currentmarker}{\pgfqpoint{-0.048611in}{0.000000in}}{\pgfqpoint{0.000000in}{0.000000in}}{%
\pgfpathmoveto{\pgfqpoint{0.000000in}{0.000000in}}%
\pgfpathlineto{\pgfqpoint{-0.048611in}{0.000000in}}%
\pgfusepath{stroke,fill}%
}%
\begin{pgfscope}%
\pgfsys@transformshift{0.800000in}{1.407559in}%
\pgfsys@useobject{currentmarker}{}%
\end{pgfscope}%
\end{pgfscope}%
\begin{pgfscope}%
\pgftext[x=0.478386in,y=1.354937in,left,base]{\fontsize{11.000000}{13.200000}\selectfont \(\displaystyle 200\)}%
\end{pgfscope}%
\begin{pgfscope}%
\pgfsetbuttcap%
\pgfsetroundjoin%
\definecolor{currentfill}{rgb}{0.000000,0.000000,0.000000}%
\pgfsetfillcolor{currentfill}%
\pgfsetlinewidth{0.803000pt}%
\definecolor{currentstroke}{rgb}{0.000000,0.000000,0.000000}%
\pgfsetstrokecolor{currentstroke}%
\pgfsetdash{}{0pt}%
\pgfsys@defobject{currentmarker}{\pgfqpoint{-0.048611in}{0.000000in}}{\pgfqpoint{0.000000in}{0.000000in}}{%
\pgfpathmoveto{\pgfqpoint{0.000000in}{0.000000in}}%
\pgfpathlineto{\pgfqpoint{-0.048611in}{0.000000in}}%
\pgfusepath{stroke,fill}%
}%
\begin{pgfscope}%
\pgfsys@transformshift{0.800000in}{2.126635in}%
\pgfsys@useobject{currentmarker}{}%
\end{pgfscope}%
\end{pgfscope}%
\begin{pgfscope}%
\pgftext[x=0.478386in,y=2.074013in,left,base]{\fontsize{11.000000}{13.200000}\selectfont \(\displaystyle 400\)}%
\end{pgfscope}%
\begin{pgfscope}%
\pgfsetbuttcap%
\pgfsetroundjoin%
\definecolor{currentfill}{rgb}{0.000000,0.000000,0.000000}%
\pgfsetfillcolor{currentfill}%
\pgfsetlinewidth{0.803000pt}%
\definecolor{currentstroke}{rgb}{0.000000,0.000000,0.000000}%
\pgfsetstrokecolor{currentstroke}%
\pgfsetdash{}{0pt}%
\pgfsys@defobject{currentmarker}{\pgfqpoint{-0.048611in}{0.000000in}}{\pgfqpoint{0.000000in}{0.000000in}}{%
\pgfpathmoveto{\pgfqpoint{0.000000in}{0.000000in}}%
\pgfpathlineto{\pgfqpoint{-0.048611in}{0.000000in}}%
\pgfusepath{stroke,fill}%
}%
\begin{pgfscope}%
\pgfsys@transformshift{0.800000in}{2.845711in}%
\pgfsys@useobject{currentmarker}{}%
\end{pgfscope}%
\end{pgfscope}%
\begin{pgfscope}%
\pgftext[x=0.478386in,y=2.793089in,left,base]{\fontsize{11.000000}{13.200000}\selectfont \(\displaystyle 600\)}%
\end{pgfscope}%
\begin{pgfscope}%
\pgfsetbuttcap%
\pgfsetroundjoin%
\definecolor{currentfill}{rgb}{0.000000,0.000000,0.000000}%
\pgfsetfillcolor{currentfill}%
\pgfsetlinewidth{0.803000pt}%
\definecolor{currentstroke}{rgb}{0.000000,0.000000,0.000000}%
\pgfsetstrokecolor{currentstroke}%
\pgfsetdash{}{0pt}%
\pgfsys@defobject{currentmarker}{\pgfqpoint{-0.048611in}{0.000000in}}{\pgfqpoint{0.000000in}{0.000000in}}{%
\pgfpathmoveto{\pgfqpoint{0.000000in}{0.000000in}}%
\pgfpathlineto{\pgfqpoint{-0.048611in}{0.000000in}}%
\pgfusepath{stroke,fill}%
}%
\begin{pgfscope}%
\pgfsys@transformshift{0.800000in}{3.564787in}%
\pgfsys@useobject{currentmarker}{}%
\end{pgfscope}%
\end{pgfscope}%
\begin{pgfscope}%
\pgftext[x=0.478386in,y=3.512165in,left,base]{\fontsize{11.000000}{13.200000}\selectfont \(\displaystyle 800\)}%
\end{pgfscope}%
\begin{pgfscope}%
\pgftext[x=0.422830in,y=2.376000in,,bottom,rotate=90.000000]{\fontsize{11.000000}{13.200000}\selectfont Latency of reads (in milliseconds)}%
\end{pgfscope}%
\begin{pgfscope}%
\pgfpathrectangle{\pgfqpoint{0.800000in}{0.528000in}}{\pgfqpoint{4.960000in}{3.696000in}}%
\pgfusepath{clip}%
\pgfsetrectcap%
\pgfsetroundjoin%
\pgfsetlinewidth{1.505625pt}%
\definecolor{currentstroke}{rgb}{0.121569,0.466667,0.705882}%
\pgfsetstrokecolor{currentstroke}%
\pgfsetdash{}{0pt}%
\pgfpathmoveto{\pgfqpoint{1.025455in}{0.974799in}}%
\pgfpathlineto{\pgfqpoint{1.039861in}{1.000878in}}%
\pgfpathlineto{\pgfqpoint{1.054267in}{1.055217in}}%
\pgfpathlineto{\pgfqpoint{1.083079in}{1.181529in}}%
\pgfpathlineto{\pgfqpoint{1.097485in}{1.267394in}}%
\pgfpathlineto{\pgfqpoint{1.111891in}{1.509569in}}%
\pgfpathlineto{\pgfqpoint{1.126297in}{1.303142in}}%
\pgfpathlineto{\pgfqpoint{1.140703in}{1.444205in}}%
\pgfpathlineto{\pgfqpoint{1.155109in}{1.419529in}}%
\pgfpathlineto{\pgfqpoint{1.169515in}{1.016621in}}%
\pgfpathlineto{\pgfqpoint{1.183921in}{1.895317in}}%
\pgfpathlineto{\pgfqpoint{1.198327in}{1.195731in}}%
\pgfpathlineto{\pgfqpoint{1.212733in}{1.125618in}}%
\pgfpathlineto{\pgfqpoint{1.227139in}{1.317745in}}%
\pgfpathlineto{\pgfqpoint{1.241545in}{1.362924in}}%
\pgfpathlineto{\pgfqpoint{1.255951in}{1.356169in}}%
\pgfpathlineto{\pgfqpoint{1.270357in}{1.670084in}}%
\pgfpathlineto{\pgfqpoint{1.284763in}{1.324278in}}%
\pgfpathlineto{\pgfqpoint{1.299169in}{1.292132in}}%
\pgfpathlineto{\pgfqpoint{1.313575in}{1.327706in}}%
\pgfpathlineto{\pgfqpoint{1.327981in}{1.407906in}}%
\pgfpathlineto{\pgfqpoint{1.342387in}{1.265158in}}%
\pgfpathlineto{\pgfqpoint{1.356793in}{1.146672in}}%
\pgfpathlineto{\pgfqpoint{1.371200in}{1.213150in}}%
\pgfpathlineto{\pgfqpoint{1.385606in}{1.122252in}}%
\pgfpathlineto{\pgfqpoint{1.400012in}{1.429308in}}%
\pgfpathlineto{\pgfqpoint{1.414418in}{1.192525in}}%
\pgfpathlineto{\pgfqpoint{1.428824in}{1.411329in}}%
\pgfpathlineto{\pgfqpoint{1.443230in}{1.219368in}}%
\pgfpathlineto{\pgfqpoint{1.457636in}{1.253372in}}%
\pgfpathlineto{\pgfqpoint{1.472042in}{1.206825in}}%
\pgfpathlineto{\pgfqpoint{1.486448in}{1.407764in}}%
\pgfpathlineto{\pgfqpoint{1.500854in}{1.214026in}}%
\pgfpathlineto{\pgfqpoint{1.515260in}{1.479147in}}%
\pgfpathlineto{\pgfqpoint{1.529666in}{1.330405in}}%
\pgfpathlineto{\pgfqpoint{1.544072in}{1.264710in}}%
\pgfpathlineto{\pgfqpoint{1.558478in}{1.261765in}}%
\pgfpathlineto{\pgfqpoint{1.572884in}{1.239006in}}%
\pgfpathlineto{\pgfqpoint{1.587290in}{1.253695in}}%
\pgfpathlineto{\pgfqpoint{1.601696in}{1.035646in}}%
\pgfpathlineto{\pgfqpoint{1.616102in}{0.963238in}}%
\pgfpathlineto{\pgfqpoint{1.630508in}{1.231792in}}%
\pgfpathlineto{\pgfqpoint{1.644914in}{1.096030in}}%
\pgfpathlineto{\pgfqpoint{1.659320in}{1.069353in}}%
\pgfpathlineto{\pgfqpoint{1.673726in}{1.128003in}}%
\pgfpathlineto{\pgfqpoint{1.688132in}{1.013090in}}%
\pgfpathlineto{\pgfqpoint{1.702538in}{1.961300in}}%
\pgfpathlineto{\pgfqpoint{1.716945in}{1.496485in}}%
\pgfpathlineto{\pgfqpoint{1.731351in}{1.167650in}}%
\pgfpathlineto{\pgfqpoint{1.745757in}{1.418093in}}%
\pgfpathlineto{\pgfqpoint{1.760163in}{1.218650in}}%
\pgfpathlineto{\pgfqpoint{1.774569in}{1.346854in}}%
\pgfpathlineto{\pgfqpoint{1.788975in}{1.296623in}}%
\pgfpathlineto{\pgfqpoint{1.803381in}{1.308694in}}%
\pgfpathlineto{\pgfqpoint{1.817787in}{1.013491in}}%
\pgfpathlineto{\pgfqpoint{1.832193in}{1.459872in}}%
\pgfpathlineto{\pgfqpoint{1.846599in}{1.302260in}}%
\pgfpathlineto{\pgfqpoint{1.861005in}{1.090814in}}%
\pgfpathlineto{\pgfqpoint{1.875411in}{1.157625in}}%
\pgfpathlineto{\pgfqpoint{1.889817in}{1.391223in}}%
\pgfpathlineto{\pgfqpoint{1.904223in}{1.322982in}}%
\pgfpathlineto{\pgfqpoint{1.918629in}{1.337137in}}%
\pgfpathlineto{\pgfqpoint{1.933035in}{1.258030in}}%
\pgfpathlineto{\pgfqpoint{1.947441in}{1.296162in}}%
\pgfpathlineto{\pgfqpoint{1.961847in}{1.717499in}}%
\pgfpathlineto{\pgfqpoint{1.976253in}{1.327619in}}%
\pgfpathlineto{\pgfqpoint{1.990659in}{1.286340in}}%
\pgfpathlineto{\pgfqpoint{2.005065in}{1.166004in}}%
\pgfpathlineto{\pgfqpoint{2.019471in}{1.540803in}}%
\pgfpathlineto{\pgfqpoint{2.048283in}{1.239062in}}%
\pgfpathlineto{\pgfqpoint{2.062690in}{1.297861in}}%
\pgfpathlineto{\pgfqpoint{2.077096in}{1.225132in}}%
\pgfpathlineto{\pgfqpoint{2.091502in}{1.199428in}}%
\pgfpathlineto{\pgfqpoint{2.105908in}{1.265793in}}%
\pgfpathlineto{\pgfqpoint{2.120314in}{1.338306in}}%
\pgfpathlineto{\pgfqpoint{2.134720in}{1.250067in}}%
\pgfpathlineto{\pgfqpoint{2.149126in}{1.279710in}}%
\pgfpathlineto{\pgfqpoint{2.163532in}{1.285658in}}%
\pgfpathlineto{\pgfqpoint{2.177938in}{1.321058in}}%
\pgfpathlineto{\pgfqpoint{2.192344in}{1.075266in}}%
\pgfpathlineto{\pgfqpoint{2.206750in}{1.399841in}}%
\pgfpathlineto{\pgfqpoint{2.221156in}{1.331528in}}%
\pgfpathlineto{\pgfqpoint{2.235562in}{1.384036in}}%
\pgfpathlineto{\pgfqpoint{2.249968in}{1.319834in}}%
\pgfpathlineto{\pgfqpoint{2.264374in}{1.125720in}}%
\pgfpathlineto{\pgfqpoint{2.278780in}{0.949634in}}%
\pgfpathlineto{\pgfqpoint{2.293186in}{1.580368in}}%
\pgfpathlineto{\pgfqpoint{2.307592in}{1.346300in}}%
\pgfpathlineto{\pgfqpoint{2.321998in}{1.308134in}}%
\pgfpathlineto{\pgfqpoint{2.336404in}{1.347396in}}%
\pgfpathlineto{\pgfqpoint{2.350810in}{1.300441in}}%
\pgfpathlineto{\pgfqpoint{2.365216in}{1.372718in}}%
\pgfpathlineto{\pgfqpoint{2.379622in}{1.231554in}}%
\pgfpathlineto{\pgfqpoint{2.394028in}{1.261301in}}%
\pgfpathlineto{\pgfqpoint{2.408435in}{1.267291in}}%
\pgfpathlineto{\pgfqpoint{2.422841in}{1.418919in}}%
\pgfpathlineto{\pgfqpoint{2.437247in}{1.298671in}}%
\pgfpathlineto{\pgfqpoint{2.451653in}{1.280011in}}%
\pgfpathlineto{\pgfqpoint{2.466059in}{1.363925in}}%
\pgfpathlineto{\pgfqpoint{2.480465in}{2.248158in}}%
\pgfpathlineto{\pgfqpoint{2.494871in}{0.897327in}}%
\pgfpathlineto{\pgfqpoint{2.509277in}{0.707861in}}%
\pgfpathlineto{\pgfqpoint{2.523683in}{0.706809in}}%
\pgfpathlineto{\pgfqpoint{2.538089in}{0.703924in}}%
\pgfpathlineto{\pgfqpoint{2.581307in}{0.702816in}}%
\pgfpathlineto{\pgfqpoint{2.595713in}{0.704477in}}%
\pgfpathlineto{\pgfqpoint{2.610119in}{0.703105in}}%
\pgfpathlineto{\pgfqpoint{2.638931in}{0.703847in}}%
\pgfpathlineto{\pgfqpoint{2.653337in}{0.703091in}}%
\pgfpathlineto{\pgfqpoint{2.667743in}{1.190042in}}%
\pgfpathlineto{\pgfqpoint{2.682149in}{1.154896in}}%
\pgfpathlineto{\pgfqpoint{2.696555in}{1.407756in}}%
\pgfpathlineto{\pgfqpoint{2.710961in}{1.285043in}}%
\pgfpathlineto{\pgfqpoint{2.725367in}{1.241567in}}%
\pgfpathlineto{\pgfqpoint{2.739773in}{1.357461in}}%
\pgfpathlineto{\pgfqpoint{2.754179in}{1.149459in}}%
\pgfpathlineto{\pgfqpoint{2.768586in}{1.189998in}}%
\pgfpathlineto{\pgfqpoint{2.782992in}{1.192666in}}%
\pgfpathlineto{\pgfqpoint{2.797398in}{1.188901in}}%
\pgfpathlineto{\pgfqpoint{2.811804in}{1.195699in}}%
\pgfpathlineto{\pgfqpoint{2.826210in}{1.248327in}}%
\pgfpathlineto{\pgfqpoint{2.840616in}{1.412335in}}%
\pgfpathlineto{\pgfqpoint{2.855022in}{1.150716in}}%
\pgfpathlineto{\pgfqpoint{2.869428in}{1.256452in}}%
\pgfpathlineto{\pgfqpoint{2.883834in}{1.352665in}}%
\pgfpathlineto{\pgfqpoint{2.898240in}{1.306070in}}%
\pgfpathlineto{\pgfqpoint{2.912646in}{1.396491in}}%
\pgfpathlineto{\pgfqpoint{2.927052in}{1.330789in}}%
\pgfpathlineto{\pgfqpoint{2.941458in}{1.376751in}}%
\pgfpathlineto{\pgfqpoint{2.955864in}{1.336656in}}%
\pgfpathlineto{\pgfqpoint{2.970270in}{1.317943in}}%
\pgfpathlineto{\pgfqpoint{2.984676in}{1.264394in}}%
\pgfpathlineto{\pgfqpoint{2.999082in}{1.024375in}}%
\pgfpathlineto{\pgfqpoint{3.013488in}{1.691995in}}%
\pgfpathlineto{\pgfqpoint{3.027894in}{1.270611in}}%
\pgfpathlineto{\pgfqpoint{3.042300in}{1.229405in}}%
\pgfpathlineto{\pgfqpoint{3.056706in}{1.418149in}}%
\pgfpathlineto{\pgfqpoint{3.071112in}{1.383904in}}%
\pgfpathlineto{\pgfqpoint{3.085518in}{1.283944in}}%
\pgfpathlineto{\pgfqpoint{3.099924in}{1.361362in}}%
\pgfpathlineto{\pgfqpoint{3.114331in}{1.305768in}}%
\pgfpathlineto{\pgfqpoint{3.128737in}{1.382432in}}%
\pgfpathlineto{\pgfqpoint{3.143143in}{1.320192in}}%
\pgfpathlineto{\pgfqpoint{3.157549in}{1.235147in}}%
\pgfpathlineto{\pgfqpoint{3.171955in}{1.308903in}}%
\pgfpathlineto{\pgfqpoint{3.186361in}{1.443614in}}%
\pgfpathlineto{\pgfqpoint{3.200767in}{1.230601in}}%
\pgfpathlineto{\pgfqpoint{3.215173in}{1.383472in}}%
\pgfpathlineto{\pgfqpoint{3.229579in}{1.300058in}}%
\pgfpathlineto{\pgfqpoint{3.243985in}{1.263071in}}%
\pgfpathlineto{\pgfqpoint{3.258391in}{1.322264in}}%
\pgfpathlineto{\pgfqpoint{3.272797in}{1.282608in}}%
\pgfpathlineto{\pgfqpoint{3.287203in}{1.354991in}}%
\pgfpathlineto{\pgfqpoint{3.301609in}{1.274962in}}%
\pgfpathlineto{\pgfqpoint{3.316015in}{1.378728in}}%
\pgfpathlineto{\pgfqpoint{3.330421in}{1.347589in}}%
\pgfpathlineto{\pgfqpoint{3.344827in}{1.280566in}}%
\pgfpathlineto{\pgfqpoint{3.359233in}{1.317019in}}%
\pgfpathlineto{\pgfqpoint{3.373639in}{1.238646in}}%
\pgfpathlineto{\pgfqpoint{3.388045in}{1.319128in}}%
\pgfpathlineto{\pgfqpoint{3.402451in}{1.225864in}}%
\pgfpathlineto{\pgfqpoint{3.416857in}{1.317279in}}%
\pgfpathlineto{\pgfqpoint{3.431263in}{1.232049in}}%
\pgfpathlineto{\pgfqpoint{3.445669in}{1.234704in}}%
\pgfpathlineto{\pgfqpoint{3.460076in}{1.272579in}}%
\pgfpathlineto{\pgfqpoint{3.474482in}{1.183955in}}%
\pgfpathlineto{\pgfqpoint{3.488888in}{1.236981in}}%
\pgfpathlineto{\pgfqpoint{3.503294in}{1.362987in}}%
\pgfpathlineto{\pgfqpoint{3.517700in}{1.269982in}}%
\pgfpathlineto{\pgfqpoint{3.532106in}{1.345084in}}%
\pgfpathlineto{\pgfqpoint{3.546512in}{1.330909in}}%
\pgfpathlineto{\pgfqpoint{3.560918in}{1.368855in}}%
\pgfpathlineto{\pgfqpoint{3.575324in}{1.330831in}}%
\pgfpathlineto{\pgfqpoint{3.589730in}{1.673039in}}%
\pgfpathlineto{\pgfqpoint{3.618542in}{1.216155in}}%
\pgfpathlineto{\pgfqpoint{3.632948in}{1.204384in}}%
\pgfpathlineto{\pgfqpoint{3.647354in}{1.780080in}}%
\pgfpathlineto{\pgfqpoint{3.661760in}{1.509620in}}%
\pgfpathlineto{\pgfqpoint{3.676166in}{1.651323in}}%
\pgfpathlineto{\pgfqpoint{3.690572in}{1.167897in}}%
\pgfpathlineto{\pgfqpoint{3.704978in}{1.230872in}}%
\pgfpathlineto{\pgfqpoint{3.719384in}{1.189768in}}%
\pgfpathlineto{\pgfqpoint{3.733790in}{1.154697in}}%
\pgfpathlineto{\pgfqpoint{3.748196in}{1.184216in}}%
\pgfpathlineto{\pgfqpoint{3.762602in}{1.155233in}}%
\pgfpathlineto{\pgfqpoint{3.777008in}{2.517926in}}%
\pgfpathlineto{\pgfqpoint{3.791414in}{1.464357in}}%
\pgfpathlineto{\pgfqpoint{3.805821in}{1.107563in}}%
\pgfpathlineto{\pgfqpoint{3.820227in}{1.600774in}}%
\pgfpathlineto{\pgfqpoint{3.834633in}{1.393431in}}%
\pgfpathlineto{\pgfqpoint{3.849039in}{1.310139in}}%
\pgfpathlineto{\pgfqpoint{3.863445in}{1.124193in}}%
\pgfpathlineto{\pgfqpoint{3.877851in}{1.142437in}}%
\pgfpathlineto{\pgfqpoint{3.892257in}{1.260257in}}%
\pgfpathlineto{\pgfqpoint{3.906663in}{1.215761in}}%
\pgfpathlineto{\pgfqpoint{3.921069in}{1.377195in}}%
\pgfpathlineto{\pgfqpoint{3.935475in}{1.170338in}}%
\pgfpathlineto{\pgfqpoint{3.949881in}{1.253835in}}%
\pgfpathlineto{\pgfqpoint{3.964287in}{1.203076in}}%
\pgfpathlineto{\pgfqpoint{3.978693in}{1.464691in}}%
\pgfpathlineto{\pgfqpoint{3.993099in}{1.613641in}}%
\pgfpathlineto{\pgfqpoint{4.007505in}{1.210346in}}%
\pgfpathlineto{\pgfqpoint{4.021911in}{1.034951in}}%
\pgfpathlineto{\pgfqpoint{4.036317in}{1.028958in}}%
\pgfpathlineto{\pgfqpoint{4.050723in}{0.984342in}}%
\pgfpathlineto{\pgfqpoint{4.065129in}{0.982051in}}%
\pgfpathlineto{\pgfqpoint{4.079535in}{1.970196in}}%
\pgfpathlineto{\pgfqpoint{4.093941in}{1.403913in}}%
\pgfpathlineto{\pgfqpoint{4.108347in}{1.518446in}}%
\pgfpathlineto{\pgfqpoint{4.122753in}{1.280563in}}%
\pgfpathlineto{\pgfqpoint{4.137159in}{1.194918in}}%
\pgfpathlineto{\pgfqpoint{4.151565in}{1.155517in}}%
\pgfpathlineto{\pgfqpoint{4.165972in}{1.688556in}}%
\pgfpathlineto{\pgfqpoint{4.180378in}{4.056000in}}%
\pgfpathlineto{\pgfqpoint{4.194784in}{1.108522in}}%
\pgfpathlineto{\pgfqpoint{4.209190in}{1.099394in}}%
\pgfpathlineto{\pgfqpoint{4.223596in}{1.256049in}}%
\pgfpathlineto{\pgfqpoint{4.238002in}{1.794532in}}%
\pgfpathlineto{\pgfqpoint{4.252408in}{1.344364in}}%
\pgfpathlineto{\pgfqpoint{4.266814in}{1.172271in}}%
\pgfpathlineto{\pgfqpoint{4.281220in}{1.234077in}}%
\pgfpathlineto{\pgfqpoint{4.295626in}{1.259017in}}%
\pgfpathlineto{\pgfqpoint{4.310032in}{1.232660in}}%
\pgfpathlineto{\pgfqpoint{4.324438in}{1.308631in}}%
\pgfpathlineto{\pgfqpoint{4.338844in}{1.309348in}}%
\pgfpathlineto{\pgfqpoint{4.353250in}{1.180535in}}%
\pgfpathlineto{\pgfqpoint{4.367656in}{1.490375in}}%
\pgfpathlineto{\pgfqpoint{4.382062in}{1.326184in}}%
\pgfpathlineto{\pgfqpoint{4.396468in}{1.323110in}}%
\pgfpathlineto{\pgfqpoint{4.410874in}{1.434227in}}%
\pgfpathlineto{\pgfqpoint{4.425280in}{1.315179in}}%
\pgfpathlineto{\pgfqpoint{4.439686in}{1.327044in}}%
\pgfpathlineto{\pgfqpoint{4.454092in}{1.314234in}}%
\pgfpathlineto{\pgfqpoint{4.468498in}{1.299311in}}%
\pgfpathlineto{\pgfqpoint{4.482904in}{1.277968in}}%
\pgfpathlineto{\pgfqpoint{4.497310in}{1.447170in}}%
\pgfpathlineto{\pgfqpoint{4.511717in}{1.013090in}}%
\pgfpathlineto{\pgfqpoint{4.526123in}{1.384481in}}%
\pgfpathlineto{\pgfqpoint{4.540529in}{1.191635in}}%
\pgfpathlineto{\pgfqpoint{4.554935in}{1.297324in}}%
\pgfpathlineto{\pgfqpoint{4.569341in}{1.305330in}}%
\pgfpathlineto{\pgfqpoint{4.583747in}{1.322815in}}%
\pgfpathlineto{\pgfqpoint{4.598153in}{1.344120in}}%
\pgfpathlineto{\pgfqpoint{4.612559in}{1.206199in}}%
\pgfpathlineto{\pgfqpoint{4.626965in}{1.377111in}}%
\pgfpathlineto{\pgfqpoint{4.641371in}{1.328787in}}%
\pgfpathlineto{\pgfqpoint{4.655777in}{1.299412in}}%
\pgfpathlineto{\pgfqpoint{4.670183in}{1.230717in}}%
\pgfpathlineto{\pgfqpoint{4.684589in}{1.232882in}}%
\pgfpathlineto{\pgfqpoint{4.698995in}{1.073815in}}%
\pgfpathlineto{\pgfqpoint{4.713401in}{1.561417in}}%
\pgfpathlineto{\pgfqpoint{4.727807in}{1.033331in}}%
\pgfpathlineto{\pgfqpoint{4.742213in}{1.133546in}}%
\pgfpathlineto{\pgfqpoint{4.756619in}{1.505595in}}%
\pgfpathlineto{\pgfqpoint{4.771025in}{1.558186in}}%
\pgfpathlineto{\pgfqpoint{4.785431in}{1.547971in}}%
\pgfpathlineto{\pgfqpoint{4.799837in}{1.313074in}}%
\pgfpathlineto{\pgfqpoint{4.814243in}{1.292502in}}%
\pgfpathlineto{\pgfqpoint{4.828649in}{1.243632in}}%
\pgfpathlineto{\pgfqpoint{4.843055in}{1.256134in}}%
\pgfpathlineto{\pgfqpoint{4.857462in}{1.399565in}}%
\pgfpathlineto{\pgfqpoint{4.871868in}{1.237110in}}%
\pgfpathlineto{\pgfqpoint{4.886274in}{1.115452in}}%
\pgfpathlineto{\pgfqpoint{4.900680in}{1.092789in}}%
\pgfpathlineto{\pgfqpoint{4.915086in}{1.321640in}}%
\pgfpathlineto{\pgfqpoint{4.929492in}{1.219702in}}%
\pgfpathlineto{\pgfqpoint{4.943898in}{1.327061in}}%
\pgfpathlineto{\pgfqpoint{4.958304in}{1.217781in}}%
\pgfpathlineto{\pgfqpoint{4.972710in}{1.409495in}}%
\pgfpathlineto{\pgfqpoint{4.987116in}{1.394307in}}%
\pgfpathlineto{\pgfqpoint{5.001522in}{1.383445in}}%
\pgfpathlineto{\pgfqpoint{5.015928in}{1.346681in}}%
\pgfpathlineto{\pgfqpoint{5.030334in}{1.339332in}}%
\pgfpathlineto{\pgfqpoint{5.044740in}{1.320543in}}%
\pgfpathlineto{\pgfqpoint{5.059146in}{1.259701in}}%
\pgfpathlineto{\pgfqpoint{5.073552in}{1.327499in}}%
\pgfpathlineto{\pgfqpoint{5.087958in}{1.244990in}}%
\pgfpathlineto{\pgfqpoint{5.102364in}{1.422552in}}%
\pgfpathlineto{\pgfqpoint{5.116770in}{1.235370in}}%
\pgfpathlineto{\pgfqpoint{5.131176in}{1.326281in}}%
\pgfpathlineto{\pgfqpoint{5.145582in}{1.107136in}}%
\pgfpathlineto{\pgfqpoint{5.159988in}{1.638730in}}%
\pgfpathlineto{\pgfqpoint{5.174394in}{1.310158in}}%
\pgfpathlineto{\pgfqpoint{5.203207in}{1.456162in}}%
\pgfpathlineto{\pgfqpoint{5.217613in}{1.392029in}}%
\pgfpathlineto{\pgfqpoint{5.232019in}{1.335734in}}%
\pgfpathlineto{\pgfqpoint{5.246425in}{1.118051in}}%
\pgfpathlineto{\pgfqpoint{5.260831in}{1.584792in}}%
\pgfpathlineto{\pgfqpoint{5.275237in}{1.500445in}}%
\pgfpathlineto{\pgfqpoint{5.289643in}{1.300032in}}%
\pgfpathlineto{\pgfqpoint{5.304049in}{1.050534in}}%
\pgfpathlineto{\pgfqpoint{5.318455in}{1.070683in}}%
\pgfpathlineto{\pgfqpoint{5.332861in}{1.898884in}}%
\pgfpathlineto{\pgfqpoint{5.347267in}{1.341882in}}%
\pgfpathlineto{\pgfqpoint{5.361673in}{1.273295in}}%
\pgfpathlineto{\pgfqpoint{5.376079in}{1.047530in}}%
\pgfpathlineto{\pgfqpoint{5.390485in}{1.040172in}}%
\pgfpathlineto{\pgfqpoint{5.404891in}{0.959619in}}%
\pgfpathlineto{\pgfqpoint{5.419297in}{0.914012in}}%
\pgfpathlineto{\pgfqpoint{5.433703in}{0.904193in}}%
\pgfpathlineto{\pgfqpoint{5.448109in}{0.824176in}}%
\pgfpathlineto{\pgfqpoint{5.462515in}{0.758861in}}%
\pgfpathlineto{\pgfqpoint{5.476921in}{0.725855in}}%
\pgfpathlineto{\pgfqpoint{5.491327in}{0.719434in}}%
\pgfpathlineto{\pgfqpoint{5.505733in}{0.723607in}}%
\pgfpathlineto{\pgfqpoint{5.520139in}{0.716261in}}%
\pgfpathlineto{\pgfqpoint{5.534545in}{0.696000in}}%
\pgfpathlineto{\pgfqpoint{5.534545in}{0.696000in}}%
\pgfusepath{stroke}%
\end{pgfscope}%
\begin{pgfscope}%
\pgfsetrectcap%
\pgfsetmiterjoin%
\pgfsetlinewidth{0.803000pt}%
\definecolor{currentstroke}{rgb}{0.000000,0.000000,0.000000}%
\pgfsetstrokecolor{currentstroke}%
\pgfsetdash{}{0pt}%
\pgfpathmoveto{\pgfqpoint{0.800000in}{0.528000in}}%
\pgfpathlineto{\pgfqpoint{0.800000in}{4.224000in}}%
\pgfusepath{stroke}%
\end{pgfscope}%
\begin{pgfscope}%
\pgfsetrectcap%
\pgfsetmiterjoin%
\pgfsetlinewidth{0.803000pt}%
\definecolor{currentstroke}{rgb}{0.000000,0.000000,0.000000}%
\pgfsetstrokecolor{currentstroke}%
\pgfsetdash{}{0pt}%
\pgfpathmoveto{\pgfqpoint{5.760000in}{0.528000in}}%
\pgfpathlineto{\pgfqpoint{5.760000in}{4.224000in}}%
\pgfusepath{stroke}%
\end{pgfscope}%
\begin{pgfscope}%
\pgfsetrectcap%
\pgfsetmiterjoin%
\pgfsetlinewidth{0.803000pt}%
\definecolor{currentstroke}{rgb}{0.000000,0.000000,0.000000}%
\pgfsetstrokecolor{currentstroke}%
\pgfsetdash{}{0pt}%
\pgfpathmoveto{\pgfqpoint{0.800000in}{0.528000in}}%
\pgfpathlineto{\pgfqpoint{5.760000in}{0.528000in}}%
\pgfusepath{stroke}%
\end{pgfscope}%
\begin{pgfscope}%
\pgfsetrectcap%
\pgfsetmiterjoin%
\pgfsetlinewidth{0.803000pt}%
\definecolor{currentstroke}{rgb}{0.000000,0.000000,0.000000}%
\pgfsetstrokecolor{currentstroke}%
\pgfsetdash{}{0pt}%
\pgfpathmoveto{\pgfqpoint{0.800000in}{4.224000in}}%
\pgfpathlineto{\pgfqpoint{5.760000in}{4.224000in}}%
\pgfusepath{stroke}%
\end{pgfscope}%
\begin{pgfscope}%
\pgfsetbuttcap%
\pgfsetmiterjoin%
\definecolor{currentfill}{rgb}{1.000000,1.000000,1.000000}%
\pgfsetfillcolor{currentfill}%
\pgfsetfillopacity{0.800000}%
\pgfsetlinewidth{1.003750pt}%
\definecolor{currentstroke}{rgb}{0.800000,0.800000,0.800000}%
\pgfsetstrokecolor{currentstroke}%
\pgfsetstrokeopacity{0.800000}%
\pgfsetdash{}{0pt}%
\pgfpathmoveto{\pgfqpoint{5.591944in}{4.025389in}}%
\pgfpathlineto{\pgfqpoint{5.653056in}{4.025389in}}%
\pgfpathquadraticcurveto{\pgfqpoint{5.683611in}{4.025389in}}{\pgfqpoint{5.683611in}{4.055944in}}%
\pgfpathlineto{\pgfqpoint{5.683611in}{4.117056in}}%
\pgfpathquadraticcurveto{\pgfqpoint{5.683611in}{4.147611in}}{\pgfqpoint{5.653056in}{4.147611in}}%
\pgfpathlineto{\pgfqpoint{5.591944in}{4.147611in}}%
\pgfpathquadraticcurveto{\pgfqpoint{5.561389in}{4.147611in}}{\pgfqpoint{5.561389in}{4.117056in}}%
\pgfpathlineto{\pgfqpoint{5.561389in}{4.055944in}}%
\pgfpathquadraticcurveto{\pgfqpoint{5.561389in}{4.025389in}}{\pgfqpoint{5.591944in}{4.025389in}}%
\pgfpathclose%
\pgfusepath{stroke,fill}%
\end{pgfscope}%
\end{pgfpicture}%
\makeatother%
\endgroup%

    \caption{Latency of read operations for every second of experiment 4 in \prettyref{tab:res-3}}
    \label{fig:rlat-pplp}
\end{figure}

\section{Discussion}

Much of the results we obtained were as we expected. Nonetheless, we recognise that our method had limitations. These limitations spanned across our methodology and resources. 
The main limitations are as follows:
\begin{enumerate}
    \item All failures were induced \textit{only} on the primary.
    \item Only one failure was induced per experiment.
    \item Timeout errors created uncertainty about document state.
    \item Uniformly random document selections.
    \item Analysis was based only on the execution history, not the internal replica logs.
\end{enumerate}

\subsection{Failures exclusively on the Primary}
The decision to induce failures exclusively on the primary arose from recognising that the primary replica is the arbiter of all write operations in a replica set. As such, by making the primary fail, we ensure that the replica set \textit{must} go through some kind of recovery flow. Unfortunately, this means our methodology did not account for any behaviours that would arise from when a secondary replica (or multiple secondaries) fail and how this would impact the durability and performance of the replica set. 

\subsection{One failure per experiment}
Our experiment was designed to be analysed in three segments - normal operation, failure and recovery. In order to ensure a distinguishable cutoff point for each segment, we only induced a single failure and fixed it after a specified time period. This fundamentally limited what kinds of errors and scenarios we could induce as part of the experiment. Specifically, we did not investigate a case of repeated or ongoing failures of a node, such as repeated restarts due to misconfiguration. This could stress how elections handle an unstable, but functional, primary.

\subsection{Timeout Errors}
Having inspected the results, we found a large number of errors being reported by the analysis script. The majority of these errors are simply timeouts and are mostly present during the time that the failure is induced in the primary. These timeouts are a result of the election procedure that must take place when the primary crashes in order for MongoDB to select a new primary replica. As there is no arbiter for write operations present during this time, no write operations can be acknowledged. If the read preference is set to \textit{Primary}, no read operations can be acknowledged either. This causes those operations to wait until the primary is available, which may take longer than the configured timeout, resulting in an error. 

We must also address that our experiment harness only queries and modifies documents that have their ID's in the list of acknowledged writes. This has the observable consequence that an update operation could be committed on the replica set but the acknowledgement not be received by the client. We can therefore reason that \textit{Create Document} operations may suffer from the same issue. This kind of ambiguity can easily mask durability failures as it prevents us from having a complete view of all documents in the replica set, in particular preventing us from querying documents that have been committed to the replica set but not entered into the list of acknowledged writes because of a timeout. This is especially problematic since durability failures tend to happen \textit{during} a failure - at the same time as the timeouts.

\subsection{Uniformly random document selection}
It is common knowledge that data is not accessed in a uniform manner. Our experiment used a simple random number generator to decide which document to read or update. As such, our workload did not represent a realistic scenario of operations performed on the database. For example, the uniform selection ensures minimal contention for documents in the queries at the database-level, meaning that we are not observing how MongoDB handles high contention during failures.

\subsection{Execution history, not replica logs}
Our methodology concentrated on detecting write losses by looking at the execution history. We did not inspect the native MongoDB logs as part of our study because we concentrated on finding immediately visible durability failures in MongoDB, from the view of the a user/client of the database. This limitation meant we could not pick up certain durability failures, such as write rollbacks on secondary nodes.

\begin{table}
    \begin{tabular}{@{}lllccc@{}}
    \toprule
    Read Preference   & Read Concern & Write Concern & Throughput & Errors & Lost Writes  \\ \midrule
    Primary           & Local        & Primary       & 763824     & 1177   & 400          \\
    Primary           & Local        & Journaled     & 828721     & 28760  & 331          \\
    Primary           & Majority     & Majority      & 798183     & 31212  & 0            \\
    Primary Preferred & Local        & Primary       & 846215     & 8142   & 141          \\
    Primary Preferred & Local        & Journaled     & 840277     & 9181   & 421          \\
    Primary Preferred & Majority     & Majority      & 765270     & 7499   & 0            \\ \bottomrule
    \end{tabular}
    \caption{Experiment results MongoDB 4.0 with \textit{Shutdown} failure on 0.3 write probability}
    \label{tab:res-1}
\end{table}

\begin{table}
    \begin{tabular}{@{}lllccc@{}}
        \toprule
        Read Preference  & Read Concern & Write Concern & Throughput & Errors & Lost Writes \\ \midrule
        Primary          & Local        & Primary       & 693570     & 101310 & 2223        \\
        Primary          & Local        & Journaled     & 647880     & 89464  & 1733        \\
        Primary          & Majority     & Majority      & 641685     & 86279  & 0           \\
        Primary Preferred & Local        & Primary       & 782852     & 66952  & 1800        \\
        Primary Preferred & Local        & Journaled     & 705703     & 50119  & 180         \\
        Primary Preferred & Majority     & Majority      & 646299     & 51933  & 0           \\ \bottomrule
        \end{tabular}
    \caption{Experiment results MongoDB 4.0 with \textit{Shutdown} failure on 0.7 write probability}
\end{table}

\begin{table}
    \begin{tabular}{@{}lllccc@{}}
        \toprule
        Read Preference  & Read Concern & Write Concern & Throughput & Errors & Lost Writes \\ \midrule
        Primary          & Local        & Primary       & 738601     & 2135   & 261         \\
        Primary          & Local        & Journaled     & 759628     & 25906  & 2834        \\
        Primary          & Majority     & Majority      & 757109     & 21838  & 0           \\
        Primary Preferred & Local        & Primary       & 805828     & 1037   & 706         \\
        Primary Preferred & Local        & Journaled     & 770923     & 11868  & 171         \\
        Primary Preferred & Majority     & Majority      & 777713     & 18867  & 0           \\ \bottomrule
        \end{tabular}
    \caption{Experiment results MongoDB 4.0 with \textit{Poweroff} failure on 0.3 write probability}
    \label{tab:res-3}
\end{table}

\begin{table}
    \begin{tabular}{@{}lllccc@{}}
        \toprule
        Read Preference  & Read Concern & Write Concern & Throughput & Errors & Lost Writes \\ \midrule
        Primary          & Local        & Primary       & 767995     & 59474  & 1763        \\
        Primary          & Local        & Journaled     & 649970     & 49109  & 1217        \\
        Primary          & Majority     & Majority      & 566651     & 50420  & 0           \\
        Primary Preferred & Local        & Primary       & 712128     & 6144   & 4540        \\
        Primary Preferred & Local        & Journaled     & 730408     & 74884  & 238         \\
        Primary Preferred & Majority     & Majority      & 534430     & 33748  & 9           \\ \bottomrule
        \end{tabular}
    \caption{Experiment results MongoDB 4.0 with \textit{Poweroff} failure on 0.7 write probability}
    \label{tab:res-4}
\end{table}

\begin{table}
    \begin{tabular}{@{}lllccc@{}}
        \toprule
        Read Preference  & Read Concern & Write Concern & Throughput & Errors & Lost Writes \\ \midrule
        Primary          & Local        & Primary       & 605527     & 666    & 70          \\
        Primary          & Local        & Journaled     & 613437     & 33398  & 352         \\
        Primary          & Majority     & Majority      & 628248     & 35591  & 0           \\
        Primary Preferred & Local        & Primary       & 672897     & 9760   & 393         \\
        Primary Preferred & Local        & Journaled     & 664797     & 8093   & 581         \\
        Primary Preferred & Majority     & Majority      & 688463     & 10962  & 0           \\ \bottomrule
        \end{tabular}
    \caption{Experiment results MongoDB 3.6-rc0 with \textit{Shutdown} failure on 0.3 write probability}
\end{table}

\begin{table}
    \begin{tabular}{@{}lllccc@{}}
    \toprule
    Read Preference  & Read Concern & Write Concern & Throughput & Errors & Lost Writes \\ \midrule
    Primary          & Local        & Primary       & 569716     & 85219  & 210         \\
    Primary          & Local        & Journaled     & 546575     & 953    & 87          \\
    Primary          & Majority     & Majority      & 412634     & 62548  & 40          \\
    Primary Preferred & Local        & Primary       & 595135     & 44845  & 1097        \\
    Primary Preferred & Local        & Journaled     & 549167     & 51327  & 387         \\
    Primary Preferred & Majority     & Majority      & 536845     & 74019  & 0           \\ \bottomrule
    \end{tabular}
    \label{tab:res-6}
    \caption{Experiment results MongoDB 3.6-rc0 with \textit{Shutdown} failure on 0.7 write probability}
\end{table}

\begin{table}
    \begin{tabular}{@{}lllccc@{}}
        \toprule
        Read Preference  & Read Concern & Write Concern & Throughput & Errors & Lost Writes \\ \midrule
        Primary          & Local        & Primary       & 663695     & 17754  & 1319        \\
        Primary          & Local        & Journaled     & 663009     & 15741  & 1815        \\
        Primary          & Majority     & Majority      & 628013     & 15211  & 0           \\
        Primary Preferred & Local        & Primary       & 706243     & 5633   & 1189        \\
        Primary Preferred & Local        & Journaled     & 676734     & 709    & 335         \\
        Primary Preferred & Majority     & Majority      & 676025     & 5204   & 0           \\ \bottomrule
        \end{tabular}
    \caption{Experiment results MongoDB 3.6-rc0 with \textit{Poweroff} failure on 0.3 write probability}
\end{table}

\begin{table}
    \begin{tabular}{@{}lllccc@{}}
        \toprule
        Read Preference  & Read Concern & Write Concern & Throughput & Errors & Lost Writes \\ \midrule
        Primary          & Local        & Primary       & 472444     & 110326 & 5589        \\
        Primary          & Local        & Journaled     & 597266     & 65285  & 457         \\
        Primary          & Majority     & Majority      & 338798     & 126149 & 2           \\
        Primary Preferred & Local        & Primary       & 581263     & 32181  & 255         \\
        Primary Preferred & Local        & Journaled     & 608816     & 36625  & 2782        \\
        Primary Preferred & Majority     & Majority      & 571397     & 34555  & 0           \\ \bottomrule
        \end{tabular}
    \caption{Experiment results MongoDB 3.6-rc0 with \textit{Poweroff} failure on 0.7 write probability}
    \label{tab:res-8}
\end{table}