\chapter{Results} \label{chap:kdur-results}

In this chapter, we present an empirical exploration of time-till-1-durability in MongoDB based on the theory developed in \prettyref{chap:kdur-theory} and proceed to evaluate the results in light of how MongoDB handles write operations.

\section{Environment}
The experiments were performed on the same harness as the one developed in \prettyref{chap:det-results}, using the \textit{Local} read concern and \textit{Primary} read preference. The harness was configured to \textit{not} induce a failure and perform only write operations (using write probability 1). 

\section{Estimating 1-durability}
We ran two experiments, each for 5 minutes. One with write concern \textit{w:1} and one with \textit{journaled}. The execution histories were processed to count the latencies of each operation. The raw results of this collection can be seen in \prettyref{fig:latencies}.

Given that there is a significant difference between the patterns of \textit{w:1} and \textit{journaled} histories, we use the journaled history and apply equation \prettyref{eq:persist} on every operation to derive the time a write becomes 1-durable. Given that VirtualBox networking is very stable, we subtract the average ping latency to the primary from every operation to determine the time that operation became 1-durable. A comparison between 1-durability and \textit{w:1} write acknowledgement can be found in \prettyref{fig:cdf}.

To further clarify the differences between \textit{w:1} and 1-durability, we present a chart of the difference between the two cumulative plots in \prettyref{fig:diff}.

\begin{figure}
    \centering        
    %% Creator: Matplotlib, PGF backend
%%
%% To include the figure in your LaTeX document, write
%%   \input{<filename>.pgf}
%%
%% Make sure the required packages are loaded in your preamble
%%   \usepackage{pgf}
%%
%% Figures using additional raster images can only be included by \input if
%% they are in the same directory as the main LaTeX file. For loading figures
%% from other directories you can use the `import` package
%%   \usepackage{import}
%% and then include the figures with
%%   \import{<path to file>}{<filename>.pgf}
%%
%% Matplotlib used the following preamble
%%   \usepackage[utf8x]{inputenc}
%%   \usepackage[T1]{fontenc}
%%   \usepackage{lmodern}
%%
\begingroup%
\makeatletter%
\begin{pgfpicture}%
\pgfpathrectangle{\pgfpointorigin}{\pgfqpoint{6.400000in}{4.800000in}}%
\pgfusepath{use as bounding box, clip}%
\begin{pgfscope}%
\pgfsetbuttcap%
\pgfsetmiterjoin%
\definecolor{currentfill}{rgb}{1.000000,1.000000,1.000000}%
\pgfsetfillcolor{currentfill}%
\pgfsetlinewidth{0.000000pt}%
\definecolor{currentstroke}{rgb}{1.000000,1.000000,1.000000}%
\pgfsetstrokecolor{currentstroke}%
\pgfsetdash{}{0pt}%
\pgfpathmoveto{\pgfqpoint{0.000000in}{0.000000in}}%
\pgfpathlineto{\pgfqpoint{6.400000in}{0.000000in}}%
\pgfpathlineto{\pgfqpoint{6.400000in}{4.800000in}}%
\pgfpathlineto{\pgfqpoint{0.000000in}{4.800000in}}%
\pgfpathclose%
\pgfusepath{fill}%
\end{pgfscope}%
\begin{pgfscope}%
\pgfsetbuttcap%
\pgfsetmiterjoin%
\definecolor{currentfill}{rgb}{1.000000,1.000000,1.000000}%
\pgfsetfillcolor{currentfill}%
\pgfsetlinewidth{0.000000pt}%
\definecolor{currentstroke}{rgb}{0.000000,0.000000,0.000000}%
\pgfsetstrokecolor{currentstroke}%
\pgfsetstrokeopacity{0.000000}%
\pgfsetdash{}{0pt}%
\pgfpathmoveto{\pgfqpoint{0.800000in}{0.528000in}}%
\pgfpathlineto{\pgfqpoint{5.760000in}{0.528000in}}%
\pgfpathlineto{\pgfqpoint{5.760000in}{4.224000in}}%
\pgfpathlineto{\pgfqpoint{0.800000in}{4.224000in}}%
\pgfpathclose%
\pgfusepath{fill}%
\end{pgfscope}%
\begin{pgfscope}%
\pgfsetbuttcap%
\pgfsetroundjoin%
\definecolor{currentfill}{rgb}{0.000000,0.000000,0.000000}%
\pgfsetfillcolor{currentfill}%
\pgfsetlinewidth{0.803000pt}%
\definecolor{currentstroke}{rgb}{0.000000,0.000000,0.000000}%
\pgfsetstrokecolor{currentstroke}%
\pgfsetdash{}{0pt}%
\pgfsys@defobject{currentmarker}{\pgfqpoint{0.000000in}{-0.048611in}}{\pgfqpoint{0.000000in}{0.000000in}}{%
\pgfpathmoveto{\pgfqpoint{0.000000in}{0.000000in}}%
\pgfpathlineto{\pgfqpoint{0.000000in}{-0.048611in}}%
\pgfusepath{stroke,fill}%
}%
\begin{pgfscope}%
\pgfsys@transformshift{1.020936in}{0.528000in}%
\pgfsys@useobject{currentmarker}{}%
\end{pgfscope}%
\end{pgfscope}%
\begin{pgfscope}%
\pgftext[x=1.020936in,y=0.430778in,,top]{\fontsize{11.000000}{13.200000}\selectfont \(\displaystyle 0\)}%
\end{pgfscope}%
\begin{pgfscope}%
\pgfsetbuttcap%
\pgfsetroundjoin%
\definecolor{currentfill}{rgb}{0.000000,0.000000,0.000000}%
\pgfsetfillcolor{currentfill}%
\pgfsetlinewidth{0.803000pt}%
\definecolor{currentstroke}{rgb}{0.000000,0.000000,0.000000}%
\pgfsetstrokecolor{currentstroke}%
\pgfsetdash{}{0pt}%
\pgfsys@defobject{currentmarker}{\pgfqpoint{0.000000in}{-0.048611in}}{\pgfqpoint{0.000000in}{0.000000in}}{%
\pgfpathmoveto{\pgfqpoint{0.000000in}{0.000000in}}%
\pgfpathlineto{\pgfqpoint{0.000000in}{-0.048611in}}%
\pgfusepath{stroke,fill}%
}%
\begin{pgfscope}%
\pgfsys@transformshift{1.924562in}{0.528000in}%
\pgfsys@useobject{currentmarker}{}%
\end{pgfscope}%
\end{pgfscope}%
\begin{pgfscope}%
\pgftext[x=1.924562in,y=0.430778in,,top]{\fontsize{11.000000}{13.200000}\selectfont \(\displaystyle 200\)}%
\end{pgfscope}%
\begin{pgfscope}%
\pgfsetbuttcap%
\pgfsetroundjoin%
\definecolor{currentfill}{rgb}{0.000000,0.000000,0.000000}%
\pgfsetfillcolor{currentfill}%
\pgfsetlinewidth{0.803000pt}%
\definecolor{currentstroke}{rgb}{0.000000,0.000000,0.000000}%
\pgfsetstrokecolor{currentstroke}%
\pgfsetdash{}{0pt}%
\pgfsys@defobject{currentmarker}{\pgfqpoint{0.000000in}{-0.048611in}}{\pgfqpoint{0.000000in}{0.000000in}}{%
\pgfpathmoveto{\pgfqpoint{0.000000in}{0.000000in}}%
\pgfpathlineto{\pgfqpoint{0.000000in}{-0.048611in}}%
\pgfusepath{stroke,fill}%
}%
\begin{pgfscope}%
\pgfsys@transformshift{2.828187in}{0.528000in}%
\pgfsys@useobject{currentmarker}{}%
\end{pgfscope}%
\end{pgfscope}%
\begin{pgfscope}%
\pgftext[x=2.828187in,y=0.430778in,,top]{\fontsize{11.000000}{13.200000}\selectfont \(\displaystyle 400\)}%
\end{pgfscope}%
\begin{pgfscope}%
\pgfsetbuttcap%
\pgfsetroundjoin%
\definecolor{currentfill}{rgb}{0.000000,0.000000,0.000000}%
\pgfsetfillcolor{currentfill}%
\pgfsetlinewidth{0.803000pt}%
\definecolor{currentstroke}{rgb}{0.000000,0.000000,0.000000}%
\pgfsetstrokecolor{currentstroke}%
\pgfsetdash{}{0pt}%
\pgfsys@defobject{currentmarker}{\pgfqpoint{0.000000in}{-0.048611in}}{\pgfqpoint{0.000000in}{0.000000in}}{%
\pgfpathmoveto{\pgfqpoint{0.000000in}{0.000000in}}%
\pgfpathlineto{\pgfqpoint{0.000000in}{-0.048611in}}%
\pgfusepath{stroke,fill}%
}%
\begin{pgfscope}%
\pgfsys@transformshift{3.731813in}{0.528000in}%
\pgfsys@useobject{currentmarker}{}%
\end{pgfscope}%
\end{pgfscope}%
\begin{pgfscope}%
\pgftext[x=3.731813in,y=0.430778in,,top]{\fontsize{11.000000}{13.200000}\selectfont \(\displaystyle 600\)}%
\end{pgfscope}%
\begin{pgfscope}%
\pgfsetbuttcap%
\pgfsetroundjoin%
\definecolor{currentfill}{rgb}{0.000000,0.000000,0.000000}%
\pgfsetfillcolor{currentfill}%
\pgfsetlinewidth{0.803000pt}%
\definecolor{currentstroke}{rgb}{0.000000,0.000000,0.000000}%
\pgfsetstrokecolor{currentstroke}%
\pgfsetdash{}{0pt}%
\pgfsys@defobject{currentmarker}{\pgfqpoint{0.000000in}{-0.048611in}}{\pgfqpoint{0.000000in}{0.000000in}}{%
\pgfpathmoveto{\pgfqpoint{0.000000in}{0.000000in}}%
\pgfpathlineto{\pgfqpoint{0.000000in}{-0.048611in}}%
\pgfusepath{stroke,fill}%
}%
\begin{pgfscope}%
\pgfsys@transformshift{4.635438in}{0.528000in}%
\pgfsys@useobject{currentmarker}{}%
\end{pgfscope}%
\end{pgfscope}%
\begin{pgfscope}%
\pgftext[x=4.635438in,y=0.430778in,,top]{\fontsize{11.000000}{13.200000}\selectfont \(\displaystyle 800\)}%
\end{pgfscope}%
\begin{pgfscope}%
\pgfsetbuttcap%
\pgfsetroundjoin%
\definecolor{currentfill}{rgb}{0.000000,0.000000,0.000000}%
\pgfsetfillcolor{currentfill}%
\pgfsetlinewidth{0.803000pt}%
\definecolor{currentstroke}{rgb}{0.000000,0.000000,0.000000}%
\pgfsetstrokecolor{currentstroke}%
\pgfsetdash{}{0pt}%
\pgfsys@defobject{currentmarker}{\pgfqpoint{0.000000in}{-0.048611in}}{\pgfqpoint{0.000000in}{0.000000in}}{%
\pgfpathmoveto{\pgfqpoint{0.000000in}{0.000000in}}%
\pgfpathlineto{\pgfqpoint{0.000000in}{-0.048611in}}%
\pgfusepath{stroke,fill}%
}%
\begin{pgfscope}%
\pgfsys@transformshift{5.539064in}{0.528000in}%
\pgfsys@useobject{currentmarker}{}%
\end{pgfscope}%
\end{pgfscope}%
\begin{pgfscope}%
\pgftext[x=5.539064in,y=0.430778in,,top]{\fontsize{11.000000}{13.200000}\selectfont \(\displaystyle 1000\)}%
\end{pgfscope}%
\begin{pgfscope}%
\pgftext[x=3.280000in,y=0.240271in,,top]{\fontsize{11.000000}{13.200000}\selectfont Latency (in milliseconds)}%
\end{pgfscope}%
\begin{pgfscope}%
\pgfsetbuttcap%
\pgfsetroundjoin%
\definecolor{currentfill}{rgb}{0.000000,0.000000,0.000000}%
\pgfsetfillcolor{currentfill}%
\pgfsetlinewidth{0.803000pt}%
\definecolor{currentstroke}{rgb}{0.000000,0.000000,0.000000}%
\pgfsetstrokecolor{currentstroke}%
\pgfsetdash{}{0pt}%
\pgfsys@defobject{currentmarker}{\pgfqpoint{-0.048611in}{0.000000in}}{\pgfqpoint{0.000000in}{0.000000in}}{%
\pgfpathmoveto{\pgfqpoint{0.000000in}{0.000000in}}%
\pgfpathlineto{\pgfqpoint{-0.048611in}{0.000000in}}%
\pgfusepath{stroke,fill}%
}%
\begin{pgfscope}%
\pgfsys@transformshift{0.800000in}{0.696000in}%
\pgfsys@useobject{currentmarker}{}%
\end{pgfscope}%
\end{pgfscope}%
\begin{pgfscope}%
\pgftext[x=0.627981in,y=0.643378in,left,base]{\fontsize{11.000000}{13.200000}\selectfont \(\displaystyle 0\)}%
\end{pgfscope}%
\begin{pgfscope}%
\pgfsetbuttcap%
\pgfsetroundjoin%
\definecolor{currentfill}{rgb}{0.000000,0.000000,0.000000}%
\pgfsetfillcolor{currentfill}%
\pgfsetlinewidth{0.803000pt}%
\definecolor{currentstroke}{rgb}{0.000000,0.000000,0.000000}%
\pgfsetstrokecolor{currentstroke}%
\pgfsetdash{}{0pt}%
\pgfsys@defobject{currentmarker}{\pgfqpoint{-0.048611in}{0.000000in}}{\pgfqpoint{0.000000in}{0.000000in}}{%
\pgfpathmoveto{\pgfqpoint{0.000000in}{0.000000in}}%
\pgfpathlineto{\pgfqpoint{-0.048611in}{0.000000in}}%
\pgfusepath{stroke,fill}%
}%
\begin{pgfscope}%
\pgfsys@transformshift{0.800000in}{1.129492in}%
\pgfsys@useobject{currentmarker}{}%
\end{pgfscope}%
\end{pgfscope}%
\begin{pgfscope}%
\pgftext[x=0.403588in,y=1.076870in,left,base]{\fontsize{11.000000}{13.200000}\selectfont \(\displaystyle 1000\)}%
\end{pgfscope}%
\begin{pgfscope}%
\pgfsetbuttcap%
\pgfsetroundjoin%
\definecolor{currentfill}{rgb}{0.000000,0.000000,0.000000}%
\pgfsetfillcolor{currentfill}%
\pgfsetlinewidth{0.803000pt}%
\definecolor{currentstroke}{rgb}{0.000000,0.000000,0.000000}%
\pgfsetstrokecolor{currentstroke}%
\pgfsetdash{}{0pt}%
\pgfsys@defobject{currentmarker}{\pgfqpoint{-0.048611in}{0.000000in}}{\pgfqpoint{0.000000in}{0.000000in}}{%
\pgfpathmoveto{\pgfqpoint{0.000000in}{0.000000in}}%
\pgfpathlineto{\pgfqpoint{-0.048611in}{0.000000in}}%
\pgfusepath{stroke,fill}%
}%
\begin{pgfscope}%
\pgfsys@transformshift{0.800000in}{1.562985in}%
\pgfsys@useobject{currentmarker}{}%
\end{pgfscope}%
\end{pgfscope}%
\begin{pgfscope}%
\pgftext[x=0.403588in,y=1.510363in,left,base]{\fontsize{11.000000}{13.200000}\selectfont \(\displaystyle 2000\)}%
\end{pgfscope}%
\begin{pgfscope}%
\pgfsetbuttcap%
\pgfsetroundjoin%
\definecolor{currentfill}{rgb}{0.000000,0.000000,0.000000}%
\pgfsetfillcolor{currentfill}%
\pgfsetlinewidth{0.803000pt}%
\definecolor{currentstroke}{rgb}{0.000000,0.000000,0.000000}%
\pgfsetstrokecolor{currentstroke}%
\pgfsetdash{}{0pt}%
\pgfsys@defobject{currentmarker}{\pgfqpoint{-0.048611in}{0.000000in}}{\pgfqpoint{0.000000in}{0.000000in}}{%
\pgfpathmoveto{\pgfqpoint{0.000000in}{0.000000in}}%
\pgfpathlineto{\pgfqpoint{-0.048611in}{0.000000in}}%
\pgfusepath{stroke,fill}%
}%
\begin{pgfscope}%
\pgfsys@transformshift{0.800000in}{1.996477in}%
\pgfsys@useobject{currentmarker}{}%
\end{pgfscope}%
\end{pgfscope}%
\begin{pgfscope}%
\pgftext[x=0.403588in,y=1.943855in,left,base]{\fontsize{11.000000}{13.200000}\selectfont \(\displaystyle 3000\)}%
\end{pgfscope}%
\begin{pgfscope}%
\pgfsetbuttcap%
\pgfsetroundjoin%
\definecolor{currentfill}{rgb}{0.000000,0.000000,0.000000}%
\pgfsetfillcolor{currentfill}%
\pgfsetlinewidth{0.803000pt}%
\definecolor{currentstroke}{rgb}{0.000000,0.000000,0.000000}%
\pgfsetstrokecolor{currentstroke}%
\pgfsetdash{}{0pt}%
\pgfsys@defobject{currentmarker}{\pgfqpoint{-0.048611in}{0.000000in}}{\pgfqpoint{0.000000in}{0.000000in}}{%
\pgfpathmoveto{\pgfqpoint{0.000000in}{0.000000in}}%
\pgfpathlineto{\pgfqpoint{-0.048611in}{0.000000in}}%
\pgfusepath{stroke,fill}%
}%
\begin{pgfscope}%
\pgfsys@transformshift{0.800000in}{2.429970in}%
\pgfsys@useobject{currentmarker}{}%
\end{pgfscope}%
\end{pgfscope}%
\begin{pgfscope}%
\pgftext[x=0.403588in,y=2.377348in,left,base]{\fontsize{11.000000}{13.200000}\selectfont \(\displaystyle 4000\)}%
\end{pgfscope}%
\begin{pgfscope}%
\pgfsetbuttcap%
\pgfsetroundjoin%
\definecolor{currentfill}{rgb}{0.000000,0.000000,0.000000}%
\pgfsetfillcolor{currentfill}%
\pgfsetlinewidth{0.803000pt}%
\definecolor{currentstroke}{rgb}{0.000000,0.000000,0.000000}%
\pgfsetstrokecolor{currentstroke}%
\pgfsetdash{}{0pt}%
\pgfsys@defobject{currentmarker}{\pgfqpoint{-0.048611in}{0.000000in}}{\pgfqpoint{0.000000in}{0.000000in}}{%
\pgfpathmoveto{\pgfqpoint{0.000000in}{0.000000in}}%
\pgfpathlineto{\pgfqpoint{-0.048611in}{0.000000in}}%
\pgfusepath{stroke,fill}%
}%
\begin{pgfscope}%
\pgfsys@transformshift{0.800000in}{2.863462in}%
\pgfsys@useobject{currentmarker}{}%
\end{pgfscope}%
\end{pgfscope}%
\begin{pgfscope}%
\pgftext[x=0.403588in,y=2.810840in,left,base]{\fontsize{11.000000}{13.200000}\selectfont \(\displaystyle 5000\)}%
\end{pgfscope}%
\begin{pgfscope}%
\pgfsetbuttcap%
\pgfsetroundjoin%
\definecolor{currentfill}{rgb}{0.000000,0.000000,0.000000}%
\pgfsetfillcolor{currentfill}%
\pgfsetlinewidth{0.803000pt}%
\definecolor{currentstroke}{rgb}{0.000000,0.000000,0.000000}%
\pgfsetstrokecolor{currentstroke}%
\pgfsetdash{}{0pt}%
\pgfsys@defobject{currentmarker}{\pgfqpoint{-0.048611in}{0.000000in}}{\pgfqpoint{0.000000in}{0.000000in}}{%
\pgfpathmoveto{\pgfqpoint{0.000000in}{0.000000in}}%
\pgfpathlineto{\pgfqpoint{-0.048611in}{0.000000in}}%
\pgfusepath{stroke,fill}%
}%
\begin{pgfscope}%
\pgfsys@transformshift{0.800000in}{3.296955in}%
\pgfsys@useobject{currentmarker}{}%
\end{pgfscope}%
\end{pgfscope}%
\begin{pgfscope}%
\pgftext[x=0.403588in,y=3.244332in,left,base]{\fontsize{11.000000}{13.200000}\selectfont \(\displaystyle 6000\)}%
\end{pgfscope}%
\begin{pgfscope}%
\pgfsetbuttcap%
\pgfsetroundjoin%
\definecolor{currentfill}{rgb}{0.000000,0.000000,0.000000}%
\pgfsetfillcolor{currentfill}%
\pgfsetlinewidth{0.803000pt}%
\definecolor{currentstroke}{rgb}{0.000000,0.000000,0.000000}%
\pgfsetstrokecolor{currentstroke}%
\pgfsetdash{}{0pt}%
\pgfsys@defobject{currentmarker}{\pgfqpoint{-0.048611in}{0.000000in}}{\pgfqpoint{0.000000in}{0.000000in}}{%
\pgfpathmoveto{\pgfqpoint{0.000000in}{0.000000in}}%
\pgfpathlineto{\pgfqpoint{-0.048611in}{0.000000in}}%
\pgfusepath{stroke,fill}%
}%
\begin{pgfscope}%
\pgfsys@transformshift{0.800000in}{3.730447in}%
\pgfsys@useobject{currentmarker}{}%
\end{pgfscope}%
\end{pgfscope}%
\begin{pgfscope}%
\pgftext[x=0.403588in,y=3.677825in,left,base]{\fontsize{11.000000}{13.200000}\selectfont \(\displaystyle 7000\)}%
\end{pgfscope}%
\begin{pgfscope}%
\pgfsetbuttcap%
\pgfsetroundjoin%
\definecolor{currentfill}{rgb}{0.000000,0.000000,0.000000}%
\pgfsetfillcolor{currentfill}%
\pgfsetlinewidth{0.803000pt}%
\definecolor{currentstroke}{rgb}{0.000000,0.000000,0.000000}%
\pgfsetstrokecolor{currentstroke}%
\pgfsetdash{}{0pt}%
\pgfsys@defobject{currentmarker}{\pgfqpoint{-0.048611in}{0.000000in}}{\pgfqpoint{0.000000in}{0.000000in}}{%
\pgfpathmoveto{\pgfqpoint{0.000000in}{0.000000in}}%
\pgfpathlineto{\pgfqpoint{-0.048611in}{0.000000in}}%
\pgfusepath{stroke,fill}%
}%
\begin{pgfscope}%
\pgfsys@transformshift{0.800000in}{4.163940in}%
\pgfsys@useobject{currentmarker}{}%
\end{pgfscope}%
\end{pgfscope}%
\begin{pgfscope}%
\pgftext[x=0.403588in,y=4.111317in,left,base]{\fontsize{11.000000}{13.200000}\selectfont \(\displaystyle 8000\)}%
\end{pgfscope}%
\begin{pgfscope}%
\pgftext[x=0.348033in,y=2.376000in,,bottom,rotate=90.000000]{\fontsize{11.000000}{13.200000}\selectfont Number of operations}%
\end{pgfscope}%
\begin{pgfscope}%
\pgfpathrectangle{\pgfqpoint{0.800000in}{0.528000in}}{\pgfqpoint{4.960000in}{3.696000in}}%
\pgfusepath{clip}%
\pgfsetrectcap%
\pgfsetroundjoin%
\pgfsetlinewidth{1.505625pt}%
\definecolor{currentstroke}{rgb}{0.121569,0.466667,0.705882}%
\pgfsetstrokecolor{currentstroke}%
\pgfsetdash{}{0pt}%
\pgfpathmoveto{\pgfqpoint{1.025455in}{1.752855in}}%
\pgfpathlineto{\pgfqpoint{1.029973in}{1.641447in}}%
\pgfpathlineto{\pgfqpoint{1.034491in}{1.632344in}}%
\pgfpathlineto{\pgfqpoint{1.039009in}{1.615871in}}%
\pgfpathlineto{\pgfqpoint{1.043527in}{1.570788in}}%
\pgfpathlineto{\pgfqpoint{1.048045in}{1.623674in}}%
\pgfpathlineto{\pgfqpoint{1.052563in}{1.613270in}}%
\pgfpathlineto{\pgfqpoint{1.061600in}{1.502296in}}%
\pgfpathlineto{\pgfqpoint{1.066118in}{1.454178in}}%
\pgfpathlineto{\pgfqpoint{1.070636in}{1.436839in}}%
\pgfpathlineto{\pgfqpoint{1.075154in}{1.357943in}}%
\pgfpathlineto{\pgfqpoint{1.079672in}{1.379184in}}%
\pgfpathlineto{\pgfqpoint{1.084190in}{1.332367in}}%
\pgfpathlineto{\pgfqpoint{1.097745in}{1.240467in}}%
\pgfpathlineto{\pgfqpoint{1.102263in}{1.184979in}}%
\pgfpathlineto{\pgfqpoint{1.106781in}{1.148133in}}%
\pgfpathlineto{\pgfqpoint{1.111299in}{1.142064in}}%
\pgfpathlineto{\pgfqpoint{1.115817in}{1.080941in}}%
\pgfpathlineto{\pgfqpoint{1.120335in}{1.077473in}}%
\pgfpathlineto{\pgfqpoint{1.124853in}{1.051897in}}%
\pgfpathlineto{\pgfqpoint{1.129371in}{1.044961in}}%
\pgfpathlineto{\pgfqpoint{1.133890in}{1.019819in}}%
\pgfpathlineto{\pgfqpoint{1.138408in}{1.002479in}}%
\pgfpathlineto{\pgfqpoint{1.147444in}{0.945692in}}%
\pgfpathlineto{\pgfqpoint{1.151962in}{0.924451in}}%
\pgfpathlineto{\pgfqpoint{1.156480in}{0.941790in}}%
\pgfpathlineto{\pgfqpoint{1.160998in}{0.890638in}}%
\pgfpathlineto{\pgfqpoint{1.165516in}{0.902342in}}%
\pgfpathlineto{\pgfqpoint{1.174553in}{0.842087in}}%
\pgfpathlineto{\pgfqpoint{1.179071in}{0.838186in}}%
\pgfpathlineto{\pgfqpoint{1.183589in}{0.841220in}}%
\pgfpathlineto{\pgfqpoint{1.188107in}{0.829082in}}%
\pgfpathlineto{\pgfqpoint{1.192625in}{0.837752in}}%
\pgfpathlineto{\pgfqpoint{1.197143in}{0.825181in}}%
\pgfpathlineto{\pgfqpoint{1.201662in}{0.835151in}}%
\pgfpathlineto{\pgfqpoint{1.206180in}{0.827782in}}%
\pgfpathlineto{\pgfqpoint{1.210698in}{0.845988in}}%
\pgfpathlineto{\pgfqpoint{1.215216in}{0.857693in}}%
\pgfpathlineto{\pgfqpoint{1.219734in}{0.886303in}}%
\pgfpathlineto{\pgfqpoint{1.224252in}{0.924451in}}%
\pgfpathlineto{\pgfqpoint{1.228770in}{0.944825in}}%
\pgfpathlineto{\pgfqpoint{1.237807in}{1.063602in}}%
\pgfpathlineto{\pgfqpoint{1.242325in}{1.126025in}}%
\pgfpathlineto{\pgfqpoint{1.246843in}{1.347973in}}%
\pgfpathlineto{\pgfqpoint{1.251361in}{1.315461in}}%
\pgfpathlineto{\pgfqpoint{1.255879in}{1.435105in}}%
\pgfpathlineto{\pgfqpoint{1.260397in}{1.496661in}}%
\pgfpathlineto{\pgfqpoint{1.264915in}{1.742884in}}%
\pgfpathlineto{\pgfqpoint{1.273952in}{2.383153in}}%
\pgfpathlineto{\pgfqpoint{1.278470in}{2.811877in}}%
\pgfpathlineto{\pgfqpoint{1.282988in}{2.925018in}}%
\pgfpathlineto{\pgfqpoint{1.287506in}{2.944959in}}%
\pgfpathlineto{\pgfqpoint{1.292024in}{3.285684in}}%
\pgfpathlineto{\pgfqpoint{1.296542in}{3.500696in}}%
\pgfpathlineto{\pgfqpoint{1.301060in}{3.795471in}}%
\pgfpathlineto{\pgfqpoint{1.305578in}{3.819747in}}%
\pgfpathlineto{\pgfqpoint{1.310097in}{3.826249in}}%
\pgfpathlineto{\pgfqpoint{1.314615in}{4.056000in}}%
\pgfpathlineto{\pgfqpoint{1.319133in}{3.809343in}}%
\pgfpathlineto{\pgfqpoint{1.323651in}{3.720477in}}%
\pgfpathlineto{\pgfqpoint{1.328169in}{3.785934in}}%
\pgfpathlineto{\pgfqpoint{1.332687in}{3.953262in}}%
\pgfpathlineto{\pgfqpoint{1.337205in}{4.017853in}}%
\pgfpathlineto{\pgfqpoint{1.341723in}{4.033458in}}%
\pgfpathlineto{\pgfqpoint{1.346242in}{3.977104in}}%
\pgfpathlineto{\pgfqpoint{1.355278in}{3.625975in}}%
\pgfpathlineto{\pgfqpoint{1.359796in}{3.322097in}}%
\pgfpathlineto{\pgfqpoint{1.364314in}{3.151735in}}%
\pgfpathlineto{\pgfqpoint{1.368832in}{2.882969in}}%
\pgfpathlineto{\pgfqpoint{1.391423in}{2.148200in}}%
\pgfpathlineto{\pgfqpoint{1.400459in}{1.983906in}}%
\pgfpathlineto{\pgfqpoint{1.404977in}{1.849957in}}%
\pgfpathlineto{\pgfqpoint{1.414013in}{1.676993in}}%
\pgfpathlineto{\pgfqpoint{1.423050in}{1.565152in}}%
\pgfpathlineto{\pgfqpoint{1.427568in}{1.466316in}}%
\pgfpathlineto{\pgfqpoint{1.432086in}{1.421666in}}%
\pgfpathlineto{\pgfqpoint{1.436604in}{1.327599in}}%
\pgfpathlineto{\pgfqpoint{1.441122in}{1.269077in}}%
\pgfpathlineto{\pgfqpoint{1.445640in}{1.277747in}}%
\pgfpathlineto{\pgfqpoint{1.450158in}{1.276013in}}%
\pgfpathlineto{\pgfqpoint{1.454677in}{1.243067in}}%
\pgfpathlineto{\pgfqpoint{1.459195in}{1.229629in}}%
\pgfpathlineto{\pgfqpoint{1.463713in}{1.184979in}}%
\pgfpathlineto{\pgfqpoint{1.468231in}{1.108251in}}%
\pgfpathlineto{\pgfqpoint{1.472749in}{1.091345in}}%
\pgfpathlineto{\pgfqpoint{1.477267in}{1.093513in}}%
\pgfpathlineto{\pgfqpoint{1.481785in}{1.081375in}}%
\pgfpathlineto{\pgfqpoint{1.490822in}{0.973435in}}%
\pgfpathlineto{\pgfqpoint{1.495340in}{0.980371in}}%
\pgfpathlineto{\pgfqpoint{1.499858in}{1.012016in}}%
\pgfpathlineto{\pgfqpoint{1.504376in}{0.963898in}}%
\pgfpathlineto{\pgfqpoint{1.508894in}{0.977337in}}%
\pgfpathlineto{\pgfqpoint{1.513412in}{1.004647in}}%
\pgfpathlineto{\pgfqpoint{1.517930in}{1.020252in}}%
\pgfpathlineto{\pgfqpoint{1.522449in}{1.025454in}}%
\pgfpathlineto{\pgfqpoint{1.526967in}{1.078774in}}%
\pgfpathlineto{\pgfqpoint{1.531485in}{1.110852in}}%
\pgfpathlineto{\pgfqpoint{1.536003in}{1.189748in}}%
\pgfpathlineto{\pgfqpoint{1.540521in}{1.138162in}}%
\pgfpathlineto{\pgfqpoint{1.545039in}{1.237432in}}%
\pgfpathlineto{\pgfqpoint{1.554075in}{1.305057in}}%
\pgfpathlineto{\pgfqpoint{1.558594in}{1.378317in}}%
\pgfpathlineto{\pgfqpoint{1.563112in}{1.471951in}}%
\pgfpathlineto{\pgfqpoint{1.567630in}{1.603300in}}%
\pgfpathlineto{\pgfqpoint{1.572148in}{1.550847in}}%
\pgfpathlineto{\pgfqpoint{1.576666in}{1.672658in}}%
\pgfpathlineto{\pgfqpoint{1.581184in}{1.702136in}}%
\pgfpathlineto{\pgfqpoint{1.585702in}{1.791435in}}%
\pgfpathlineto{\pgfqpoint{1.590220in}{1.761091in}}%
\pgfpathlineto{\pgfqpoint{1.599257in}{1.924084in}}%
\pgfpathlineto{\pgfqpoint{1.612811in}{2.146899in}}%
\pgfpathlineto{\pgfqpoint{1.617329in}{2.258307in}}%
\pgfpathlineto{\pgfqpoint{1.621847in}{2.423467in}}%
\pgfpathlineto{\pgfqpoint{1.626365in}{2.496294in}}%
\pgfpathlineto{\pgfqpoint{1.630884in}{2.638480in}}%
\pgfpathlineto{\pgfqpoint{1.635402in}{2.709572in}}%
\pgfpathlineto{\pgfqpoint{1.639920in}{2.857827in}}%
\pgfpathlineto{\pgfqpoint{1.644438in}{2.914181in}}%
\pgfpathlineto{\pgfqpoint{1.648956in}{3.095814in}}%
\pgfpathlineto{\pgfqpoint{1.653474in}{2.964033in}}%
\pgfpathlineto{\pgfqpoint{1.657992in}{3.016485in}}%
\pgfpathlineto{\pgfqpoint{1.662510in}{3.015185in}}%
\pgfpathlineto{\pgfqpoint{1.667029in}{2.932821in}}%
\pgfpathlineto{\pgfqpoint{1.671547in}{2.747286in}}%
\pgfpathlineto{\pgfqpoint{1.676065in}{2.692233in}}%
\pgfpathlineto{\pgfqpoint{1.685101in}{2.489792in}}%
\pgfpathlineto{\pgfqpoint{1.689619in}{2.528373in}}%
\pgfpathlineto{\pgfqpoint{1.694137in}{2.403527in}}%
\pgfpathlineto{\pgfqpoint{1.698655in}{2.352808in}}%
\pgfpathlineto{\pgfqpoint{1.703174in}{2.323764in}}%
\pgfpathlineto{\pgfqpoint{1.707692in}{2.322464in}}%
\pgfpathlineto{\pgfqpoint{1.716728in}{2.035925in}}%
\pgfpathlineto{\pgfqpoint{1.721246in}{2.004280in}}%
\pgfpathlineto{\pgfqpoint{1.725764in}{1.848656in}}%
\pgfpathlineto{\pgfqpoint{1.730282in}{1.737249in}}%
\pgfpathlineto{\pgfqpoint{1.734801in}{1.731180in}}%
\pgfpathlineto{\pgfqpoint{1.739319in}{1.646215in}}%
\pgfpathlineto{\pgfqpoint{1.748355in}{1.534808in}}%
\pgfpathlineto{\pgfqpoint{1.752873in}{1.513133in}}%
\pgfpathlineto{\pgfqpoint{1.757391in}{1.441607in}}%
\pgfpathlineto{\pgfqpoint{1.766427in}{1.376583in}}%
\pgfpathlineto{\pgfqpoint{1.770946in}{1.371815in}}%
\pgfpathlineto{\pgfqpoint{1.775464in}{1.348406in}}%
\pgfpathlineto{\pgfqpoint{1.779982in}{1.292919in}}%
\pgfpathlineto{\pgfqpoint{1.784500in}{1.255205in}}%
\pgfpathlineto{\pgfqpoint{1.789018in}{1.234831in}}%
\pgfpathlineto{\pgfqpoint{1.793536in}{1.250870in}}%
\pgfpathlineto{\pgfqpoint{1.798054in}{1.214890in}}%
\pgfpathlineto{\pgfqpoint{1.802572in}{1.282949in}}%
\pgfpathlineto{\pgfqpoint{1.807091in}{1.157669in}}%
\pgfpathlineto{\pgfqpoint{1.811609in}{1.167206in}}%
\pgfpathlineto{\pgfqpoint{1.820645in}{1.200585in}}%
\pgfpathlineto{\pgfqpoint{1.825163in}{1.212290in}}%
\pgfpathlineto{\pgfqpoint{1.829681in}{1.228762in}}%
\pgfpathlineto{\pgfqpoint{1.834199in}{1.258240in}}%
\pgfpathlineto{\pgfqpoint{1.838717in}{1.209689in}}%
\pgfpathlineto{\pgfqpoint{1.843236in}{1.213156in}}%
\pgfpathlineto{\pgfqpoint{1.847754in}{1.293353in}}%
\pgfpathlineto{\pgfqpoint{1.852272in}{1.214890in}}%
\pgfpathlineto{\pgfqpoint{1.856790in}{1.219225in}}%
\pgfpathlineto{\pgfqpoint{1.861308in}{1.263008in}}%
\pgfpathlineto{\pgfqpoint{1.865826in}{1.239166in}}%
\pgfpathlineto{\pgfqpoint{1.870344in}{1.230063in}}%
\pgfpathlineto{\pgfqpoint{1.874862in}{1.254338in}}%
\pgfpathlineto{\pgfqpoint{1.879381in}{1.372248in}}%
\pgfpathlineto{\pgfqpoint{1.883899in}{1.419499in}}%
\pgfpathlineto{\pgfqpoint{1.888417in}{1.355342in}}%
\pgfpathlineto{\pgfqpoint{1.892935in}{1.405194in}}%
\pgfpathlineto{\pgfqpoint{1.897453in}{1.474552in}}%
\pgfpathlineto{\pgfqpoint{1.901971in}{1.505764in}}%
\pgfpathlineto{\pgfqpoint{1.906489in}{1.510532in}}%
\pgfpathlineto{\pgfqpoint{1.911007in}{1.559950in}}%
\pgfpathlineto{\pgfqpoint{1.915526in}{1.534808in}}%
\pgfpathlineto{\pgfqpoint{1.920044in}{1.527005in}}%
\pgfpathlineto{\pgfqpoint{1.924562in}{1.527872in}}%
\pgfpathlineto{\pgfqpoint{1.929080in}{1.539576in}}%
\pgfpathlineto{\pgfqpoint{1.933598in}{1.578157in}}%
\pgfpathlineto{\pgfqpoint{1.938116in}{1.629743in}}%
\pgfpathlineto{\pgfqpoint{1.942634in}{1.619339in}}%
\pgfpathlineto{\pgfqpoint{1.947152in}{1.644915in}}%
\pgfpathlineto{\pgfqpoint{1.951671in}{1.618905in}}%
\pgfpathlineto{\pgfqpoint{1.956189in}{1.713840in}}%
\pgfpathlineto{\pgfqpoint{1.960707in}{1.715574in}}%
\pgfpathlineto{\pgfqpoint{1.965225in}{1.683929in}}%
\pgfpathlineto{\pgfqpoint{1.969743in}{1.739416in}}%
\pgfpathlineto{\pgfqpoint{1.974261in}{1.658353in}}%
\pgfpathlineto{\pgfqpoint{1.978779in}{1.670925in}}%
\pgfpathlineto{\pgfqpoint{1.983298in}{1.674392in}}%
\pgfpathlineto{\pgfqpoint{1.987816in}{1.630176in}}%
\pgfpathlineto{\pgfqpoint{1.992334in}{1.604600in}}%
\pgfpathlineto{\pgfqpoint{1.996852in}{1.597664in}}%
\pgfpathlineto{\pgfqpoint{2.001370in}{1.546512in}}%
\pgfpathlineto{\pgfqpoint{2.005888in}{1.535675in}}%
\pgfpathlineto{\pgfqpoint{2.010406in}{1.495794in}}%
\pgfpathlineto{\pgfqpoint{2.014924in}{1.471085in}}%
\pgfpathlineto{\pgfqpoint{2.019443in}{1.459814in}}%
\pgfpathlineto{\pgfqpoint{2.023961in}{1.429469in}}%
\pgfpathlineto{\pgfqpoint{2.028479in}{1.416031in}}%
\pgfpathlineto{\pgfqpoint{2.032997in}{1.307224in}}%
\pgfpathlineto{\pgfqpoint{2.037515in}{1.298121in}}%
\pgfpathlineto{\pgfqpoint{2.042033in}{1.270811in}}%
\pgfpathlineto{\pgfqpoint{2.046551in}{1.305057in}}%
\pgfpathlineto{\pgfqpoint{2.055588in}{1.193216in}}%
\pgfpathlineto{\pgfqpoint{2.060106in}{1.197551in}}%
\pgfpathlineto{\pgfqpoint{2.064624in}{1.167640in}}%
\pgfpathlineto{\pgfqpoint{2.069142in}{1.180211in}}%
\pgfpathlineto{\pgfqpoint{2.078178in}{1.181078in}}%
\pgfpathlineto{\pgfqpoint{2.082696in}{1.118222in}}%
\pgfpathlineto{\pgfqpoint{2.087214in}{1.116921in}}%
\pgfpathlineto{\pgfqpoint{2.096251in}{1.060567in}}%
\pgfpathlineto{\pgfqpoint{2.100769in}{1.091345in}}%
\pgfpathlineto{\pgfqpoint{2.105287in}{1.045828in}}%
\pgfpathlineto{\pgfqpoint{2.109805in}{1.063168in}}%
\pgfpathlineto{\pgfqpoint{2.118841in}{1.065769in}}%
\pgfpathlineto{\pgfqpoint{2.123359in}{1.065769in}}%
\pgfpathlineto{\pgfqpoint{2.127878in}{1.069670in}}%
\pgfpathlineto{\pgfqpoint{2.132396in}{1.080508in}}%
\pgfpathlineto{\pgfqpoint{2.136914in}{1.110852in}}%
\pgfpathlineto{\pgfqpoint{2.141432in}{1.132527in}}%
\pgfpathlineto{\pgfqpoint{2.145950in}{1.087877in}}%
\pgfpathlineto{\pgfqpoint{2.150468in}{1.098281in}}%
\pgfpathlineto{\pgfqpoint{2.154986in}{1.043661in}}%
\pgfpathlineto{\pgfqpoint{2.159504in}{1.046262in}}%
\pgfpathlineto{\pgfqpoint{2.164023in}{1.050597in}}%
\pgfpathlineto{\pgfqpoint{2.168541in}{1.042360in}}%
\pgfpathlineto{\pgfqpoint{2.173059in}{1.054065in}}%
\pgfpathlineto{\pgfqpoint{2.177577in}{1.044094in}}%
\pgfpathlineto{\pgfqpoint{2.182095in}{1.062735in}}%
\pgfpathlineto{\pgfqpoint{2.186613in}{1.045828in}}%
\pgfpathlineto{\pgfqpoint{2.191131in}{1.065336in}}%
\pgfpathlineto{\pgfqpoint{2.195649in}{1.025888in}}%
\pgfpathlineto{\pgfqpoint{2.200168in}{1.018518in}}%
\pgfpathlineto{\pgfqpoint{2.209204in}{1.017651in}}%
\pgfpathlineto{\pgfqpoint{2.213722in}{1.010282in}}%
\pgfpathlineto{\pgfqpoint{2.218240in}{0.992509in}}%
\pgfpathlineto{\pgfqpoint{2.222758in}{1.012449in}}%
\pgfpathlineto{\pgfqpoint{2.227276in}{1.044528in}}%
\pgfpathlineto{\pgfqpoint{2.231794in}{1.058833in}}%
\pgfpathlineto{\pgfqpoint{2.236313in}{1.053631in}}%
\pgfpathlineto{\pgfqpoint{2.240831in}{1.053631in}}%
\pgfpathlineto{\pgfqpoint{2.245349in}{1.078340in}}%
\pgfpathlineto{\pgfqpoint{2.249867in}{1.061868in}}%
\pgfpathlineto{\pgfqpoint{2.254385in}{1.104783in}}%
\pgfpathlineto{\pgfqpoint{2.258903in}{1.101315in}}%
\pgfpathlineto{\pgfqpoint{2.263421in}{1.121256in}}%
\pgfpathlineto{\pgfqpoint{2.267940in}{1.091345in}}%
\pgfpathlineto{\pgfqpoint{2.272458in}{1.104350in}}%
\pgfpathlineto{\pgfqpoint{2.276976in}{1.075739in}}%
\pgfpathlineto{\pgfqpoint{2.281494in}{1.073138in}}%
\pgfpathlineto{\pgfqpoint{2.286012in}{1.079641in}}%
\pgfpathlineto{\pgfqpoint{2.290530in}{1.070537in}}%
\pgfpathlineto{\pgfqpoint{2.295048in}{1.037592in}}%
\pgfpathlineto{\pgfqpoint{2.299566in}{1.016351in}}%
\pgfpathlineto{\pgfqpoint{2.304085in}{1.087010in}}%
\pgfpathlineto{\pgfqpoint{2.308603in}{1.055799in}}%
\pgfpathlineto{\pgfqpoint{2.313121in}{1.049296in}}%
\pgfpathlineto{\pgfqpoint{2.317639in}{1.044961in}}%
\pgfpathlineto{\pgfqpoint{2.322157in}{1.050597in}}%
\pgfpathlineto{\pgfqpoint{2.331193in}{0.992509in}}%
\pgfpathlineto{\pgfqpoint{2.335711in}{0.999445in}}%
\pgfpathlineto{\pgfqpoint{2.340230in}{0.978637in}}%
\pgfpathlineto{\pgfqpoint{2.344748in}{0.994676in}}%
\pgfpathlineto{\pgfqpoint{2.353784in}{0.953928in}}%
\pgfpathlineto{\pgfqpoint{2.358302in}{0.948293in}}%
\pgfpathlineto{\pgfqpoint{2.362820in}{1.014617in}}%
\pgfpathlineto{\pgfqpoint{2.367338in}{0.963031in}}%
\pgfpathlineto{\pgfqpoint{2.371856in}{0.946559in}}%
\pgfpathlineto{\pgfqpoint{2.376375in}{0.925751in}}%
\pgfpathlineto{\pgfqpoint{2.380893in}{0.924884in}}%
\pgfpathlineto{\pgfqpoint{2.389929in}{0.917081in}}%
\pgfpathlineto{\pgfqpoint{2.394447in}{0.940923in}}%
\pgfpathlineto{\pgfqpoint{2.398965in}{0.937022in}}%
\pgfpathlineto{\pgfqpoint{2.403483in}{0.937889in}}%
\pgfpathlineto{\pgfqpoint{2.408001in}{0.903209in}}%
\pgfpathlineto{\pgfqpoint{2.412520in}{0.904943in}}%
\pgfpathlineto{\pgfqpoint{2.417038in}{0.909278in}}%
\pgfpathlineto{\pgfqpoint{2.421556in}{0.923150in}}%
\pgfpathlineto{\pgfqpoint{2.426074in}{0.915781in}}%
\pgfpathlineto{\pgfqpoint{2.430592in}{0.949593in}}%
\pgfpathlineto{\pgfqpoint{2.439628in}{0.933987in}}%
\pgfpathlineto{\pgfqpoint{2.444146in}{0.941357in}}%
\pgfpathlineto{\pgfqpoint{2.448665in}{0.920116in}}%
\pgfpathlineto{\pgfqpoint{2.453183in}{0.887604in}}%
\pgfpathlineto{\pgfqpoint{2.457701in}{0.885003in}}%
\pgfpathlineto{\pgfqpoint{2.466737in}{0.915347in}}%
\pgfpathlineto{\pgfqpoint{2.471255in}{0.906677in}}%
\pgfpathlineto{\pgfqpoint{2.475773in}{0.875899in}}%
\pgfpathlineto{\pgfqpoint{2.484810in}{0.877633in}}%
\pgfpathlineto{\pgfqpoint{2.493846in}{0.866363in}}%
\pgfpathlineto{\pgfqpoint{2.498364in}{0.878934in}}%
\pgfpathlineto{\pgfqpoint{2.502882in}{0.871564in}}%
\pgfpathlineto{\pgfqpoint{2.507400in}{0.881101in}}%
\pgfpathlineto{\pgfqpoint{2.511918in}{0.867230in}}%
\pgfpathlineto{\pgfqpoint{2.516437in}{0.889771in}}%
\pgfpathlineto{\pgfqpoint{2.520955in}{0.878934in}}%
\pgfpathlineto{\pgfqpoint{2.525473in}{0.881535in}}%
\pgfpathlineto{\pgfqpoint{2.529991in}{0.880668in}}%
\pgfpathlineto{\pgfqpoint{2.534509in}{0.864629in}}%
\pgfpathlineto{\pgfqpoint{2.539027in}{0.874599in}}%
\pgfpathlineto{\pgfqpoint{2.543545in}{0.860727in}}%
\pgfpathlineto{\pgfqpoint{2.548063in}{0.868963in}}%
\pgfpathlineto{\pgfqpoint{2.552582in}{0.857259in}}%
\pgfpathlineto{\pgfqpoint{2.557100in}{0.876333in}}%
\pgfpathlineto{\pgfqpoint{2.561618in}{0.872865in}}%
\pgfpathlineto{\pgfqpoint{2.566136in}{0.862028in}}%
\pgfpathlineto{\pgfqpoint{2.575172in}{0.876333in}}%
\pgfpathlineto{\pgfqpoint{2.579690in}{0.878500in}}%
\pgfpathlineto{\pgfqpoint{2.584208in}{0.876766in}}%
\pgfpathlineto{\pgfqpoint{2.588727in}{0.857259in}}%
\pgfpathlineto{\pgfqpoint{2.593245in}{0.849890in}}%
\pgfpathlineto{\pgfqpoint{2.597763in}{0.853791in}}%
\pgfpathlineto{\pgfqpoint{2.602281in}{0.869830in}}%
\pgfpathlineto{\pgfqpoint{2.611317in}{0.865496in}}%
\pgfpathlineto{\pgfqpoint{2.615835in}{0.868963in}}%
\pgfpathlineto{\pgfqpoint{2.620353in}{0.856392in}}%
\pgfpathlineto{\pgfqpoint{2.624872in}{0.862028in}}%
\pgfpathlineto{\pgfqpoint{2.629390in}{0.849023in}}%
\pgfpathlineto{\pgfqpoint{2.633908in}{0.866363in}}%
\pgfpathlineto{\pgfqpoint{2.642944in}{0.844254in}}%
\pgfpathlineto{\pgfqpoint{2.647462in}{0.835585in}}%
\pgfpathlineto{\pgfqpoint{2.651980in}{0.814343in}}%
\pgfpathlineto{\pgfqpoint{2.656498in}{0.825181in}}%
\pgfpathlineto{\pgfqpoint{2.661017in}{0.842087in}}%
\pgfpathlineto{\pgfqpoint{2.665535in}{0.824314in}}%
\pgfpathlineto{\pgfqpoint{2.670053in}{0.831250in}}%
\pgfpathlineto{\pgfqpoint{2.674571in}{0.841220in}}%
\pgfpathlineto{\pgfqpoint{2.679089in}{0.836885in}}%
\pgfpathlineto{\pgfqpoint{2.683607in}{0.829082in}}%
\pgfpathlineto{\pgfqpoint{2.688125in}{0.825614in}}%
\pgfpathlineto{\pgfqpoint{2.692643in}{0.806541in}}%
\pgfpathlineto{\pgfqpoint{2.697162in}{0.819112in}}%
\pgfpathlineto{\pgfqpoint{2.701680in}{0.821279in}}%
\pgfpathlineto{\pgfqpoint{2.706198in}{0.836885in}}%
\pgfpathlineto{\pgfqpoint{2.710716in}{0.820846in}}%
\pgfpathlineto{\pgfqpoint{2.715234in}{0.823447in}}%
\pgfpathlineto{\pgfqpoint{2.719752in}{0.813476in}}%
\pgfpathlineto{\pgfqpoint{2.724270in}{0.808708in}}%
\pgfpathlineto{\pgfqpoint{2.728788in}{0.819979in}}%
\pgfpathlineto{\pgfqpoint{2.733307in}{0.810009in}}%
\pgfpathlineto{\pgfqpoint{2.737825in}{0.811309in}}%
\pgfpathlineto{\pgfqpoint{2.742343in}{0.821279in}}%
\pgfpathlineto{\pgfqpoint{2.746861in}{0.813476in}}%
\pgfpathlineto{\pgfqpoint{2.751379in}{0.791802in}}%
\pgfpathlineto{\pgfqpoint{2.755897in}{0.811742in}}%
\pgfpathlineto{\pgfqpoint{2.760415in}{0.817811in}}%
\pgfpathlineto{\pgfqpoint{2.764934in}{0.818678in}}%
\pgfpathlineto{\pgfqpoint{2.769452in}{0.826481in}}%
\pgfpathlineto{\pgfqpoint{2.773970in}{0.816077in}}%
\pgfpathlineto{\pgfqpoint{2.778488in}{0.817378in}}%
\pgfpathlineto{\pgfqpoint{2.783006in}{0.799605in}}%
\pgfpathlineto{\pgfqpoint{2.787524in}{0.797871in}}%
\pgfpathlineto{\pgfqpoint{2.792042in}{0.785299in}}%
\pgfpathlineto{\pgfqpoint{2.796560in}{0.790501in}}%
\pgfpathlineto{\pgfqpoint{2.801079in}{0.804373in}}%
\pgfpathlineto{\pgfqpoint{2.805597in}{0.780965in}}%
\pgfpathlineto{\pgfqpoint{2.810115in}{0.780531in}}%
\pgfpathlineto{\pgfqpoint{2.814633in}{0.792235in}}%
\pgfpathlineto{\pgfqpoint{2.819151in}{0.781398in}}%
\pgfpathlineto{\pgfqpoint{2.823669in}{0.774462in}}%
\pgfpathlineto{\pgfqpoint{2.837224in}{0.787033in}}%
\pgfpathlineto{\pgfqpoint{2.841742in}{0.786600in}}%
\pgfpathlineto{\pgfqpoint{2.846260in}{0.783565in}}%
\pgfpathlineto{\pgfqpoint{2.850778in}{0.768827in}}%
\pgfpathlineto{\pgfqpoint{2.855296in}{0.784432in}}%
\pgfpathlineto{\pgfqpoint{2.864332in}{0.781398in}}%
\pgfpathlineto{\pgfqpoint{2.868850in}{0.789201in}}%
\pgfpathlineto{\pgfqpoint{2.873369in}{0.774896in}}%
\pgfpathlineto{\pgfqpoint{2.877887in}{0.776196in}}%
\pgfpathlineto{\pgfqpoint{2.882405in}{0.797871in}}%
\pgfpathlineto{\pgfqpoint{2.886923in}{0.786166in}}%
\pgfpathlineto{\pgfqpoint{2.891441in}{0.787900in}}%
\pgfpathlineto{\pgfqpoint{2.895959in}{0.785733in}}%
\pgfpathlineto{\pgfqpoint{2.900477in}{0.773595in}}%
\pgfpathlineto{\pgfqpoint{2.904995in}{0.780531in}}%
\pgfpathlineto{\pgfqpoint{2.909514in}{0.790501in}}%
\pgfpathlineto{\pgfqpoint{2.914032in}{0.786166in}}%
\pgfpathlineto{\pgfqpoint{2.918550in}{0.761024in}}%
\pgfpathlineto{\pgfqpoint{2.923068in}{0.770561in}}%
\pgfpathlineto{\pgfqpoint{2.927586in}{0.764492in}}%
\pgfpathlineto{\pgfqpoint{2.932104in}{0.776630in}}%
\pgfpathlineto{\pgfqpoint{2.936622in}{0.768393in}}%
\pgfpathlineto{\pgfqpoint{2.941140in}{0.774462in}}%
\pgfpathlineto{\pgfqpoint{2.945659in}{0.786600in}}%
\pgfpathlineto{\pgfqpoint{2.950177in}{0.791368in}}%
\pgfpathlineto{\pgfqpoint{2.954695in}{0.770994in}}%
\pgfpathlineto{\pgfqpoint{2.959213in}{0.773595in}}%
\pgfpathlineto{\pgfqpoint{2.963731in}{0.766659in}}%
\pgfpathlineto{\pgfqpoint{2.968249in}{0.784866in}}%
\pgfpathlineto{\pgfqpoint{2.972767in}{0.761024in}}%
\pgfpathlineto{\pgfqpoint{2.977285in}{0.788767in}}%
\pgfpathlineto{\pgfqpoint{2.981804in}{0.789201in}}%
\pgfpathlineto{\pgfqpoint{2.986322in}{0.797004in}}%
\pgfpathlineto{\pgfqpoint{2.990840in}{0.780531in}}%
\pgfpathlineto{\pgfqpoint{2.995358in}{0.795703in}}%
\pgfpathlineto{\pgfqpoint{2.999876in}{0.792235in}}%
\pgfpathlineto{\pgfqpoint{3.004394in}{0.795270in}}%
\pgfpathlineto{\pgfqpoint{3.013430in}{0.763625in}}%
\pgfpathlineto{\pgfqpoint{3.017949in}{0.772728in}}%
\pgfpathlineto{\pgfqpoint{3.022467in}{0.771428in}}%
\pgfpathlineto{\pgfqpoint{3.026985in}{0.780098in}}%
\pgfpathlineto{\pgfqpoint{3.031503in}{0.769694in}}%
\pgfpathlineto{\pgfqpoint{3.036021in}{0.769260in}}%
\pgfpathlineto{\pgfqpoint{3.040539in}{0.778364in}}%
\pgfpathlineto{\pgfqpoint{3.049576in}{0.780965in}}%
\pgfpathlineto{\pgfqpoint{3.054094in}{0.778364in}}%
\pgfpathlineto{\pgfqpoint{3.063130in}{0.770127in}}%
\pgfpathlineto{\pgfqpoint{3.067648in}{0.762758in}}%
\pgfpathlineto{\pgfqpoint{3.072166in}{0.757556in}}%
\pgfpathlineto{\pgfqpoint{3.076684in}{0.760157in}}%
\pgfpathlineto{\pgfqpoint{3.081202in}{0.747152in}}%
\pgfpathlineto{\pgfqpoint{3.085721in}{0.751054in}}%
\pgfpathlineto{\pgfqpoint{3.090239in}{0.762324in}}%
\pgfpathlineto{\pgfqpoint{3.094757in}{0.759723in}}%
\pgfpathlineto{\pgfqpoint{3.099275in}{0.747152in}}%
\pgfpathlineto{\pgfqpoint{3.112829in}{0.743684in}}%
\pgfpathlineto{\pgfqpoint{3.117347in}{0.754521in}}%
\pgfpathlineto{\pgfqpoint{3.121866in}{0.754521in}}%
\pgfpathlineto{\pgfqpoint{3.130902in}{0.741950in}}%
\pgfpathlineto{\pgfqpoint{3.135420in}{0.761024in}}%
\pgfpathlineto{\pgfqpoint{3.139938in}{0.752354in}}%
\pgfpathlineto{\pgfqpoint{3.148974in}{0.744985in}}%
\pgfpathlineto{\pgfqpoint{3.153492in}{0.741517in}}%
\pgfpathlineto{\pgfqpoint{3.158011in}{0.749320in}}%
\pgfpathlineto{\pgfqpoint{3.162529in}{0.741517in}}%
\pgfpathlineto{\pgfqpoint{3.167047in}{0.747586in}}%
\pgfpathlineto{\pgfqpoint{3.171565in}{0.750620in}}%
\pgfpathlineto{\pgfqpoint{3.176083in}{0.742384in}}%
\pgfpathlineto{\pgfqpoint{3.180601in}{0.739783in}}%
\pgfpathlineto{\pgfqpoint{3.185119in}{0.740216in}}%
\pgfpathlineto{\pgfqpoint{3.189637in}{0.750620in}}%
\pgfpathlineto{\pgfqpoint{3.198674in}{0.738482in}}%
\pgfpathlineto{\pgfqpoint{3.207710in}{0.739783in}}%
\pgfpathlineto{\pgfqpoint{3.212228in}{0.751054in}}%
\pgfpathlineto{\pgfqpoint{3.216746in}{0.743684in}}%
\pgfpathlineto{\pgfqpoint{3.225782in}{0.760590in}}%
\pgfpathlineto{\pgfqpoint{3.230301in}{0.774462in}}%
\pgfpathlineto{\pgfqpoint{3.234819in}{0.757556in}}%
\pgfpathlineto{\pgfqpoint{3.239337in}{0.757556in}}%
\pgfpathlineto{\pgfqpoint{3.243855in}{0.764492in}}%
\pgfpathlineto{\pgfqpoint{3.248373in}{0.762324in}}%
\pgfpathlineto{\pgfqpoint{3.252891in}{0.767526in}}%
\pgfpathlineto{\pgfqpoint{3.257409in}{0.753221in}}%
\pgfpathlineto{\pgfqpoint{3.261927in}{0.772295in}}%
\pgfpathlineto{\pgfqpoint{3.266446in}{0.772728in}}%
\pgfpathlineto{\pgfqpoint{3.270964in}{0.757556in}}%
\pgfpathlineto{\pgfqpoint{3.275482in}{0.772728in}}%
\pgfpathlineto{\pgfqpoint{3.280000in}{0.750187in}}%
\pgfpathlineto{\pgfqpoint{3.284518in}{0.766659in}}%
\pgfpathlineto{\pgfqpoint{3.289036in}{0.757556in}}%
\pgfpathlineto{\pgfqpoint{3.293554in}{0.755388in}}%
\pgfpathlineto{\pgfqpoint{3.298073in}{0.745852in}}%
\pgfpathlineto{\pgfqpoint{3.302591in}{0.755822in}}%
\pgfpathlineto{\pgfqpoint{3.307109in}{0.745852in}}%
\pgfpathlineto{\pgfqpoint{3.311627in}{0.752354in}}%
\pgfpathlineto{\pgfqpoint{3.320663in}{0.750187in}}%
\pgfpathlineto{\pgfqpoint{3.325181in}{0.754088in}}%
\pgfpathlineto{\pgfqpoint{3.329699in}{0.743251in}}%
\pgfpathlineto{\pgfqpoint{3.334218in}{0.759723in}}%
\pgfpathlineto{\pgfqpoint{3.338736in}{0.757122in}}%
\pgfpathlineto{\pgfqpoint{3.343254in}{0.751487in}}%
\pgfpathlineto{\pgfqpoint{3.347772in}{0.766226in}}%
\pgfpathlineto{\pgfqpoint{3.352290in}{0.751487in}}%
\pgfpathlineto{\pgfqpoint{3.356808in}{0.770127in}}%
\pgfpathlineto{\pgfqpoint{3.361326in}{0.773595in}}%
\pgfpathlineto{\pgfqpoint{3.365844in}{0.764925in}}%
\pgfpathlineto{\pgfqpoint{3.374881in}{0.765792in}}%
\pgfpathlineto{\pgfqpoint{3.383917in}{0.744985in}}%
\pgfpathlineto{\pgfqpoint{3.388435in}{0.743684in}}%
\pgfpathlineto{\pgfqpoint{3.392953in}{0.741083in}}%
\pgfpathlineto{\pgfqpoint{3.397471in}{0.731980in}}%
\pgfpathlineto{\pgfqpoint{3.401989in}{0.741083in}}%
\pgfpathlineto{\pgfqpoint{3.406508in}{0.737182in}}%
\pgfpathlineto{\pgfqpoint{3.411026in}{0.746285in}}%
\pgfpathlineto{\pgfqpoint{3.415544in}{0.751921in}}%
\pgfpathlineto{\pgfqpoint{3.420062in}{0.754521in}}%
\pgfpathlineto{\pgfqpoint{3.424580in}{0.751921in}}%
\pgfpathlineto{\pgfqpoint{3.429098in}{0.739783in}}%
\pgfpathlineto{\pgfqpoint{3.433616in}{0.750187in}}%
\pgfpathlineto{\pgfqpoint{3.438134in}{0.753221in}}%
\pgfpathlineto{\pgfqpoint{3.442653in}{0.748886in}}%
\pgfpathlineto{\pgfqpoint{3.447171in}{0.751921in}}%
\pgfpathlineto{\pgfqpoint{3.451689in}{0.748453in}}%
\pgfpathlineto{\pgfqpoint{3.456207in}{0.751054in}}%
\pgfpathlineto{\pgfqpoint{3.460725in}{0.749320in}}%
\pgfpathlineto{\pgfqpoint{3.465243in}{0.732847in}}%
\pgfpathlineto{\pgfqpoint{3.469761in}{0.728512in}}%
\pgfpathlineto{\pgfqpoint{3.474279in}{0.744118in}}%
\pgfpathlineto{\pgfqpoint{3.478798in}{0.744118in}}%
\pgfpathlineto{\pgfqpoint{3.483316in}{0.735014in}}%
\pgfpathlineto{\pgfqpoint{3.487834in}{0.744118in}}%
\pgfpathlineto{\pgfqpoint{3.492352in}{0.736748in}}%
\pgfpathlineto{\pgfqpoint{3.496870in}{0.733714in}}%
\pgfpathlineto{\pgfqpoint{3.501388in}{0.735014in}}%
\pgfpathlineto{\pgfqpoint{3.505906in}{0.738916in}}%
\pgfpathlineto{\pgfqpoint{3.510424in}{0.739349in}}%
\pgfpathlineto{\pgfqpoint{3.514943in}{0.735448in}}%
\pgfpathlineto{\pgfqpoint{3.519461in}{0.738916in}}%
\pgfpathlineto{\pgfqpoint{3.523979in}{0.748453in}}%
\pgfpathlineto{\pgfqpoint{3.533015in}{0.741517in}}%
\pgfpathlineto{\pgfqpoint{3.537533in}{0.741083in}}%
\pgfpathlineto{\pgfqpoint{3.542051in}{0.734147in}}%
\pgfpathlineto{\pgfqpoint{3.546570in}{0.730246in}}%
\pgfpathlineto{\pgfqpoint{3.551088in}{0.731113in}}%
\pgfpathlineto{\pgfqpoint{3.555606in}{0.727211in}}%
\pgfpathlineto{\pgfqpoint{3.560124in}{0.730246in}}%
\pgfpathlineto{\pgfqpoint{3.564642in}{0.726344in}}%
\pgfpathlineto{\pgfqpoint{3.569160in}{0.733714in}}%
\pgfpathlineto{\pgfqpoint{3.573678in}{0.748453in}}%
\pgfpathlineto{\pgfqpoint{3.578196in}{0.735014in}}%
\pgfpathlineto{\pgfqpoint{3.582715in}{0.748886in}}%
\pgfpathlineto{\pgfqpoint{3.587233in}{0.752354in}}%
\pgfpathlineto{\pgfqpoint{3.591751in}{0.741950in}}%
\pgfpathlineto{\pgfqpoint{3.596269in}{0.755388in}}%
\pgfpathlineto{\pgfqpoint{3.600787in}{0.746719in}}%
\pgfpathlineto{\pgfqpoint{3.605305in}{0.753654in}}%
\pgfpathlineto{\pgfqpoint{3.609823in}{0.757989in}}%
\pgfpathlineto{\pgfqpoint{3.614341in}{0.749320in}}%
\pgfpathlineto{\pgfqpoint{3.618860in}{0.748886in}}%
\pgfpathlineto{\pgfqpoint{3.623378in}{0.739783in}}%
\pgfpathlineto{\pgfqpoint{3.627896in}{0.735448in}}%
\pgfpathlineto{\pgfqpoint{3.632414in}{0.744551in}}%
\pgfpathlineto{\pgfqpoint{3.636932in}{0.741517in}}%
\pgfpathlineto{\pgfqpoint{3.645968in}{0.726344in}}%
\pgfpathlineto{\pgfqpoint{3.650486in}{0.738482in}}%
\pgfpathlineto{\pgfqpoint{3.655005in}{0.733280in}}%
\pgfpathlineto{\pgfqpoint{3.664041in}{0.746719in}}%
\pgfpathlineto{\pgfqpoint{3.668559in}{0.732847in}}%
\pgfpathlineto{\pgfqpoint{3.673077in}{0.748453in}}%
\pgfpathlineto{\pgfqpoint{3.677595in}{0.754955in}}%
\pgfpathlineto{\pgfqpoint{3.682113in}{0.781398in}}%
\pgfpathlineto{\pgfqpoint{3.691150in}{0.749753in}}%
\pgfpathlineto{\pgfqpoint{3.695668in}{0.740216in}}%
\pgfpathlineto{\pgfqpoint{3.700186in}{0.737615in}}%
\pgfpathlineto{\pgfqpoint{3.704704in}{0.728078in}}%
\pgfpathlineto{\pgfqpoint{3.709222in}{0.736315in}}%
\pgfpathlineto{\pgfqpoint{3.713740in}{0.736748in}}%
\pgfpathlineto{\pgfqpoint{3.718258in}{0.731113in}}%
\pgfpathlineto{\pgfqpoint{3.722776in}{0.727645in}}%
\pgfpathlineto{\pgfqpoint{3.727295in}{0.739349in}}%
\pgfpathlineto{\pgfqpoint{3.731813in}{0.746719in}}%
\pgfpathlineto{\pgfqpoint{3.736331in}{0.741083in}}%
\pgfpathlineto{\pgfqpoint{3.740849in}{0.725477in}}%
\pgfpathlineto{\pgfqpoint{3.745367in}{0.742384in}}%
\pgfpathlineto{\pgfqpoint{3.749885in}{0.740216in}}%
\pgfpathlineto{\pgfqpoint{3.754403in}{0.759290in}}%
\pgfpathlineto{\pgfqpoint{3.758921in}{0.735448in}}%
\pgfpathlineto{\pgfqpoint{3.763440in}{0.732413in}}%
\pgfpathlineto{\pgfqpoint{3.767958in}{0.727645in}}%
\pgfpathlineto{\pgfqpoint{3.772476in}{0.734581in}}%
\pgfpathlineto{\pgfqpoint{3.776994in}{0.725044in}}%
\pgfpathlineto{\pgfqpoint{3.781512in}{0.734147in}}%
\pgfpathlineto{\pgfqpoint{3.786030in}{0.731113in}}%
\pgfpathlineto{\pgfqpoint{3.790548in}{0.723744in}}%
\pgfpathlineto{\pgfqpoint{3.795066in}{0.722443in}}%
\pgfpathlineto{\pgfqpoint{3.799585in}{0.742817in}}%
\pgfpathlineto{\pgfqpoint{3.804103in}{0.754521in}}%
\pgfpathlineto{\pgfqpoint{3.808621in}{0.732413in}}%
\pgfpathlineto{\pgfqpoint{3.813139in}{0.724611in}}%
\pgfpathlineto{\pgfqpoint{3.817657in}{0.727211in}}%
\pgfpathlineto{\pgfqpoint{3.822175in}{0.724611in}}%
\pgfpathlineto{\pgfqpoint{3.826693in}{0.731546in}}%
\pgfpathlineto{\pgfqpoint{3.831212in}{0.719842in}}%
\pgfpathlineto{\pgfqpoint{3.835730in}{0.723310in}}%
\pgfpathlineto{\pgfqpoint{3.840248in}{0.720709in}}%
\pgfpathlineto{\pgfqpoint{3.844766in}{0.730246in}}%
\pgfpathlineto{\pgfqpoint{3.849284in}{0.733280in}}%
\pgfpathlineto{\pgfqpoint{3.853802in}{0.720709in}}%
\pgfpathlineto{\pgfqpoint{3.858320in}{0.725911in}}%
\pgfpathlineto{\pgfqpoint{3.862838in}{0.728078in}}%
\pgfpathlineto{\pgfqpoint{3.867357in}{0.732413in}}%
\pgfpathlineto{\pgfqpoint{3.871875in}{0.729379in}}%
\pgfpathlineto{\pgfqpoint{3.876393in}{0.746285in}}%
\pgfpathlineto{\pgfqpoint{3.880911in}{0.774896in}}%
\pgfpathlineto{\pgfqpoint{3.885429in}{0.739349in}}%
\pgfpathlineto{\pgfqpoint{3.889947in}{0.744551in}}%
\pgfpathlineto{\pgfqpoint{3.894465in}{0.738049in}}%
\pgfpathlineto{\pgfqpoint{3.903502in}{0.739349in}}%
\pgfpathlineto{\pgfqpoint{3.908020in}{0.739349in}}%
\pgfpathlineto{\pgfqpoint{3.912538in}{0.749753in}}%
\pgfpathlineto{\pgfqpoint{3.917056in}{0.742384in}}%
\pgfpathlineto{\pgfqpoint{3.921574in}{0.760590in}}%
\pgfpathlineto{\pgfqpoint{3.926092in}{0.770994in}}%
\pgfpathlineto{\pgfqpoint{3.930610in}{0.754521in}}%
\pgfpathlineto{\pgfqpoint{3.935128in}{0.745418in}}%
\pgfpathlineto{\pgfqpoint{3.939647in}{0.762758in}}%
\pgfpathlineto{\pgfqpoint{3.944165in}{0.732847in}}%
\pgfpathlineto{\pgfqpoint{3.948683in}{0.724177in}}%
\pgfpathlineto{\pgfqpoint{3.953201in}{0.725477in}}%
\pgfpathlineto{\pgfqpoint{3.957719in}{0.721576in}}%
\pgfpathlineto{\pgfqpoint{3.962237in}{0.725044in}}%
\pgfpathlineto{\pgfqpoint{3.966755in}{0.723744in}}%
\pgfpathlineto{\pgfqpoint{3.971273in}{0.732413in}}%
\pgfpathlineto{\pgfqpoint{3.975792in}{0.738049in}}%
\pgfpathlineto{\pgfqpoint{3.980310in}{0.732847in}}%
\pgfpathlineto{\pgfqpoint{3.989346in}{0.731980in}}%
\pgfpathlineto{\pgfqpoint{3.993864in}{0.725911in}}%
\pgfpathlineto{\pgfqpoint{3.998382in}{0.723744in}}%
\pgfpathlineto{\pgfqpoint{4.002900in}{0.724177in}}%
\pgfpathlineto{\pgfqpoint{4.007418in}{0.719409in}}%
\pgfpathlineto{\pgfqpoint{4.011937in}{0.718975in}}%
\pgfpathlineto{\pgfqpoint{4.016455in}{0.723310in}}%
\pgfpathlineto{\pgfqpoint{4.025491in}{0.724177in}}%
\pgfpathlineto{\pgfqpoint{4.030009in}{0.722877in}}%
\pgfpathlineto{\pgfqpoint{4.034527in}{0.712906in}}%
\pgfpathlineto{\pgfqpoint{4.043563in}{0.722010in}}%
\pgfpathlineto{\pgfqpoint{4.048082in}{0.715074in}}%
\pgfpathlineto{\pgfqpoint{4.052600in}{0.715074in}}%
\pgfpathlineto{\pgfqpoint{4.057118in}{0.719842in}}%
\pgfpathlineto{\pgfqpoint{4.061636in}{0.715074in}}%
\pgfpathlineto{\pgfqpoint{4.066154in}{0.718975in}}%
\pgfpathlineto{\pgfqpoint{4.070672in}{0.710305in}}%
\pgfpathlineto{\pgfqpoint{4.075190in}{0.710739in}}%
\pgfpathlineto{\pgfqpoint{4.079709in}{0.716374in}}%
\pgfpathlineto{\pgfqpoint{4.084227in}{0.712906in}}%
\pgfpathlineto{\pgfqpoint{4.088745in}{0.713340in}}%
\pgfpathlineto{\pgfqpoint{4.093263in}{0.717675in}}%
\pgfpathlineto{\pgfqpoint{4.097781in}{0.712473in}}%
\pgfpathlineto{\pgfqpoint{4.102299in}{0.719842in}}%
\pgfpathlineto{\pgfqpoint{4.111335in}{0.719842in}}%
\pgfpathlineto{\pgfqpoint{4.115854in}{0.730679in}}%
\pgfpathlineto{\pgfqpoint{4.120372in}{0.725911in}}%
\pgfpathlineto{\pgfqpoint{4.124890in}{0.726778in}}%
\pgfpathlineto{\pgfqpoint{4.129408in}{0.717241in}}%
\pgfpathlineto{\pgfqpoint{4.138444in}{0.721143in}}%
\pgfpathlineto{\pgfqpoint{4.142962in}{0.742384in}}%
\pgfpathlineto{\pgfqpoint{4.147480in}{0.715074in}}%
\pgfpathlineto{\pgfqpoint{4.151999in}{0.713773in}}%
\pgfpathlineto{\pgfqpoint{4.156517in}{0.721576in}}%
\pgfpathlineto{\pgfqpoint{4.161035in}{0.720276in}}%
\pgfpathlineto{\pgfqpoint{4.165553in}{0.725477in}}%
\pgfpathlineto{\pgfqpoint{4.170071in}{0.726778in}}%
\pgfpathlineto{\pgfqpoint{4.174589in}{0.741950in}}%
\pgfpathlineto{\pgfqpoint{4.179107in}{0.725911in}}%
\pgfpathlineto{\pgfqpoint{4.183625in}{0.715941in}}%
\pgfpathlineto{\pgfqpoint{4.188144in}{0.716808in}}%
\pgfpathlineto{\pgfqpoint{4.192662in}{0.726778in}}%
\pgfpathlineto{\pgfqpoint{4.197180in}{0.728078in}}%
\pgfpathlineto{\pgfqpoint{4.201698in}{0.719409in}}%
\pgfpathlineto{\pgfqpoint{4.206216in}{0.751054in}}%
\pgfpathlineto{\pgfqpoint{4.210734in}{0.736315in}}%
\pgfpathlineto{\pgfqpoint{4.219770in}{0.721143in}}%
\pgfpathlineto{\pgfqpoint{4.224289in}{0.726344in}}%
\pgfpathlineto{\pgfqpoint{4.228807in}{0.723310in}}%
\pgfpathlineto{\pgfqpoint{4.233325in}{0.717675in}}%
\pgfpathlineto{\pgfqpoint{4.237843in}{0.729812in}}%
\pgfpathlineto{\pgfqpoint{4.242361in}{0.728512in}}%
\pgfpathlineto{\pgfqpoint{4.246879in}{0.722010in}}%
\pgfpathlineto{\pgfqpoint{4.251397in}{0.726344in}}%
\pgfpathlineto{\pgfqpoint{4.260434in}{0.716808in}}%
\pgfpathlineto{\pgfqpoint{4.273988in}{0.724177in}}%
\pgfpathlineto{\pgfqpoint{4.278506in}{0.727645in}}%
\pgfpathlineto{\pgfqpoint{4.283024in}{0.733714in}}%
\pgfpathlineto{\pgfqpoint{4.287542in}{0.732413in}}%
\pgfpathlineto{\pgfqpoint{4.296579in}{0.716374in}}%
\pgfpathlineto{\pgfqpoint{4.301097in}{0.725911in}}%
\pgfpathlineto{\pgfqpoint{4.305615in}{0.723310in}}%
\pgfpathlineto{\pgfqpoint{4.310133in}{0.727645in}}%
\pgfpathlineto{\pgfqpoint{4.314651in}{0.721143in}}%
\pgfpathlineto{\pgfqpoint{4.319169in}{0.722443in}}%
\pgfpathlineto{\pgfqpoint{4.323687in}{0.722443in}}%
\pgfpathlineto{\pgfqpoint{4.328206in}{0.721143in}}%
\pgfpathlineto{\pgfqpoint{4.332724in}{0.715507in}}%
\pgfpathlineto{\pgfqpoint{4.337242in}{0.726778in}}%
\pgfpathlineto{\pgfqpoint{4.341760in}{0.717241in}}%
\pgfpathlineto{\pgfqpoint{4.346278in}{0.717675in}}%
\pgfpathlineto{\pgfqpoint{4.350796in}{0.712906in}}%
\pgfpathlineto{\pgfqpoint{4.355314in}{0.715507in}}%
\pgfpathlineto{\pgfqpoint{4.359832in}{0.721143in}}%
\pgfpathlineto{\pgfqpoint{4.364351in}{0.718108in}}%
\pgfpathlineto{\pgfqpoint{4.373387in}{0.708571in}}%
\pgfpathlineto{\pgfqpoint{4.377905in}{0.710739in}}%
\pgfpathlineto{\pgfqpoint{4.382423in}{0.710305in}}%
\pgfpathlineto{\pgfqpoint{4.386941in}{0.706404in}}%
\pgfpathlineto{\pgfqpoint{4.391459in}{0.709438in}}%
\pgfpathlineto{\pgfqpoint{4.395977in}{0.705970in}}%
\pgfpathlineto{\pgfqpoint{4.400496in}{0.705970in}}%
\pgfpathlineto{\pgfqpoint{4.405014in}{0.708138in}}%
\pgfpathlineto{\pgfqpoint{4.409532in}{0.708571in}}%
\pgfpathlineto{\pgfqpoint{4.414050in}{0.715507in}}%
\pgfpathlineto{\pgfqpoint{4.418568in}{0.710305in}}%
\pgfpathlineto{\pgfqpoint{4.423086in}{0.711172in}}%
\pgfpathlineto{\pgfqpoint{4.427604in}{0.713773in}}%
\pgfpathlineto{\pgfqpoint{4.432122in}{0.731980in}}%
\pgfpathlineto{\pgfqpoint{4.441159in}{0.711172in}}%
\pgfpathlineto{\pgfqpoint{4.445677in}{0.711172in}}%
\pgfpathlineto{\pgfqpoint{4.450195in}{0.706837in}}%
\pgfpathlineto{\pgfqpoint{4.459231in}{0.708571in}}%
\pgfpathlineto{\pgfqpoint{4.463749in}{0.705537in}}%
\pgfpathlineto{\pgfqpoint{4.468267in}{0.708138in}}%
\pgfpathlineto{\pgfqpoint{4.472786in}{0.713340in}}%
\pgfpathlineto{\pgfqpoint{4.477304in}{0.706837in}}%
\pgfpathlineto{\pgfqpoint{4.481822in}{0.712906in}}%
\pgfpathlineto{\pgfqpoint{4.486340in}{0.710739in}}%
\pgfpathlineto{\pgfqpoint{4.490858in}{0.717241in}}%
\pgfpathlineto{\pgfqpoint{4.495376in}{0.712906in}}%
\pgfpathlineto{\pgfqpoint{4.499894in}{0.713340in}}%
\pgfpathlineto{\pgfqpoint{4.504412in}{0.711606in}}%
\pgfpathlineto{\pgfqpoint{4.513449in}{0.718542in}}%
\pgfpathlineto{\pgfqpoint{4.517967in}{0.716808in}}%
\pgfpathlineto{\pgfqpoint{4.522485in}{0.712039in}}%
\pgfpathlineto{\pgfqpoint{4.527003in}{0.719842in}}%
\pgfpathlineto{\pgfqpoint{4.531521in}{0.721576in}}%
\pgfpathlineto{\pgfqpoint{4.540557in}{0.711606in}}%
\pgfpathlineto{\pgfqpoint{4.545076in}{0.710739in}}%
\pgfpathlineto{\pgfqpoint{4.549594in}{0.712039in}}%
\pgfpathlineto{\pgfqpoint{4.554112in}{0.712039in}}%
\pgfpathlineto{\pgfqpoint{4.558630in}{0.713340in}}%
\pgfpathlineto{\pgfqpoint{4.563148in}{0.711172in}}%
\pgfpathlineto{\pgfqpoint{4.572184in}{0.713340in}}%
\pgfpathlineto{\pgfqpoint{4.576702in}{0.710305in}}%
\pgfpathlineto{\pgfqpoint{4.581221in}{0.716374in}}%
\pgfpathlineto{\pgfqpoint{4.585739in}{0.725044in}}%
\pgfpathlineto{\pgfqpoint{4.590257in}{0.709005in}}%
\pgfpathlineto{\pgfqpoint{4.599293in}{0.730246in}}%
\pgfpathlineto{\pgfqpoint{4.603811in}{0.739783in}}%
\pgfpathlineto{\pgfqpoint{4.608329in}{0.717241in}}%
\pgfpathlineto{\pgfqpoint{4.612848in}{0.733280in}}%
\pgfpathlineto{\pgfqpoint{4.621884in}{0.718108in}}%
\pgfpathlineto{\pgfqpoint{4.626402in}{0.713340in}}%
\pgfpathlineto{\pgfqpoint{4.630920in}{0.720276in}}%
\pgfpathlineto{\pgfqpoint{4.635438in}{0.711172in}}%
\pgfpathlineto{\pgfqpoint{4.639956in}{0.709872in}}%
\pgfpathlineto{\pgfqpoint{4.644474in}{0.712906in}}%
\pgfpathlineto{\pgfqpoint{4.648993in}{0.709438in}}%
\pgfpathlineto{\pgfqpoint{4.658029in}{0.728512in}}%
\pgfpathlineto{\pgfqpoint{4.662547in}{0.706404in}}%
\pgfpathlineto{\pgfqpoint{4.667065in}{0.708571in}}%
\pgfpathlineto{\pgfqpoint{4.676101in}{0.705970in}}%
\pgfpathlineto{\pgfqpoint{4.680619in}{0.709872in}}%
\pgfpathlineto{\pgfqpoint{4.685138in}{0.708571in}}%
\pgfpathlineto{\pgfqpoint{4.689656in}{0.712473in}}%
\pgfpathlineto{\pgfqpoint{4.694174in}{0.709872in}}%
\pgfpathlineto{\pgfqpoint{4.698692in}{0.702069in}}%
\pgfpathlineto{\pgfqpoint{4.707728in}{0.707271in}}%
\pgfpathlineto{\pgfqpoint{4.712246in}{0.706837in}}%
\pgfpathlineto{\pgfqpoint{4.716764in}{0.709438in}}%
\pgfpathlineto{\pgfqpoint{4.725801in}{0.708571in}}%
\pgfpathlineto{\pgfqpoint{4.730319in}{0.704670in}}%
\pgfpathlineto{\pgfqpoint{4.734837in}{0.707704in}}%
\pgfpathlineto{\pgfqpoint{4.739355in}{0.714207in}}%
\pgfpathlineto{\pgfqpoint{4.748391in}{0.707271in}}%
\pgfpathlineto{\pgfqpoint{4.752909in}{0.704670in}}%
\pgfpathlineto{\pgfqpoint{4.757428in}{0.703803in}}%
\pgfpathlineto{\pgfqpoint{4.761946in}{0.709005in}}%
\pgfpathlineto{\pgfqpoint{4.766464in}{0.709005in}}%
\pgfpathlineto{\pgfqpoint{4.770982in}{0.712473in}}%
\pgfpathlineto{\pgfqpoint{4.780018in}{0.707704in}}%
\pgfpathlineto{\pgfqpoint{4.784536in}{0.709438in}}%
\pgfpathlineto{\pgfqpoint{4.789054in}{0.714207in}}%
\pgfpathlineto{\pgfqpoint{4.793573in}{0.724611in}}%
\pgfpathlineto{\pgfqpoint{4.798091in}{0.716374in}}%
\pgfpathlineto{\pgfqpoint{4.802609in}{0.715507in}}%
\pgfpathlineto{\pgfqpoint{4.807127in}{0.707704in}}%
\pgfpathlineto{\pgfqpoint{4.811645in}{0.709005in}}%
\pgfpathlineto{\pgfqpoint{4.816163in}{0.706404in}}%
\pgfpathlineto{\pgfqpoint{4.820681in}{0.717241in}}%
\pgfpathlineto{\pgfqpoint{4.825199in}{0.716374in}}%
\pgfpathlineto{\pgfqpoint{4.829718in}{0.711172in}}%
\pgfpathlineto{\pgfqpoint{4.838754in}{0.711172in}}%
\pgfpathlineto{\pgfqpoint{4.843272in}{0.712039in}}%
\pgfpathlineto{\pgfqpoint{4.847790in}{0.709005in}}%
\pgfpathlineto{\pgfqpoint{4.852308in}{0.712473in}}%
\pgfpathlineto{\pgfqpoint{4.856826in}{0.710739in}}%
\pgfpathlineto{\pgfqpoint{4.861345in}{0.707271in}}%
\pgfpathlineto{\pgfqpoint{4.865863in}{0.713773in}}%
\pgfpathlineto{\pgfqpoint{4.874899in}{0.713340in}}%
\pgfpathlineto{\pgfqpoint{4.879417in}{0.706837in}}%
\pgfpathlineto{\pgfqpoint{4.883935in}{0.712906in}}%
\pgfpathlineto{\pgfqpoint{4.888453in}{0.709438in}}%
\pgfpathlineto{\pgfqpoint{4.892971in}{0.709005in}}%
\pgfpathlineto{\pgfqpoint{4.897490in}{0.706837in}}%
\pgfpathlineto{\pgfqpoint{4.902008in}{0.708571in}}%
\pgfpathlineto{\pgfqpoint{4.920080in}{0.705537in}}%
\pgfpathlineto{\pgfqpoint{4.924598in}{0.702069in}}%
\pgfpathlineto{\pgfqpoint{4.929116in}{0.702936in}}%
\pgfpathlineto{\pgfqpoint{4.933635in}{0.714207in}}%
\pgfpathlineto{\pgfqpoint{4.942671in}{0.725477in}}%
\pgfpathlineto{\pgfqpoint{4.947189in}{0.707704in}}%
\pgfpathlineto{\pgfqpoint{4.956225in}{0.712039in}}%
\pgfpathlineto{\pgfqpoint{4.960743in}{0.709005in}}%
\pgfpathlineto{\pgfqpoint{4.965261in}{0.711172in}}%
\pgfpathlineto{\pgfqpoint{4.969780in}{0.712039in}}%
\pgfpathlineto{\pgfqpoint{4.974298in}{0.716374in}}%
\pgfpathlineto{\pgfqpoint{4.978816in}{0.712906in}}%
\pgfpathlineto{\pgfqpoint{4.983334in}{0.715941in}}%
\pgfpathlineto{\pgfqpoint{4.987852in}{0.710305in}}%
\pgfpathlineto{\pgfqpoint{4.996888in}{0.707271in}}%
\pgfpathlineto{\pgfqpoint{5.001406in}{0.707704in}}%
\pgfpathlineto{\pgfqpoint{5.005925in}{0.704670in}}%
\pgfpathlineto{\pgfqpoint{5.010443in}{0.706837in}}%
\pgfpathlineto{\pgfqpoint{5.014961in}{0.704670in}}%
\pgfpathlineto{\pgfqpoint{5.019479in}{0.703803in}}%
\pgfpathlineto{\pgfqpoint{5.023997in}{0.712906in}}%
\pgfpathlineto{\pgfqpoint{5.028515in}{0.706837in}}%
\pgfpathlineto{\pgfqpoint{5.033033in}{0.707704in}}%
\pgfpathlineto{\pgfqpoint{5.037551in}{0.712473in}}%
\pgfpathlineto{\pgfqpoint{5.046588in}{0.707704in}}%
\pgfpathlineto{\pgfqpoint{5.051106in}{0.711606in}}%
\pgfpathlineto{\pgfqpoint{5.060142in}{0.728945in}}%
\pgfpathlineto{\pgfqpoint{5.064660in}{0.722443in}}%
\pgfpathlineto{\pgfqpoint{5.069178in}{0.722443in}}%
\pgfpathlineto{\pgfqpoint{5.073696in}{0.718542in}}%
\pgfpathlineto{\pgfqpoint{5.078215in}{0.711172in}}%
\pgfpathlineto{\pgfqpoint{5.082733in}{0.713773in}}%
\pgfpathlineto{\pgfqpoint{5.087251in}{0.718542in}}%
\pgfpathlineto{\pgfqpoint{5.091769in}{0.713773in}}%
\pgfpathlineto{\pgfqpoint{5.105323in}{0.708571in}}%
\pgfpathlineto{\pgfqpoint{5.109842in}{0.710739in}}%
\pgfpathlineto{\pgfqpoint{5.114360in}{0.708571in}}%
\pgfpathlineto{\pgfqpoint{5.118878in}{0.708571in}}%
\pgfpathlineto{\pgfqpoint{5.123396in}{0.712906in}}%
\pgfpathlineto{\pgfqpoint{5.132432in}{0.709872in}}%
\pgfpathlineto{\pgfqpoint{5.136950in}{0.710739in}}%
\pgfpathlineto{\pgfqpoint{5.141468in}{0.708571in}}%
\pgfpathlineto{\pgfqpoint{5.145987in}{0.711172in}}%
\pgfpathlineto{\pgfqpoint{5.150505in}{0.708138in}}%
\pgfpathlineto{\pgfqpoint{5.155023in}{0.712906in}}%
\pgfpathlineto{\pgfqpoint{5.159541in}{0.706837in}}%
\pgfpathlineto{\pgfqpoint{5.164059in}{0.707704in}}%
\pgfpathlineto{\pgfqpoint{5.168577in}{0.704236in}}%
\pgfpathlineto{\pgfqpoint{5.173095in}{0.703803in}}%
\pgfpathlineto{\pgfqpoint{5.186650in}{0.706404in}}%
\pgfpathlineto{\pgfqpoint{5.191168in}{0.703369in}}%
\pgfpathlineto{\pgfqpoint{5.195686in}{0.709438in}}%
\pgfpathlineto{\pgfqpoint{5.200204in}{0.709438in}}%
\pgfpathlineto{\pgfqpoint{5.204722in}{0.705970in}}%
\pgfpathlineto{\pgfqpoint{5.209240in}{0.705103in}}%
\pgfpathlineto{\pgfqpoint{5.213758in}{0.708571in}}%
\pgfpathlineto{\pgfqpoint{5.218277in}{0.709005in}}%
\pgfpathlineto{\pgfqpoint{5.222795in}{0.708138in}}%
\pgfpathlineto{\pgfqpoint{5.227313in}{0.709005in}}%
\pgfpathlineto{\pgfqpoint{5.231831in}{0.702502in}}%
\pgfpathlineto{\pgfqpoint{5.236349in}{0.705970in}}%
\pgfpathlineto{\pgfqpoint{5.240867in}{0.705970in}}%
\pgfpathlineto{\pgfqpoint{5.245385in}{0.707271in}}%
\pgfpathlineto{\pgfqpoint{5.254422in}{0.701635in}}%
\pgfpathlineto{\pgfqpoint{5.258940in}{0.705537in}}%
\pgfpathlineto{\pgfqpoint{5.263458in}{0.702502in}}%
\pgfpathlineto{\pgfqpoint{5.272494in}{0.706837in}}%
\pgfpathlineto{\pgfqpoint{5.277012in}{0.701635in}}%
\pgfpathlineto{\pgfqpoint{5.286048in}{0.710739in}}%
\pgfpathlineto{\pgfqpoint{5.290567in}{0.705537in}}%
\pgfpathlineto{\pgfqpoint{5.299603in}{0.705537in}}%
\pgfpathlineto{\pgfqpoint{5.304121in}{0.712473in}}%
\pgfpathlineto{\pgfqpoint{5.308639in}{0.710739in}}%
\pgfpathlineto{\pgfqpoint{5.313157in}{0.718108in}}%
\pgfpathlineto{\pgfqpoint{5.322193in}{0.704236in}}%
\pgfpathlineto{\pgfqpoint{5.340266in}{0.702069in}}%
\pgfpathlineto{\pgfqpoint{5.344784in}{0.704236in}}%
\pgfpathlineto{\pgfqpoint{5.349302in}{0.710739in}}%
\pgfpathlineto{\pgfqpoint{5.353820in}{0.704236in}}%
\pgfpathlineto{\pgfqpoint{5.358338in}{0.704670in}}%
\pgfpathlineto{\pgfqpoint{5.362857in}{0.703369in}}%
\pgfpathlineto{\pgfqpoint{5.367375in}{0.703803in}}%
\pgfpathlineto{\pgfqpoint{5.376411in}{0.707271in}}%
\pgfpathlineto{\pgfqpoint{5.380929in}{0.707271in}}%
\pgfpathlineto{\pgfqpoint{5.385447in}{0.700768in}}%
\pgfpathlineto{\pgfqpoint{5.389965in}{0.703803in}}%
\pgfpathlineto{\pgfqpoint{5.394484in}{0.702069in}}%
\pgfpathlineto{\pgfqpoint{5.399002in}{0.702502in}}%
\pgfpathlineto{\pgfqpoint{5.403520in}{0.700768in}}%
\pgfpathlineto{\pgfqpoint{5.408038in}{0.702502in}}%
\pgfpathlineto{\pgfqpoint{5.412556in}{0.701635in}}%
\pgfpathlineto{\pgfqpoint{5.417074in}{0.702502in}}%
\pgfpathlineto{\pgfqpoint{5.426110in}{0.710739in}}%
\pgfpathlineto{\pgfqpoint{5.430629in}{0.713340in}}%
\pgfpathlineto{\pgfqpoint{5.439665in}{0.711606in}}%
\pgfpathlineto{\pgfqpoint{5.444183in}{0.707271in}}%
\pgfpathlineto{\pgfqpoint{5.448701in}{0.712039in}}%
\pgfpathlineto{\pgfqpoint{5.457737in}{0.716374in}}%
\pgfpathlineto{\pgfqpoint{5.462255in}{0.715941in}}%
\pgfpathlineto{\pgfqpoint{5.466774in}{0.709438in}}%
\pgfpathlineto{\pgfqpoint{5.475810in}{0.702936in}}%
\pgfpathlineto{\pgfqpoint{5.480328in}{0.702069in}}%
\pgfpathlineto{\pgfqpoint{5.484846in}{0.702502in}}%
\pgfpathlineto{\pgfqpoint{5.489364in}{0.701635in}}%
\pgfpathlineto{\pgfqpoint{5.502919in}{0.703369in}}%
\pgfpathlineto{\pgfqpoint{5.507437in}{0.703803in}}%
\pgfpathlineto{\pgfqpoint{5.511955in}{0.699901in}}%
\pgfpathlineto{\pgfqpoint{5.516473in}{0.704236in}}%
\pgfpathlineto{\pgfqpoint{5.520991in}{0.702936in}}%
\pgfpathlineto{\pgfqpoint{5.525509in}{0.704670in}}%
\pgfpathlineto{\pgfqpoint{5.534545in}{0.702069in}}%
\pgfpathlineto{\pgfqpoint{5.534545in}{0.702069in}}%
\pgfusepath{stroke}%
\end{pgfscope}%
\begin{pgfscope}%
\pgfpathrectangle{\pgfqpoint{0.800000in}{0.528000in}}{\pgfqpoint{4.960000in}{3.696000in}}%
\pgfusepath{clip}%
\pgfsetrectcap%
\pgfsetroundjoin%
\pgfsetlinewidth{1.505625pt}%
\definecolor{currentstroke}{rgb}{1.000000,0.498039,0.054902}%
\pgfsetstrokecolor{currentstroke}%
\pgfsetdash{}{0pt}%
\pgfpathmoveto{\pgfqpoint{1.025455in}{0.968233in}}%
\pgfpathlineto{\pgfqpoint{1.029973in}{1.132960in}}%
\pgfpathlineto{\pgfqpoint{1.039009in}{0.957396in}}%
\pgfpathlineto{\pgfqpoint{1.043527in}{0.887170in}}%
\pgfpathlineto{\pgfqpoint{1.048045in}{0.867230in}}%
\pgfpathlineto{\pgfqpoint{1.052563in}{0.820412in}}%
\pgfpathlineto{\pgfqpoint{1.057081in}{0.799605in}}%
\pgfpathlineto{\pgfqpoint{1.061600in}{0.790501in}}%
\pgfpathlineto{\pgfqpoint{1.066118in}{0.787900in}}%
\pgfpathlineto{\pgfqpoint{1.075154in}{0.764492in}}%
\pgfpathlineto{\pgfqpoint{1.079672in}{0.764058in}}%
\pgfpathlineto{\pgfqpoint{1.088708in}{0.740650in}}%
\pgfpathlineto{\pgfqpoint{1.093226in}{0.748453in}}%
\pgfpathlineto{\pgfqpoint{1.097745in}{0.741950in}}%
\pgfpathlineto{\pgfqpoint{1.111299in}{0.771861in}}%
\pgfpathlineto{\pgfqpoint{1.115817in}{0.771861in}}%
\pgfpathlineto{\pgfqpoint{1.120335in}{0.775763in}}%
\pgfpathlineto{\pgfqpoint{1.124853in}{0.774029in}}%
\pgfpathlineto{\pgfqpoint{1.129371in}{0.774029in}}%
\pgfpathlineto{\pgfqpoint{1.133890in}{0.777497in}}%
\pgfpathlineto{\pgfqpoint{1.138408in}{0.769694in}}%
\pgfpathlineto{\pgfqpoint{1.142926in}{0.767960in}}%
\pgfpathlineto{\pgfqpoint{1.147444in}{0.762324in}}%
\pgfpathlineto{\pgfqpoint{1.151962in}{0.768393in}}%
\pgfpathlineto{\pgfqpoint{1.156480in}{0.768393in}}%
\pgfpathlineto{\pgfqpoint{1.160998in}{0.776630in}}%
\pgfpathlineto{\pgfqpoint{1.165516in}{0.761891in}}%
\pgfpathlineto{\pgfqpoint{1.170035in}{0.773595in}}%
\pgfpathlineto{\pgfqpoint{1.174553in}{0.776630in}}%
\pgfpathlineto{\pgfqpoint{1.179071in}{0.776630in}}%
\pgfpathlineto{\pgfqpoint{1.183589in}{0.767093in}}%
\pgfpathlineto{\pgfqpoint{1.188107in}{0.773162in}}%
\pgfpathlineto{\pgfqpoint{1.192625in}{0.774462in}}%
\pgfpathlineto{\pgfqpoint{1.197143in}{0.773162in}}%
\pgfpathlineto{\pgfqpoint{1.201662in}{0.767526in}}%
\pgfpathlineto{\pgfqpoint{1.206180in}{0.773595in}}%
\pgfpathlineto{\pgfqpoint{1.210698in}{0.784432in}}%
\pgfpathlineto{\pgfqpoint{1.215216in}{0.774462in}}%
\pgfpathlineto{\pgfqpoint{1.219734in}{0.769694in}}%
\pgfpathlineto{\pgfqpoint{1.224252in}{0.775763in}}%
\pgfpathlineto{\pgfqpoint{1.228770in}{0.773595in}}%
\pgfpathlineto{\pgfqpoint{1.233288in}{0.774462in}}%
\pgfpathlineto{\pgfqpoint{1.242325in}{0.783565in}}%
\pgfpathlineto{\pgfqpoint{1.251361in}{0.828215in}}%
\pgfpathlineto{\pgfqpoint{1.260397in}{0.879801in}}%
\pgfpathlineto{\pgfqpoint{1.264915in}{0.917515in}}%
\pgfpathlineto{\pgfqpoint{1.269433in}{0.986440in}}%
\pgfpathlineto{\pgfqpoint{1.273952in}{1.002479in}}%
\pgfpathlineto{\pgfqpoint{1.278470in}{1.083976in}}%
\pgfpathlineto{\pgfqpoint{1.287506in}{1.011583in}}%
\pgfpathlineto{\pgfqpoint{1.292024in}{1.028489in}}%
\pgfpathlineto{\pgfqpoint{1.301060in}{1.232664in}}%
\pgfpathlineto{\pgfqpoint{1.305578in}{1.156802in}}%
\pgfpathlineto{\pgfqpoint{1.310097in}{1.129059in}}%
\pgfpathlineto{\pgfqpoint{1.314615in}{1.166339in}}%
\pgfpathlineto{\pgfqpoint{1.323651in}{1.068370in}}%
\pgfpathlineto{\pgfqpoint{1.332687in}{1.061868in}}%
\pgfpathlineto{\pgfqpoint{1.337205in}{1.031090in}}%
\pgfpathlineto{\pgfqpoint{1.341723in}{1.056666in}}%
\pgfpathlineto{\pgfqpoint{1.346242in}{1.015050in}}%
\pgfpathlineto{\pgfqpoint{1.350760in}{1.055799in}}%
\pgfpathlineto{\pgfqpoint{1.355278in}{1.051897in}}%
\pgfpathlineto{\pgfqpoint{1.359796in}{1.025021in}}%
\pgfpathlineto{\pgfqpoint{1.364314in}{1.051030in}}%
\pgfpathlineto{\pgfqpoint{1.368832in}{1.086143in}}%
\pgfpathlineto{\pgfqpoint{1.373350in}{1.087010in}}%
\pgfpathlineto{\pgfqpoint{1.377868in}{1.112586in}}%
\pgfpathlineto{\pgfqpoint{1.382387in}{1.096547in}}%
\pgfpathlineto{\pgfqpoint{1.386905in}{1.096114in}}%
\pgfpathlineto{\pgfqpoint{1.391423in}{1.069670in}}%
\pgfpathlineto{\pgfqpoint{1.400459in}{1.056666in}}%
\pgfpathlineto{\pgfqpoint{1.409495in}{1.081808in}}%
\pgfpathlineto{\pgfqpoint{1.414013in}{1.110852in}}%
\pgfpathlineto{\pgfqpoint{1.418532in}{1.102182in}}%
\pgfpathlineto{\pgfqpoint{1.427568in}{1.091779in}}%
\pgfpathlineto{\pgfqpoint{1.432086in}{1.103916in}}%
\pgfpathlineto{\pgfqpoint{1.436604in}{1.094813in}}%
\pgfpathlineto{\pgfqpoint{1.441122in}{1.146832in}}%
\pgfpathlineto{\pgfqpoint{1.445640in}{1.144665in}}%
\pgfpathlineto{\pgfqpoint{1.450158in}{1.126891in}}%
\pgfpathlineto{\pgfqpoint{1.454677in}{1.160270in}}%
\pgfpathlineto{\pgfqpoint{1.459195in}{1.165906in}}%
\pgfpathlineto{\pgfqpoint{1.463713in}{1.165906in}}%
\pgfpathlineto{\pgfqpoint{1.468231in}{1.214023in}}%
\pgfpathlineto{\pgfqpoint{1.472749in}{1.211423in}}%
\pgfpathlineto{\pgfqpoint{1.477267in}{1.207088in}}%
\pgfpathlineto{\pgfqpoint{1.481785in}{1.185413in}}%
\pgfpathlineto{\pgfqpoint{1.486304in}{1.153335in}}%
\pgfpathlineto{\pgfqpoint{1.490822in}{1.174576in}}%
\pgfpathlineto{\pgfqpoint{1.495340in}{1.136428in}}%
\pgfpathlineto{\pgfqpoint{1.499858in}{1.139463in}}%
\pgfpathlineto{\pgfqpoint{1.504376in}{1.132960in}}%
\pgfpathlineto{\pgfqpoint{1.508894in}{1.124291in}}%
\pgfpathlineto{\pgfqpoint{1.513412in}{1.166773in}}%
\pgfpathlineto{\pgfqpoint{1.517930in}{1.195383in}}%
\pgfpathlineto{\pgfqpoint{1.522449in}{1.178044in}}%
\pgfpathlineto{\pgfqpoint{1.526967in}{1.177610in}}%
\pgfpathlineto{\pgfqpoint{1.531485in}{1.192349in}}%
\pgfpathlineto{\pgfqpoint{1.540521in}{1.171975in}}%
\pgfpathlineto{\pgfqpoint{1.545039in}{1.133827in}}%
\pgfpathlineto{\pgfqpoint{1.549557in}{1.132093in}}%
\pgfpathlineto{\pgfqpoint{1.554075in}{1.123424in}}%
\pgfpathlineto{\pgfqpoint{1.563112in}{1.295087in}}%
\pgfpathlineto{\pgfqpoint{1.567630in}{1.291619in}}%
\pgfpathlineto{\pgfqpoint{1.572148in}{1.359243in}}%
\pgfpathlineto{\pgfqpoint{1.576666in}{1.370948in}}%
\pgfpathlineto{\pgfqpoint{1.581184in}{1.437706in}}%
\pgfpathlineto{\pgfqpoint{1.585702in}{1.390455in}}%
\pgfpathlineto{\pgfqpoint{1.590220in}{1.460247in}}%
\pgfpathlineto{\pgfqpoint{1.594739in}{1.429903in}}%
\pgfpathlineto{\pgfqpoint{1.599257in}{1.419499in}}%
\pgfpathlineto{\pgfqpoint{1.603775in}{1.366179in}}%
\pgfpathlineto{\pgfqpoint{1.608293in}{1.330633in}}%
\pgfpathlineto{\pgfqpoint{1.612811in}{1.279047in}}%
\pgfpathlineto{\pgfqpoint{1.617329in}{1.331933in}}%
\pgfpathlineto{\pgfqpoint{1.621847in}{1.322397in}}%
\pgfpathlineto{\pgfqpoint{1.626365in}{1.281215in}}%
\pgfpathlineto{\pgfqpoint{1.630884in}{1.181078in}}%
\pgfpathlineto{\pgfqpoint{1.635402in}{1.214023in}}%
\pgfpathlineto{\pgfqpoint{1.639920in}{1.281648in}}%
\pgfpathlineto{\pgfqpoint{1.644438in}{1.226595in}}%
\pgfpathlineto{\pgfqpoint{1.653474in}{1.053198in}}%
\pgfpathlineto{\pgfqpoint{1.657992in}{1.078340in}}%
\pgfpathlineto{\pgfqpoint{1.662510in}{1.197551in}}%
\pgfpathlineto{\pgfqpoint{1.667029in}{1.093946in}}%
\pgfpathlineto{\pgfqpoint{1.671547in}{1.127325in}}%
\pgfpathlineto{\pgfqpoint{1.676065in}{1.184546in}}%
\pgfpathlineto{\pgfqpoint{1.680583in}{1.150734in}}%
\pgfpathlineto{\pgfqpoint{1.685101in}{1.205787in}}%
\pgfpathlineto{\pgfqpoint{1.689619in}{1.289018in}}%
\pgfpathlineto{\pgfqpoint{1.694137in}{1.243501in}}%
\pgfpathlineto{\pgfqpoint{1.698655in}{1.316328in}}%
\pgfpathlineto{\pgfqpoint{1.703174in}{1.290318in}}%
\pgfpathlineto{\pgfqpoint{1.712210in}{1.387420in}}%
\pgfpathlineto{\pgfqpoint{1.716728in}{1.412130in}}%
\pgfpathlineto{\pgfqpoint{1.721246in}{1.407795in}}%
\pgfpathlineto{\pgfqpoint{1.725764in}{1.448543in}}%
\pgfpathlineto{\pgfqpoint{1.730282in}{1.467617in}}%
\pgfpathlineto{\pgfqpoint{1.734801in}{1.494927in}}%
\pgfpathlineto{\pgfqpoint{1.739319in}{1.601999in}}%
\pgfpathlineto{\pgfqpoint{1.743837in}{1.567320in}}%
\pgfpathlineto{\pgfqpoint{1.748355in}{1.559083in}}%
\pgfpathlineto{\pgfqpoint{1.752873in}{1.712106in}}%
\pgfpathlineto{\pgfqpoint{1.757391in}{1.773662in}}%
\pgfpathlineto{\pgfqpoint{1.761909in}{1.804007in}}%
\pgfpathlineto{\pgfqpoint{1.766427in}{1.880735in}}%
\pgfpathlineto{\pgfqpoint{1.770946in}{2.005147in}}%
\pgfpathlineto{\pgfqpoint{1.775464in}{2.172475in}}%
\pgfpathlineto{\pgfqpoint{1.779982in}{2.163372in}}%
\pgfpathlineto{\pgfqpoint{1.784500in}{2.127826in}}%
\pgfpathlineto{\pgfqpoint{1.789018in}{2.033324in}}%
\pgfpathlineto{\pgfqpoint{1.793536in}{2.206288in}}%
\pgfpathlineto{\pgfqpoint{1.798054in}{2.194583in}}%
\pgfpathlineto{\pgfqpoint{1.807091in}{2.317262in}}%
\pgfpathlineto{\pgfqpoint{1.811609in}{2.395290in}}%
\pgfpathlineto{\pgfqpoint{1.816127in}{2.437773in}}%
\pgfpathlineto{\pgfqpoint{1.820645in}{2.706105in}}%
\pgfpathlineto{\pgfqpoint{1.829681in}{2.531407in}}%
\pgfpathlineto{\pgfqpoint{1.834199in}{2.579958in}}%
\pgfpathlineto{\pgfqpoint{1.838717in}{2.640647in}}%
\pgfpathlineto{\pgfqpoint{1.843236in}{2.570855in}}%
\pgfpathlineto{\pgfqpoint{1.847754in}{2.602066in}}%
\pgfpathlineto{\pgfqpoint{1.852272in}{2.655819in}}%
\pgfpathlineto{\pgfqpoint{1.856790in}{2.553949in}}%
\pgfpathlineto{\pgfqpoint{1.861308in}{2.561752in}}%
\pgfpathlineto{\pgfqpoint{1.865826in}{2.597731in}}%
\pgfpathlineto{\pgfqpoint{1.874862in}{2.361045in}}%
\pgfpathlineto{\pgfqpoint{1.879381in}{2.498895in}}%
\pgfpathlineto{\pgfqpoint{1.883899in}{2.409596in}}%
\pgfpathlineto{\pgfqpoint{1.888417in}{2.486324in}}%
\pgfpathlineto{\pgfqpoint{1.892935in}{2.358010in}}%
\pgfpathlineto{\pgfqpoint{1.897453in}{2.281715in}}%
\pgfpathlineto{\pgfqpoint{1.901971in}{2.238366in}}%
\pgfpathlineto{\pgfqpoint{1.906489in}{2.106584in}}%
\pgfpathlineto{\pgfqpoint{1.915526in}{2.293420in}}%
\pgfpathlineto{\pgfqpoint{1.920044in}{2.171175in}}%
\pgfpathlineto{\pgfqpoint{1.924562in}{2.097915in}}%
\pgfpathlineto{\pgfqpoint{1.933598in}{1.991709in}}%
\pgfpathlineto{\pgfqpoint{1.938116in}{1.965266in}}%
\pgfpathlineto{\pgfqpoint{1.942634in}{1.948793in}}%
\pgfpathlineto{\pgfqpoint{1.951671in}{1.908478in}}%
\pgfpathlineto{\pgfqpoint{1.956189in}{1.873799in}}%
\pgfpathlineto{\pgfqpoint{1.960707in}{1.908912in}}%
\pgfpathlineto{\pgfqpoint{1.965225in}{1.810509in}}%
\pgfpathlineto{\pgfqpoint{1.969743in}{1.908478in}}%
\pgfpathlineto{\pgfqpoint{1.974261in}{1.881602in}}%
\pgfpathlineto{\pgfqpoint{1.978779in}{1.966566in}}%
\pgfpathlineto{\pgfqpoint{1.983298in}{1.969601in}}%
\pgfpathlineto{\pgfqpoint{1.987816in}{1.989541in}}%
\pgfpathlineto{\pgfqpoint{1.992334in}{1.953562in}}%
\pgfpathlineto{\pgfqpoint{1.996852in}{1.949227in}}%
\pgfpathlineto{\pgfqpoint{2.001370in}{1.971768in}}%
\pgfpathlineto{\pgfqpoint{2.005888in}{1.943591in}}%
\pgfpathlineto{\pgfqpoint{2.010406in}{1.959630in}}%
\pgfpathlineto{\pgfqpoint{2.014924in}{2.026388in}}%
\pgfpathlineto{\pgfqpoint{2.019443in}{2.066270in}}%
\pgfpathlineto{\pgfqpoint{2.023961in}{2.071038in}}%
\pgfpathlineto{\pgfqpoint{2.028479in}{2.187648in}}%
\pgfpathlineto{\pgfqpoint{2.032997in}{2.212790in}}%
\pgfpathlineto{\pgfqpoint{2.037515in}{2.079708in}}%
\pgfpathlineto{\pgfqpoint{2.042033in}{2.157737in}}%
\pgfpathlineto{\pgfqpoint{2.046551in}{2.062802in}}%
\pgfpathlineto{\pgfqpoint{2.060106in}{2.215825in}}%
\pgfpathlineto{\pgfqpoint{2.064624in}{2.275213in}}%
\pgfpathlineto{\pgfqpoint{2.069142in}{2.178111in}}%
\pgfpathlineto{\pgfqpoint{2.073660in}{2.282582in}}%
\pgfpathlineto{\pgfqpoint{2.078178in}{2.210623in}}%
\pgfpathlineto{\pgfqpoint{2.082696in}{2.239667in}}%
\pgfpathlineto{\pgfqpoint{2.087214in}{2.220593in}}%
\pgfpathlineto{\pgfqpoint{2.091733in}{2.221893in}}%
\pgfpathlineto{\pgfqpoint{2.096251in}{2.234898in}}%
\pgfpathlineto{\pgfqpoint{2.100769in}{2.255272in}}%
\pgfpathlineto{\pgfqpoint{2.105287in}{2.159471in}}%
\pgfpathlineto{\pgfqpoint{2.109805in}{2.238800in}}%
\pgfpathlineto{\pgfqpoint{2.114323in}{2.232731in}}%
\pgfpathlineto{\pgfqpoint{2.118841in}{2.175076in}}%
\pgfpathlineto{\pgfqpoint{2.123359in}{2.165973in}}%
\pgfpathlineto{\pgfqpoint{2.127878in}{2.185914in}}%
\pgfpathlineto{\pgfqpoint{2.132396in}{2.141697in}}%
\pgfpathlineto{\pgfqpoint{2.136914in}{2.245736in}}%
\pgfpathlineto{\pgfqpoint{2.145950in}{2.145599in}}%
\pgfpathlineto{\pgfqpoint{2.150468in}{2.191115in}}%
\pgfpathlineto{\pgfqpoint{2.154986in}{2.023354in}}%
\pgfpathlineto{\pgfqpoint{2.159504in}{2.025088in}}%
\pgfpathlineto{\pgfqpoint{2.164023in}{1.916715in}}%
\pgfpathlineto{\pgfqpoint{2.168541in}{1.985640in}}%
\pgfpathlineto{\pgfqpoint{2.173059in}{1.966133in}}%
\pgfpathlineto{\pgfqpoint{2.177577in}{1.940557in}}%
\pgfpathlineto{\pgfqpoint{2.182095in}{1.885503in}}%
\pgfpathlineto{\pgfqpoint{2.186613in}{1.901543in}}%
\pgfpathlineto{\pgfqpoint{2.191131in}{1.889838in}}%
\pgfpathlineto{\pgfqpoint{2.195649in}{1.929720in}}%
\pgfpathlineto{\pgfqpoint{2.200168in}{1.920616in}}%
\pgfpathlineto{\pgfqpoint{2.204686in}{1.796204in}}%
\pgfpathlineto{\pgfqpoint{2.209204in}{1.710806in}}%
\pgfpathlineto{\pgfqpoint{2.213722in}{1.728579in}}%
\pgfpathlineto{\pgfqpoint{2.218240in}{1.628009in}}%
\pgfpathlineto{\pgfqpoint{2.222758in}{1.683062in}}%
\pgfpathlineto{\pgfqpoint{2.227276in}{1.651851in}}%
\pgfpathlineto{\pgfqpoint{2.231794in}{1.697801in}}%
\pgfpathlineto{\pgfqpoint{2.236313in}{1.577290in}}%
\pgfpathlineto{\pgfqpoint{2.245349in}{1.555616in}}%
\pgfpathlineto{\pgfqpoint{2.249867in}{1.548246in}}%
\pgfpathlineto{\pgfqpoint{2.254385in}{1.582926in}}%
\pgfpathlineto{\pgfqpoint{2.258903in}{1.541744in}}%
\pgfpathlineto{\pgfqpoint{2.263421in}{1.511833in}}%
\pgfpathlineto{\pgfqpoint{2.267940in}{1.556483in}}%
\pgfpathlineto{\pgfqpoint{2.272458in}{1.512700in}}%
\pgfpathlineto{\pgfqpoint{2.276976in}{1.533507in}}%
\pgfpathlineto{\pgfqpoint{2.281494in}{1.424701in}}%
\pgfpathlineto{\pgfqpoint{2.286012in}{1.488858in}}%
\pgfpathlineto{\pgfqpoint{2.295048in}{1.396524in}}%
\pgfpathlineto{\pgfqpoint{2.299566in}{1.439006in}}%
\pgfpathlineto{\pgfqpoint{2.304085in}{1.324564in}}%
\pgfpathlineto{\pgfqpoint{2.308603in}{1.311559in}}%
\pgfpathlineto{\pgfqpoint{2.313121in}{1.238299in}}%
\pgfpathlineto{\pgfqpoint{2.317639in}{1.226595in}}%
\pgfpathlineto{\pgfqpoint{2.322157in}{1.191482in}}%
\pgfpathlineto{\pgfqpoint{2.326675in}{1.167206in}}%
\pgfpathlineto{\pgfqpoint{2.331193in}{1.201019in}}%
\pgfpathlineto{\pgfqpoint{2.340230in}{1.139463in}}%
\pgfpathlineto{\pgfqpoint{2.344748in}{1.138596in}}%
\pgfpathlineto{\pgfqpoint{2.349266in}{1.151601in}}%
\pgfpathlineto{\pgfqpoint{2.353784in}{1.142931in}}%
\pgfpathlineto{\pgfqpoint{2.358302in}{1.130793in}}%
\pgfpathlineto{\pgfqpoint{2.362820in}{1.122557in}}%
\pgfpathlineto{\pgfqpoint{2.371856in}{1.153335in}}%
\pgfpathlineto{\pgfqpoint{2.376375in}{1.112586in}}%
\pgfpathlineto{\pgfqpoint{2.380893in}{1.057966in}}%
\pgfpathlineto{\pgfqpoint{2.385411in}{1.087877in}}%
\pgfpathlineto{\pgfqpoint{2.389929in}{1.051897in}}%
\pgfpathlineto{\pgfqpoint{2.394447in}{1.036292in}}%
\pgfpathlineto{\pgfqpoint{2.398965in}{1.088311in}}%
\pgfpathlineto{\pgfqpoint{2.403483in}{1.071838in}}%
\pgfpathlineto{\pgfqpoint{2.408001in}{1.082675in}}%
\pgfpathlineto{\pgfqpoint{2.412520in}{1.083976in}}%
\pgfpathlineto{\pgfqpoint{2.417038in}{1.069237in}}%
\pgfpathlineto{\pgfqpoint{2.421556in}{1.100015in}}%
\pgfpathlineto{\pgfqpoint{2.426074in}{1.066203in}}%
\pgfpathlineto{\pgfqpoint{2.430592in}{1.055365in}}%
\pgfpathlineto{\pgfqpoint{2.435110in}{1.068804in}}%
\pgfpathlineto{\pgfqpoint{2.439628in}{1.106084in}}%
\pgfpathlineto{\pgfqpoint{2.444146in}{1.016784in}}%
\pgfpathlineto{\pgfqpoint{2.453183in}{1.009849in}}%
\pgfpathlineto{\pgfqpoint{2.457701in}{1.013316in}}%
\pgfpathlineto{\pgfqpoint{2.462219in}{1.019385in}}%
\pgfpathlineto{\pgfqpoint{2.471255in}{1.048863in}}%
\pgfpathlineto{\pgfqpoint{2.475773in}{1.070971in}}%
\pgfpathlineto{\pgfqpoint{2.480291in}{1.014183in}}%
\pgfpathlineto{\pgfqpoint{2.484810in}{0.993809in}}%
\pgfpathlineto{\pgfqpoint{2.489328in}{1.021553in}}%
\pgfpathlineto{\pgfqpoint{2.493846in}{1.018085in}}%
\pgfpathlineto{\pgfqpoint{2.498364in}{0.988607in}}%
\pgfpathlineto{\pgfqpoint{2.502882in}{0.980371in}}%
\pgfpathlineto{\pgfqpoint{2.507400in}{0.957396in}}%
\pgfpathlineto{\pgfqpoint{2.511918in}{0.975169in}}%
\pgfpathlineto{\pgfqpoint{2.516437in}{1.004647in}}%
\pgfpathlineto{\pgfqpoint{2.520955in}{0.969534in}}%
\pgfpathlineto{\pgfqpoint{2.525473in}{0.970834in}}%
\pgfpathlineto{\pgfqpoint{2.529991in}{0.986006in}}%
\pgfpathlineto{\pgfqpoint{2.534509in}{0.988174in}}%
\pgfpathlineto{\pgfqpoint{2.539027in}{0.987740in}}%
\pgfpathlineto{\pgfqpoint{2.543545in}{0.961297in}}%
\pgfpathlineto{\pgfqpoint{2.548063in}{0.957396in}}%
\pgfpathlineto{\pgfqpoint{2.552582in}{0.947426in}}%
\pgfpathlineto{\pgfqpoint{2.557100in}{0.961297in}}%
\pgfpathlineto{\pgfqpoint{2.561618in}{0.955662in}}%
\pgfpathlineto{\pgfqpoint{2.566136in}{0.920116in}}%
\pgfpathlineto{\pgfqpoint{2.570654in}{0.915781in}}%
\pgfpathlineto{\pgfqpoint{2.575172in}{0.930953in}}%
\pgfpathlineto{\pgfqpoint{2.579690in}{0.939189in}}%
\pgfpathlineto{\pgfqpoint{2.584208in}{0.894540in}}%
\pgfpathlineto{\pgfqpoint{2.588727in}{0.893673in}}%
\pgfpathlineto{\pgfqpoint{2.593245in}{0.914914in}}%
\pgfpathlineto{\pgfqpoint{2.597763in}{0.886737in}}%
\pgfpathlineto{\pgfqpoint{2.602281in}{0.880234in}}%
\pgfpathlineto{\pgfqpoint{2.606799in}{0.868963in}}%
\pgfpathlineto{\pgfqpoint{2.611317in}{0.889338in}}%
\pgfpathlineto{\pgfqpoint{2.615835in}{0.850323in}}%
\pgfpathlineto{\pgfqpoint{2.620353in}{0.892806in}}%
\pgfpathlineto{\pgfqpoint{2.624872in}{0.865929in}}%
\pgfpathlineto{\pgfqpoint{2.629390in}{0.882835in}}%
\pgfpathlineto{\pgfqpoint{2.633908in}{0.891939in}}%
\pgfpathlineto{\pgfqpoint{2.638426in}{0.883702in}}%
\pgfpathlineto{\pgfqpoint{2.642944in}{0.900608in}}%
\pgfpathlineto{\pgfqpoint{2.647462in}{0.904076in}}%
\pgfpathlineto{\pgfqpoint{2.651980in}{0.905810in}}%
\pgfpathlineto{\pgfqpoint{2.656498in}{0.876766in}}%
\pgfpathlineto{\pgfqpoint{2.661017in}{0.886303in}}%
\pgfpathlineto{\pgfqpoint{2.665535in}{0.853358in}}%
\pgfpathlineto{\pgfqpoint{2.670053in}{0.834718in}}%
\pgfpathlineto{\pgfqpoint{2.674571in}{0.842954in}}%
\pgfpathlineto{\pgfqpoint{2.679089in}{0.891939in}}%
\pgfpathlineto{\pgfqpoint{2.683607in}{0.865062in}}%
\pgfpathlineto{\pgfqpoint{2.688125in}{0.866796in}}%
\pgfpathlineto{\pgfqpoint{2.697162in}{0.829082in}}%
\pgfpathlineto{\pgfqpoint{2.701680in}{0.823447in}}%
\pgfpathlineto{\pgfqpoint{2.706198in}{0.824314in}}%
\pgfpathlineto{\pgfqpoint{2.710716in}{0.821279in}}%
\pgfpathlineto{\pgfqpoint{2.715234in}{0.813476in}}%
\pgfpathlineto{\pgfqpoint{2.719752in}{0.810875in}}%
\pgfpathlineto{\pgfqpoint{2.724270in}{0.813043in}}%
\pgfpathlineto{\pgfqpoint{2.733307in}{0.820412in}}%
\pgfpathlineto{\pgfqpoint{2.737825in}{0.818678in}}%
\pgfpathlineto{\pgfqpoint{2.742343in}{0.800038in}}%
\pgfpathlineto{\pgfqpoint{2.746861in}{0.806541in}}%
\pgfpathlineto{\pgfqpoint{2.751379in}{0.822146in}}%
\pgfpathlineto{\pgfqpoint{2.755897in}{0.821713in}}%
\pgfpathlineto{\pgfqpoint{2.760415in}{0.794836in}}%
\pgfpathlineto{\pgfqpoint{2.764934in}{0.810875in}}%
\pgfpathlineto{\pgfqpoint{2.769452in}{0.837752in}}%
\pgfpathlineto{\pgfqpoint{2.773970in}{0.833851in}}%
\pgfpathlineto{\pgfqpoint{2.778488in}{0.794403in}}%
\pgfpathlineto{\pgfqpoint{2.783006in}{0.795270in}}%
\pgfpathlineto{\pgfqpoint{2.796560in}{0.777930in}}%
\pgfpathlineto{\pgfqpoint{2.801079in}{0.782698in}}%
\pgfpathlineto{\pgfqpoint{2.810115in}{0.760590in}}%
\pgfpathlineto{\pgfqpoint{2.814633in}{0.761457in}}%
\pgfpathlineto{\pgfqpoint{2.819151in}{0.766659in}}%
\pgfpathlineto{\pgfqpoint{2.823669in}{0.774029in}}%
\pgfpathlineto{\pgfqpoint{2.828187in}{0.774029in}}%
\pgfpathlineto{\pgfqpoint{2.837224in}{0.788334in}}%
\pgfpathlineto{\pgfqpoint{2.841742in}{0.792669in}}%
\pgfpathlineto{\pgfqpoint{2.846260in}{0.806107in}}%
\pgfpathlineto{\pgfqpoint{2.855296in}{0.777497in}}%
\pgfpathlineto{\pgfqpoint{2.859814in}{0.792669in}}%
\pgfpathlineto{\pgfqpoint{2.864332in}{0.794836in}}%
\pgfpathlineto{\pgfqpoint{2.868850in}{0.793969in}}%
\pgfpathlineto{\pgfqpoint{2.873369in}{0.794836in}}%
\pgfpathlineto{\pgfqpoint{2.882405in}{0.779231in}}%
\pgfpathlineto{\pgfqpoint{2.886923in}{0.780531in}}%
\pgfpathlineto{\pgfqpoint{2.891441in}{0.790501in}}%
\pgfpathlineto{\pgfqpoint{2.895959in}{0.804373in}}%
\pgfpathlineto{\pgfqpoint{2.900477in}{0.787033in}}%
\pgfpathlineto{\pgfqpoint{2.904995in}{0.782698in}}%
\pgfpathlineto{\pgfqpoint{2.909514in}{0.789634in}}%
\pgfpathlineto{\pgfqpoint{2.914032in}{0.808708in}}%
\pgfpathlineto{\pgfqpoint{2.918550in}{0.787033in}}%
\pgfpathlineto{\pgfqpoint{2.923068in}{0.793536in}}%
\pgfpathlineto{\pgfqpoint{2.927586in}{0.806107in}}%
\pgfpathlineto{\pgfqpoint{2.932104in}{0.794836in}}%
\pgfpathlineto{\pgfqpoint{2.936622in}{0.800038in}}%
\pgfpathlineto{\pgfqpoint{2.941140in}{0.814343in}}%
\pgfpathlineto{\pgfqpoint{2.945659in}{0.811742in}}%
\pgfpathlineto{\pgfqpoint{2.950177in}{0.806541in}}%
\pgfpathlineto{\pgfqpoint{2.954695in}{0.803940in}}%
\pgfpathlineto{\pgfqpoint{2.963731in}{0.783565in}}%
\pgfpathlineto{\pgfqpoint{2.968249in}{0.781832in}}%
\pgfpathlineto{\pgfqpoint{2.972767in}{0.767093in}}%
\pgfpathlineto{\pgfqpoint{2.977285in}{0.768393in}}%
\pgfpathlineto{\pgfqpoint{2.990840in}{0.790068in}}%
\pgfpathlineto{\pgfqpoint{2.995358in}{0.787467in}}%
\pgfpathlineto{\pgfqpoint{2.999876in}{0.786600in}}%
\pgfpathlineto{\pgfqpoint{3.004394in}{0.777930in}}%
\pgfpathlineto{\pgfqpoint{3.008912in}{0.775763in}}%
\pgfpathlineto{\pgfqpoint{3.013430in}{0.775329in}}%
\pgfpathlineto{\pgfqpoint{3.017949in}{0.781832in}}%
\pgfpathlineto{\pgfqpoint{3.022467in}{0.774462in}}%
\pgfpathlineto{\pgfqpoint{3.031503in}{0.782698in}}%
\pgfpathlineto{\pgfqpoint{3.036021in}{0.807841in}}%
\pgfpathlineto{\pgfqpoint{3.040539in}{0.780531in}}%
\pgfpathlineto{\pgfqpoint{3.045057in}{0.785733in}}%
\pgfpathlineto{\pgfqpoint{3.049576in}{0.784866in}}%
\pgfpathlineto{\pgfqpoint{3.054094in}{0.800038in}}%
\pgfpathlineto{\pgfqpoint{3.058612in}{0.788767in}}%
\pgfpathlineto{\pgfqpoint{3.063130in}{0.781398in}}%
\pgfpathlineto{\pgfqpoint{3.067648in}{0.783132in}}%
\pgfpathlineto{\pgfqpoint{3.076684in}{0.756689in}}%
\pgfpathlineto{\pgfqpoint{3.081202in}{0.763191in}}%
\pgfpathlineto{\pgfqpoint{3.094757in}{0.770994in}}%
\pgfpathlineto{\pgfqpoint{3.099275in}{0.766226in}}%
\pgfpathlineto{\pgfqpoint{3.103793in}{0.764492in}}%
\pgfpathlineto{\pgfqpoint{3.108311in}{0.770127in}}%
\pgfpathlineto{\pgfqpoint{3.112829in}{0.787900in}}%
\pgfpathlineto{\pgfqpoint{3.117347in}{0.784432in}}%
\pgfpathlineto{\pgfqpoint{3.121866in}{0.771861in}}%
\pgfpathlineto{\pgfqpoint{3.126384in}{0.766659in}}%
\pgfpathlineto{\pgfqpoint{3.130902in}{0.770994in}}%
\pgfpathlineto{\pgfqpoint{3.135420in}{0.760157in}}%
\pgfpathlineto{\pgfqpoint{3.139938in}{0.759723in}}%
\pgfpathlineto{\pgfqpoint{3.144456in}{0.762758in}}%
\pgfpathlineto{\pgfqpoint{3.148974in}{0.754521in}}%
\pgfpathlineto{\pgfqpoint{3.153492in}{0.762758in}}%
\pgfpathlineto{\pgfqpoint{3.158011in}{0.779664in}}%
\pgfpathlineto{\pgfqpoint{3.162529in}{0.775329in}}%
\pgfpathlineto{\pgfqpoint{3.167047in}{0.758856in}}%
\pgfpathlineto{\pgfqpoint{3.171565in}{0.752788in}}%
\pgfpathlineto{\pgfqpoint{3.176083in}{0.744118in}}%
\pgfpathlineto{\pgfqpoint{3.180601in}{0.744551in}}%
\pgfpathlineto{\pgfqpoint{3.185119in}{0.752354in}}%
\pgfpathlineto{\pgfqpoint{3.189637in}{0.742817in}}%
\pgfpathlineto{\pgfqpoint{3.194156in}{0.741517in}}%
\pgfpathlineto{\pgfqpoint{3.198674in}{0.752788in}}%
\pgfpathlineto{\pgfqpoint{3.203192in}{0.753654in}}%
\pgfpathlineto{\pgfqpoint{3.212228in}{0.737182in}}%
\pgfpathlineto{\pgfqpoint{3.216746in}{0.732847in}}%
\pgfpathlineto{\pgfqpoint{3.221264in}{0.741950in}}%
\pgfpathlineto{\pgfqpoint{3.225782in}{0.729379in}}%
\pgfpathlineto{\pgfqpoint{3.230301in}{0.747586in}}%
\pgfpathlineto{\pgfqpoint{3.234819in}{0.748019in}}%
\pgfpathlineto{\pgfqpoint{3.239337in}{0.763625in}}%
\pgfpathlineto{\pgfqpoint{3.243855in}{0.763625in}}%
\pgfpathlineto{\pgfqpoint{3.248373in}{0.761024in}}%
\pgfpathlineto{\pgfqpoint{3.252891in}{0.751921in}}%
\pgfpathlineto{\pgfqpoint{3.257409in}{0.746285in}}%
\pgfpathlineto{\pgfqpoint{3.261927in}{0.754521in}}%
\pgfpathlineto{\pgfqpoint{3.266446in}{0.769260in}}%
\pgfpathlineto{\pgfqpoint{3.270964in}{0.761891in}}%
\pgfpathlineto{\pgfqpoint{3.275482in}{0.761457in}}%
\pgfpathlineto{\pgfqpoint{3.284518in}{0.746719in}}%
\pgfpathlineto{\pgfqpoint{3.289036in}{0.752788in}}%
\pgfpathlineto{\pgfqpoint{3.298073in}{0.747152in}}%
\pgfpathlineto{\pgfqpoint{3.302591in}{0.761891in}}%
\pgfpathlineto{\pgfqpoint{3.307109in}{0.758856in}}%
\pgfpathlineto{\pgfqpoint{3.311627in}{0.761024in}}%
\pgfpathlineto{\pgfqpoint{3.316145in}{0.759290in}}%
\pgfpathlineto{\pgfqpoint{3.325181in}{0.764925in}}%
\pgfpathlineto{\pgfqpoint{3.329699in}{0.765359in}}%
\pgfpathlineto{\pgfqpoint{3.334218in}{0.770994in}}%
\pgfpathlineto{\pgfqpoint{3.338736in}{0.761891in}}%
\pgfpathlineto{\pgfqpoint{3.343254in}{0.766659in}}%
\pgfpathlineto{\pgfqpoint{3.347772in}{0.788767in}}%
\pgfpathlineto{\pgfqpoint{3.352290in}{0.764925in}}%
\pgfpathlineto{\pgfqpoint{3.356808in}{0.756689in}}%
\pgfpathlineto{\pgfqpoint{3.361326in}{0.754088in}}%
\pgfpathlineto{\pgfqpoint{3.365844in}{0.759723in}}%
\pgfpathlineto{\pgfqpoint{3.370363in}{0.748019in}}%
\pgfpathlineto{\pgfqpoint{3.374881in}{0.740650in}}%
\pgfpathlineto{\pgfqpoint{3.383917in}{0.739783in}}%
\pgfpathlineto{\pgfqpoint{3.388435in}{0.748019in}}%
\pgfpathlineto{\pgfqpoint{3.392953in}{0.740650in}}%
\pgfpathlineto{\pgfqpoint{3.397471in}{0.738049in}}%
\pgfpathlineto{\pgfqpoint{3.401989in}{0.726778in}}%
\pgfpathlineto{\pgfqpoint{3.406508in}{0.730246in}}%
\pgfpathlineto{\pgfqpoint{3.411026in}{0.728512in}}%
\pgfpathlineto{\pgfqpoint{3.415544in}{0.738916in}}%
\pgfpathlineto{\pgfqpoint{3.420062in}{0.744551in}}%
\pgfpathlineto{\pgfqpoint{3.424580in}{0.746285in}}%
\pgfpathlineto{\pgfqpoint{3.429098in}{0.735014in}}%
\pgfpathlineto{\pgfqpoint{3.433616in}{0.746719in}}%
\pgfpathlineto{\pgfqpoint{3.438134in}{0.740650in}}%
\pgfpathlineto{\pgfqpoint{3.442653in}{0.724611in}}%
\pgfpathlineto{\pgfqpoint{3.447171in}{0.734581in}}%
\pgfpathlineto{\pgfqpoint{3.451689in}{0.728512in}}%
\pgfpathlineto{\pgfqpoint{3.460725in}{0.735448in}}%
\pgfpathlineto{\pgfqpoint{3.465243in}{0.741950in}}%
\pgfpathlineto{\pgfqpoint{3.469761in}{0.737182in}}%
\pgfpathlineto{\pgfqpoint{3.474279in}{0.734147in}}%
\pgfpathlineto{\pgfqpoint{3.483316in}{0.735448in}}%
\pgfpathlineto{\pgfqpoint{3.487834in}{0.732413in}}%
\pgfpathlineto{\pgfqpoint{3.492352in}{0.724611in}}%
\pgfpathlineto{\pgfqpoint{3.496870in}{0.729379in}}%
\pgfpathlineto{\pgfqpoint{3.501388in}{0.725911in}}%
\pgfpathlineto{\pgfqpoint{3.505906in}{0.734147in}}%
\pgfpathlineto{\pgfqpoint{3.510424in}{0.719842in}}%
\pgfpathlineto{\pgfqpoint{3.514943in}{0.728945in}}%
\pgfpathlineto{\pgfqpoint{3.519461in}{0.726344in}}%
\pgfpathlineto{\pgfqpoint{3.523979in}{0.715507in}}%
\pgfpathlineto{\pgfqpoint{3.528497in}{0.721576in}}%
\pgfpathlineto{\pgfqpoint{3.533015in}{0.723310in}}%
\pgfpathlineto{\pgfqpoint{3.542051in}{0.747586in}}%
\pgfpathlineto{\pgfqpoint{3.551088in}{0.728945in}}%
\pgfpathlineto{\pgfqpoint{3.555606in}{0.725911in}}%
\pgfpathlineto{\pgfqpoint{3.560124in}{0.729812in}}%
\pgfpathlineto{\pgfqpoint{3.564642in}{0.730679in}}%
\pgfpathlineto{\pgfqpoint{3.573678in}{0.754955in}}%
\pgfpathlineto{\pgfqpoint{3.578196in}{0.729812in}}%
\pgfpathlineto{\pgfqpoint{3.582715in}{0.735881in}}%
\pgfpathlineto{\pgfqpoint{3.587233in}{0.730246in}}%
\pgfpathlineto{\pgfqpoint{3.591751in}{0.739783in}}%
\pgfpathlineto{\pgfqpoint{3.600787in}{0.723310in}}%
\pgfpathlineto{\pgfqpoint{3.605305in}{0.729812in}}%
\pgfpathlineto{\pgfqpoint{3.609823in}{0.732847in}}%
\pgfpathlineto{\pgfqpoint{3.618860in}{0.720709in}}%
\pgfpathlineto{\pgfqpoint{3.623378in}{0.727211in}}%
\pgfpathlineto{\pgfqpoint{3.632414in}{0.726778in}}%
\pgfpathlineto{\pgfqpoint{3.636932in}{0.730246in}}%
\pgfpathlineto{\pgfqpoint{3.641450in}{0.729812in}}%
\pgfpathlineto{\pgfqpoint{3.645968in}{0.731980in}}%
\pgfpathlineto{\pgfqpoint{3.650486in}{0.727211in}}%
\pgfpathlineto{\pgfqpoint{3.655005in}{0.726344in}}%
\pgfpathlineto{\pgfqpoint{3.659523in}{0.722443in}}%
\pgfpathlineto{\pgfqpoint{3.664041in}{0.727645in}}%
\pgfpathlineto{\pgfqpoint{3.668559in}{0.731113in}}%
\pgfpathlineto{\pgfqpoint{3.673077in}{0.728945in}}%
\pgfpathlineto{\pgfqpoint{3.677595in}{0.723310in}}%
\pgfpathlineto{\pgfqpoint{3.682113in}{0.726344in}}%
\pgfpathlineto{\pgfqpoint{3.686631in}{0.737182in}}%
\pgfpathlineto{\pgfqpoint{3.691150in}{0.727645in}}%
\pgfpathlineto{\pgfqpoint{3.695668in}{0.724177in}}%
\pgfpathlineto{\pgfqpoint{3.700186in}{0.723744in}}%
\pgfpathlineto{\pgfqpoint{3.704704in}{0.730246in}}%
\pgfpathlineto{\pgfqpoint{3.709222in}{0.715074in}}%
\pgfpathlineto{\pgfqpoint{3.713740in}{0.728512in}}%
\pgfpathlineto{\pgfqpoint{3.718258in}{0.729379in}}%
\pgfpathlineto{\pgfqpoint{3.722776in}{0.724611in}}%
\pgfpathlineto{\pgfqpoint{3.727295in}{0.716808in}}%
\pgfpathlineto{\pgfqpoint{3.731813in}{0.725044in}}%
\pgfpathlineto{\pgfqpoint{3.736331in}{0.722443in}}%
\pgfpathlineto{\pgfqpoint{3.740849in}{0.728078in}}%
\pgfpathlineto{\pgfqpoint{3.745367in}{0.727211in}}%
\pgfpathlineto{\pgfqpoint{3.749885in}{0.722010in}}%
\pgfpathlineto{\pgfqpoint{3.754403in}{0.720276in}}%
\pgfpathlineto{\pgfqpoint{3.763440in}{0.742817in}}%
\pgfpathlineto{\pgfqpoint{3.767958in}{0.731980in}}%
\pgfpathlineto{\pgfqpoint{3.772476in}{0.729379in}}%
\pgfpathlineto{\pgfqpoint{3.776994in}{0.731980in}}%
\pgfpathlineto{\pgfqpoint{3.781512in}{0.718975in}}%
\pgfpathlineto{\pgfqpoint{3.786030in}{0.717675in}}%
\pgfpathlineto{\pgfqpoint{3.790548in}{0.718542in}}%
\pgfpathlineto{\pgfqpoint{3.795066in}{0.725911in}}%
\pgfpathlineto{\pgfqpoint{3.799585in}{0.730246in}}%
\pgfpathlineto{\pgfqpoint{3.804103in}{0.725477in}}%
\pgfpathlineto{\pgfqpoint{3.808621in}{0.729812in}}%
\pgfpathlineto{\pgfqpoint{3.813139in}{0.724177in}}%
\pgfpathlineto{\pgfqpoint{3.817657in}{0.716374in}}%
\pgfpathlineto{\pgfqpoint{3.822175in}{0.714640in}}%
\pgfpathlineto{\pgfqpoint{3.826693in}{0.717675in}}%
\pgfpathlineto{\pgfqpoint{3.831212in}{0.718542in}}%
\pgfpathlineto{\pgfqpoint{3.835730in}{0.713340in}}%
\pgfpathlineto{\pgfqpoint{3.840248in}{0.718542in}}%
\pgfpathlineto{\pgfqpoint{3.844766in}{0.716374in}}%
\pgfpathlineto{\pgfqpoint{3.849284in}{0.717241in}}%
\pgfpathlineto{\pgfqpoint{3.853802in}{0.713773in}}%
\pgfpathlineto{\pgfqpoint{3.862838in}{0.715941in}}%
\pgfpathlineto{\pgfqpoint{3.867357in}{0.719842in}}%
\pgfpathlineto{\pgfqpoint{3.871875in}{0.715074in}}%
\pgfpathlineto{\pgfqpoint{3.876393in}{0.716808in}}%
\pgfpathlineto{\pgfqpoint{3.880911in}{0.712039in}}%
\pgfpathlineto{\pgfqpoint{3.885429in}{0.718542in}}%
\pgfpathlineto{\pgfqpoint{3.889947in}{0.721143in}}%
\pgfpathlineto{\pgfqpoint{3.894465in}{0.715507in}}%
\pgfpathlineto{\pgfqpoint{3.898983in}{0.712906in}}%
\pgfpathlineto{\pgfqpoint{3.903502in}{0.715507in}}%
\pgfpathlineto{\pgfqpoint{3.908020in}{0.715941in}}%
\pgfpathlineto{\pgfqpoint{3.912538in}{0.718542in}}%
\pgfpathlineto{\pgfqpoint{3.917056in}{0.715507in}}%
\pgfpathlineto{\pgfqpoint{3.921574in}{0.710739in}}%
\pgfpathlineto{\pgfqpoint{3.926092in}{0.707704in}}%
\pgfpathlineto{\pgfqpoint{3.930610in}{0.707704in}}%
\pgfpathlineto{\pgfqpoint{3.935128in}{0.720276in}}%
\pgfpathlineto{\pgfqpoint{3.939647in}{0.712906in}}%
\pgfpathlineto{\pgfqpoint{3.948683in}{0.711172in}}%
\pgfpathlineto{\pgfqpoint{3.953201in}{0.715941in}}%
\pgfpathlineto{\pgfqpoint{3.957719in}{0.705537in}}%
\pgfpathlineto{\pgfqpoint{3.962237in}{0.719842in}}%
\pgfpathlineto{\pgfqpoint{3.966755in}{0.708571in}}%
\pgfpathlineto{\pgfqpoint{3.971273in}{0.705970in}}%
\pgfpathlineto{\pgfqpoint{3.975792in}{0.705970in}}%
\pgfpathlineto{\pgfqpoint{3.984828in}{0.707704in}}%
\pgfpathlineto{\pgfqpoint{3.989346in}{0.706837in}}%
\pgfpathlineto{\pgfqpoint{3.993864in}{0.702069in}}%
\pgfpathlineto{\pgfqpoint{3.998382in}{0.702936in}}%
\pgfpathlineto{\pgfqpoint{4.002900in}{0.706404in}}%
\pgfpathlineto{\pgfqpoint{4.007418in}{0.704670in}}%
\pgfpathlineto{\pgfqpoint{4.011937in}{0.704670in}}%
\pgfpathlineto{\pgfqpoint{4.016455in}{0.714640in}}%
\pgfpathlineto{\pgfqpoint{4.020973in}{0.708138in}}%
\pgfpathlineto{\pgfqpoint{4.025491in}{0.713773in}}%
\pgfpathlineto{\pgfqpoint{4.030009in}{0.715074in}}%
\pgfpathlineto{\pgfqpoint{4.034527in}{0.712906in}}%
\pgfpathlineto{\pgfqpoint{4.043563in}{0.705103in}}%
\pgfpathlineto{\pgfqpoint{4.048082in}{0.708571in}}%
\pgfpathlineto{\pgfqpoint{4.052600in}{0.707704in}}%
\pgfpathlineto{\pgfqpoint{4.061636in}{0.711606in}}%
\pgfpathlineto{\pgfqpoint{4.066154in}{0.710739in}}%
\pgfpathlineto{\pgfqpoint{4.079709in}{0.715074in}}%
\pgfpathlineto{\pgfqpoint{4.084227in}{0.712039in}}%
\pgfpathlineto{\pgfqpoint{4.088745in}{0.707271in}}%
\pgfpathlineto{\pgfqpoint{4.097781in}{0.715941in}}%
\pgfpathlineto{\pgfqpoint{4.102299in}{0.715941in}}%
\pgfpathlineto{\pgfqpoint{4.106817in}{0.712906in}}%
\pgfpathlineto{\pgfqpoint{4.111335in}{0.715507in}}%
\pgfpathlineto{\pgfqpoint{4.120372in}{0.711172in}}%
\pgfpathlineto{\pgfqpoint{4.124890in}{0.711606in}}%
\pgfpathlineto{\pgfqpoint{4.129408in}{0.715941in}}%
\pgfpathlineto{\pgfqpoint{4.133926in}{0.715074in}}%
\pgfpathlineto{\pgfqpoint{4.138444in}{0.719409in}}%
\pgfpathlineto{\pgfqpoint{4.142962in}{0.716374in}}%
\pgfpathlineto{\pgfqpoint{4.147480in}{0.707704in}}%
\pgfpathlineto{\pgfqpoint{4.156517in}{0.715941in}}%
\pgfpathlineto{\pgfqpoint{4.161035in}{0.712473in}}%
\pgfpathlineto{\pgfqpoint{4.165553in}{0.712906in}}%
\pgfpathlineto{\pgfqpoint{4.170071in}{0.705103in}}%
\pgfpathlineto{\pgfqpoint{4.174589in}{0.706404in}}%
\pgfpathlineto{\pgfqpoint{4.179107in}{0.703803in}}%
\pgfpathlineto{\pgfqpoint{4.183625in}{0.704670in}}%
\pgfpathlineto{\pgfqpoint{4.188144in}{0.704236in}}%
\pgfpathlineto{\pgfqpoint{4.192662in}{0.701635in}}%
\pgfpathlineto{\pgfqpoint{4.206216in}{0.701635in}}%
\pgfpathlineto{\pgfqpoint{4.210734in}{0.706837in}}%
\pgfpathlineto{\pgfqpoint{4.215252in}{0.702502in}}%
\pgfpathlineto{\pgfqpoint{4.219770in}{0.714207in}}%
\pgfpathlineto{\pgfqpoint{4.224289in}{0.708571in}}%
\pgfpathlineto{\pgfqpoint{4.228807in}{0.709005in}}%
\pgfpathlineto{\pgfqpoint{4.233325in}{0.708138in}}%
\pgfpathlineto{\pgfqpoint{4.242361in}{0.708571in}}%
\pgfpathlineto{\pgfqpoint{4.246879in}{0.706837in}}%
\pgfpathlineto{\pgfqpoint{4.251397in}{0.711606in}}%
\pgfpathlineto{\pgfqpoint{4.255915in}{0.709438in}}%
\pgfpathlineto{\pgfqpoint{4.260434in}{0.709438in}}%
\pgfpathlineto{\pgfqpoint{4.264952in}{0.712473in}}%
\pgfpathlineto{\pgfqpoint{4.269470in}{0.705970in}}%
\pgfpathlineto{\pgfqpoint{4.273988in}{0.710739in}}%
\pgfpathlineto{\pgfqpoint{4.278506in}{0.709438in}}%
\pgfpathlineto{\pgfqpoint{4.283024in}{0.702936in}}%
\pgfpathlineto{\pgfqpoint{4.287542in}{0.706837in}}%
\pgfpathlineto{\pgfqpoint{4.292060in}{0.702502in}}%
\pgfpathlineto{\pgfqpoint{4.296579in}{0.704236in}}%
\pgfpathlineto{\pgfqpoint{4.301097in}{0.703369in}}%
\pgfpathlineto{\pgfqpoint{4.305615in}{0.705970in}}%
\pgfpathlineto{\pgfqpoint{4.310133in}{0.700335in}}%
\pgfpathlineto{\pgfqpoint{4.314651in}{0.706837in}}%
\pgfpathlineto{\pgfqpoint{4.319169in}{0.701635in}}%
\pgfpathlineto{\pgfqpoint{4.323687in}{0.699468in}}%
\pgfpathlineto{\pgfqpoint{4.328206in}{0.702936in}}%
\pgfpathlineto{\pgfqpoint{4.337242in}{0.697734in}}%
\pgfpathlineto{\pgfqpoint{4.341760in}{0.699901in}}%
\pgfpathlineto{\pgfqpoint{4.346278in}{0.698601in}}%
\pgfpathlineto{\pgfqpoint{4.350796in}{0.701202in}}%
\pgfpathlineto{\pgfqpoint{4.359832in}{0.700335in}}%
\pgfpathlineto{\pgfqpoint{4.364351in}{0.701635in}}%
\pgfpathlineto{\pgfqpoint{4.373387in}{0.698167in}}%
\pgfpathlineto{\pgfqpoint{4.377905in}{0.700768in}}%
\pgfpathlineto{\pgfqpoint{4.382423in}{0.700335in}}%
\pgfpathlineto{\pgfqpoint{4.391459in}{0.705970in}}%
\pgfpathlineto{\pgfqpoint{4.400496in}{0.704670in}}%
\pgfpathlineto{\pgfqpoint{4.414050in}{0.703803in}}%
\pgfpathlineto{\pgfqpoint{4.418568in}{0.699468in}}%
\pgfpathlineto{\pgfqpoint{4.423086in}{0.704670in}}%
\pgfpathlineto{\pgfqpoint{4.427604in}{0.700335in}}%
\pgfpathlineto{\pgfqpoint{4.432122in}{0.707704in}}%
\pgfpathlineto{\pgfqpoint{4.436641in}{0.706404in}}%
\pgfpathlineto{\pgfqpoint{4.445677in}{0.709438in}}%
\pgfpathlineto{\pgfqpoint{4.450195in}{0.715507in}}%
\pgfpathlineto{\pgfqpoint{4.454713in}{0.707271in}}%
\pgfpathlineto{\pgfqpoint{4.463749in}{0.709872in}}%
\pgfpathlineto{\pgfqpoint{4.468267in}{0.706404in}}%
\pgfpathlineto{\pgfqpoint{4.472786in}{0.710305in}}%
\pgfpathlineto{\pgfqpoint{4.477304in}{0.709005in}}%
\pgfpathlineto{\pgfqpoint{4.481822in}{0.704670in}}%
\pgfpathlineto{\pgfqpoint{4.495376in}{0.714640in}}%
\pgfpathlineto{\pgfqpoint{4.499894in}{0.707704in}}%
\pgfpathlineto{\pgfqpoint{4.508931in}{0.718975in}}%
\pgfpathlineto{\pgfqpoint{4.513449in}{0.712906in}}%
\pgfpathlineto{\pgfqpoint{4.517967in}{0.714640in}}%
\pgfpathlineto{\pgfqpoint{4.522485in}{0.715074in}}%
\pgfpathlineto{\pgfqpoint{4.527003in}{0.701635in}}%
\pgfpathlineto{\pgfqpoint{4.531521in}{0.699468in}}%
\pgfpathlineto{\pgfqpoint{4.545076in}{0.704236in}}%
\pgfpathlineto{\pgfqpoint{4.549594in}{0.701635in}}%
\pgfpathlineto{\pgfqpoint{4.558630in}{0.703803in}}%
\pgfpathlineto{\pgfqpoint{4.563148in}{0.698167in}}%
\pgfpathlineto{\pgfqpoint{4.567666in}{0.702936in}}%
\pgfpathlineto{\pgfqpoint{4.581221in}{0.700335in}}%
\pgfpathlineto{\pgfqpoint{4.585739in}{0.702502in}}%
\pgfpathlineto{\pgfqpoint{4.590257in}{0.699901in}}%
\pgfpathlineto{\pgfqpoint{4.594775in}{0.698601in}}%
\pgfpathlineto{\pgfqpoint{4.599293in}{0.703369in}}%
\pgfpathlineto{\pgfqpoint{4.603811in}{0.702069in}}%
\pgfpathlineto{\pgfqpoint{4.608329in}{0.704236in}}%
\pgfpathlineto{\pgfqpoint{4.626402in}{0.702502in}}%
\pgfpathlineto{\pgfqpoint{4.630920in}{0.703803in}}%
\pgfpathlineto{\pgfqpoint{4.635438in}{0.699468in}}%
\pgfpathlineto{\pgfqpoint{4.639956in}{0.722010in}}%
\pgfpathlineto{\pgfqpoint{4.644474in}{0.703803in}}%
\pgfpathlineto{\pgfqpoint{4.648993in}{0.700768in}}%
\pgfpathlineto{\pgfqpoint{4.653511in}{0.700335in}}%
\pgfpathlineto{\pgfqpoint{4.658029in}{0.704236in}}%
\pgfpathlineto{\pgfqpoint{4.662547in}{0.702502in}}%
\pgfpathlineto{\pgfqpoint{4.671583in}{0.705537in}}%
\pgfpathlineto{\pgfqpoint{4.676101in}{0.703803in}}%
\pgfpathlineto{\pgfqpoint{4.680619in}{0.700768in}}%
\pgfpathlineto{\pgfqpoint{4.685138in}{0.705537in}}%
\pgfpathlineto{\pgfqpoint{4.689656in}{0.706837in}}%
\pgfpathlineto{\pgfqpoint{4.694174in}{0.705537in}}%
\pgfpathlineto{\pgfqpoint{4.698692in}{0.711606in}}%
\pgfpathlineto{\pgfqpoint{4.703210in}{0.720276in}}%
\pgfpathlineto{\pgfqpoint{4.707728in}{0.705537in}}%
\pgfpathlineto{\pgfqpoint{4.712246in}{0.712473in}}%
\pgfpathlineto{\pgfqpoint{4.716764in}{0.706404in}}%
\pgfpathlineto{\pgfqpoint{4.721283in}{0.705970in}}%
\pgfpathlineto{\pgfqpoint{4.725801in}{0.710305in}}%
\pgfpathlineto{\pgfqpoint{4.730319in}{0.703369in}}%
\pgfpathlineto{\pgfqpoint{4.734837in}{0.706404in}}%
\pgfpathlineto{\pgfqpoint{4.739355in}{0.712473in}}%
\pgfpathlineto{\pgfqpoint{4.743873in}{0.706404in}}%
\pgfpathlineto{\pgfqpoint{4.748391in}{0.702502in}}%
\pgfpathlineto{\pgfqpoint{4.752909in}{0.704670in}}%
\pgfpathlineto{\pgfqpoint{4.757428in}{0.702069in}}%
\pgfpathlineto{\pgfqpoint{4.766464in}{0.702936in}}%
\pgfpathlineto{\pgfqpoint{4.770982in}{0.701202in}}%
\pgfpathlineto{\pgfqpoint{4.775500in}{0.700768in}}%
\pgfpathlineto{\pgfqpoint{4.780018in}{0.705537in}}%
\pgfpathlineto{\pgfqpoint{4.784536in}{0.704236in}}%
\pgfpathlineto{\pgfqpoint{4.789054in}{0.700768in}}%
\pgfpathlineto{\pgfqpoint{4.793573in}{0.703369in}}%
\pgfpathlineto{\pgfqpoint{4.798091in}{0.703803in}}%
\pgfpathlineto{\pgfqpoint{4.802609in}{0.708138in}}%
\pgfpathlineto{\pgfqpoint{4.807127in}{0.703803in}}%
\pgfpathlineto{\pgfqpoint{4.811645in}{0.708571in}}%
\pgfpathlineto{\pgfqpoint{4.816163in}{0.708571in}}%
\pgfpathlineto{\pgfqpoint{4.820681in}{0.705537in}}%
\pgfpathlineto{\pgfqpoint{4.825199in}{0.699901in}}%
\pgfpathlineto{\pgfqpoint{4.829718in}{0.698601in}}%
\pgfpathlineto{\pgfqpoint{4.834236in}{0.701635in}}%
\pgfpathlineto{\pgfqpoint{4.838754in}{0.700768in}}%
\pgfpathlineto{\pgfqpoint{4.843272in}{0.701202in}}%
\pgfpathlineto{\pgfqpoint{4.847790in}{0.700335in}}%
\pgfpathlineto{\pgfqpoint{4.856826in}{0.702936in}}%
\pgfpathlineto{\pgfqpoint{4.861345in}{0.700335in}}%
\pgfpathlineto{\pgfqpoint{4.865863in}{0.702936in}}%
\pgfpathlineto{\pgfqpoint{4.870381in}{0.702069in}}%
\pgfpathlineto{\pgfqpoint{4.874899in}{0.707271in}}%
\pgfpathlineto{\pgfqpoint{4.883935in}{0.702069in}}%
\pgfpathlineto{\pgfqpoint{4.888453in}{0.705103in}}%
\pgfpathlineto{\pgfqpoint{4.897490in}{0.700768in}}%
\pgfpathlineto{\pgfqpoint{4.902008in}{0.703803in}}%
\pgfpathlineto{\pgfqpoint{4.906526in}{0.699034in}}%
\pgfpathlineto{\pgfqpoint{4.920080in}{0.698167in}}%
\pgfpathlineto{\pgfqpoint{4.929116in}{0.703369in}}%
\pgfpathlineto{\pgfqpoint{4.933635in}{0.700768in}}%
\pgfpathlineto{\pgfqpoint{4.938153in}{0.702502in}}%
\pgfpathlineto{\pgfqpoint{4.942671in}{0.706837in}}%
\pgfpathlineto{\pgfqpoint{4.947189in}{0.703803in}}%
\pgfpathlineto{\pgfqpoint{4.951707in}{0.705970in}}%
\pgfpathlineto{\pgfqpoint{4.956225in}{0.706837in}}%
\pgfpathlineto{\pgfqpoint{4.960743in}{0.702936in}}%
\pgfpathlineto{\pgfqpoint{4.965261in}{0.701635in}}%
\pgfpathlineto{\pgfqpoint{4.974298in}{0.702936in}}%
\pgfpathlineto{\pgfqpoint{4.978816in}{0.701635in}}%
\pgfpathlineto{\pgfqpoint{4.983334in}{0.704236in}}%
\pgfpathlineto{\pgfqpoint{4.992370in}{0.704236in}}%
\pgfpathlineto{\pgfqpoint{4.996888in}{0.702936in}}%
\pgfpathlineto{\pgfqpoint{5.001406in}{0.700335in}}%
\pgfpathlineto{\pgfqpoint{5.005925in}{0.706404in}}%
\pgfpathlineto{\pgfqpoint{5.010443in}{0.698601in}}%
\pgfpathlineto{\pgfqpoint{5.028515in}{0.703369in}}%
\pgfpathlineto{\pgfqpoint{5.033033in}{0.702936in}}%
\pgfpathlineto{\pgfqpoint{5.042070in}{0.705970in}}%
\pgfpathlineto{\pgfqpoint{5.046588in}{0.700335in}}%
\pgfpathlineto{\pgfqpoint{5.051106in}{0.698167in}}%
\pgfpathlineto{\pgfqpoint{5.055624in}{0.699901in}}%
\pgfpathlineto{\pgfqpoint{5.060142in}{0.697300in}}%
\pgfpathlineto{\pgfqpoint{5.064660in}{0.699034in}}%
\pgfpathlineto{\pgfqpoint{5.069178in}{0.699468in}}%
\pgfpathlineto{\pgfqpoint{5.073696in}{0.697734in}}%
\pgfpathlineto{\pgfqpoint{5.078215in}{0.699034in}}%
\pgfpathlineto{\pgfqpoint{5.082733in}{0.697734in}}%
\pgfpathlineto{\pgfqpoint{5.087251in}{0.697734in}}%
\pgfpathlineto{\pgfqpoint{5.091769in}{0.699034in}}%
\pgfpathlineto{\pgfqpoint{5.096287in}{0.706837in}}%
\pgfpathlineto{\pgfqpoint{5.100805in}{0.712039in}}%
\pgfpathlineto{\pgfqpoint{5.109842in}{0.714640in}}%
\pgfpathlineto{\pgfqpoint{5.114360in}{0.709872in}}%
\pgfpathlineto{\pgfqpoint{5.118878in}{0.703369in}}%
\pgfpathlineto{\pgfqpoint{5.123396in}{0.706404in}}%
\pgfpathlineto{\pgfqpoint{5.127914in}{0.718542in}}%
\pgfpathlineto{\pgfqpoint{5.132432in}{0.717675in}}%
\pgfpathlineto{\pgfqpoint{5.136950in}{0.702069in}}%
\pgfpathlineto{\pgfqpoint{5.145987in}{0.702069in}}%
\pgfpathlineto{\pgfqpoint{5.155023in}{0.697734in}}%
\pgfpathlineto{\pgfqpoint{5.159541in}{0.699034in}}%
\pgfpathlineto{\pgfqpoint{5.164059in}{0.701635in}}%
\pgfpathlineto{\pgfqpoint{5.168577in}{0.701202in}}%
\pgfpathlineto{\pgfqpoint{5.173095in}{0.699468in}}%
\pgfpathlineto{\pgfqpoint{5.177613in}{0.699901in}}%
\pgfpathlineto{\pgfqpoint{5.182132in}{0.706404in}}%
\pgfpathlineto{\pgfqpoint{5.191168in}{0.699468in}}%
\pgfpathlineto{\pgfqpoint{5.195686in}{0.699034in}}%
\pgfpathlineto{\pgfqpoint{5.200204in}{0.700335in}}%
\pgfpathlineto{\pgfqpoint{5.204722in}{0.697734in}}%
\pgfpathlineto{\pgfqpoint{5.209240in}{0.699034in}}%
\pgfpathlineto{\pgfqpoint{5.213758in}{0.697300in}}%
\pgfpathlineto{\pgfqpoint{5.218277in}{0.698601in}}%
\pgfpathlineto{\pgfqpoint{5.227313in}{0.698167in}}%
\pgfpathlineto{\pgfqpoint{5.231831in}{0.699901in}}%
\pgfpathlineto{\pgfqpoint{5.240867in}{0.696867in}}%
\pgfpathlineto{\pgfqpoint{5.245385in}{0.698167in}}%
\pgfpathlineto{\pgfqpoint{5.249903in}{0.697734in}}%
\pgfpathlineto{\pgfqpoint{5.263458in}{0.701635in}}%
\pgfpathlineto{\pgfqpoint{5.277012in}{0.699468in}}%
\pgfpathlineto{\pgfqpoint{5.281530in}{0.702502in}}%
\pgfpathlineto{\pgfqpoint{5.286048in}{0.699468in}}%
\pgfpathlineto{\pgfqpoint{5.299603in}{0.696433in}}%
\pgfpathlineto{\pgfqpoint{5.304121in}{0.697300in}}%
\pgfpathlineto{\pgfqpoint{5.308639in}{0.696867in}}%
\pgfpathlineto{\pgfqpoint{5.313157in}{0.699901in}}%
\pgfpathlineto{\pgfqpoint{5.317675in}{0.697300in}}%
\pgfpathlineto{\pgfqpoint{5.322193in}{0.697300in}}%
\pgfpathlineto{\pgfqpoint{5.326712in}{0.699468in}}%
\pgfpathlineto{\pgfqpoint{5.331230in}{0.696000in}}%
\pgfpathlineto{\pgfqpoint{5.335748in}{0.698601in}}%
\pgfpathlineto{\pgfqpoint{5.353820in}{0.699034in}}%
\pgfpathlineto{\pgfqpoint{5.358338in}{0.702502in}}%
\pgfpathlineto{\pgfqpoint{5.362857in}{0.697734in}}%
\pgfpathlineto{\pgfqpoint{5.367375in}{0.703369in}}%
\pgfpathlineto{\pgfqpoint{5.371893in}{0.702502in}}%
\pgfpathlineto{\pgfqpoint{5.380929in}{0.699034in}}%
\pgfpathlineto{\pgfqpoint{5.389965in}{0.698167in}}%
\pgfpathlineto{\pgfqpoint{5.394484in}{0.699468in}}%
\pgfpathlineto{\pgfqpoint{5.399002in}{0.698167in}}%
\pgfpathlineto{\pgfqpoint{5.403520in}{0.699034in}}%
\pgfpathlineto{\pgfqpoint{5.417074in}{0.696000in}}%
\pgfpathlineto{\pgfqpoint{5.421592in}{0.698601in}}%
\pgfpathlineto{\pgfqpoint{5.426110in}{0.697300in}}%
\pgfpathlineto{\pgfqpoint{5.435147in}{0.698601in}}%
\pgfpathlineto{\pgfqpoint{5.448701in}{0.696867in}}%
\pgfpathlineto{\pgfqpoint{5.453219in}{0.699034in}}%
\pgfpathlineto{\pgfqpoint{5.457737in}{0.699034in}}%
\pgfpathlineto{\pgfqpoint{5.462255in}{0.697734in}}%
\pgfpathlineto{\pgfqpoint{5.466774in}{0.699034in}}%
\pgfpathlineto{\pgfqpoint{5.471292in}{0.696000in}}%
\pgfpathlineto{\pgfqpoint{5.480328in}{0.698601in}}%
\pgfpathlineto{\pgfqpoint{5.484846in}{0.697734in}}%
\pgfpathlineto{\pgfqpoint{5.489364in}{0.698601in}}%
\pgfpathlineto{\pgfqpoint{5.493882in}{0.698167in}}%
\pgfpathlineto{\pgfqpoint{5.498400in}{0.700335in}}%
\pgfpathlineto{\pgfqpoint{5.502919in}{0.697734in}}%
\pgfpathlineto{\pgfqpoint{5.507437in}{0.699901in}}%
\pgfpathlineto{\pgfqpoint{5.511955in}{0.699901in}}%
\pgfpathlineto{\pgfqpoint{5.516473in}{0.696433in}}%
\pgfpathlineto{\pgfqpoint{5.520991in}{0.697300in}}%
\pgfpathlineto{\pgfqpoint{5.525509in}{0.696867in}}%
\pgfpathlineto{\pgfqpoint{5.534545in}{0.697734in}}%
\pgfpathlineto{\pgfqpoint{5.534545in}{0.697734in}}%
\pgfusepath{stroke}%
\end{pgfscope}%
\begin{pgfscope}%
\pgfsetrectcap%
\pgfsetmiterjoin%
\pgfsetlinewidth{0.803000pt}%
\definecolor{currentstroke}{rgb}{0.000000,0.000000,0.000000}%
\pgfsetstrokecolor{currentstroke}%
\pgfsetdash{}{0pt}%
\pgfpathmoveto{\pgfqpoint{0.800000in}{0.528000in}}%
\pgfpathlineto{\pgfqpoint{0.800000in}{4.224000in}}%
\pgfusepath{stroke}%
\end{pgfscope}%
\begin{pgfscope}%
\pgfsetrectcap%
\pgfsetmiterjoin%
\pgfsetlinewidth{0.803000pt}%
\definecolor{currentstroke}{rgb}{0.000000,0.000000,0.000000}%
\pgfsetstrokecolor{currentstroke}%
\pgfsetdash{}{0pt}%
\pgfpathmoveto{\pgfqpoint{5.760000in}{0.528000in}}%
\pgfpathlineto{\pgfqpoint{5.760000in}{4.224000in}}%
\pgfusepath{stroke}%
\end{pgfscope}%
\begin{pgfscope}%
\pgfsetrectcap%
\pgfsetmiterjoin%
\pgfsetlinewidth{0.803000pt}%
\definecolor{currentstroke}{rgb}{0.000000,0.000000,0.000000}%
\pgfsetstrokecolor{currentstroke}%
\pgfsetdash{}{0pt}%
\pgfpathmoveto{\pgfqpoint{0.800000in}{0.528000in}}%
\pgfpathlineto{\pgfqpoint{5.760000in}{0.528000in}}%
\pgfusepath{stroke}%
\end{pgfscope}%
\begin{pgfscope}%
\pgfsetrectcap%
\pgfsetmiterjoin%
\pgfsetlinewidth{0.803000pt}%
\definecolor{currentstroke}{rgb}{0.000000,0.000000,0.000000}%
\pgfsetstrokecolor{currentstroke}%
\pgfsetdash{}{0pt}%
\pgfpathmoveto{\pgfqpoint{0.800000in}{4.224000in}}%
\pgfpathlineto{\pgfqpoint{5.760000in}{4.224000in}}%
\pgfusepath{stroke}%
\end{pgfscope}%
\begin{pgfscope}%
\pgfsetbuttcap%
\pgfsetmiterjoin%
\definecolor{currentfill}{rgb}{1.000000,1.000000,1.000000}%
\pgfsetfillcolor{currentfill}%
\pgfsetfillopacity{0.800000}%
\pgfsetlinewidth{1.003750pt}%
\definecolor{currentstroke}{rgb}{0.800000,0.800000,0.800000}%
\pgfsetstrokecolor{currentstroke}%
\pgfsetstrokeopacity{0.800000}%
\pgfsetdash{}{0pt}%
\pgfpathmoveto{\pgfqpoint{3.591379in}{3.675698in}}%
\pgfpathlineto{\pgfqpoint{5.653056in}{3.675698in}}%
\pgfpathquadraticcurveto{\pgfqpoint{5.683611in}{3.675698in}}{\pgfqpoint{5.683611in}{3.706254in}}%
\pgfpathlineto{\pgfqpoint{5.683611in}{4.117056in}}%
\pgfpathquadraticcurveto{\pgfqpoint{5.683611in}{4.147611in}}{\pgfqpoint{5.653056in}{4.147611in}}%
\pgfpathlineto{\pgfqpoint{3.591379in}{4.147611in}}%
\pgfpathquadraticcurveto{\pgfqpoint{3.560823in}{4.147611in}}{\pgfqpoint{3.560823in}{4.117056in}}%
\pgfpathlineto{\pgfqpoint{3.560823in}{3.706254in}}%
\pgfpathquadraticcurveto{\pgfqpoint{3.560823in}{3.675698in}}{\pgfqpoint{3.591379in}{3.675698in}}%
\pgfpathclose%
\pgfusepath{stroke,fill}%
\end{pgfscope}%
\begin{pgfscope}%
\pgfsetrectcap%
\pgfsetroundjoin%
\pgfsetlinewidth{1.505625pt}%
\definecolor{currentstroke}{rgb}{0.121569,0.466667,0.705882}%
\pgfsetstrokecolor{currentstroke}%
\pgfsetdash{}{0pt}%
\pgfpathmoveto{\pgfqpoint{3.621934in}{4.033028in}}%
\pgfpathlineto{\pgfqpoint{3.927490in}{4.033028in}}%
\pgfusepath{stroke}%
\end{pgfscope}%
\begin{pgfscope}%
\pgftext[x=4.049712in,y=3.979556in,left,base]{\fontsize{11.000000}{13.200000}\selectfont Primary write concern}%
\end{pgfscope}%
\begin{pgfscope}%
\pgfsetrectcap%
\pgfsetroundjoin%
\pgfsetlinewidth{1.505625pt}%
\definecolor{currentstroke}{rgb}{1.000000,0.498039,0.054902}%
\pgfsetstrokecolor{currentstroke}%
\pgfsetdash{}{0pt}%
\pgfpathmoveto{\pgfqpoint{3.621934in}{3.819988in}}%
\pgfpathlineto{\pgfqpoint{3.927490in}{3.819988in}}%
\pgfusepath{stroke}%
\end{pgfscope}%
\begin{pgfscope}%
\pgftext[x=4.049712in,y=3.766516in,left,base]{\fontsize{11.000000}{13.200000}\selectfont Journaled write concern}%
\end{pgfscope}%
\end{pgfpicture}%
\makeatother%
\endgroup%

    \label{fig:latencies}
    \caption{A frequency graph of the number of write operations acknowledged at latencies of 1-1000ms.}
\end{figure}

\begin{figure}
    \centering        
    %% Creator: Matplotlib, PGF backend
%%
%% To include the figure in your LaTeX document, write
%%   \input{<filename>.pgf}
%%
%% Make sure the required packages are loaded in your preamble
%%   \usepackage{pgf}
%%
%% Figures using additional raster images can only be included by \input if
%% they are in the same directory as the main LaTeX file. For loading figures
%% from other directories you can use the `import` package
%%   \usepackage{import}
%% and then include the figures with
%%   \import{<path to file>}{<filename>.pgf}
%%
%% Matplotlib used the following preamble
%%   \usepackage[utf8x]{inputenc}
%%   \usepackage[T1]{fontenc}
%%   \usepackage{lmodern}
%%
\begingroup%
\makeatletter%
\begin{pgfpicture}%
\pgfpathrectangle{\pgfpointorigin}{\pgfqpoint{6.400000in}{4.800000in}}%
\pgfusepath{use as bounding box, clip}%
\begin{pgfscope}%
\pgfsetbuttcap%
\pgfsetmiterjoin%
\definecolor{currentfill}{rgb}{1.000000,1.000000,1.000000}%
\pgfsetfillcolor{currentfill}%
\pgfsetlinewidth{0.000000pt}%
\definecolor{currentstroke}{rgb}{1.000000,1.000000,1.000000}%
\pgfsetstrokecolor{currentstroke}%
\pgfsetdash{}{0pt}%
\pgfpathmoveto{\pgfqpoint{0.000000in}{0.000000in}}%
\pgfpathlineto{\pgfqpoint{6.400000in}{0.000000in}}%
\pgfpathlineto{\pgfqpoint{6.400000in}{4.800000in}}%
\pgfpathlineto{\pgfqpoint{0.000000in}{4.800000in}}%
\pgfpathclose%
\pgfusepath{fill}%
\end{pgfscope}%
\begin{pgfscope}%
\pgfsetbuttcap%
\pgfsetmiterjoin%
\definecolor{currentfill}{rgb}{1.000000,1.000000,1.000000}%
\pgfsetfillcolor{currentfill}%
\pgfsetlinewidth{0.000000pt}%
\definecolor{currentstroke}{rgb}{0.000000,0.000000,0.000000}%
\pgfsetstrokecolor{currentstroke}%
\pgfsetstrokeopacity{0.000000}%
\pgfsetdash{}{0pt}%
\pgfpathmoveto{\pgfqpoint{0.800000in}{0.528000in}}%
\pgfpathlineto{\pgfqpoint{5.760000in}{0.528000in}}%
\pgfpathlineto{\pgfqpoint{5.760000in}{4.224000in}}%
\pgfpathlineto{\pgfqpoint{0.800000in}{4.224000in}}%
\pgfpathclose%
\pgfusepath{fill}%
\end{pgfscope}%
\begin{pgfscope}%
\pgfsetbuttcap%
\pgfsetroundjoin%
\definecolor{currentfill}{rgb}{0.000000,0.000000,0.000000}%
\pgfsetfillcolor{currentfill}%
\pgfsetlinewidth{0.803000pt}%
\definecolor{currentstroke}{rgb}{0.000000,0.000000,0.000000}%
\pgfsetstrokecolor{currentstroke}%
\pgfsetdash{}{0pt}%
\pgfsys@defobject{currentmarker}{\pgfqpoint{0.000000in}{-0.048611in}}{\pgfqpoint{0.000000in}{0.000000in}}{%
\pgfpathmoveto{\pgfqpoint{0.000000in}{0.000000in}}%
\pgfpathlineto{\pgfqpoint{0.000000in}{-0.048611in}}%
\pgfusepath{stroke,fill}%
}%
\begin{pgfscope}%
\pgfsys@transformshift{1.025455in}{0.528000in}%
\pgfsys@useobject{currentmarker}{}%
\end{pgfscope}%
\end{pgfscope}%
\begin{pgfscope}%
\pgftext[x=1.025455in,y=0.430778in,,top]{\fontsize{11.000000}{13.200000}\selectfont \(\displaystyle 0\)}%
\end{pgfscope}%
\begin{pgfscope}%
\pgfsetbuttcap%
\pgfsetroundjoin%
\definecolor{currentfill}{rgb}{0.000000,0.000000,0.000000}%
\pgfsetfillcolor{currentfill}%
\pgfsetlinewidth{0.803000pt}%
\definecolor{currentstroke}{rgb}{0.000000,0.000000,0.000000}%
\pgfsetstrokecolor{currentstroke}%
\pgfsetdash{}{0pt}%
\pgfsys@defobject{currentmarker}{\pgfqpoint{0.000000in}{-0.048611in}}{\pgfqpoint{0.000000in}{0.000000in}}{%
\pgfpathmoveto{\pgfqpoint{0.000000in}{0.000000in}}%
\pgfpathlineto{\pgfqpoint{0.000000in}{-0.048611in}}%
\pgfusepath{stroke,fill}%
}%
\begin{pgfscope}%
\pgfsys@transformshift{1.928175in}{0.528000in}%
\pgfsys@useobject{currentmarker}{}%
\end{pgfscope}%
\end{pgfscope}%
\begin{pgfscope}%
\pgftext[x=1.928175in,y=0.430778in,,top]{\fontsize{11.000000}{13.200000}\selectfont \(\displaystyle 200\)}%
\end{pgfscope}%
\begin{pgfscope}%
\pgfsetbuttcap%
\pgfsetroundjoin%
\definecolor{currentfill}{rgb}{0.000000,0.000000,0.000000}%
\pgfsetfillcolor{currentfill}%
\pgfsetlinewidth{0.803000pt}%
\definecolor{currentstroke}{rgb}{0.000000,0.000000,0.000000}%
\pgfsetstrokecolor{currentstroke}%
\pgfsetdash{}{0pt}%
\pgfsys@defobject{currentmarker}{\pgfqpoint{0.000000in}{-0.048611in}}{\pgfqpoint{0.000000in}{0.000000in}}{%
\pgfpathmoveto{\pgfqpoint{0.000000in}{0.000000in}}%
\pgfpathlineto{\pgfqpoint{0.000000in}{-0.048611in}}%
\pgfusepath{stroke,fill}%
}%
\begin{pgfscope}%
\pgfsys@transformshift{2.830896in}{0.528000in}%
\pgfsys@useobject{currentmarker}{}%
\end{pgfscope}%
\end{pgfscope}%
\begin{pgfscope}%
\pgftext[x=2.830896in,y=0.430778in,,top]{\fontsize{11.000000}{13.200000}\selectfont \(\displaystyle 400\)}%
\end{pgfscope}%
\begin{pgfscope}%
\pgfsetbuttcap%
\pgfsetroundjoin%
\definecolor{currentfill}{rgb}{0.000000,0.000000,0.000000}%
\pgfsetfillcolor{currentfill}%
\pgfsetlinewidth{0.803000pt}%
\definecolor{currentstroke}{rgb}{0.000000,0.000000,0.000000}%
\pgfsetstrokecolor{currentstroke}%
\pgfsetdash{}{0pt}%
\pgfsys@defobject{currentmarker}{\pgfqpoint{0.000000in}{-0.048611in}}{\pgfqpoint{0.000000in}{0.000000in}}{%
\pgfpathmoveto{\pgfqpoint{0.000000in}{0.000000in}}%
\pgfpathlineto{\pgfqpoint{0.000000in}{-0.048611in}}%
\pgfusepath{stroke,fill}%
}%
\begin{pgfscope}%
\pgfsys@transformshift{3.733617in}{0.528000in}%
\pgfsys@useobject{currentmarker}{}%
\end{pgfscope}%
\end{pgfscope}%
\begin{pgfscope}%
\pgftext[x=3.733617in,y=0.430778in,,top]{\fontsize{11.000000}{13.200000}\selectfont \(\displaystyle 600\)}%
\end{pgfscope}%
\begin{pgfscope}%
\pgfsetbuttcap%
\pgfsetroundjoin%
\definecolor{currentfill}{rgb}{0.000000,0.000000,0.000000}%
\pgfsetfillcolor{currentfill}%
\pgfsetlinewidth{0.803000pt}%
\definecolor{currentstroke}{rgb}{0.000000,0.000000,0.000000}%
\pgfsetstrokecolor{currentstroke}%
\pgfsetdash{}{0pt}%
\pgfsys@defobject{currentmarker}{\pgfqpoint{0.000000in}{-0.048611in}}{\pgfqpoint{0.000000in}{0.000000in}}{%
\pgfpathmoveto{\pgfqpoint{0.000000in}{0.000000in}}%
\pgfpathlineto{\pgfqpoint{0.000000in}{-0.048611in}}%
\pgfusepath{stroke,fill}%
}%
\begin{pgfscope}%
\pgfsys@transformshift{4.636338in}{0.528000in}%
\pgfsys@useobject{currentmarker}{}%
\end{pgfscope}%
\end{pgfscope}%
\begin{pgfscope}%
\pgftext[x=4.636338in,y=0.430778in,,top]{\fontsize{11.000000}{13.200000}\selectfont \(\displaystyle 800\)}%
\end{pgfscope}%
\begin{pgfscope}%
\pgfsetbuttcap%
\pgfsetroundjoin%
\definecolor{currentfill}{rgb}{0.000000,0.000000,0.000000}%
\pgfsetfillcolor{currentfill}%
\pgfsetlinewidth{0.803000pt}%
\definecolor{currentstroke}{rgb}{0.000000,0.000000,0.000000}%
\pgfsetstrokecolor{currentstroke}%
\pgfsetdash{}{0pt}%
\pgfsys@defobject{currentmarker}{\pgfqpoint{0.000000in}{-0.048611in}}{\pgfqpoint{0.000000in}{0.000000in}}{%
\pgfpathmoveto{\pgfqpoint{0.000000in}{0.000000in}}%
\pgfpathlineto{\pgfqpoint{0.000000in}{-0.048611in}}%
\pgfusepath{stroke,fill}%
}%
\begin{pgfscope}%
\pgfsys@transformshift{5.539059in}{0.528000in}%
\pgfsys@useobject{currentmarker}{}%
\end{pgfscope}%
\end{pgfscope}%
\begin{pgfscope}%
\pgftext[x=5.539059in,y=0.430778in,,top]{\fontsize{11.000000}{13.200000}\selectfont \(\displaystyle 1000\)}%
\end{pgfscope}%
\begin{pgfscope}%
\pgftext[x=3.280000in,y=0.240271in,,top]{\fontsize{11.000000}{13.200000}\selectfont Latency (in milliseconds)}%
\end{pgfscope}%
\begin{pgfscope}%
\pgfsetbuttcap%
\pgfsetroundjoin%
\definecolor{currentfill}{rgb}{0.000000,0.000000,0.000000}%
\pgfsetfillcolor{currentfill}%
\pgfsetlinewidth{0.803000pt}%
\definecolor{currentstroke}{rgb}{0.000000,0.000000,0.000000}%
\pgfsetstrokecolor{currentstroke}%
\pgfsetdash{}{0pt}%
\pgfsys@defobject{currentmarker}{\pgfqpoint{-0.048611in}{0.000000in}}{\pgfqpoint{0.000000in}{0.000000in}}{%
\pgfpathmoveto{\pgfqpoint{0.000000in}{0.000000in}}%
\pgfpathlineto{\pgfqpoint{-0.048611in}{0.000000in}}%
\pgfusepath{stroke,fill}%
}%
\begin{pgfscope}%
\pgfsys@transformshift{0.800000in}{0.695957in}%
\pgfsys@useobject{currentmarker}{}%
\end{pgfscope}%
\end{pgfscope}%
\begin{pgfscope}%
\pgftext[x=0.511629in,y=0.643335in,left,base]{\fontsize{11.000000}{13.200000}\selectfont \(\displaystyle 0.0\)}%
\end{pgfscope}%
\begin{pgfscope}%
\pgfsetbuttcap%
\pgfsetroundjoin%
\definecolor{currentfill}{rgb}{0.000000,0.000000,0.000000}%
\pgfsetfillcolor{currentfill}%
\pgfsetlinewidth{0.803000pt}%
\definecolor{currentstroke}{rgb}{0.000000,0.000000,0.000000}%
\pgfsetstrokecolor{currentstroke}%
\pgfsetdash{}{0pt}%
\pgfsys@defobject{currentmarker}{\pgfqpoint{-0.048611in}{0.000000in}}{\pgfqpoint{0.000000in}{0.000000in}}{%
\pgfpathmoveto{\pgfqpoint{0.000000in}{0.000000in}}%
\pgfpathlineto{\pgfqpoint{-0.048611in}{0.000000in}}%
\pgfusepath{stroke,fill}%
}%
\begin{pgfscope}%
\pgfsys@transformshift{0.800000in}{1.367966in}%
\pgfsys@useobject{currentmarker}{}%
\end{pgfscope}%
\end{pgfscope}%
\begin{pgfscope}%
\pgftext[x=0.511629in,y=1.315343in,left,base]{\fontsize{11.000000}{13.200000}\selectfont \(\displaystyle 0.2\)}%
\end{pgfscope}%
\begin{pgfscope}%
\pgfsetbuttcap%
\pgfsetroundjoin%
\definecolor{currentfill}{rgb}{0.000000,0.000000,0.000000}%
\pgfsetfillcolor{currentfill}%
\pgfsetlinewidth{0.803000pt}%
\definecolor{currentstroke}{rgb}{0.000000,0.000000,0.000000}%
\pgfsetstrokecolor{currentstroke}%
\pgfsetdash{}{0pt}%
\pgfsys@defobject{currentmarker}{\pgfqpoint{-0.048611in}{0.000000in}}{\pgfqpoint{0.000000in}{0.000000in}}{%
\pgfpathmoveto{\pgfqpoint{0.000000in}{0.000000in}}%
\pgfpathlineto{\pgfqpoint{-0.048611in}{0.000000in}}%
\pgfusepath{stroke,fill}%
}%
\begin{pgfscope}%
\pgfsys@transformshift{0.800000in}{2.039974in}%
\pgfsys@useobject{currentmarker}{}%
\end{pgfscope}%
\end{pgfscope}%
\begin{pgfscope}%
\pgftext[x=0.511629in,y=1.987352in,left,base]{\fontsize{11.000000}{13.200000}\selectfont \(\displaystyle 0.4\)}%
\end{pgfscope}%
\begin{pgfscope}%
\pgfsetbuttcap%
\pgfsetroundjoin%
\definecolor{currentfill}{rgb}{0.000000,0.000000,0.000000}%
\pgfsetfillcolor{currentfill}%
\pgfsetlinewidth{0.803000pt}%
\definecolor{currentstroke}{rgb}{0.000000,0.000000,0.000000}%
\pgfsetstrokecolor{currentstroke}%
\pgfsetdash{}{0pt}%
\pgfsys@defobject{currentmarker}{\pgfqpoint{-0.048611in}{0.000000in}}{\pgfqpoint{0.000000in}{0.000000in}}{%
\pgfpathmoveto{\pgfqpoint{0.000000in}{0.000000in}}%
\pgfpathlineto{\pgfqpoint{-0.048611in}{0.000000in}}%
\pgfusepath{stroke,fill}%
}%
\begin{pgfscope}%
\pgfsys@transformshift{0.800000in}{2.711983in}%
\pgfsys@useobject{currentmarker}{}%
\end{pgfscope}%
\end{pgfscope}%
\begin{pgfscope}%
\pgftext[x=0.511629in,y=2.659361in,left,base]{\fontsize{11.000000}{13.200000}\selectfont \(\displaystyle 0.6\)}%
\end{pgfscope}%
\begin{pgfscope}%
\pgfsetbuttcap%
\pgfsetroundjoin%
\definecolor{currentfill}{rgb}{0.000000,0.000000,0.000000}%
\pgfsetfillcolor{currentfill}%
\pgfsetlinewidth{0.803000pt}%
\definecolor{currentstroke}{rgb}{0.000000,0.000000,0.000000}%
\pgfsetstrokecolor{currentstroke}%
\pgfsetdash{}{0pt}%
\pgfsys@defobject{currentmarker}{\pgfqpoint{-0.048611in}{0.000000in}}{\pgfqpoint{0.000000in}{0.000000in}}{%
\pgfpathmoveto{\pgfqpoint{0.000000in}{0.000000in}}%
\pgfpathlineto{\pgfqpoint{-0.048611in}{0.000000in}}%
\pgfusepath{stroke,fill}%
}%
\begin{pgfscope}%
\pgfsys@transformshift{0.800000in}{3.383991in}%
\pgfsys@useobject{currentmarker}{}%
\end{pgfscope}%
\end{pgfscope}%
\begin{pgfscope}%
\pgftext[x=0.511629in,y=3.331369in,left,base]{\fontsize{11.000000}{13.200000}\selectfont \(\displaystyle 0.8\)}%
\end{pgfscope}%
\begin{pgfscope}%
\pgfsetbuttcap%
\pgfsetroundjoin%
\definecolor{currentfill}{rgb}{0.000000,0.000000,0.000000}%
\pgfsetfillcolor{currentfill}%
\pgfsetlinewidth{0.803000pt}%
\definecolor{currentstroke}{rgb}{0.000000,0.000000,0.000000}%
\pgfsetstrokecolor{currentstroke}%
\pgfsetdash{}{0pt}%
\pgfsys@defobject{currentmarker}{\pgfqpoint{-0.048611in}{0.000000in}}{\pgfqpoint{0.000000in}{0.000000in}}{%
\pgfpathmoveto{\pgfqpoint{0.000000in}{0.000000in}}%
\pgfpathlineto{\pgfqpoint{-0.048611in}{0.000000in}}%
\pgfusepath{stroke,fill}%
}%
\begin{pgfscope}%
\pgfsys@transformshift{0.800000in}{4.056000in}%
\pgfsys@useobject{currentmarker}{}%
\end{pgfscope}%
\end{pgfscope}%
\begin{pgfscope}%
\pgftext[x=0.511629in,y=4.003378in,left,base]{\fontsize{11.000000}{13.200000}\selectfont \(\displaystyle 1.0\)}%
\end{pgfscope}%
\begin{pgfscope}%
\pgftext[x=0.456074in,y=2.376000in,,bottom,rotate=90.000000]{\fontsize{11.000000}{13.200000}\selectfont Fraction of writes}%
\end{pgfscope}%
\begin{pgfscope}%
\pgfpathrectangle{\pgfqpoint{0.800000in}{0.528000in}}{\pgfqpoint{4.960000in}{3.696000in}}%
\pgfusepath{clip}%
\pgfsetrectcap%
\pgfsetroundjoin%
\pgfsetlinewidth{1.505625pt}%
\definecolor{currentstroke}{rgb}{0.121569,0.466667,0.705882}%
\pgfsetstrokecolor{currentstroke}%
\pgfsetdash{}{0pt}%
\pgfpathmoveto{\pgfqpoint{1.025455in}{0.704476in}}%
\pgfpathlineto{\pgfqpoint{1.038995in}{0.737974in}}%
\pgfpathlineto{\pgfqpoint{1.066077in}{0.798427in}}%
\pgfpathlineto{\pgfqpoint{1.084131in}{0.830847in}}%
\pgfpathlineto{\pgfqpoint{1.102186in}{0.857783in}}%
\pgfpathlineto{\pgfqpoint{1.115727in}{0.873595in}}%
\pgfpathlineto{\pgfqpoint{1.138295in}{0.894058in}}%
\pgfpathlineto{\pgfqpoint{1.151835in}{0.903545in}}%
\pgfpathlineto{\pgfqpoint{1.174403in}{0.915498in}}%
\pgfpathlineto{\pgfqpoint{1.228567in}{0.936460in}}%
\pgfpathlineto{\pgfqpoint{1.237594in}{0.942750in}}%
\pgfpathlineto{\pgfqpoint{1.246621in}{0.951842in}}%
\pgfpathlineto{\pgfqpoint{1.260162in}{0.974761in}}%
\pgfpathlineto{\pgfqpoint{1.264676in}{0.983887in}}%
\pgfpathlineto{\pgfqpoint{1.269189in}{0.995821in}}%
\pgfpathlineto{\pgfqpoint{1.278216in}{1.030500in}}%
\pgfpathlineto{\pgfqpoint{1.291757in}{1.105663in}}%
\pgfpathlineto{\pgfqpoint{1.300784in}{1.167154in}}%
\pgfpathlineto{\pgfqpoint{1.341407in}{1.492259in}}%
\pgfpathlineto{\pgfqpoint{1.359461in}{1.636246in}}%
\pgfpathlineto{\pgfqpoint{1.373002in}{1.719103in}}%
\pgfpathlineto{\pgfqpoint{1.386543in}{1.783070in}}%
\pgfpathlineto{\pgfqpoint{1.400084in}{1.833087in}}%
\pgfpathlineto{\pgfqpoint{1.409111in}{1.860922in}}%
\pgfpathlineto{\pgfqpoint{1.422652in}{1.894815in}}%
\pgfpathlineto{\pgfqpoint{1.431679in}{1.913503in}}%
\pgfpathlineto{\pgfqpoint{1.440706in}{1.928975in}}%
\pgfpathlineto{\pgfqpoint{1.467788in}{1.966643in}}%
\pgfpathlineto{\pgfqpoint{1.490356in}{1.988563in}}%
\pgfpathlineto{\pgfqpoint{1.521951in}{2.012044in}}%
\pgfpathlineto{\pgfqpoint{1.530978in}{2.020163in}}%
\pgfpathlineto{\pgfqpoint{1.544519in}{2.035561in}}%
\pgfpathlineto{\pgfqpoint{1.558060in}{2.055262in}}%
\pgfpathlineto{\pgfqpoint{1.567087in}{2.071885in}}%
\pgfpathlineto{\pgfqpoint{1.585142in}{2.114573in}}%
\pgfpathlineto{\pgfqpoint{1.603196in}{2.166330in}}%
\pgfpathlineto{\pgfqpoint{1.616737in}{2.213406in}}%
\pgfpathlineto{\pgfqpoint{1.625764in}{2.250907in}}%
\pgfpathlineto{\pgfqpoint{1.639305in}{2.316524in}}%
\pgfpathlineto{\pgfqpoint{1.657359in}{2.419660in}}%
\pgfpathlineto{\pgfqpoint{1.675414in}{2.521429in}}%
\pgfpathlineto{\pgfqpoint{1.693468in}{2.607152in}}%
\pgfpathlineto{\pgfqpoint{1.711522in}{2.682597in}}%
\pgfpathlineto{\pgfqpoint{1.725063in}{2.729639in}}%
\pgfpathlineto{\pgfqpoint{1.738604in}{2.766448in}}%
\pgfpathlineto{\pgfqpoint{1.752145in}{2.797020in}}%
\pgfpathlineto{\pgfqpoint{1.765686in}{2.822943in}}%
\pgfpathlineto{\pgfqpoint{1.783740in}{2.852646in}}%
\pgfpathlineto{\pgfqpoint{1.810822in}{2.889355in}}%
\pgfpathlineto{\pgfqpoint{1.828876in}{2.911938in}}%
\pgfpathlineto{\pgfqpoint{1.878526in}{2.979966in}}%
\pgfpathlineto{\pgfqpoint{1.901094in}{3.020396in}}%
\pgfpathlineto{\pgfqpoint{1.932689in}{3.086893in}}%
\pgfpathlineto{\pgfqpoint{1.955257in}{3.139454in}}%
\pgfpathlineto{\pgfqpoint{1.991366in}{3.229719in}}%
\pgfpathlineto{\pgfqpoint{2.013934in}{3.278738in}}%
\pgfpathlineto{\pgfqpoint{2.031988in}{3.312848in}}%
\pgfpathlineto{\pgfqpoint{2.054556in}{3.346474in}}%
\pgfpathlineto{\pgfqpoint{2.086152in}{3.384622in}}%
\pgfpathlineto{\pgfqpoint{2.104206in}{3.402589in}}%
\pgfpathlineto{\pgfqpoint{2.131288in}{3.427657in}}%
\pgfpathlineto{\pgfqpoint{2.185451in}{3.478973in}}%
\pgfpathlineto{\pgfqpoint{2.217046in}{3.505523in}}%
\pgfpathlineto{\pgfqpoint{2.230587in}{3.516483in}}%
\pgfpathlineto{\pgfqpoint{2.262182in}{3.546581in}}%
\pgfpathlineto{\pgfqpoint{2.284750in}{3.569218in}}%
\pgfpathlineto{\pgfqpoint{2.311832in}{3.593964in}}%
\pgfpathlineto{\pgfqpoint{2.334400in}{3.613058in}}%
\pgfpathlineto{\pgfqpoint{2.356968in}{3.629246in}}%
\pgfpathlineto{\pgfqpoint{2.411131in}{3.662625in}}%
\pgfpathlineto{\pgfqpoint{2.433699in}{3.675423in}}%
\pgfpathlineto{\pgfqpoint{2.456267in}{3.688464in}}%
\pgfpathlineto{\pgfqpoint{2.546539in}{3.730485in}}%
\pgfpathlineto{\pgfqpoint{2.641325in}{3.770871in}}%
\pgfpathlineto{\pgfqpoint{2.681947in}{3.784905in}}%
\pgfpathlineto{\pgfqpoint{2.731597in}{3.800510in}}%
\pgfpathlineto{\pgfqpoint{2.812842in}{3.822934in}}%
\pgfpathlineto{\pgfqpoint{2.853464in}{3.831744in}}%
\pgfpathlineto{\pgfqpoint{2.952764in}{3.852933in}}%
\pgfpathlineto{\pgfqpoint{2.997900in}{3.862604in}}%
\pgfpathlineto{\pgfqpoint{3.029495in}{3.869240in}}%
\pgfpathlineto{\pgfqpoint{3.106226in}{3.882498in}}%
\pgfpathlineto{\pgfqpoint{3.237121in}{3.899768in}}%
\pgfpathlineto{\pgfqpoint{3.426692in}{3.928309in}}%
\pgfpathlineto{\pgfqpoint{3.634318in}{3.952369in}}%
\pgfpathlineto{\pgfqpoint{3.792294in}{3.969570in}}%
\pgfpathlineto{\pgfqpoint{3.859998in}{3.975203in}}%
\pgfpathlineto{\pgfqpoint{3.887080in}{3.978331in}}%
\pgfpathlineto{\pgfqpoint{3.981866in}{3.988896in}}%
\pgfpathlineto{\pgfqpoint{4.031515in}{3.992483in}}%
\pgfpathlineto{\pgfqpoint{4.121787in}{3.997336in}}%
\pgfpathlineto{\pgfqpoint{4.329413in}{4.012372in}}%
\pgfpathlineto{\pgfqpoint{4.482876in}{4.018821in}}%
\pgfpathlineto{\pgfqpoint{4.595716in}{4.024113in}}%
\pgfpathlineto{\pgfqpoint{4.640852in}{4.027063in}}%
\pgfpathlineto{\pgfqpoint{4.758205in}{4.030992in}}%
\pgfpathlineto{\pgfqpoint{4.970345in}{4.039041in}}%
\pgfpathlineto{\pgfqpoint{5.029022in}{4.040988in}}%
\pgfpathlineto{\pgfqpoint{5.083185in}{4.043671in}}%
\pgfpathlineto{\pgfqpoint{5.141862in}{4.045924in}}%
\pgfpathlineto{\pgfqpoint{5.245675in}{4.048815in}}%
\pgfpathlineto{\pgfqpoint{5.444273in}{4.053796in}}%
\pgfpathlineto{\pgfqpoint{5.516491in}{4.055674in}}%
\pgfpathlineto{\pgfqpoint{5.534545in}{4.056000in}}%
\pgfpathlineto{\pgfqpoint{5.534545in}{4.056000in}}%
\pgfusepath{stroke}%
\end{pgfscope}%
\begin{pgfscope}%
\pgfpathrectangle{\pgfqpoint{0.800000in}{0.528000in}}{\pgfqpoint{4.960000in}{3.696000in}}%
\pgfusepath{clip}%
\pgfsetrectcap%
\pgfsetroundjoin%
\pgfsetlinewidth{1.505625pt}%
\definecolor{currentstroke}{rgb}{1.000000,0.498039,0.054902}%
\pgfsetstrokecolor{currentstroke}%
\pgfsetdash{}{0pt}%
\pgfpathmoveto{\pgfqpoint{1.025455in}{0.696000in}}%
\pgfpathlineto{\pgfqpoint{1.029968in}{0.699379in}}%
\pgfpathlineto{\pgfqpoint{1.038995in}{0.709069in}}%
\pgfpathlineto{\pgfqpoint{1.048023in}{0.714686in}}%
\pgfpathlineto{\pgfqpoint{1.057050in}{0.718356in}}%
\pgfpathlineto{\pgfqpoint{1.079618in}{0.723795in}}%
\pgfpathlineto{\pgfqpoint{1.210512in}{0.749320in}}%
\pgfpathlineto{\pgfqpoint{1.260162in}{0.762313in}}%
\pgfpathlineto{\pgfqpoint{1.269189in}{0.767344in}}%
\pgfpathlineto{\pgfqpoint{1.282730in}{0.779568in}}%
\pgfpathlineto{\pgfqpoint{1.300784in}{0.797404in}}%
\pgfpathlineto{\pgfqpoint{1.309812in}{0.809785in}}%
\pgfpathlineto{\pgfqpoint{1.332380in}{0.835352in}}%
\pgfpathlineto{\pgfqpoint{1.359461in}{0.861372in}}%
\pgfpathlineto{\pgfqpoint{1.373002in}{0.874705in}}%
\pgfpathlineto{\pgfqpoint{1.445220in}{0.953425in}}%
\pgfpathlineto{\pgfqpoint{1.472301in}{0.988199in}}%
\pgfpathlineto{\pgfqpoint{1.494869in}{1.018630in}}%
\pgfpathlineto{\pgfqpoint{1.526465in}{1.058364in}}%
\pgfpathlineto{\pgfqpoint{1.571601in}{1.119755in}}%
\pgfpathlineto{\pgfqpoint{1.589655in}{1.154189in}}%
\pgfpathlineto{\pgfqpoint{1.607710in}{1.190082in}}%
\pgfpathlineto{\pgfqpoint{1.630278in}{1.228127in}}%
\pgfpathlineto{\pgfqpoint{1.643818in}{1.247846in}}%
\pgfpathlineto{\pgfqpoint{1.652846in}{1.259828in}}%
\pgfpathlineto{\pgfqpoint{1.661873in}{1.269007in}}%
\pgfpathlineto{\pgfqpoint{1.670900in}{1.280171in}}%
\pgfpathlineto{\pgfqpoint{1.684441in}{1.297233in}}%
\pgfpathlineto{\pgfqpoint{1.697982in}{1.317716in}}%
\pgfpathlineto{\pgfqpoint{1.720550in}{1.358279in}}%
\pgfpathlineto{\pgfqpoint{1.738604in}{1.395947in}}%
\pgfpathlineto{\pgfqpoint{1.761172in}{1.454706in}}%
\pgfpathlineto{\pgfqpoint{1.774713in}{1.499412in}}%
\pgfpathlineto{\pgfqpoint{1.815335in}{1.668217in}}%
\pgfpathlineto{\pgfqpoint{1.824363in}{1.714784in}}%
\pgfpathlineto{\pgfqpoint{1.874012in}{1.972058in}}%
\pgfpathlineto{\pgfqpoint{1.919148in}{2.174098in}}%
\pgfpathlineto{\pgfqpoint{1.932689in}{2.226524in}}%
\pgfpathlineto{\pgfqpoint{1.955257in}{2.304228in}}%
\pgfpathlineto{\pgfqpoint{1.986852in}{2.409076in}}%
\pgfpathlineto{\pgfqpoint{2.031988in}{2.572398in}}%
\pgfpathlineto{\pgfqpoint{2.050043in}{2.643505in}}%
\pgfpathlineto{\pgfqpoint{2.068097in}{2.717658in}}%
\pgfpathlineto{\pgfqpoint{2.099692in}{2.850667in}}%
\pgfpathlineto{\pgfqpoint{2.131288in}{2.981502in}}%
\pgfpathlineto{\pgfqpoint{2.167396in}{3.121893in}}%
\pgfpathlineto{\pgfqpoint{2.189965in}{3.198838in}}%
\pgfpathlineto{\pgfqpoint{2.212533in}{3.270419in}}%
\pgfpathlineto{\pgfqpoint{2.235101in}{3.331352in}}%
\pgfpathlineto{\pgfqpoint{2.271209in}{3.416636in}}%
\pgfpathlineto{\pgfqpoint{2.293777in}{3.465281in}}%
\pgfpathlineto{\pgfqpoint{2.311832in}{3.498639in}}%
\pgfpathlineto{\pgfqpoint{2.329886in}{3.523954in}}%
\pgfpathlineto{\pgfqpoint{2.343427in}{3.541580in}}%
\pgfpathlineto{\pgfqpoint{2.384050in}{3.589772in}}%
\pgfpathlineto{\pgfqpoint{2.424672in}{3.632073in}}%
\pgfpathlineto{\pgfqpoint{2.456267in}{3.662655in}}%
\pgfpathlineto{\pgfqpoint{2.474322in}{3.679178in}}%
\pgfpathlineto{\pgfqpoint{2.483349in}{3.687782in}}%
\pgfpathlineto{\pgfqpoint{2.546539in}{3.738164in}}%
\pgfpathlineto{\pgfqpoint{2.573621in}{3.756554in}}%
\pgfpathlineto{\pgfqpoint{2.596189in}{3.770123in}}%
\pgfpathlineto{\pgfqpoint{2.668407in}{3.807157in}}%
\pgfpathlineto{\pgfqpoint{2.686461in}{3.815233in}}%
\pgfpathlineto{\pgfqpoint{2.704515in}{3.822496in}}%
\pgfpathlineto{\pgfqpoint{2.745138in}{3.835845in}}%
\pgfpathlineto{\pgfqpoint{2.808328in}{3.854230in}}%
\pgfpathlineto{\pgfqpoint{2.839924in}{3.860874in}}%
\pgfpathlineto{\pgfqpoint{3.011441in}{3.905494in}}%
\pgfpathlineto{\pgfqpoint{3.052063in}{3.915281in}}%
\pgfpathlineto{\pgfqpoint{3.074631in}{3.920791in}}%
\pgfpathlineto{\pgfqpoint{3.110740in}{3.927721in}}%
\pgfpathlineto{\pgfqpoint{3.137821in}{3.933504in}}%
\pgfpathlineto{\pgfqpoint{3.210039in}{3.945266in}}%
\pgfpathlineto{\pgfqpoint{3.246148in}{3.950184in}}%
\pgfpathlineto{\pgfqpoint{3.381556in}{3.973303in}}%
\pgfpathlineto{\pgfqpoint{3.489883in}{3.985355in}}%
\pgfpathlineto{\pgfqpoint{3.571127in}{3.992807in}}%
\pgfpathlineto{\pgfqpoint{3.593696in}{3.995422in}}%
\pgfpathlineto{\pgfqpoint{3.692995in}{4.004171in}}%
\pgfpathlineto{\pgfqpoint{3.891593in}{4.018466in}}%
\pgfpathlineto{\pgfqpoint{3.990893in}{4.022781in}}%
\pgfpathlineto{\pgfqpoint{4.054083in}{4.024912in}}%
\pgfpathlineto{\pgfqpoint{4.162410in}{4.030018in}}%
\pgfpathlineto{\pgfqpoint{4.207546in}{4.031056in}}%
\pgfpathlineto{\pgfqpoint{4.383576in}{4.035210in}}%
\pgfpathlineto{\pgfqpoint{4.496416in}{4.038584in}}%
\pgfpathlineto{\pgfqpoint{4.622797in}{4.041548in}}%
\pgfpathlineto{\pgfqpoint{4.731124in}{4.044572in}}%
\pgfpathlineto{\pgfqpoint{4.866532in}{4.047364in}}%
\pgfpathlineto{\pgfqpoint{5.408165in}{4.055295in}}%
\pgfpathlineto{\pgfqpoint{5.534545in}{4.056000in}}%
\pgfpathlineto{\pgfqpoint{5.534545in}{4.056000in}}%
\pgfusepath{stroke}%
\end{pgfscope}%
\begin{pgfscope}%
\pgfsetrectcap%
\pgfsetmiterjoin%
\pgfsetlinewidth{0.803000pt}%
\definecolor{currentstroke}{rgb}{0.000000,0.000000,0.000000}%
\pgfsetstrokecolor{currentstroke}%
\pgfsetdash{}{0pt}%
\pgfpathmoveto{\pgfqpoint{0.800000in}{0.528000in}}%
\pgfpathlineto{\pgfqpoint{0.800000in}{4.224000in}}%
\pgfusepath{stroke}%
\end{pgfscope}%
\begin{pgfscope}%
\pgfsetrectcap%
\pgfsetmiterjoin%
\pgfsetlinewidth{0.803000pt}%
\definecolor{currentstroke}{rgb}{0.000000,0.000000,0.000000}%
\pgfsetstrokecolor{currentstroke}%
\pgfsetdash{}{0pt}%
\pgfpathmoveto{\pgfqpoint{5.760000in}{0.528000in}}%
\pgfpathlineto{\pgfqpoint{5.760000in}{4.224000in}}%
\pgfusepath{stroke}%
\end{pgfscope}%
\begin{pgfscope}%
\pgfsetrectcap%
\pgfsetmiterjoin%
\pgfsetlinewidth{0.803000pt}%
\definecolor{currentstroke}{rgb}{0.000000,0.000000,0.000000}%
\pgfsetstrokecolor{currentstroke}%
\pgfsetdash{}{0pt}%
\pgfpathmoveto{\pgfqpoint{0.800000in}{0.528000in}}%
\pgfpathlineto{\pgfqpoint{5.760000in}{0.528000in}}%
\pgfusepath{stroke}%
\end{pgfscope}%
\begin{pgfscope}%
\pgfsetrectcap%
\pgfsetmiterjoin%
\pgfsetlinewidth{0.803000pt}%
\definecolor{currentstroke}{rgb}{0.000000,0.000000,0.000000}%
\pgfsetstrokecolor{currentstroke}%
\pgfsetdash{}{0pt}%
\pgfpathmoveto{\pgfqpoint{0.800000in}{4.224000in}}%
\pgfpathlineto{\pgfqpoint{5.760000in}{4.224000in}}%
\pgfusepath{stroke}%
\end{pgfscope}%
\begin{pgfscope}%
\pgfsetbuttcap%
\pgfsetmiterjoin%
\definecolor{currentfill}{rgb}{1.000000,1.000000,1.000000}%
\pgfsetfillcolor{currentfill}%
\pgfsetfillopacity{0.800000}%
\pgfsetlinewidth{1.003750pt}%
\definecolor{currentstroke}{rgb}{0.800000,0.800000,0.800000}%
\pgfsetstrokecolor{currentstroke}%
\pgfsetstrokeopacity{0.800000}%
\pgfsetdash{}{0pt}%
\pgfpathmoveto{\pgfqpoint{3.695263in}{0.604389in}}%
\pgfpathlineto{\pgfqpoint{5.653056in}{0.604389in}}%
\pgfpathquadraticcurveto{\pgfqpoint{5.683611in}{0.604389in}}{\pgfqpoint{5.683611in}{0.634944in}}%
\pgfpathlineto{\pgfqpoint{5.683611in}{1.045746in}}%
\pgfpathquadraticcurveto{\pgfqpoint{5.683611in}{1.076302in}}{\pgfqpoint{5.653056in}{1.076302in}}%
\pgfpathlineto{\pgfqpoint{3.695263in}{1.076302in}}%
\pgfpathquadraticcurveto{\pgfqpoint{3.664707in}{1.076302in}}{\pgfqpoint{3.664707in}{1.045746in}}%
\pgfpathlineto{\pgfqpoint{3.664707in}{0.634944in}}%
\pgfpathquadraticcurveto{\pgfqpoint{3.664707in}{0.604389in}}{\pgfqpoint{3.695263in}{0.604389in}}%
\pgfpathclose%
\pgfusepath{stroke,fill}%
\end{pgfscope}%
\begin{pgfscope}%
\pgfsetrectcap%
\pgfsetroundjoin%
\pgfsetlinewidth{1.505625pt}%
\definecolor{currentstroke}{rgb}{0.121569,0.466667,0.705882}%
\pgfsetstrokecolor{currentstroke}%
\pgfsetdash{}{0pt}%
\pgfpathmoveto{\pgfqpoint{3.725818in}{0.961719in}}%
\pgfpathlineto{\pgfqpoint{4.031374in}{0.961719in}}%
\pgfusepath{stroke}%
\end{pgfscope}%
\begin{pgfscope}%
\pgftext[x=4.153596in,y=0.908246in,left,base]{\fontsize{11.000000}{13.200000}\selectfont Primary write concern}%
\end{pgfscope}%
\begin{pgfscope}%
\pgfsetrectcap%
\pgfsetroundjoin%
\pgfsetlinewidth{1.505625pt}%
\definecolor{currentstroke}{rgb}{1.000000,0.498039,0.054902}%
\pgfsetstrokecolor{currentstroke}%
\pgfsetdash{}{0pt}%
\pgfpathmoveto{\pgfqpoint{3.725818in}{0.748679in}}%
\pgfpathlineto{\pgfqpoint{4.031374in}{0.748679in}}%
\pgfusepath{stroke}%
\end{pgfscope}%
\begin{pgfscope}%
\pgftext[x=4.153596in,y=0.695207in,left,base]{\fontsize{11.000000}{13.200000}\selectfont Estimated as durable}%
\end{pgfscope}%
\end{pgfpicture}%
\makeatother%
\endgroup%

    \label{fig:cdf}
    \caption{A cumulative frequency graph of the writes which become acknowledged and 1-durable at latencies 1-1000ms.}
\end{figure}

\begin{figure}
    \centering
    %% Creator: Matplotlib, PGF backend
%%
%% To include the figure in your LaTeX document, write
%%   \input{<filename>.pgf}
%%
%% Make sure the required packages are loaded in your preamble
%%   \usepackage{pgf}
%%
%% Figures using additional raster images can only be included by \input if
%% they are in the same directory as the main LaTeX file. For loading figures
%% from other directories you can use the `import` package
%%   \usepackage{import}
%% and then include the figures with
%%   \import{<path to file>}{<filename>.pgf}
%%
%% Matplotlib used the following preamble
%%   \usepackage[utf8x]{inputenc}
%%   \usepackage[T1]{fontenc}
%%   \usepackage{lmodern}
%%
\begingroup%
\makeatletter%
\begin{pgfpicture}%
\pgfpathrectangle{\pgfpointorigin}{\pgfqpoint{6.400000in}{4.800000in}}%
\pgfusepath{use as bounding box, clip}%
\begin{pgfscope}%
\pgfsetbuttcap%
\pgfsetmiterjoin%
\definecolor{currentfill}{rgb}{1.000000,1.000000,1.000000}%
\pgfsetfillcolor{currentfill}%
\pgfsetlinewidth{0.000000pt}%
\definecolor{currentstroke}{rgb}{1.000000,1.000000,1.000000}%
\pgfsetstrokecolor{currentstroke}%
\pgfsetdash{}{0pt}%
\pgfpathmoveto{\pgfqpoint{0.000000in}{0.000000in}}%
\pgfpathlineto{\pgfqpoint{6.400000in}{0.000000in}}%
\pgfpathlineto{\pgfqpoint{6.400000in}{4.800000in}}%
\pgfpathlineto{\pgfqpoint{0.000000in}{4.800000in}}%
\pgfpathclose%
\pgfusepath{fill}%
\end{pgfscope}%
\begin{pgfscope}%
\pgfsetbuttcap%
\pgfsetmiterjoin%
\definecolor{currentfill}{rgb}{1.000000,1.000000,1.000000}%
\pgfsetfillcolor{currentfill}%
\pgfsetlinewidth{0.000000pt}%
\definecolor{currentstroke}{rgb}{0.000000,0.000000,0.000000}%
\pgfsetstrokecolor{currentstroke}%
\pgfsetstrokeopacity{0.000000}%
\pgfsetdash{}{0pt}%
\pgfpathmoveto{\pgfqpoint{0.800000in}{0.528000in}}%
\pgfpathlineto{\pgfqpoint{5.760000in}{0.528000in}}%
\pgfpathlineto{\pgfqpoint{5.760000in}{4.224000in}}%
\pgfpathlineto{\pgfqpoint{0.800000in}{4.224000in}}%
\pgfpathclose%
\pgfusepath{fill}%
\end{pgfscope}%
\begin{pgfscope}%
\pgfsetbuttcap%
\pgfsetroundjoin%
\definecolor{currentfill}{rgb}{0.000000,0.000000,0.000000}%
\pgfsetfillcolor{currentfill}%
\pgfsetlinewidth{0.803000pt}%
\definecolor{currentstroke}{rgb}{0.000000,0.000000,0.000000}%
\pgfsetstrokecolor{currentstroke}%
\pgfsetdash{}{0pt}%
\pgfsys@defobject{currentmarker}{\pgfqpoint{0.000000in}{-0.048611in}}{\pgfqpoint{0.000000in}{0.000000in}}{%
\pgfpathmoveto{\pgfqpoint{0.000000in}{0.000000in}}%
\pgfpathlineto{\pgfqpoint{0.000000in}{-0.048611in}}%
\pgfusepath{stroke,fill}%
}%
\begin{pgfscope}%
\pgfsys@transformshift{1.025455in}{0.528000in}%
\pgfsys@useobject{currentmarker}{}%
\end{pgfscope}%
\end{pgfscope}%
\begin{pgfscope}%
\definecolor{textcolor}{rgb}{0.000000,0.000000,0.000000}%
\pgfsetstrokecolor{textcolor}%
\pgfsetfillcolor{textcolor}%
\pgftext[x=1.025455in,y=0.430778in,,top]{\color{textcolor}\fontsize{11.000000}{13.200000}\selectfont \(\displaystyle 0\)}%
\end{pgfscope}%
\begin{pgfscope}%
\pgfsetbuttcap%
\pgfsetroundjoin%
\definecolor{currentfill}{rgb}{0.000000,0.000000,0.000000}%
\pgfsetfillcolor{currentfill}%
\pgfsetlinewidth{0.803000pt}%
\definecolor{currentstroke}{rgb}{0.000000,0.000000,0.000000}%
\pgfsetstrokecolor{currentstroke}%
\pgfsetdash{}{0pt}%
\pgfsys@defobject{currentmarker}{\pgfqpoint{0.000000in}{-0.048611in}}{\pgfqpoint{0.000000in}{0.000000in}}{%
\pgfpathmoveto{\pgfqpoint{0.000000in}{0.000000in}}%
\pgfpathlineto{\pgfqpoint{0.000000in}{-0.048611in}}%
\pgfusepath{stroke,fill}%
}%
\begin{pgfscope}%
\pgfsys@transformshift{1.928175in}{0.528000in}%
\pgfsys@useobject{currentmarker}{}%
\end{pgfscope}%
\end{pgfscope}%
\begin{pgfscope}%
\definecolor{textcolor}{rgb}{0.000000,0.000000,0.000000}%
\pgfsetstrokecolor{textcolor}%
\pgfsetfillcolor{textcolor}%
\pgftext[x=1.928175in,y=0.430778in,,top]{\color{textcolor}\fontsize{11.000000}{13.200000}\selectfont \(\displaystyle 200\)}%
\end{pgfscope}%
\begin{pgfscope}%
\pgfsetbuttcap%
\pgfsetroundjoin%
\definecolor{currentfill}{rgb}{0.000000,0.000000,0.000000}%
\pgfsetfillcolor{currentfill}%
\pgfsetlinewidth{0.803000pt}%
\definecolor{currentstroke}{rgb}{0.000000,0.000000,0.000000}%
\pgfsetstrokecolor{currentstroke}%
\pgfsetdash{}{0pt}%
\pgfsys@defobject{currentmarker}{\pgfqpoint{0.000000in}{-0.048611in}}{\pgfqpoint{0.000000in}{0.000000in}}{%
\pgfpathmoveto{\pgfqpoint{0.000000in}{0.000000in}}%
\pgfpathlineto{\pgfqpoint{0.000000in}{-0.048611in}}%
\pgfusepath{stroke,fill}%
}%
\begin{pgfscope}%
\pgfsys@transformshift{2.830896in}{0.528000in}%
\pgfsys@useobject{currentmarker}{}%
\end{pgfscope}%
\end{pgfscope}%
\begin{pgfscope}%
\definecolor{textcolor}{rgb}{0.000000,0.000000,0.000000}%
\pgfsetstrokecolor{textcolor}%
\pgfsetfillcolor{textcolor}%
\pgftext[x=2.830896in,y=0.430778in,,top]{\color{textcolor}\fontsize{11.000000}{13.200000}\selectfont \(\displaystyle 400\)}%
\end{pgfscope}%
\begin{pgfscope}%
\pgfsetbuttcap%
\pgfsetroundjoin%
\definecolor{currentfill}{rgb}{0.000000,0.000000,0.000000}%
\pgfsetfillcolor{currentfill}%
\pgfsetlinewidth{0.803000pt}%
\definecolor{currentstroke}{rgb}{0.000000,0.000000,0.000000}%
\pgfsetstrokecolor{currentstroke}%
\pgfsetdash{}{0pt}%
\pgfsys@defobject{currentmarker}{\pgfqpoint{0.000000in}{-0.048611in}}{\pgfqpoint{0.000000in}{0.000000in}}{%
\pgfpathmoveto{\pgfqpoint{0.000000in}{0.000000in}}%
\pgfpathlineto{\pgfqpoint{0.000000in}{-0.048611in}}%
\pgfusepath{stroke,fill}%
}%
\begin{pgfscope}%
\pgfsys@transformshift{3.733617in}{0.528000in}%
\pgfsys@useobject{currentmarker}{}%
\end{pgfscope}%
\end{pgfscope}%
\begin{pgfscope}%
\definecolor{textcolor}{rgb}{0.000000,0.000000,0.000000}%
\pgfsetstrokecolor{textcolor}%
\pgfsetfillcolor{textcolor}%
\pgftext[x=3.733617in,y=0.430778in,,top]{\color{textcolor}\fontsize{11.000000}{13.200000}\selectfont \(\displaystyle 600\)}%
\end{pgfscope}%
\begin{pgfscope}%
\pgfsetbuttcap%
\pgfsetroundjoin%
\definecolor{currentfill}{rgb}{0.000000,0.000000,0.000000}%
\pgfsetfillcolor{currentfill}%
\pgfsetlinewidth{0.803000pt}%
\definecolor{currentstroke}{rgb}{0.000000,0.000000,0.000000}%
\pgfsetstrokecolor{currentstroke}%
\pgfsetdash{}{0pt}%
\pgfsys@defobject{currentmarker}{\pgfqpoint{0.000000in}{-0.048611in}}{\pgfqpoint{0.000000in}{0.000000in}}{%
\pgfpathmoveto{\pgfqpoint{0.000000in}{0.000000in}}%
\pgfpathlineto{\pgfqpoint{0.000000in}{-0.048611in}}%
\pgfusepath{stroke,fill}%
}%
\begin{pgfscope}%
\pgfsys@transformshift{4.636338in}{0.528000in}%
\pgfsys@useobject{currentmarker}{}%
\end{pgfscope}%
\end{pgfscope}%
\begin{pgfscope}%
\definecolor{textcolor}{rgb}{0.000000,0.000000,0.000000}%
\pgfsetstrokecolor{textcolor}%
\pgfsetfillcolor{textcolor}%
\pgftext[x=4.636338in,y=0.430778in,,top]{\color{textcolor}\fontsize{11.000000}{13.200000}\selectfont \(\displaystyle 800\)}%
\end{pgfscope}%
\begin{pgfscope}%
\pgfsetbuttcap%
\pgfsetroundjoin%
\definecolor{currentfill}{rgb}{0.000000,0.000000,0.000000}%
\pgfsetfillcolor{currentfill}%
\pgfsetlinewidth{0.803000pt}%
\definecolor{currentstroke}{rgb}{0.000000,0.000000,0.000000}%
\pgfsetstrokecolor{currentstroke}%
\pgfsetdash{}{0pt}%
\pgfsys@defobject{currentmarker}{\pgfqpoint{0.000000in}{-0.048611in}}{\pgfqpoint{0.000000in}{0.000000in}}{%
\pgfpathmoveto{\pgfqpoint{0.000000in}{0.000000in}}%
\pgfpathlineto{\pgfqpoint{0.000000in}{-0.048611in}}%
\pgfusepath{stroke,fill}%
}%
\begin{pgfscope}%
\pgfsys@transformshift{5.539059in}{0.528000in}%
\pgfsys@useobject{currentmarker}{}%
\end{pgfscope}%
\end{pgfscope}%
\begin{pgfscope}%
\definecolor{textcolor}{rgb}{0.000000,0.000000,0.000000}%
\pgfsetstrokecolor{textcolor}%
\pgfsetfillcolor{textcolor}%
\pgftext[x=5.539059in,y=0.430778in,,top]{\color{textcolor}\fontsize{11.000000}{13.200000}\selectfont \(\displaystyle 1000\)}%
\end{pgfscope}%
\begin{pgfscope}%
\definecolor{textcolor}{rgb}{0.000000,0.000000,0.000000}%
\pgfsetstrokecolor{textcolor}%
\pgfsetfillcolor{textcolor}%
\pgftext[x=3.280000in,y=0.240271in,,top]{\color{textcolor}\fontsize{11.000000}{13.200000}\selectfont Latency (in milliseconds)}%
\end{pgfscope}%
\begin{pgfscope}%
\pgfsetbuttcap%
\pgfsetroundjoin%
\definecolor{currentfill}{rgb}{0.000000,0.000000,0.000000}%
\pgfsetfillcolor{currentfill}%
\pgfsetlinewidth{0.803000pt}%
\definecolor{currentstroke}{rgb}{0.000000,0.000000,0.000000}%
\pgfsetstrokecolor{currentstroke}%
\pgfsetdash{}{0pt}%
\pgfsys@defobject{currentmarker}{\pgfqpoint{-0.048611in}{0.000000in}}{\pgfqpoint{0.000000in}{0.000000in}}{%
\pgfpathmoveto{\pgfqpoint{0.000000in}{0.000000in}}%
\pgfpathlineto{\pgfqpoint{-0.048611in}{0.000000in}}%
\pgfusepath{stroke,fill}%
}%
\begin{pgfscope}%
\pgfsys@transformshift{0.800000in}{0.696000in}%
\pgfsys@useobject{currentmarker}{}%
\end{pgfscope}%
\end{pgfscope}%
\begin{pgfscope}%
\definecolor{textcolor}{rgb}{0.000000,0.000000,0.000000}%
\pgfsetstrokecolor{textcolor}%
\pgfsetfillcolor{textcolor}%
\pgftext[x=0.436832in,y=0.643378in,left,base]{\color{textcolor}\fontsize{11.000000}{13.200000}\selectfont \(\displaystyle 0.00\)}%
\end{pgfscope}%
\begin{pgfscope}%
\pgfsetbuttcap%
\pgfsetroundjoin%
\definecolor{currentfill}{rgb}{0.000000,0.000000,0.000000}%
\pgfsetfillcolor{currentfill}%
\pgfsetlinewidth{0.803000pt}%
\definecolor{currentstroke}{rgb}{0.000000,0.000000,0.000000}%
\pgfsetstrokecolor{currentstroke}%
\pgfsetdash{}{0pt}%
\pgfsys@defobject{currentmarker}{\pgfqpoint{-0.048611in}{0.000000in}}{\pgfqpoint{0.000000in}{0.000000in}}{%
\pgfpathmoveto{\pgfqpoint{0.000000in}{0.000000in}}%
\pgfpathlineto{\pgfqpoint{-0.048611in}{0.000000in}}%
\pgfusepath{stroke,fill}%
}%
\begin{pgfscope}%
\pgfsys@transformshift{0.800000in}{1.107884in}%
\pgfsys@useobject{currentmarker}{}%
\end{pgfscope}%
\end{pgfscope}%
\begin{pgfscope}%
\definecolor{textcolor}{rgb}{0.000000,0.000000,0.000000}%
\pgfsetstrokecolor{textcolor}%
\pgfsetfillcolor{textcolor}%
\pgftext[x=0.436832in,y=1.055262in,left,base]{\color{textcolor}\fontsize{11.000000}{13.200000}\selectfont \(\displaystyle 0.05\)}%
\end{pgfscope}%
\begin{pgfscope}%
\pgfsetbuttcap%
\pgfsetroundjoin%
\definecolor{currentfill}{rgb}{0.000000,0.000000,0.000000}%
\pgfsetfillcolor{currentfill}%
\pgfsetlinewidth{0.803000pt}%
\definecolor{currentstroke}{rgb}{0.000000,0.000000,0.000000}%
\pgfsetstrokecolor{currentstroke}%
\pgfsetdash{}{0pt}%
\pgfsys@defobject{currentmarker}{\pgfqpoint{-0.048611in}{0.000000in}}{\pgfqpoint{0.000000in}{0.000000in}}{%
\pgfpathmoveto{\pgfqpoint{0.000000in}{0.000000in}}%
\pgfpathlineto{\pgfqpoint{-0.048611in}{0.000000in}}%
\pgfusepath{stroke,fill}%
}%
\begin{pgfscope}%
\pgfsys@transformshift{0.800000in}{1.519768in}%
\pgfsys@useobject{currentmarker}{}%
\end{pgfscope}%
\end{pgfscope}%
\begin{pgfscope}%
\definecolor{textcolor}{rgb}{0.000000,0.000000,0.000000}%
\pgfsetstrokecolor{textcolor}%
\pgfsetfillcolor{textcolor}%
\pgftext[x=0.436832in,y=1.467145in,left,base]{\color{textcolor}\fontsize{11.000000}{13.200000}\selectfont \(\displaystyle 0.10\)}%
\end{pgfscope}%
\begin{pgfscope}%
\pgfsetbuttcap%
\pgfsetroundjoin%
\definecolor{currentfill}{rgb}{0.000000,0.000000,0.000000}%
\pgfsetfillcolor{currentfill}%
\pgfsetlinewidth{0.803000pt}%
\definecolor{currentstroke}{rgb}{0.000000,0.000000,0.000000}%
\pgfsetstrokecolor{currentstroke}%
\pgfsetdash{}{0pt}%
\pgfsys@defobject{currentmarker}{\pgfqpoint{-0.048611in}{0.000000in}}{\pgfqpoint{0.000000in}{0.000000in}}{%
\pgfpathmoveto{\pgfqpoint{0.000000in}{0.000000in}}%
\pgfpathlineto{\pgfqpoint{-0.048611in}{0.000000in}}%
\pgfusepath{stroke,fill}%
}%
\begin{pgfscope}%
\pgfsys@transformshift{0.800000in}{1.931652in}%
\pgfsys@useobject{currentmarker}{}%
\end{pgfscope}%
\end{pgfscope}%
\begin{pgfscope}%
\definecolor{textcolor}{rgb}{0.000000,0.000000,0.000000}%
\pgfsetstrokecolor{textcolor}%
\pgfsetfillcolor{textcolor}%
\pgftext[x=0.436832in,y=1.879029in,left,base]{\color{textcolor}\fontsize{11.000000}{13.200000}\selectfont \(\displaystyle 0.15\)}%
\end{pgfscope}%
\begin{pgfscope}%
\pgfsetbuttcap%
\pgfsetroundjoin%
\definecolor{currentfill}{rgb}{0.000000,0.000000,0.000000}%
\pgfsetfillcolor{currentfill}%
\pgfsetlinewidth{0.803000pt}%
\definecolor{currentstroke}{rgb}{0.000000,0.000000,0.000000}%
\pgfsetstrokecolor{currentstroke}%
\pgfsetdash{}{0pt}%
\pgfsys@defobject{currentmarker}{\pgfqpoint{-0.048611in}{0.000000in}}{\pgfqpoint{0.000000in}{0.000000in}}{%
\pgfpathmoveto{\pgfqpoint{0.000000in}{0.000000in}}%
\pgfpathlineto{\pgfqpoint{-0.048611in}{0.000000in}}%
\pgfusepath{stroke,fill}%
}%
\begin{pgfscope}%
\pgfsys@transformshift{0.800000in}{2.343535in}%
\pgfsys@useobject{currentmarker}{}%
\end{pgfscope}%
\end{pgfscope}%
\begin{pgfscope}%
\definecolor{textcolor}{rgb}{0.000000,0.000000,0.000000}%
\pgfsetstrokecolor{textcolor}%
\pgfsetfillcolor{textcolor}%
\pgftext[x=0.436832in,y=2.290913in,left,base]{\color{textcolor}\fontsize{11.000000}{13.200000}\selectfont \(\displaystyle 0.20\)}%
\end{pgfscope}%
\begin{pgfscope}%
\pgfsetbuttcap%
\pgfsetroundjoin%
\definecolor{currentfill}{rgb}{0.000000,0.000000,0.000000}%
\pgfsetfillcolor{currentfill}%
\pgfsetlinewidth{0.803000pt}%
\definecolor{currentstroke}{rgb}{0.000000,0.000000,0.000000}%
\pgfsetstrokecolor{currentstroke}%
\pgfsetdash{}{0pt}%
\pgfsys@defobject{currentmarker}{\pgfqpoint{-0.048611in}{0.000000in}}{\pgfqpoint{0.000000in}{0.000000in}}{%
\pgfpathmoveto{\pgfqpoint{0.000000in}{0.000000in}}%
\pgfpathlineto{\pgfqpoint{-0.048611in}{0.000000in}}%
\pgfusepath{stroke,fill}%
}%
\begin{pgfscope}%
\pgfsys@transformshift{0.800000in}{2.755419in}%
\pgfsys@useobject{currentmarker}{}%
\end{pgfscope}%
\end{pgfscope}%
\begin{pgfscope}%
\definecolor{textcolor}{rgb}{0.000000,0.000000,0.000000}%
\pgfsetstrokecolor{textcolor}%
\pgfsetfillcolor{textcolor}%
\pgftext[x=0.436832in,y=2.702797in,left,base]{\color{textcolor}\fontsize{11.000000}{13.200000}\selectfont \(\displaystyle 0.25\)}%
\end{pgfscope}%
\begin{pgfscope}%
\pgfsetbuttcap%
\pgfsetroundjoin%
\definecolor{currentfill}{rgb}{0.000000,0.000000,0.000000}%
\pgfsetfillcolor{currentfill}%
\pgfsetlinewidth{0.803000pt}%
\definecolor{currentstroke}{rgb}{0.000000,0.000000,0.000000}%
\pgfsetstrokecolor{currentstroke}%
\pgfsetdash{}{0pt}%
\pgfsys@defobject{currentmarker}{\pgfqpoint{-0.048611in}{0.000000in}}{\pgfqpoint{0.000000in}{0.000000in}}{%
\pgfpathmoveto{\pgfqpoint{0.000000in}{0.000000in}}%
\pgfpathlineto{\pgfqpoint{-0.048611in}{0.000000in}}%
\pgfusepath{stroke,fill}%
}%
\begin{pgfscope}%
\pgfsys@transformshift{0.800000in}{3.167303in}%
\pgfsys@useobject{currentmarker}{}%
\end{pgfscope}%
\end{pgfscope}%
\begin{pgfscope}%
\definecolor{textcolor}{rgb}{0.000000,0.000000,0.000000}%
\pgfsetstrokecolor{textcolor}%
\pgfsetfillcolor{textcolor}%
\pgftext[x=0.436832in,y=3.114681in,left,base]{\color{textcolor}\fontsize{11.000000}{13.200000}\selectfont \(\displaystyle 0.30\)}%
\end{pgfscope}%
\begin{pgfscope}%
\pgfsetbuttcap%
\pgfsetroundjoin%
\definecolor{currentfill}{rgb}{0.000000,0.000000,0.000000}%
\pgfsetfillcolor{currentfill}%
\pgfsetlinewidth{0.803000pt}%
\definecolor{currentstroke}{rgb}{0.000000,0.000000,0.000000}%
\pgfsetstrokecolor{currentstroke}%
\pgfsetdash{}{0pt}%
\pgfsys@defobject{currentmarker}{\pgfqpoint{-0.048611in}{0.000000in}}{\pgfqpoint{0.000000in}{0.000000in}}{%
\pgfpathmoveto{\pgfqpoint{0.000000in}{0.000000in}}%
\pgfpathlineto{\pgfqpoint{-0.048611in}{0.000000in}}%
\pgfusepath{stroke,fill}%
}%
\begin{pgfscope}%
\pgfsys@transformshift{0.800000in}{3.579187in}%
\pgfsys@useobject{currentmarker}{}%
\end{pgfscope}%
\end{pgfscope}%
\begin{pgfscope}%
\definecolor{textcolor}{rgb}{0.000000,0.000000,0.000000}%
\pgfsetstrokecolor{textcolor}%
\pgfsetfillcolor{textcolor}%
\pgftext[x=0.436832in,y=3.526565in,left,base]{\color{textcolor}\fontsize{11.000000}{13.200000}\selectfont \(\displaystyle 0.35\)}%
\end{pgfscope}%
\begin{pgfscope}%
\pgfsetbuttcap%
\pgfsetroundjoin%
\definecolor{currentfill}{rgb}{0.000000,0.000000,0.000000}%
\pgfsetfillcolor{currentfill}%
\pgfsetlinewidth{0.803000pt}%
\definecolor{currentstroke}{rgb}{0.000000,0.000000,0.000000}%
\pgfsetstrokecolor{currentstroke}%
\pgfsetdash{}{0pt}%
\pgfsys@defobject{currentmarker}{\pgfqpoint{-0.048611in}{0.000000in}}{\pgfqpoint{0.000000in}{0.000000in}}{%
\pgfpathmoveto{\pgfqpoint{0.000000in}{0.000000in}}%
\pgfpathlineto{\pgfqpoint{-0.048611in}{0.000000in}}%
\pgfusepath{stroke,fill}%
}%
\begin{pgfscope}%
\pgfsys@transformshift{0.800000in}{3.991071in}%
\pgfsys@useobject{currentmarker}{}%
\end{pgfscope}%
\end{pgfscope}%
\begin{pgfscope}%
\definecolor{textcolor}{rgb}{0.000000,0.000000,0.000000}%
\pgfsetstrokecolor{textcolor}%
\pgfsetfillcolor{textcolor}%
\pgftext[x=0.436832in,y=3.938449in,left,base]{\color{textcolor}\fontsize{11.000000}{13.200000}\selectfont \(\displaystyle 0.40\)}%
\end{pgfscope}%
\begin{pgfscope}%
\definecolor{textcolor}{rgb}{0.000000,0.000000,0.000000}%
\pgfsetstrokecolor{textcolor}%
\pgfsetfillcolor{textcolor}%
\pgftext[x=0.381277in,y=2.376000in,,bottom,rotate=90.000000]{\color{textcolor}\fontsize{11.000000}{13.200000}\selectfont Fraction of writes}%
\end{pgfscope}%
\begin{pgfscope}%
\pgfpathrectangle{\pgfqpoint{0.800000in}{0.528000in}}{\pgfqpoint{4.960000in}{3.696000in}}%
\pgfusepath{clip}%
\pgfsetrectcap%
\pgfsetroundjoin%
\pgfsetlinewidth{1.505625pt}%
\definecolor{currentstroke}{rgb}{0.121569,0.466667,0.705882}%
\pgfsetstrokecolor{currentstroke}%
\pgfsetdash{}{0pt}%
\pgfpathmoveto{\pgfqpoint{1.025455in}{0.716780in}}%
\pgfpathlineto{\pgfqpoint{1.029968in}{0.738032in}}%
\pgfpathlineto{\pgfqpoint{1.038995in}{0.766864in}}%
\pgfpathlineto{\pgfqpoint{1.052536in}{0.823963in}}%
\pgfpathlineto{\pgfqpoint{1.066077in}{0.886279in}}%
\pgfpathlineto{\pgfqpoint{1.084131in}{0.956384in}}%
\pgfpathlineto{\pgfqpoint{1.097672in}{1.002562in}}%
\pgfpathlineto{\pgfqpoint{1.106699in}{1.028317in}}%
\pgfpathlineto{\pgfqpoint{1.120240in}{1.057462in}}%
\pgfpathlineto{\pgfqpoint{1.133781in}{1.080645in}}%
\pgfpathlineto{\pgfqpoint{1.147322in}{1.099065in}}%
\pgfpathlineto{\pgfqpoint{1.160863in}{1.112873in}}%
\pgfpathlineto{\pgfqpoint{1.178917in}{1.123734in}}%
\pgfpathlineto{\pgfqpoint{1.219540in}{1.138022in}}%
\pgfpathlineto{\pgfqpoint{1.228567in}{1.145055in}}%
\pgfpathlineto{\pgfqpoint{1.237594in}{1.155729in}}%
\pgfpathlineto{\pgfqpoint{1.246621in}{1.172822in}}%
\pgfpathlineto{\pgfqpoint{1.264676in}{1.233631in}}%
\pgfpathlineto{\pgfqpoint{1.269189in}{1.256147in}}%
\pgfpathlineto{\pgfqpoint{1.273703in}{1.285179in}}%
\pgfpathlineto{\pgfqpoint{1.278216in}{1.323003in}}%
\pgfpathlineto{\pgfqpoint{1.291757in}{1.475225in}}%
\pgfpathlineto{\pgfqpoint{1.305298in}{1.672789in}}%
\pgfpathlineto{\pgfqpoint{1.336893in}{2.202542in}}%
\pgfpathlineto{\pgfqpoint{1.359461in}{2.595726in}}%
\pgfpathlineto{\pgfqpoint{1.368488in}{2.716931in}}%
\pgfpathlineto{\pgfqpoint{1.377516in}{2.811085in}}%
\pgfpathlineto{\pgfqpoint{1.386543in}{2.886238in}}%
\pgfpathlineto{\pgfqpoint{1.400084in}{2.974144in}}%
\pgfpathlineto{\pgfqpoint{1.409111in}{3.020093in}}%
\pgfpathlineto{\pgfqpoint{1.418138in}{3.052910in}}%
\pgfpathlineto{\pgfqpoint{1.427165in}{3.078565in}}%
\pgfpathlineto{\pgfqpoint{1.436193in}{3.095917in}}%
\pgfpathlineto{\pgfqpoint{1.440706in}{3.101432in}}%
\pgfpathlineto{\pgfqpoint{1.458761in}{3.110593in}}%
\pgfpathlineto{\pgfqpoint{1.463274in}{3.111207in}}%
\pgfpathlineto{\pgfqpoint{1.467788in}{3.110574in}}%
\pgfpathlineto{\pgfqpoint{1.490356in}{3.088506in}}%
\pgfpathlineto{\pgfqpoint{1.499383in}{3.076242in}}%
\pgfpathlineto{\pgfqpoint{1.508410in}{3.065770in}}%
\pgfpathlineto{\pgfqpoint{1.521951in}{3.048765in}}%
\pgfpathlineto{\pgfqpoint{1.530978in}{3.039346in}}%
\pgfpathlineto{\pgfqpoint{1.535492in}{3.035836in}}%
\pgfpathlineto{\pgfqpoint{1.544519in}{3.032708in}}%
\pgfpathlineto{\pgfqpoint{1.553546in}{3.037395in}}%
\pgfpathlineto{\pgfqpoint{1.567087in}{3.048421in}}%
\pgfpathlineto{\pgfqpoint{1.571601in}{3.055653in}}%
\pgfpathlineto{\pgfqpoint{1.576114in}{3.059361in}}%
\pgfpathlineto{\pgfqpoint{1.594169in}{3.087657in}}%
\pgfpathlineto{\pgfqpoint{1.603196in}{3.109819in}}%
\pgfpathlineto{\pgfqpoint{1.612223in}{3.144982in}}%
\pgfpathlineto{\pgfqpoint{1.621250in}{3.192098in}}%
\pgfpathlineto{\pgfqpoint{1.630278in}{3.253819in}}%
\pgfpathlineto{\pgfqpoint{1.643818in}{3.376448in}}%
\pgfpathlineto{\pgfqpoint{1.652846in}{3.476129in}}%
\pgfpathlineto{\pgfqpoint{1.675414in}{3.726013in}}%
\pgfpathlineto{\pgfqpoint{1.688954in}{3.840753in}}%
\pgfpathlineto{\pgfqpoint{1.702495in}{3.932402in}}%
\pgfpathlineto{\pgfqpoint{1.711522in}{3.985607in}}%
\pgfpathlineto{\pgfqpoint{1.720550in}{4.021546in}}%
\pgfpathlineto{\pgfqpoint{1.725063in}{4.036448in}}%
\pgfpathlineto{\pgfqpoint{1.729577in}{4.045762in}}%
\pgfpathlineto{\pgfqpoint{1.738604in}{4.056000in}}%
\pgfpathlineto{\pgfqpoint{1.747631in}{4.053429in}}%
\pgfpathlineto{\pgfqpoint{1.752145in}{4.050607in}}%
\pgfpathlineto{\pgfqpoint{1.756658in}{4.042524in}}%
\pgfpathlineto{\pgfqpoint{1.765686in}{4.016733in}}%
\pgfpathlineto{\pgfqpoint{1.774713in}{3.978752in}}%
\pgfpathlineto{\pgfqpoint{1.797281in}{3.840060in}}%
\pgfpathlineto{\pgfqpoint{1.810822in}{3.741523in}}%
\pgfpathlineto{\pgfqpoint{1.824363in}{3.616583in}}%
\pgfpathlineto{\pgfqpoint{1.860471in}{3.274399in}}%
\pgfpathlineto{\pgfqpoint{1.878526in}{3.116378in}}%
\pgfpathlineto{\pgfqpoint{1.892067in}{3.012439in}}%
\pgfpathlineto{\pgfqpoint{1.901094in}{2.955191in}}%
\pgfpathlineto{\pgfqpoint{1.910121in}{2.910728in}}%
\pgfpathlineto{\pgfqpoint{1.932689in}{2.805329in}}%
\pgfpathlineto{\pgfqpoint{1.941716in}{2.778026in}}%
\pgfpathlineto{\pgfqpoint{1.955257in}{2.743689in}}%
\pgfpathlineto{\pgfqpoint{1.977825in}{2.704658in}}%
\pgfpathlineto{\pgfqpoint{1.991366in}{2.668574in}}%
\pgfpathlineto{\pgfqpoint{2.004907in}{2.627711in}}%
\pgfpathlineto{\pgfqpoint{2.013934in}{2.597114in}}%
\pgfpathlineto{\pgfqpoint{2.027475in}{2.536598in}}%
\pgfpathlineto{\pgfqpoint{2.041016in}{2.456979in}}%
\pgfpathlineto{\pgfqpoint{2.063584in}{2.313609in}}%
\pgfpathlineto{\pgfqpoint{2.158369in}{1.589666in}}%
\pgfpathlineto{\pgfqpoint{2.176424in}{1.473569in}}%
\pgfpathlineto{\pgfqpoint{2.203505in}{1.309993in}}%
\pgfpathlineto{\pgfqpoint{2.217046in}{1.240975in}}%
\pgfpathlineto{\pgfqpoint{2.239614in}{1.143196in}}%
\pgfpathlineto{\pgfqpoint{2.257669in}{1.079998in}}%
\pgfpathlineto{\pgfqpoint{2.280237in}{1.009201in}}%
\pgfpathlineto{\pgfqpoint{2.298291in}{0.960237in}}%
\pgfpathlineto{\pgfqpoint{2.302805in}{0.946580in}}%
\pgfpathlineto{\pgfqpoint{2.316345in}{0.923076in}}%
\pgfpathlineto{\pgfqpoint{2.325373in}{0.911516in}}%
\pgfpathlineto{\pgfqpoint{2.329886in}{0.906166in}}%
\pgfpathlineto{\pgfqpoint{2.338913in}{0.893214in}}%
\pgfpathlineto{\pgfqpoint{2.375022in}{0.846886in}}%
\pgfpathlineto{\pgfqpoint{2.384050in}{0.836013in}}%
\pgfpathlineto{\pgfqpoint{2.393077in}{0.825725in}}%
\pgfpathlineto{\pgfqpoint{2.397590in}{0.822214in}}%
\pgfpathlineto{\pgfqpoint{2.420158in}{0.794997in}}%
\pgfpathlineto{\pgfqpoint{2.429186in}{0.783928in}}%
\pgfpathlineto{\pgfqpoint{2.438213in}{0.775580in}}%
\pgfpathlineto{\pgfqpoint{2.442726in}{0.769753in}}%
\pgfpathlineto{\pgfqpoint{2.451754in}{0.763469in}}%
\pgfpathlineto{\pgfqpoint{2.474322in}{0.741831in}}%
\pgfpathlineto{\pgfqpoint{2.483349in}{0.730806in}}%
\pgfpathlineto{\pgfqpoint{2.492376in}{0.721807in}}%
\pgfpathlineto{\pgfqpoint{2.501403in}{0.712976in}}%
\pgfpathlineto{\pgfqpoint{2.519458in}{0.698762in}}%
\pgfpathlineto{\pgfqpoint{2.523971in}{0.696000in}}%
\pgfpathlineto{\pgfqpoint{5.534545in}{0.696000in}}%
\pgfpathlineto{\pgfqpoint{5.534545in}{0.696000in}}%
\pgfusepath{stroke}%
\end{pgfscope}%
\begin{pgfscope}%
\pgfsetrectcap%
\pgfsetmiterjoin%
\pgfsetlinewidth{0.803000pt}%
\definecolor{currentstroke}{rgb}{0.000000,0.000000,0.000000}%
\pgfsetstrokecolor{currentstroke}%
\pgfsetdash{}{0pt}%
\pgfpathmoveto{\pgfqpoint{0.800000in}{0.528000in}}%
\pgfpathlineto{\pgfqpoint{0.800000in}{4.224000in}}%
\pgfusepath{stroke}%
\end{pgfscope}%
\begin{pgfscope}%
\pgfsetrectcap%
\pgfsetmiterjoin%
\pgfsetlinewidth{0.803000pt}%
\definecolor{currentstroke}{rgb}{0.000000,0.000000,0.000000}%
\pgfsetstrokecolor{currentstroke}%
\pgfsetdash{}{0pt}%
\pgfpathmoveto{\pgfqpoint{5.760000in}{0.528000in}}%
\pgfpathlineto{\pgfqpoint{5.760000in}{4.224000in}}%
\pgfusepath{stroke}%
\end{pgfscope}%
\begin{pgfscope}%
\pgfsetrectcap%
\pgfsetmiterjoin%
\pgfsetlinewidth{0.803000pt}%
\definecolor{currentstroke}{rgb}{0.000000,0.000000,0.000000}%
\pgfsetstrokecolor{currentstroke}%
\pgfsetdash{}{0pt}%
\pgfpathmoveto{\pgfqpoint{0.800000in}{0.528000in}}%
\pgfpathlineto{\pgfqpoint{5.760000in}{0.528000in}}%
\pgfusepath{stroke}%
\end{pgfscope}%
\begin{pgfscope}%
\pgfsetrectcap%
\pgfsetmiterjoin%
\pgfsetlinewidth{0.803000pt}%
\definecolor{currentstroke}{rgb}{0.000000,0.000000,0.000000}%
\pgfsetstrokecolor{currentstroke}%
\pgfsetdash{}{0pt}%
\pgfpathmoveto{\pgfqpoint{0.800000in}{4.224000in}}%
\pgfpathlineto{\pgfqpoint{5.760000in}{4.224000in}}%
\pgfusepath{stroke}%
\end{pgfscope}%
\begin{pgfscope}%
\pgfsetbuttcap%
\pgfsetmiterjoin%
\definecolor{currentfill}{rgb}{1.000000,1.000000,1.000000}%
\pgfsetfillcolor{currentfill}%
\pgfsetfillopacity{0.800000}%
\pgfsetlinewidth{1.003750pt}%
\definecolor{currentstroke}{rgb}{0.800000,0.800000,0.800000}%
\pgfsetstrokecolor{currentstroke}%
\pgfsetstrokeopacity{0.800000}%
\pgfsetdash{}{0pt}%
\pgfpathmoveto{\pgfqpoint{5.591944in}{4.025389in}}%
\pgfpathlineto{\pgfqpoint{5.653056in}{4.025389in}}%
\pgfpathquadraticcurveto{\pgfqpoint{5.683611in}{4.025389in}}{\pgfqpoint{5.683611in}{4.055944in}}%
\pgfpathlineto{\pgfqpoint{5.683611in}{4.117056in}}%
\pgfpathquadraticcurveto{\pgfqpoint{5.683611in}{4.147611in}}{\pgfqpoint{5.653056in}{4.147611in}}%
\pgfpathlineto{\pgfqpoint{5.591944in}{4.147611in}}%
\pgfpathquadraticcurveto{\pgfqpoint{5.561389in}{4.147611in}}{\pgfqpoint{5.561389in}{4.117056in}}%
\pgfpathlineto{\pgfqpoint{5.561389in}{4.055944in}}%
\pgfpathquadraticcurveto{\pgfqpoint{5.561389in}{4.025389in}}{\pgfqpoint{5.591944in}{4.025389in}}%
\pgfpathclose%
\pgfusepath{stroke,fill}%
\end{pgfscope}%
\end{pgfpicture}%
\makeatother%
\endgroup%

    \label{fig:diff}
    \caption{Difference in plots of \prettyref{fig:cdf}.}
\end{figure}

\section{Discussion}

We observe a distinct difference in the pattern of plots in \prettyref{fig:latencies}. In particular, w:1 is shown to have multiple peaks on the latencies of which the operations tend to be acknowledged. This is likely to be due to MongoDB's need to persist the operations with w:1 write concern separately from acknowledging them, which results in certain operations being blocked due to needing to wait until the batch is persisted, before the operation can be applied to the data and acknowledged. It is likely that due to this "waiting" that peaks in latency tend to form, as MongoDB forces operations to wait while it attempts to persist a batch of older operations. This behaviour is not seen in the journaled write concern because every write is offloaded onto the WiredTiger journal, which is then solely in charge of deciding when to persist those operations, separate from MongoDB. We should emphasize that this is conjecture and more research is required to fully understand this behaviour.

Additionally, we see that the peaks and troughs of w:1 and journaled plots respectively align from ~170ms to 300ms. We do not know why these alignments occur, but postulate once again that the WritedTiger journal may be playing a role in this behaviour. This further highlights the need for further research, specifically into the intrinsics of WiredTiger and its interaction with MongoDB.

We also note that the journaled execution has a higher average latency, clumping around 200ms, which is expected given the additional work required by the journaling process before operation can be acknowledged. However, journaled execution also seems more stable than its w:1 counterpart, with fewer operations appearing with more than 500ms latency. This stability can also be attributed to WiredTiger, as MongoDB can offload the responsibility of persisting writes onto a separate application. The instability of w:1 is therefore a consequence of MongoDB needing to decide when to persist operations, forcing the incoming operations to wait. Specifically, if an incoming operation is found to be dependent on an operation currently being persisted, the operation will need to wait until after the flush to disk has been performed in order to be applied to memory and acknowledged.

Moving on to the cumulative distribution of w:1 acknowledgements and estimated 1-durability in \prettyref{fig:cdf}, we find very predictable results, as w:1 operations get acknowledged long before those operations become 1-durable. We see an interesting result after 300ms, where the fraction of 1-durable writes overtakes w:1. This is likely due to our estimation technique, as we used the execution history of journaled writes to estimate 1-durability. As such, the increased instability of operations with the w:1 write concern in the higher latencies is the cause of this behaviour.

The differences between the two plots as shown in \prettyref{fig:diff} show a similar pattern. The difference is small early on as most operations are not acknowledged by either write concern in the earlier latencies but increases rapidly until 200ms as the majority w:1 operations get acknowledged but few of the journaled operations do. The difference then drops quite sharply towards zero by 300ms as most of the journaled operations are now being acknowledged. We chose not to graph the negative difference (where 1-durable overtakes w:1).